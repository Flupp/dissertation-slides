\documentclass{beamer}

% SPDX-License-Identifier: CC-BY-4.0 OR MIT-0
% Copyright 2018 Toni Dietze
%
\usefonttheme{professionalfonts}

% LuaLaTeX specific packages
\usepackage{fontspec}
	\defaultfontfeatures{Ligatures=TeX}
\usepackage{polyglossia}
	\setdefaultlanguage{english}
\usepackage{amsmath}  % has to be loaded before unicode-math
\usepackage[math-style=ISO]{unicode-math}
	\setmathfont{Latin Modern Math}
% 	\setmathfont[range={\mathcal,\mathbfcal},StylisticSet=1]{xits-math.otf}
% 	\setmathfont[range={"029F5}]{XITS Math}  % ⧵
% 	\setmathfont[range={\mathscr,\mathbfscr},StylisticSet=1]{Latin Modern Math}  % make \mathscr use the correct font

\usepackage[noend]{algpseudocode}
	\algrenewcommand\algorithmicrequire{\textbf{Input:}}
	\algrenewcommand\algorithmicensure{\textbf{Output:}}
\usepackage[backend=biber, maxbibnames=42, maxcitenames=42, sorting=ynt, style=authoryear]{biblatex}
\usepackage{csquotes}
\usepackage{mathtools}
\usepackage{media9}
\usepackage{scalerel}
\usepackage{standalone}
\usepackage{tikz}
	\usetikzlibrary{arrows.meta}
	\usetikzlibrary{backgrounds}
	\usetikzlibrary{calc}
	\usetikzlibrary{decorations}
	\usetikzlibrary{decorations.pathmorphing}
	\usetikzlibrary{decorations.pathreplacing}
	\usetikzlibrary{fadings}
	\usetikzlibrary{fit}
	\usetikzlibrary{graphs}
	\usetikzlibrary{graphdrawing}
	\usetikzlibrary{intersections}
	\usetikzlibrary{positioning}
	\usetikzlibrary{quotes}
	\usetikzlibrary{shadows.blur}
	\usetikzlibrary{shapes.arrows}
	\usetikzlibrary{shapes.geometric}
	\usegdlibrary{trees}
\usepackage{xifthen}
\usepackage{xspace}

\usepackage{pgfplots}
	\pgfplotsset
		{ compat = 1.15
		, /pgf/number format/1000 sep = {\,}
		, /pgf/number format/assume math mode = true
		, every axis plot/.append style =
			{ mark options = {fill opacity = 0.25}
			}
		}
	\usepgfplotslibrary{groupplots}
\usepackage{pgfplotstable}

\hypersetup
	{ bookmarksopen
	, pdflang = en
	, unicode
	}


%%%%%%%%%%%%%%%%%%%%%%%%%%%%%%%%%%%%%%%%%%%%%%%%%%%%%%%%%%%%%%%%%%%%%%%%%%%%%%


% always show bad boxes
%\overfullrule=1em


%%%%%%%%%%%%%%%%%%%%%%%%%%%%%%%%%%%%%%%%%%%%%%%%%%%%%%%%%%%%%%%%%%%%%%%%%%%%%%
% biblatex
%%%%%%%%%%%%%%%%%%%%%%%%%%%%%%%%%%%%%%%%%%%%%%%%%%%%%%%%%%%%%%%%%%%%%%%%%%%%%%

\addbibresource{slides-dissertation-defense.bib}
% \renewcommand*{\finalnamedelim}{\addcomma\space}
% \setlength{\bibitemsep}{1em}
% 
\AtEveryBibitem{% Clean up the bibtex rather than editing it
 \clearlist{address}
 \clearfield{date}
 \clearfield{eprint}
 \clearfield{isbn}
 \clearfield{issn}
 \clearlist{language}
 \clearlist{location}
 \clearfield{month}
 \clearfield{series}
%  \clearfield{url}
%  \clearfield{doi}
 \clearfield{organization}

%  \ifentrytype{book}{}{% Remove stuff except for books
%   \clearfield{booktitle}
%   \clearfield{pages}
  \clearlist{publisher}
  \clearname{editor}
%  }
}
% do not print url if doi is present
% http://tex.stackexchange.com/questions/154864/biblatex-use-doi-only-if-there-is-no-url
\DeclareSourcemap{
	\maps[datatype=bibtex]{
		\map{
			\step[fieldsource=doi,final]
			\step[fieldset=url,null]
}	}	}
%
% remove qoutes around titles
\DeclareFieldFormat
	[article,inbook,incollection,inproceedings,patent,thesis,unpublished]
	{title}{#1\isdot}
% 
% \DeclareFieldFormat{url}{\mkbibacro{URL}\addcolon\addnbspace\url{#1}}
% 
% \DeclareNameAlias{sortname}{first-last}
% 
\renewbibmacro{in:}{\ifentrytype{article}{}{}}


%%%%%%%%%%%%%%%%%%%%%%%%%%%%%%%%%%%%%%%%%%%%%%%%%%%%%%%%%%%%%%%%%%%%%%%%%%%%%%
% beamer
%%%%%%%%%%%%%%%%%%%%%%%%%%%%%%%%%%%%%%%%%%%%%%%%%%%%%%%%%%%%%%%%%%%%%%%%%%%%%%

\useoutertheme{infolines}
\makeatletter
% based on
% /usr/share/texmf-dist/tex/latex/beamer/beamerouterthemeinfolines.sty
\setbeamertemplate{footline}
{%
	\leavevmode%
	\hbox{%
	\begin{beamercolorbox}[wd=.333333\paperwidth,ht=2.25ex,dp=1ex,center]{author in head/foot}%
		\usebeamerfont{author in head/foot}\insertshortauthor\expandafter\beamer@ifempty\expandafter{\beamer@shortinstitute}{}{~~(\insertshortinstitute)}
	\end{beamercolorbox}%
	\begin{beamercolorbox}[wd=.333333\paperwidth,ht=2.25ex,dp=1ex,center]{title in head/foot}%
		\usebeamerfont{title in head/foot}\insertshorttitle
	\end{beamercolorbox}%
	\begin{beamercolorbox}[wd=.333333\paperwidth,ht=2.25ex,dp=1ex,right]{date in head/foot}%
		\usebeamerfont{date in head/foot}%
		\hfill\insertshortdate\hfill\hfill%
		%\hspace*{2ex}%
		%\insertshortdate%
		%\hspace{0pt plus 1 filll}%
		%(\insertframenumber.\insertoverlaynumber{} / \insertmainframenumber)%
		%\hspace{0pt plus 1 filll}%
		\phantom{000}\llap{\insertpagenumber} / \insertpresentationendpage%
		\hspace*{2ex}%
	\end{beamercolorbox}}%
	\vskip0pt%
}
\makeatother
\useinnertheme{circles}
\beamertemplatenavigationsymbolsempty
\setbeamertemplate{bibliography item}{}
\setbeamertemplate{headline}[default]

\input{tudcolors.tex}
\setbeamercolor*{alerted text}{fg=HKS07K100}
\usecolortheme[named=HKS41K100]{structure}

\setbeamercolor*{palette primary}{use=structure,fg=white,bg=structure.fg}
\setbeamercolor*{palette secondary}{use=structure,fg=white,bg=structure.fg!80}
\setbeamercolor*{palette tertiary}{use=structure,fg=white,bg=structure.fg!60}
\setbeamercolor*{palette quaternary}{fg=white,bg=black}

\setbeamercolor*{sidebar}{use=structure,bg=structure.fg}

\setbeamercolor*{palette sidebar primary}{use=structure,fg=structure.fg!20}
\setbeamercolor*{palette sidebar secondary}{fg=white}
\setbeamercolor*{palette sidebar tertiary}{use=structure,fg=structure.fg!40}
\setbeamercolor*{palette sidebar quaternary}{fg=white}

\setbeamercolor*{titlelike}{parent=palette primary}

\setbeamercolor*{separation line}{}
\setbeamercolor*{fine separation line}{}

\setbeamercolor{block title}{use=structure,fg=white,bg=structure.fg}
\setbeamercolor{block title alerted}{use=alerted text,fg=white,bg=alerted text.fg!75!black}
\setbeamercolor{block title example}{use=example text,fg=white,bg=example text.fg!75!black}

\setbeamercolor{block body}{parent=normal text,use=block title,bg=block title.bg!10!bg}
\setbeamercolor{block body alerted}{parent=normal text,use=block title alerted,bg=block title alerted.bg!10!bg}
\setbeamercolor{block body example}{parent=normal text,use=block title example,bg=block title example.bg!10!bg}

% \setbeamertemplate{itemize items}[default]


%%%%%%%%%%%%%%%%%%%%%%%%%%%%%%%%%%%%%%%%%%%%%%%%%%%%%%%%%%%%%%%%%%%%%%%%%%%%%%
% TikZ
%%%%%%%%%%%%%%%%%%%%%%%%%%%%%%%%%%%%%%%%%%%%%%%%%%%%%%%%%%%%%%%%%%%%%%%%%%%%%%

\tikzset
	{ > = Stealth
	}


%%%%%%%%%%%%%%%%%%%%%%%%%%%%%%%%%%%%%%%%%%%%%%%%%%%%%%%%%%%%%%%%%%%%%%%%%%%%%%
% general commands and styles
%%%%%%%%%%%%%%%%%%%%%%%%%%%%%%%%%%%%%%%%%%%%%%%%%%%%%%%%%%%%%%%%%%%%%%%%%%%%%%

% \delegateStyle and \inheritStyle command
% usage: \delegateStyle{… \inheritStyle{…} …}
% example: \(X_{\delegateStyle{\fbox{\inheritStyle{X}}}}\)
% Save the current style and regain it in the argument.
% This works both for math and text mode, and can be nested.
% Acknowledgments: Based on \ThisStyle and \SavedStyle from scalerel package.
\makeatletter
\newcommand*{\@inheritStyle@D}[1]{\(\displaystyle      #1\)}
\newcommand*{\@inheritStyle@T}[1]{\(\textstyle         #1\)}
\newcommand*{\@inheritStyle@S}[1]{\(\scriptstyle       #1\)}
\newcommand*{\@inheritStyle@s}[1]{\(\scriptscriptstyle #1\)}
\newcommand*{\@inheritStyle@t}[1]{#1}
\newcommand*{\inheritStyle}{\csname @inheritStyle@\@inheritStyleSwitch\endcsname}
\newcommand*{\delegateStyle}[1]{%
	\ifmmode%
		\mathchoice%
		{\edef\@inheritStyleSwitch{D}#1}%
		{\edef\@inheritStyleSwitch{T}#1}%
		{\edef\@inheritStyleSwitch{S}#1}%
		{\edef\@inheritStyleSwitch{s}#1}%
	\else%
		\edef\@inheritStyleSwitch{t}#1%
	\fi%
}
\makeatother


% \oalt command
% requires: \delegateStyle and \inheritStyle command
% usage: \oalt<…>[…]{…}{…} (cf. \alt)
% Behaves like \alt, but reserves space according to largest overlays.
% The optional argument defines the alignment inside the reserved space;
% it is one of c, l, r, s (cf. \makebox); the default is c.
\makeatletter
\newlength{\oalt@dp}
\newlength{\oalt@ht}
\newlength{\oalt@wd}
\newbox{\oalt@a}
\newbox{\oalt@b}
\newbox{\oalt@empty}
\newcommand<>*{\oalt}[3][c]{%
	\delegateStyle{%
		% based on \setto… in /usr/share/texmf-dist/tex/latex/base/latex.ltx
		\setbox\oalt@a\hbox{\inheritStyle{#2}}%
		\setbox\oalt@b\hbox{\inheritStyle{#3}}%
		\pgfmathsetlength{\oalt@dp}{max(\dp\oalt@a,\dp\oalt@b)}%
		\pgfmathsetlength{\oalt@ht}{max(\ht\oalt@a,\ht\oalt@b)}%
		\pgfmathsetlength{\oalt@wd}{max(\wd\oalt@a,\wd\oalt@b)}%
		\raisebox{0pt}[\oalt@ht][\oalt@dp]{%
			\makebox[\oalt@wd][#1]{%
				\alt#4{\unhbox\oalt@a}{\unhbox\oalt@b}%
			}%
		}%
		\setbox\oalt@a\box\oalt@empty%
		\setbox\oalt@b\box\oalt@empty%
	}%
}
\makeatother


% \otemporal command
% requires: \delegateStyle and \inheritStyle command
% usage: \otemporal<…>[…]{…}{…}{…} (cf. \temporal)
% Behaves like \temporal, but reserves space according to largest overlays.
% The optional argument defines the alignment inside the reserved space;
% it is one of c, l, r, s (cf. \makebox); the default is c.
\makeatletter
\newlength{\ot@dp}
\newlength{\ot@ht}
\newlength{\ot@wd}
\newbox{\ot@a}
\newbox{\ot@b}
\newbox{\ot@c}
\newbox{\ot@empty}
\newcommand<>*{\otemporal}[4][c]{%
	\delegateStyle{%
		% based on \setto… in /usr/share/texmf-dist/tex/latex/base/latex.ltx
		\setbox\ot@a\hbox{\inheritStyle{#2}}%
		\setbox\ot@b\hbox{\inheritStyle{#3}}%
		\setbox\ot@c\hbox{\inheritStyle{#4}}%
		\pgfmathsetlength{\ot@dp}{max(\dp\ot@a,\dp\ot@b,\dp\ot@c)}%
		\pgfmathsetlength{\ot@ht}{max(\ht\ot@a,\ht\ot@b,\ht\ot@c)}%
		\pgfmathsetlength{\ot@wd}{max(\wd\ot@a,\wd\ot@b,\wd\ot@c)}%
		\raisebox{0pt}[\ot@ht][\ot@dp]{%
			\makebox[\ot@wd][#1]{%
				\temporal#5{\unhbox\ot@a}{\unhbox\ot@b}{\unhbox\ot@c}%
			}%
		}%
		\setbox\ot@a\box\ot@empty%
		\setbox\ot@b\box\ot@empty%
		\setbox\ot@c\box\ot@empty%
	}%
}
\makeatother


% Resize delimiters like braces, brackets, etc.
% Parameters: size, left delimiter, formula, right delimiter
% Example: \delim2({\frac{1}{2}})
\newcommand*{\delim}[4]{%
	\ifcase#1%
		#2#3#4%
	\or%
		\bigl#2#3\bigr#4%
	\or%
		\Bigl#2#3\Bigr#4%
	\or%
		\biggl#2#3\biggr#4%
	\or%
		\Biggl#2#3\Biggr#4%
	\else%
		\left#2#3\right#4%
	\fi%
}


% similar to \fullcite, but using the formatting of \printbibliography
\newcommand*{\printfullcite}[1]{%
	\begin{refsection}%
		\nocite{#1}%
		\DeclareNameAlias{author}{first-last}%
		\printbibliography[heading = none]%
	\end{refsection}%
}


\colorlet{light alert}{HKS07K60}
\tikzset{alert.bg/.style={rounded corners, fill=light alert}}
\tikzset{every picture/.style={line cap=round, semithick}}
% http://tex.stackexchange.com/questions/6135/how-to-make-beamer-overlays-with-tikz-node
\tikzset{onslide/.code args={<#1>#2}{\only<#1>{\pgfkeysalso{#2}}}}
\tikzset{invisible/.code args={<#1>}{\alt<#1>{\pgfkeysalso{transparent}}{\pgfkeysalso{opaque}}}}
\tikzset{uncover/.code args={<#1>}{\alt<#1>{\pgfkeysalso{opaque}}{\pgfkeysalso{opacity=0.25}}}}
\tikzset{visible/.code args={<#1>}{\alt<#1>{\pgfkeysalso{opaque}}{\pgfkeysalso{transparent}}}}
\tikzset{vuncover/.code args=%
	{<#1><#2>}%
	{\alt<#1>%
		{\alt<#2>%
			{\pgfkeysalso{opaque}}%
			{\pgfkeysalso{opacity=0.25}}%
		}{\pgfkeysalso{transparent}}%
	}%
}

\newcommand<%
	>{\tikzhighlight}[2][]{%
	\delegateStyle{\alt#3%
		{\tikz[baseline=0, anchor=base, inner sep=0.2em, text height=, text depth=]{\node[alert.bg, #1]{\inheritStyle{#2}};}}%
		{\tikz[baseline=0, anchor=base, inner sep=0.2em, text height=, text depth=]{\node[#1, fill=none]{\inheritStyle{#2}};}}%
	}%
}

\newcommand{\mathhighlight}{\tikzhighlight}

\newcommand<>{\mhl}[2][]{\mathhighlight#3[inner sep=0.2em, #1]{#2}}


\newcommand<>{\inlineblock}[2][]{{%
	\usebeamercolor*[fg]{block body}%
	\tikzhighlight#3[fill=block body.bg, #1]{#2}%
}}


% a small letter s for plurals of abbreviations
\newcommand*{\s}{{\scriptsize s}\xspace}


\newcommand<>*{\sout}[2][opacity=0.75, ultra thick]{%
	\delegateStyle{%
		\tikz[baseline=0, anchor=base, inner sep=0, outer sep=0]{
			\useasboundingbox node (n) {\inheritStyle{#2}};
			\only#3{
				\node (h) {\inheritStyle{\ifmmode\mathstrut\else\strut\fi}};
				\draw[#1] (n.west |- {$(h.south)!0.5!(h.north)$}) -- (n.east |- {$(h.south)!0.5!(h.north)$});
			}
		}%
	}%
}


% tight style
% Sets outer sep to default inner sep and inner sep to 0.
% Use this style for nodes that are neither drawn nor filled to prevent
% unwanted growth of the bounding box.
\tikzset{tight/.style={inner sep=0, outer sep=0.3333em}}


% rounded tree edges style
% usage: rounded tree edges={⟨direction⟩}{⟨looseness⟩}{⟨strength⟩}
\tikzset{
	rounded tree edges/.style n args={3}{
	edge from parent path={
	let
		\n{direction}={#1},
		\n{looseness}={#2},
		\n{strength}={#3},
		\p1=(\tikzparentnode),
		\p2=(\tikzchildnode),
		\p3=(\n{direction}:1pt),
		\p4=(\x2 - \x1, \y2 - \y1),
		\n{dist}={veclen(\p4)},
		\p4=(\x4 / \n{dist}, \y4 / \n{dist}),
		\n{angle}={atan2(\y4, \x4)},
		\n{delta}={Mod(\n{angle} - \n{direction}, 360)},
		\n{delta}={\n{delta} > 180 ? \n{delta} - 360  : \n{delta}},
		\n{delta}={\n{delta} >  90 ?  180 - \n{delta} : \n{delta}},
		\n{delta}={\n{delta} < -90 ? -180 - \n{delta} : \n{delta}}
	in (\tikzparentnode) .. controls
		+(    \n{angle}+\n{strength}*\n{delta}:\n{looseness}*0.3915*\n{dist}) and
		+(180+\n{angle}-\n{strength}*\n{delta}:\n{looseness}*0.3915*\n{dist}) ..
		(\tikzchildnode)
	}
	}
}


% Tear out snippets from PDFs.
% Usage: \tear[…]{file.pdf}
% The optional parameter is the same as for \includegraphics.
% Useful Arguments:
%   * page=‹pagenumber›
%   * trim=‹left› ‹bottom› ‹right› ‹top›
%   * width=0.98\linewidth
\newcommand*{\tear}[2][]{%
	\begin{tikzpicture}
		\node
			[ blur shadow
			, clip
			, decorate
			, decoration=random steps
			, draw
			, inner sep=0
			, preaction={fill=white}% hide the shadow if paper is transparent
			] {\includegraphics[#1]{#2}};
	\end{tikzpicture}%
}


\makeatletter
\newcommand*{\timeline}[3][0]{%
	\ifcsname timeline@cmd@#3\endcsname%
		\@timeline[#1]{#2}{#3}%
		\PackageWarning{timeline}{redefining timeline \@backslashchar\string#3}%
	\else%
		\ifcsname#3\endcsname%
			\errmessage{Command \@backslashchar\string#3 already defined}%
		\else%
			\@timeline[#1]{#2}{#3}%
		\fi%
	\fi%
}%
\newcommand*{\@timeline}[3][0]{%
	% mark command as timeline command – they can be overwritten
	\expandafter\def\csname timeline@cmd@#3\endcsname{}%
	\setcounter{@timeline}{#1}%
	\def\timeline@cmd{#3}%
	\timeline@reset%
	\timeline@append{0}%
	\@tfor\timeline@next:=#2\do{%
		\if\timeline@next+%
			\stepcounter{@timeline}%
			\timeline@append{,\the@timeline}%
		\else\if\timeline@next-%
			\stepcounter{@timeline}%
		\else%
			%\timeline@append{\timeline@next}%
			\GenericError{}{\protect\timeline: ignoring unknown character: \timeline@next}%
		\fi\fi%
	}%
}%
% \newcommand*{\tl}[1]{%
% 	\ifcsname timeline@cmd@#1\endcsname%
% 		\csname timeline@cmd@#1\endcsname%
% 	\else%
% 		0%
% 		%\GenericError{}{\protect\tl: timeline not defined: #1}%
% 	\fi%
% }%
\newcounter{@timeline}%
\def\timeline@reset{%
	\expandafter\def\csname\timeline@cmd\endcsname{}%
}%
\def\timeline@append#1{%
	\expandafter\edef\csname\timeline@cmd\endcsname{%
		\csname\timeline@cmd\endcsname#1%
	}%
}%
\makeatother


\newcommand*{\xminus}[1]{%
	\mathrel{\tikz[baseline={([yshift=-0.25em]n.south)}, inner sep=0, outer sep=0.2em]{%
		\node (n) {\(\scriptstyle #1\)};
		\draw (n.south west) -- (n.south east);
	}}%
}
\newcommand*{\tikzrightarrow}[1]{%
	\mathrel{\tikz[baseline={([yshift=-0.25em]n.south)}, inner sep=0, outer sep=0.2em]{%
		\node (n) {\(\scriptstyle #1\)};
		\draw[->, > = Computer Modern Rightarrow, line width = 0.4pt] (n.south west) -- (n.south east);
	}}%
}


%%%%%%%%%%%%%%%%%%%%%%%%%%%%%%%%%%%%%%%%%%%%%%%%%%%%%%%%%%%%%%%%%%%%%%%%%%%%%%
% document specific commands
%%%%%%%%%%%%%%%%%%%%%%%%%%%%%%%%%%%%%%%%%%%%%%%%%%%%%%%%%%%%%%%%%%%%%%%%%%%%%%

\newcommand<>*{\mycite}[1]{\uncover#2{{\color{HKS57K100}[\cite{#1}]}}}


\newcommand{\statetree}[1]{
	\tikz
	[ anchor=base
	, baseline=(current bounding box.center)
	, level distance=2em
	, sibling distance=2em
	]{
		\matrix
		[ draw=nt
		, edge from parent/.style={draw=black}
		, inner sep=0
		, nodes={inner sep=0.2em, rounded corners=0}
		, rounded corners
		] {#1\\}
	}
}


\newcommand*{\mylargeleaf}[1]{{\LARGE\color{HKS41K70}#1}}

\definecolor{state s}{named}{HKS57K80}
\definecolor{state t}{named}{HKS41K70}
\newcommand*{\stateS}[1]{{\color{state s}#1}}
\newcommand*{\stateT}[1]{{\color{state t}#1}}

\tikzset{
	subtree/.style =
		{ fill=lightgray
		, inner sep=0.02em
		, isosceles triangle apex angle=60
		, shape=isosceles triangle
		, shape border rotate=90
		}
	, state/.style = {circle, draw, inner sep=0.1em}
	, trans/.style = {rectangle, draw}
}

\newcommand*{\srBool}{\mathbb{B}}
\newcommand*{\srProb}{ℙ}


%%%%%%%%%%%%%%%%%%%%%%%%%%%%%%%%%%%%%%%%%%%%%%%%%%%%%%%%%%%%%%%%%%%%%%%%%%%%%%
% commands for specific notations
%%%%%%%%%%%%%%%%%%%%%%%%%%%%%%%%%%%%%%%%%%%%%%%%%%%%%%%%%%%%%%%%%%%%%%%%%%%%%%

\DeclareMathOperator*{\argmax}{argmax}

\newcommand*{\cardinality}[1]{\lvert#1\rvert}
\newcommand*{\corpussize}[1]{\lvert#1\rvert}

\DeclareMathOperator{\crispOp}{crisp}
\newcommand*        {\crisp}[2][0]{\crispOp\delim{#1}({#2})}

\DeclareMathOperator{\lhsOp}{lhs}
\newcommand*{\lhs}[1]{\lhsOp(#1)}

\DeclareMathOperator{\lklhdOp}{L}
\newcommand*{\lklhd}[2]{\lklhdOp(#1 ∣ #2)}

\DeclareMathOperator{\mleOp}{mle}
\newcommand*{\mle}[2][]{%
	\ifthenelse{\isempty{#1}}{%
		\mleOp(#2)%
	}{%
		\mleOp_{#1}(#2)%
	}%
}

\DeclareMathOperator{\mrg}{merge}

% CVD: color vision deficiencies
\definecolor{CVD light red}   {HTML}{FF8080}
\definecolor{CVD light yellow}{HTML}{FFFF80}
\definecolor{CVD light green} {HTML}{40FFC0}

\definecolor{nt}{named}{HKS41K70}
\newcommand*{\nt}[1]{{\color{nt}#1}}

% set of all probability distributions over #1
\DeclareMathOperator{\pdsOp}{Pd}
\newcommand*{\pds}[1]{\pdsOp(#1)}

\DeclareMathOperator{\positionsOp}{pos}
\newcommand*{\positions}[1]{\positionsOp(#1)}

\DeclareMathOperator{\rankOp}{rk}
\newcommand*{\rank}[1]{\rankOp(#1)}

\DeclareMathOperator{\runsOp}{run}
\newcommand*{\runs}[2][]{%
	\ifthenelse%
		{\isempty{#1}}%
		{\runsOp(#2)}%
		{\runsOp_{#1}(#2)}%
}

\newcommand*{\semantics}[1]{⟦#1⟧}

\DeclareMathOperator{\splt}{split}

\newcommand*{\subtree}[2]{#1|_{#2}}

\DeclareMathOperator{\supportOp}{supp}
\newcommand*{\support}[1]{\supportOp(#1)}

\newcommand*{\symId}{\textsc{\color{gray}Id}}
\newcommand*{\symCons}{\textsc{\color{gray}Cons}}
\newcommand*{\symFlip}{\textsc{\color{gray}Flip}}
\newcommand*{\symNull}{\textsc{\color{gray}Null}}
\newcommand*{\symNullR}{\textsc{\color{gray}N\(\overline{\textsc{ull}}\)}}
\newcommand*{\symSnoc}{\textsc{\color{gray}Snoc}}

\newcommand*{\transWTA}[4][]{#3 \xrightarrow{#1} #2(#4)}

\DeclareMathOperator{\uniqueRunOp}{r}
\newcommand*{\uniqueRun}[2][]{%
	\ifthenelse%
		{\isempty{#1}}%
		{\uniqueRunOp^{#2}}%
		{\uniqueRunOp_{\!#1}^{#2}}%
}

\DeclareMathOperator{\treesOp}{T}
\newcommand*{\trees}[2][]{%
	\ifthenelse%
		{\isempty{#1}}%
		{\treesOp_{\!#2}}%
		{\treesOp_{\!#2}(#1)}%
}
\DeclareMathOperator{\treesUOp}{U}
\newcommand*{\treesU}[2][]{%
	\ifthenelse%
		{\isempty{#1}}%
		{\treesUOp_{#2}}%
		{\treesUOp_{#2}(#1)}%
}


%%%%%%%%%%%%%%%%%%%%%%%%%%%%%%%%%%%%%%%%%%%%%%%%%%%%%%%%%%%%%%%%%%%%%%%%%%%%%%
% metadata
%%%%%%%%%%%%%%%%%%%%%%%%%%%%%%%%%%%%%%%%%%%%%%%%%%%%%%%%%%%%%%%%%%%%%%%%%%%%%%

\ifstandalonebeamer\else
	\title[Defense of Dissertation]{A Formal View on Training of Weighted Tree Automata by Likelihood-Driven State Splitting and Merging}
	\subtitle{Defense of Dissertation}
\fi
\author{Toni Dietze}
\institute[TU Dresden]{%
	\href{https://www.orchid.inf.tu-dresden.de/index.en/}{Chair for Foundations of Programming}
\\	\href{https://tu-dresden.de/ing/informatik/thi}{Institute of Theoretical Computer Science}
\\	\href{https://tu-dresden.de/ing/informatik}{Faculty of Computer Science}
\\	\href{https://tu-dresden.de/}{Technische Universität Dresden}
\\	01062 Dresden, Germany
}
\date[2018-09-27]{September 27, 2018}


% \includeonlyframes{current}
% [label=current]


\begin{document}

\begin{frame}[plain]
	\maketitle
\end{frame}


\section{Natural Language Processing}

\begin{frame}{\secname}
	% SPDX-License-Identifier: CC-BY-4.0
% Copyright 2018 Toni Dietze
\documentclass[beamer]{standalone}
% SPDX-License-Identifier: CC-BY-4.0 OR MIT-0
% Copyright 2018 Toni Dietze
%
\usefonttheme{professionalfonts}

% LuaLaTeX specific packages
\usepackage{fontspec}
	\defaultfontfeatures{Ligatures=TeX}
\usepackage{polyglossia}
	\setdefaultlanguage{english}
\usepackage{amsmath}  % has to be loaded before unicode-math
\usepackage[math-style=ISO]{unicode-math}
	\setmathfont{Latin Modern Math}
% 	\setmathfont[range={\mathcal,\mathbfcal},StylisticSet=1]{xits-math.otf}
% 	\setmathfont[range={"029F5}]{XITS Math}  % ⧵
% 	\setmathfont[range={\mathscr,\mathbfscr},StylisticSet=1]{Latin Modern Math}  % make \mathscr use the correct font

\usepackage[noend]{algpseudocode}
	\algrenewcommand\algorithmicrequire{\textbf{Input:}}
	\algrenewcommand\algorithmicensure{\textbf{Output:}}
\usepackage[backend=biber, maxbibnames=42, maxcitenames=42, sorting=ynt, style=authoryear]{biblatex}
\usepackage{csquotes}
\usepackage{mathtools}
\usepackage{media9}
\usepackage{scalerel}
\usepackage{standalone}
\usepackage{tikz}
	\usetikzlibrary{arrows.meta}
	\usetikzlibrary{backgrounds}
	\usetikzlibrary{calc}
	\usetikzlibrary{decorations}
	\usetikzlibrary{decorations.pathmorphing}
	\usetikzlibrary{decorations.pathreplacing}
	\usetikzlibrary{fadings}
	\usetikzlibrary{fit}
	\usetikzlibrary{graphs}
	\usetikzlibrary{graphdrawing}
	\usetikzlibrary{intersections}
	\usetikzlibrary{positioning}
	\usetikzlibrary{quotes}
	\usetikzlibrary{shadows.blur}
	\usetikzlibrary{shapes.arrows}
	\usetikzlibrary{shapes.geometric}
	\usegdlibrary{trees}
\usepackage{xifthen}
\usepackage{xspace}

\usepackage{pgfplots}
	\pgfplotsset
		{ compat = 1.15
		, /pgf/number format/1000 sep = {\,}
		, /pgf/number format/assume math mode = true
		, every axis plot/.append style =
			{ mark options = {fill opacity = 0.25}
			}
		}
	\usepgfplotslibrary{groupplots}
\usepackage{pgfplotstable}

\hypersetup
	{ bookmarksopen
	, pdflang = en
	, unicode
	}


%%%%%%%%%%%%%%%%%%%%%%%%%%%%%%%%%%%%%%%%%%%%%%%%%%%%%%%%%%%%%%%%%%%%%%%%%%%%%%


% always show bad boxes
%\overfullrule=1em


%%%%%%%%%%%%%%%%%%%%%%%%%%%%%%%%%%%%%%%%%%%%%%%%%%%%%%%%%%%%%%%%%%%%%%%%%%%%%%
% biblatex
%%%%%%%%%%%%%%%%%%%%%%%%%%%%%%%%%%%%%%%%%%%%%%%%%%%%%%%%%%%%%%%%%%%%%%%%%%%%%%

\addbibresource{slides-dissertation-defense.bib}
% \renewcommand*{\finalnamedelim}{\addcomma\space}
% \setlength{\bibitemsep}{1em}
% 
\AtEveryBibitem{% Clean up the bibtex rather than editing it
 \clearlist{address}
 \clearfield{date}
 \clearfield{eprint}
 \clearfield{isbn}
 \clearfield{issn}
 \clearlist{language}
 \clearlist{location}
 \clearfield{month}
 \clearfield{series}
%  \clearfield{url}
%  \clearfield{doi}
 \clearfield{organization}

%  \ifentrytype{book}{}{% Remove stuff except for books
%   \clearfield{booktitle}
%   \clearfield{pages}
  \clearlist{publisher}
  \clearname{editor}
%  }
}
% do not print url if doi is present
% http://tex.stackexchange.com/questions/154864/biblatex-use-doi-only-if-there-is-no-url
\DeclareSourcemap{
	\maps[datatype=bibtex]{
		\map{
			\step[fieldsource=doi,final]
			\step[fieldset=url,null]
}	}	}
%
% remove qoutes around titles
\DeclareFieldFormat
	[article,inbook,incollection,inproceedings,patent,thesis,unpublished]
	{title}{#1\isdot}
% 
% \DeclareFieldFormat{url}{\mkbibacro{URL}\addcolon\addnbspace\url{#1}}
% 
% \DeclareNameAlias{sortname}{first-last}
% 
\renewbibmacro{in:}{\ifentrytype{article}{}{}}


%%%%%%%%%%%%%%%%%%%%%%%%%%%%%%%%%%%%%%%%%%%%%%%%%%%%%%%%%%%%%%%%%%%%%%%%%%%%%%
% beamer
%%%%%%%%%%%%%%%%%%%%%%%%%%%%%%%%%%%%%%%%%%%%%%%%%%%%%%%%%%%%%%%%%%%%%%%%%%%%%%

\useoutertheme{infolines}
\makeatletter
% based on
% /usr/share/texmf-dist/tex/latex/beamer/beamerouterthemeinfolines.sty
\setbeamertemplate{footline}
{%
	\leavevmode%
	\hbox{%
	\begin{beamercolorbox}[wd=.333333\paperwidth,ht=2.25ex,dp=1ex,center]{author in head/foot}%
		\usebeamerfont{author in head/foot}\insertshortauthor\expandafter\beamer@ifempty\expandafter{\beamer@shortinstitute}{}{~~(\insertshortinstitute)}
	\end{beamercolorbox}%
	\begin{beamercolorbox}[wd=.333333\paperwidth,ht=2.25ex,dp=1ex,center]{title in head/foot}%
		\usebeamerfont{title in head/foot}\insertshorttitle
	\end{beamercolorbox}%
	\begin{beamercolorbox}[wd=.333333\paperwidth,ht=2.25ex,dp=1ex,right]{date in head/foot}%
		\usebeamerfont{date in head/foot}%
		\hfill\insertshortdate\hfill\hfill%
		%\hspace*{2ex}%
		%\insertshortdate%
		%\hspace{0pt plus 1 filll}%
		%(\insertframenumber.\insertoverlaynumber{} / \insertmainframenumber)%
		%\hspace{0pt plus 1 filll}%
		\phantom{000}\llap{\insertpagenumber} / \insertpresentationendpage%
		\hspace*{2ex}%
	\end{beamercolorbox}}%
	\vskip0pt%
}
\makeatother
\useinnertheme{circles}
\beamertemplatenavigationsymbolsempty
\setbeamertemplate{bibliography item}{}
\setbeamertemplate{headline}[default]

\input{tudcolors.tex}
\setbeamercolor*{alerted text}{fg=HKS07K100}
\usecolortheme[named=HKS41K100]{structure}

\setbeamercolor*{palette primary}{use=structure,fg=white,bg=structure.fg}
\setbeamercolor*{palette secondary}{use=structure,fg=white,bg=structure.fg!80}
\setbeamercolor*{palette tertiary}{use=structure,fg=white,bg=structure.fg!60}
\setbeamercolor*{palette quaternary}{fg=white,bg=black}

\setbeamercolor*{sidebar}{use=structure,bg=structure.fg}

\setbeamercolor*{palette sidebar primary}{use=structure,fg=structure.fg!20}
\setbeamercolor*{palette sidebar secondary}{fg=white}
\setbeamercolor*{palette sidebar tertiary}{use=structure,fg=structure.fg!40}
\setbeamercolor*{palette sidebar quaternary}{fg=white}

\setbeamercolor*{titlelike}{parent=palette primary}

\setbeamercolor*{separation line}{}
\setbeamercolor*{fine separation line}{}

\setbeamercolor{block title}{use=structure,fg=white,bg=structure.fg}
\setbeamercolor{block title alerted}{use=alerted text,fg=white,bg=alerted text.fg!75!black}
\setbeamercolor{block title example}{use=example text,fg=white,bg=example text.fg!75!black}

\setbeamercolor{block body}{parent=normal text,use=block title,bg=block title.bg!10!bg}
\setbeamercolor{block body alerted}{parent=normal text,use=block title alerted,bg=block title alerted.bg!10!bg}
\setbeamercolor{block body example}{parent=normal text,use=block title example,bg=block title example.bg!10!bg}

% \setbeamertemplate{itemize items}[default]


%%%%%%%%%%%%%%%%%%%%%%%%%%%%%%%%%%%%%%%%%%%%%%%%%%%%%%%%%%%%%%%%%%%%%%%%%%%%%%
% TikZ
%%%%%%%%%%%%%%%%%%%%%%%%%%%%%%%%%%%%%%%%%%%%%%%%%%%%%%%%%%%%%%%%%%%%%%%%%%%%%%

\tikzset
	{ > = Stealth
	}


%%%%%%%%%%%%%%%%%%%%%%%%%%%%%%%%%%%%%%%%%%%%%%%%%%%%%%%%%%%%%%%%%%%%%%%%%%%%%%
% general commands and styles
%%%%%%%%%%%%%%%%%%%%%%%%%%%%%%%%%%%%%%%%%%%%%%%%%%%%%%%%%%%%%%%%%%%%%%%%%%%%%%

% \delegateStyle and \inheritStyle command
% usage: \delegateStyle{… \inheritStyle{…} …}
% example: \(X_{\delegateStyle{\fbox{\inheritStyle{X}}}}\)
% Save the current style and regain it in the argument.
% This works both for math and text mode, and can be nested.
% Acknowledgments: Based on \ThisStyle and \SavedStyle from scalerel package.
\makeatletter
\newcommand*{\@inheritStyle@D}[1]{\(\displaystyle      #1\)}
\newcommand*{\@inheritStyle@T}[1]{\(\textstyle         #1\)}
\newcommand*{\@inheritStyle@S}[1]{\(\scriptstyle       #1\)}
\newcommand*{\@inheritStyle@s}[1]{\(\scriptscriptstyle #1\)}
\newcommand*{\@inheritStyle@t}[1]{#1}
\newcommand*{\inheritStyle}{\csname @inheritStyle@\@inheritStyleSwitch\endcsname}
\newcommand*{\delegateStyle}[1]{%
	\ifmmode%
		\mathchoice%
		{\edef\@inheritStyleSwitch{D}#1}%
		{\edef\@inheritStyleSwitch{T}#1}%
		{\edef\@inheritStyleSwitch{S}#1}%
		{\edef\@inheritStyleSwitch{s}#1}%
	\else%
		\edef\@inheritStyleSwitch{t}#1%
	\fi%
}
\makeatother


% \oalt command
% requires: \delegateStyle and \inheritStyle command
% usage: \oalt<…>[…]{…}{…} (cf. \alt)
% Behaves like \alt, but reserves space according to largest overlays.
% The optional argument defines the alignment inside the reserved space;
% it is one of c, l, r, s (cf. \makebox); the default is c.
\makeatletter
\newlength{\oalt@dp}
\newlength{\oalt@ht}
\newlength{\oalt@wd}
\newbox{\oalt@a}
\newbox{\oalt@b}
\newbox{\oalt@empty}
\newcommand<>*{\oalt}[3][c]{%
	\delegateStyle{%
		% based on \setto… in /usr/share/texmf-dist/tex/latex/base/latex.ltx
		\setbox\oalt@a\hbox{\inheritStyle{#2}}%
		\setbox\oalt@b\hbox{\inheritStyle{#3}}%
		\pgfmathsetlength{\oalt@dp}{max(\dp\oalt@a,\dp\oalt@b)}%
		\pgfmathsetlength{\oalt@ht}{max(\ht\oalt@a,\ht\oalt@b)}%
		\pgfmathsetlength{\oalt@wd}{max(\wd\oalt@a,\wd\oalt@b)}%
		\raisebox{0pt}[\oalt@ht][\oalt@dp]{%
			\makebox[\oalt@wd][#1]{%
				\alt#4{\unhbox\oalt@a}{\unhbox\oalt@b}%
			}%
		}%
		\setbox\oalt@a\box\oalt@empty%
		\setbox\oalt@b\box\oalt@empty%
	}%
}
\makeatother


% \otemporal command
% requires: \delegateStyle and \inheritStyle command
% usage: \otemporal<…>[…]{…}{…}{…} (cf. \temporal)
% Behaves like \temporal, but reserves space according to largest overlays.
% The optional argument defines the alignment inside the reserved space;
% it is one of c, l, r, s (cf. \makebox); the default is c.
\makeatletter
\newlength{\ot@dp}
\newlength{\ot@ht}
\newlength{\ot@wd}
\newbox{\ot@a}
\newbox{\ot@b}
\newbox{\ot@c}
\newbox{\ot@empty}
\newcommand<>*{\otemporal}[4][c]{%
	\delegateStyle{%
		% based on \setto… in /usr/share/texmf-dist/tex/latex/base/latex.ltx
		\setbox\ot@a\hbox{\inheritStyle{#2}}%
		\setbox\ot@b\hbox{\inheritStyle{#3}}%
		\setbox\ot@c\hbox{\inheritStyle{#4}}%
		\pgfmathsetlength{\ot@dp}{max(\dp\ot@a,\dp\ot@b,\dp\ot@c)}%
		\pgfmathsetlength{\ot@ht}{max(\ht\ot@a,\ht\ot@b,\ht\ot@c)}%
		\pgfmathsetlength{\ot@wd}{max(\wd\ot@a,\wd\ot@b,\wd\ot@c)}%
		\raisebox{0pt}[\ot@ht][\ot@dp]{%
			\makebox[\ot@wd][#1]{%
				\temporal#5{\unhbox\ot@a}{\unhbox\ot@b}{\unhbox\ot@c}%
			}%
		}%
		\setbox\ot@a\box\ot@empty%
		\setbox\ot@b\box\ot@empty%
		\setbox\ot@c\box\ot@empty%
	}%
}
\makeatother


% Resize delimiters like braces, brackets, etc.
% Parameters: size, left delimiter, formula, right delimiter
% Example: \delim2({\frac{1}{2}})
\newcommand*{\delim}[4]{%
	\ifcase#1%
		#2#3#4%
	\or%
		\bigl#2#3\bigr#4%
	\or%
		\Bigl#2#3\Bigr#4%
	\or%
		\biggl#2#3\biggr#4%
	\or%
		\Biggl#2#3\Biggr#4%
	\else%
		\left#2#3\right#4%
	\fi%
}


% similar to \fullcite, but using the formatting of \printbibliography
\newcommand*{\printfullcite}[1]{%
	\begin{refsection}%
		\nocite{#1}%
		\DeclareNameAlias{author}{first-last}%
		\printbibliography[heading = none]%
	\end{refsection}%
}


\colorlet{light alert}{HKS07K60}
\tikzset{alert.bg/.style={rounded corners, fill=light alert}}
\tikzset{every picture/.style={line cap=round, semithick}}
% http://tex.stackexchange.com/questions/6135/how-to-make-beamer-overlays-with-tikz-node
\tikzset{onslide/.code args={<#1>#2}{\only<#1>{\pgfkeysalso{#2}}}}
\tikzset{invisible/.code args={<#1>}{\alt<#1>{\pgfkeysalso{transparent}}{\pgfkeysalso{opaque}}}}
\tikzset{uncover/.code args={<#1>}{\alt<#1>{\pgfkeysalso{opaque}}{\pgfkeysalso{opacity=0.25}}}}
\tikzset{visible/.code args={<#1>}{\alt<#1>{\pgfkeysalso{opaque}}{\pgfkeysalso{transparent}}}}
\tikzset{vuncover/.code args=%
	{<#1><#2>}%
	{\alt<#1>%
		{\alt<#2>%
			{\pgfkeysalso{opaque}}%
			{\pgfkeysalso{opacity=0.25}}%
		}{\pgfkeysalso{transparent}}%
	}%
}

\newcommand<%
	>{\tikzhighlight}[2][]{%
	\delegateStyle{\alt#3%
		{\tikz[baseline=0, anchor=base, inner sep=0.2em, text height=, text depth=]{\node[alert.bg, #1]{\inheritStyle{#2}};}}%
		{\tikz[baseline=0, anchor=base, inner sep=0.2em, text height=, text depth=]{\node[#1, fill=none]{\inheritStyle{#2}};}}%
	}%
}

\newcommand{\mathhighlight}{\tikzhighlight}

\newcommand<>{\mhl}[2][]{\mathhighlight#3[inner sep=0.2em, #1]{#2}}


\newcommand<>{\inlineblock}[2][]{{%
	\usebeamercolor*[fg]{block body}%
	\tikzhighlight#3[fill=block body.bg, #1]{#2}%
}}


% a small letter s for plurals of abbreviations
\newcommand*{\s}{{\scriptsize s}\xspace}


\newcommand<>*{\sout}[2][opacity=0.75, ultra thick]{%
	\delegateStyle{%
		\tikz[baseline=0, anchor=base, inner sep=0, outer sep=0]{
			\useasboundingbox node (n) {\inheritStyle{#2}};
			\only#3{
				\node (h) {\inheritStyle{\ifmmode\mathstrut\else\strut\fi}};
				\draw[#1] (n.west |- {$(h.south)!0.5!(h.north)$}) -- (n.east |- {$(h.south)!0.5!(h.north)$});
			}
		}%
	}%
}


% tight style
% Sets outer sep to default inner sep and inner sep to 0.
% Use this style for nodes that are neither drawn nor filled to prevent
% unwanted growth of the bounding box.
\tikzset{tight/.style={inner sep=0, outer sep=0.3333em}}


% rounded tree edges style
% usage: rounded tree edges={⟨direction⟩}{⟨looseness⟩}{⟨strength⟩}
\tikzset{
	rounded tree edges/.style n args={3}{
	edge from parent path={
	let
		\n{direction}={#1},
		\n{looseness}={#2},
		\n{strength}={#3},
		\p1=(\tikzparentnode),
		\p2=(\tikzchildnode),
		\p3=(\n{direction}:1pt),
		\p4=(\x2 - \x1, \y2 - \y1),
		\n{dist}={veclen(\p4)},
		\p4=(\x4 / \n{dist}, \y4 / \n{dist}),
		\n{angle}={atan2(\y4, \x4)},
		\n{delta}={Mod(\n{angle} - \n{direction}, 360)},
		\n{delta}={\n{delta} > 180 ? \n{delta} - 360  : \n{delta}},
		\n{delta}={\n{delta} >  90 ?  180 - \n{delta} : \n{delta}},
		\n{delta}={\n{delta} < -90 ? -180 - \n{delta} : \n{delta}}
	in (\tikzparentnode) .. controls
		+(    \n{angle}+\n{strength}*\n{delta}:\n{looseness}*0.3915*\n{dist}) and
		+(180+\n{angle}-\n{strength}*\n{delta}:\n{looseness}*0.3915*\n{dist}) ..
		(\tikzchildnode)
	}
	}
}


% Tear out snippets from PDFs.
% Usage: \tear[…]{file.pdf}
% The optional parameter is the same as for \includegraphics.
% Useful Arguments:
%   * page=‹pagenumber›
%   * trim=‹left› ‹bottom› ‹right› ‹top›
%   * width=0.98\linewidth
\newcommand*{\tear}[2][]{%
	\begin{tikzpicture}
		\node
			[ blur shadow
			, clip
			, decorate
			, decoration=random steps
			, draw
			, inner sep=0
			, preaction={fill=white}% hide the shadow if paper is transparent
			] {\includegraphics[#1]{#2}};
	\end{tikzpicture}%
}


\makeatletter
\newcommand*{\timeline}[3][0]{%
	\ifcsname timeline@cmd@#3\endcsname%
		\@timeline[#1]{#2}{#3}%
		\PackageWarning{timeline}{redefining timeline \@backslashchar\string#3}%
	\else%
		\ifcsname#3\endcsname%
			\errmessage{Command \@backslashchar\string#3 already defined}%
		\else%
			\@timeline[#1]{#2}{#3}%
		\fi%
	\fi%
}%
\newcommand*{\@timeline}[3][0]{%
	% mark command as timeline command – they can be overwritten
	\expandafter\def\csname timeline@cmd@#3\endcsname{}%
	\setcounter{@timeline}{#1}%
	\def\timeline@cmd{#3}%
	\timeline@reset%
	\timeline@append{0}%
	\@tfor\timeline@next:=#2\do{%
		\if\timeline@next+%
			\stepcounter{@timeline}%
			\timeline@append{,\the@timeline}%
		\else\if\timeline@next-%
			\stepcounter{@timeline}%
		\else%
			%\timeline@append{\timeline@next}%
			\GenericError{}{\protect\timeline: ignoring unknown character: \timeline@next}%
		\fi\fi%
	}%
}%
% \newcommand*{\tl}[1]{%
% 	\ifcsname timeline@cmd@#1\endcsname%
% 		\csname timeline@cmd@#1\endcsname%
% 	\else%
% 		0%
% 		%\GenericError{}{\protect\tl: timeline not defined: #1}%
% 	\fi%
% }%
\newcounter{@timeline}%
\def\timeline@reset{%
	\expandafter\def\csname\timeline@cmd\endcsname{}%
}%
\def\timeline@append#1{%
	\expandafter\edef\csname\timeline@cmd\endcsname{%
		\csname\timeline@cmd\endcsname#1%
	}%
}%
\makeatother


\newcommand*{\xminus}[1]{%
	\mathrel{\tikz[baseline={([yshift=-0.25em]n.south)}, inner sep=0, outer sep=0.2em]{%
		\node (n) {\(\scriptstyle #1\)};
		\draw (n.south west) -- (n.south east);
	}}%
}
\newcommand*{\tikzrightarrow}[1]{%
	\mathrel{\tikz[baseline={([yshift=-0.25em]n.south)}, inner sep=0, outer sep=0.2em]{%
		\node (n) {\(\scriptstyle #1\)};
		\draw[->, > = Computer Modern Rightarrow, line width = 0.4pt] (n.south west) -- (n.south east);
	}}%
}


%%%%%%%%%%%%%%%%%%%%%%%%%%%%%%%%%%%%%%%%%%%%%%%%%%%%%%%%%%%%%%%%%%%%%%%%%%%%%%
% document specific commands
%%%%%%%%%%%%%%%%%%%%%%%%%%%%%%%%%%%%%%%%%%%%%%%%%%%%%%%%%%%%%%%%%%%%%%%%%%%%%%

\newcommand<>*{\mycite}[1]{\uncover#2{{\color{HKS57K100}[\cite{#1}]}}}


\newcommand{\statetree}[1]{
	\tikz
	[ anchor=base
	, baseline=(current bounding box.center)
	, level distance=2em
	, sibling distance=2em
	]{
		\matrix
		[ draw=nt
		, edge from parent/.style={draw=black}
		, inner sep=0
		, nodes={inner sep=0.2em, rounded corners=0}
		, rounded corners
		] {#1\\}
	}
}


\newcommand*{\mylargeleaf}[1]{{\LARGE\color{HKS41K70}#1}}

\definecolor{state s}{named}{HKS57K80}
\definecolor{state t}{named}{HKS41K70}
\newcommand*{\stateS}[1]{{\color{state s}#1}}
\newcommand*{\stateT}[1]{{\color{state t}#1}}

\tikzset{
	subtree/.style =
		{ fill=lightgray
		, inner sep=0.02em
		, isosceles triangle apex angle=60
		, shape=isosceles triangle
		, shape border rotate=90
		}
	, state/.style = {circle, draw, inner sep=0.1em}
	, trans/.style = {rectangle, draw}
}

\newcommand*{\srBool}{\mathbb{B}}
\newcommand*{\srProb}{ℙ}


%%%%%%%%%%%%%%%%%%%%%%%%%%%%%%%%%%%%%%%%%%%%%%%%%%%%%%%%%%%%%%%%%%%%%%%%%%%%%%
% commands for specific notations
%%%%%%%%%%%%%%%%%%%%%%%%%%%%%%%%%%%%%%%%%%%%%%%%%%%%%%%%%%%%%%%%%%%%%%%%%%%%%%

\DeclareMathOperator*{\argmax}{argmax}

\newcommand*{\cardinality}[1]{\lvert#1\rvert}
\newcommand*{\corpussize}[1]{\lvert#1\rvert}

\DeclareMathOperator{\crispOp}{crisp}
\newcommand*        {\crisp}[2][0]{\crispOp\delim{#1}({#2})}

\DeclareMathOperator{\lhsOp}{lhs}
\newcommand*{\lhs}[1]{\lhsOp(#1)}

\DeclareMathOperator{\lklhdOp}{L}
\newcommand*{\lklhd}[2]{\lklhdOp(#1 ∣ #2)}

\DeclareMathOperator{\mleOp}{mle}
\newcommand*{\mle}[2][]{%
	\ifthenelse{\isempty{#1}}{%
		\mleOp(#2)%
	}{%
		\mleOp_{#1}(#2)%
	}%
}

\DeclareMathOperator{\mrg}{merge}

% CVD: color vision deficiencies
\definecolor{CVD light red}   {HTML}{FF8080}
\definecolor{CVD light yellow}{HTML}{FFFF80}
\definecolor{CVD light green} {HTML}{40FFC0}

\definecolor{nt}{named}{HKS41K70}
\newcommand*{\nt}[1]{{\color{nt}#1}}

% set of all probability distributions over #1
\DeclareMathOperator{\pdsOp}{Pd}
\newcommand*{\pds}[1]{\pdsOp(#1)}

\DeclareMathOperator{\positionsOp}{pos}
\newcommand*{\positions}[1]{\positionsOp(#1)}

\DeclareMathOperator{\rankOp}{rk}
\newcommand*{\rank}[1]{\rankOp(#1)}

\DeclareMathOperator{\runsOp}{run}
\newcommand*{\runs}[2][]{%
	\ifthenelse%
		{\isempty{#1}}%
		{\runsOp(#2)}%
		{\runsOp_{#1}(#2)}%
}

\newcommand*{\semantics}[1]{⟦#1⟧}

\DeclareMathOperator{\splt}{split}

\newcommand*{\subtree}[2]{#1|_{#2}}

\DeclareMathOperator{\supportOp}{supp}
\newcommand*{\support}[1]{\supportOp(#1)}

\newcommand*{\symId}{\textsc{\color{gray}Id}}
\newcommand*{\symCons}{\textsc{\color{gray}Cons}}
\newcommand*{\symFlip}{\textsc{\color{gray}Flip}}
\newcommand*{\symNull}{\textsc{\color{gray}Null}}
\newcommand*{\symNullR}{\textsc{\color{gray}N\(\overline{\textsc{ull}}\)}}
\newcommand*{\symSnoc}{\textsc{\color{gray}Snoc}}

\newcommand*{\transWTA}[4][]{#3 \xrightarrow{#1} #2(#4)}

\DeclareMathOperator{\uniqueRunOp}{r}
\newcommand*{\uniqueRun}[2][]{%
	\ifthenelse%
		{\isempty{#1}}%
		{\uniqueRunOp^{#2}}%
		{\uniqueRunOp_{\!#1}^{#2}}%
}

\DeclareMathOperator{\treesOp}{T}
\newcommand*{\trees}[2][]{%
	\ifthenelse%
		{\isempty{#1}}%
		{\treesOp_{\!#2}}%
		{\treesOp_{\!#2}(#1)}%
}
\DeclareMathOperator{\treesUOp}{U}
\newcommand*{\treesU}[2][]{%
	\ifthenelse%
		{\isempty{#1}}%
		{\treesUOp_{#2}}%
		{\treesUOp_{#2}(#1)}%
}


%%%%%%%%%%%%%%%%%%%%%%%%%%%%%%%%%%%%%%%%%%%%%%%%%%%%%%%%%%%%%%%%%%%%%%%%%%%%%%
% metadata
%%%%%%%%%%%%%%%%%%%%%%%%%%%%%%%%%%%%%%%%%%%%%%%%%%%%%%%%%%%%%%%%%%%%%%%%%%%%%%

\ifstandalonebeamer\else
	\title[Defense of Dissertation]{A Formal View on Training of Weighted Tree Automata by Likelihood-Driven State Splitting and Merging}
	\subtitle{Defense of Dissertation}
\fi
\author{Toni Dietze}
\institute[TU Dresden]{%
	\href{https://www.orchid.inf.tu-dresden.de/index.en/}{Chair for Foundations of Programming}
\\	\href{https://tu-dresden.de/ing/informatik/thi}{Institute of Theoretical Computer Science}
\\	\href{https://tu-dresden.de/ing/informatik}{Faculty of Computer Science}
\\	\href{https://tu-dresden.de/}{Technische Universität Dresden}
\\	01062 Dresden, Germany
}
\date[2018-09-27]{September 27, 2018}

\title{\jobname}
\begin{document}
\newcommand*{\myheight}{75.2mm}%
\newcommand*{\mywidth} {42.3mm}%
\newcommand*{\mygraphics}[1]{\includegraphics[height = \myheight]{#1}}%
\newcommand*{\myvideo}[1]{%
	\includemedia
		[ width = \mywidth
		, height = \myheight
		, addresource = #1
		, activate = pageopen
		, playbutton = none
		, flashvars =
			{ source = #1
			& loop = true
			& hideBar = true
			}
		]{}{VPlayer.swf}%
}%
\begin{standaloneframe}{\jobname}
	\begin{columns}<2->
	\column{13em}
		\setcounter{beamerpauses}{2}%
		\begin{itemize}[<+-|alert@+>]
		\item grammar checking
		\item handwriting recognition
		\item translation
		\item speech recognition
		\item news article generation
		%\item Screen Reader
		\end{itemize}
	\column{\mywidth}
		\begin{overprint}
		\onslide<-5>
			\setbeamercolor{black}{bg = black, fg = white}%
			\begin{beamercolorbox}[ht = \myheight, rounded = true, shadow = true]{black}%
				\setcounter{beamerpauses}{2}%
				\only<+>{\mygraphics{graphics-Android-Google-Docs-grammar-checker.png}}%
				\only<+>{\myvideo{video-Android-Google-Handwriting-Input.mp4}}%
				\only<+>{\mygraphics{graphics-Android-Google-Translate.png}}%
				\only<+>{\myvideo{video-Android-Gboard-speech-recognition.mp4}}%
			\end{beamercolorbox}%
		\onslide<6>
			\raisebox{8mm}[\myheight][0pt]{%
				\href{https://openclipart.org/detail/182846/newspaper}{%
					\includegraphics[width = \mywidth]{graphics-newspaper.pdf}%
				}%
			}%
		\end{overprint}
	\end{columns}
\end{standaloneframe}
\end{document}

\end{frame}


\section{Motivation}

\begin{frame}{\secname}
	\begin{columns}
	\column{0.45\textwidth}
		\centering
		\begin{tikzpicture}[anchor=base, level distance=2.5em, level 1/.style={sibling distance=6em}, level 2/.style={sibling distance=4em}]
			\node at (9em, 7em)  {S}
				child { node {NP}
					child { node {NNP}
						child { node {John}
						}
					}
				}
				child { node {VP}
					child { node {VPZ}
						child { node {loves}
						}
					}
					child { node {NP}
						child { node {NNP}
							child { node {Mary}
							}
						}
					}
				};
		\end{tikzpicture}
	\column{0.45\textwidth}
		\begin{itemize}
			\item[S] simple declarative clause
			\item[NP] noun phrase
			\item[VP] verb phrase
			\item[NNP] proper noun, singular
			\item[VPZ] verb, present tense, 3rd person singular
		\end{itemize}
	\end{columns}

	\begin{block}{examples for treebanks/tree corpora}
		\begin{description}[Penn Treebank]
		\item [Penn Treebank]
			50,000 English trees based on Wall Street Journal
		\item[TIGER]
			50,000 German trees based on Frankfurter Rundschau
		\end{description}
	\end{block}
\end{frame}


\begin{frame}{\secname{}}
	\begin{center}
		\begin{tikzpicture}
[ tri/.style =
	{ inner sep=0.12em
	, shape border rotate=90
	, isosceles triangle
	, isosceles triangle apex angle=60
	, draw
	} 
]
	\matrix[anchor=base, column sep=1em, row sep=3em] (m) {
		\node[anchor = base east, align = right] {corpus: \\ {\footnotesize(training data)}\phantom{:}};
	&&	\node[tri, fill = CVD light green ] {B};
	&	\node[tri, fill = CVD light green ] {C};
	&&	\node[tri, fill = CVD light green ] {E};
	\\	\node[anchor=base east] {$\trees{\varSigma}\colon$};
	&	\node[tri, fill = CVD light red] (A) {A};
	&	\node[tri, fill = CVD light green ] (B) {B};
	&	\node[tri, fill = CVD light green ] (C) {C};
	&	\node[tri, fill = CVD light yellow] (D) {D};
	&	\node[tri, fill = CVD light green ] (E) {E};
	&	\node[tri, fill = CVD light red   ] (F) {F};
	&	\node[tri, fill = CVD light yellow] (G) {G};
	&	\node {$\dots$};
	\\	\node[anchor=base east, visible=<2->] {probabilities:};
	&	\node<4-> (pA) {0};
	&	\node<2-> (pB) {0.2};
	&	\node<2-> (pC) {0.1};
	&	\node<3-> (pD) {0.1};
	&	\node<2-> (pE) {0.4};
	&	\node<4-> (pF) {0};
	&	\node<3-> (pG) {0.1};
	&	\node<2->  {$\dots$};
	\\
	};
	\begin{scope}
		[ |->
		, > = Computer Modern Rightarrow
		, line width = 0.4pt
		, shorten < = 0.75em
		, shorten > = 0.5em
		, node distance = 4em
		]
		\draw<4-> (A) -- (pA);
		\draw<2-> (B) -- (pB);
		\draw<2-> (C) -- (pC);
		\draw<3-> (D) -- (pD);
		\draw<2-> (E) -- (pE);
		\draw<4-> (F) -- (pF);
		\draw<3-> (G) -- (pG);
	\end{scope}
\end{tikzpicture}

	\end{center}
	\begin{overprint}
	\onslide<5>
		\[
			\newcommand{\snaketree}
				{\tikz[baseline=0]{\draw[decoration={snake, amplitude=0.1em, segment length=0.4em}] decorate {(0, 0) -- ++(2em, 0)} -- ++(-1em, 1.5em) -- cycle;}}
			\operatorname{parse}(\tikz[baseline=0]{\draw[decorate, decoration={snake, amplitude=0.1em, segment length=0.4em}] (0, 0) -- ++(2em, 0);})
			= \argmax_{\snaketree}
			P(\snaketree)
		\]
	\onslide<6->
		\begin{center}
			We will use weighted tree automata.
		\end{center}
	\end{overprint}
\end{frame}


\section{Outline}

\begin{frame}<1-3>[label=toc]{Outline}
	\documentclass[beamer]{standalone}
% SPDX-License-Identifier: CC-BY-4.0 OR MIT-0
% Copyright 2018 Toni Dietze
%
\usefonttheme{professionalfonts}

% LuaLaTeX specific packages
\usepackage{fontspec}
	\defaultfontfeatures{Ligatures=TeX}
\usepackage{polyglossia}
	\setdefaultlanguage{english}
\usepackage{amsmath}  % has to be loaded before unicode-math
\usepackage[math-style=ISO]{unicode-math}
	\setmathfont{Latin Modern Math}
% 	\setmathfont[range={\mathcal,\mathbfcal},StylisticSet=1]{xits-math.otf}
% 	\setmathfont[range={"029F5}]{XITS Math}  % ⧵
% 	\setmathfont[range={\mathscr,\mathbfscr},StylisticSet=1]{Latin Modern Math}  % make \mathscr use the correct font

\usepackage[noend]{algpseudocode}
	\algrenewcommand\algorithmicrequire{\textbf{Input:}}
	\algrenewcommand\algorithmicensure{\textbf{Output:}}
\usepackage[backend=biber, maxbibnames=42, maxcitenames=42, sorting=ynt, style=authoryear]{biblatex}
\usepackage{csquotes}
\usepackage{mathtools}
\usepackage{media9}
\usepackage{scalerel}
\usepackage{standalone}
\usepackage{tikz}
	\usetikzlibrary{arrows.meta}
	\usetikzlibrary{backgrounds}
	\usetikzlibrary{calc}
	\usetikzlibrary{decorations}
	\usetikzlibrary{decorations.pathmorphing}
	\usetikzlibrary{decorations.pathreplacing}
	\usetikzlibrary{fadings}
	\usetikzlibrary{fit}
	\usetikzlibrary{graphs}
	\usetikzlibrary{graphdrawing}
	\usetikzlibrary{intersections}
	\usetikzlibrary{positioning}
	\usetikzlibrary{quotes}
	\usetikzlibrary{shadows.blur}
	\usetikzlibrary{shapes.arrows}
	\usetikzlibrary{shapes.geometric}
	\usegdlibrary{trees}
\usepackage{xifthen}
\usepackage{xspace}

\usepackage{pgfplots}
	\pgfplotsset
		{ compat = 1.15
		, /pgf/number format/1000 sep = {\,}
		, /pgf/number format/assume math mode = true
		, every axis plot/.append style =
			{ mark options = {fill opacity = 0.25}
			}
		}
	\usepgfplotslibrary{groupplots}
\usepackage{pgfplotstable}

\hypersetup
	{ bookmarksopen
	, pdflang = en
	, unicode
	}


%%%%%%%%%%%%%%%%%%%%%%%%%%%%%%%%%%%%%%%%%%%%%%%%%%%%%%%%%%%%%%%%%%%%%%%%%%%%%%


% always show bad boxes
%\overfullrule=1em


%%%%%%%%%%%%%%%%%%%%%%%%%%%%%%%%%%%%%%%%%%%%%%%%%%%%%%%%%%%%%%%%%%%%%%%%%%%%%%
% biblatex
%%%%%%%%%%%%%%%%%%%%%%%%%%%%%%%%%%%%%%%%%%%%%%%%%%%%%%%%%%%%%%%%%%%%%%%%%%%%%%

\addbibresource{slides-dissertation-defense.bib}
% \renewcommand*{\finalnamedelim}{\addcomma\space}
% \setlength{\bibitemsep}{1em}
% 
\AtEveryBibitem{% Clean up the bibtex rather than editing it
 \clearlist{address}
 \clearfield{date}
 \clearfield{eprint}
 \clearfield{isbn}
 \clearfield{issn}
 \clearlist{language}
 \clearlist{location}
 \clearfield{month}
 \clearfield{series}
%  \clearfield{url}
%  \clearfield{doi}
 \clearfield{organization}

%  \ifentrytype{book}{}{% Remove stuff except for books
%   \clearfield{booktitle}
%   \clearfield{pages}
  \clearlist{publisher}
  \clearname{editor}
%  }
}
% do not print url if doi is present
% http://tex.stackexchange.com/questions/154864/biblatex-use-doi-only-if-there-is-no-url
\DeclareSourcemap{
	\maps[datatype=bibtex]{
		\map{
			\step[fieldsource=doi,final]
			\step[fieldset=url,null]
}	}	}
%
% remove qoutes around titles
\DeclareFieldFormat
	[article,inbook,incollection,inproceedings,patent,thesis,unpublished]
	{title}{#1\isdot}
% 
% \DeclareFieldFormat{url}{\mkbibacro{URL}\addcolon\addnbspace\url{#1}}
% 
% \DeclareNameAlias{sortname}{first-last}
% 
\renewbibmacro{in:}{\ifentrytype{article}{}{}}


%%%%%%%%%%%%%%%%%%%%%%%%%%%%%%%%%%%%%%%%%%%%%%%%%%%%%%%%%%%%%%%%%%%%%%%%%%%%%%
% beamer
%%%%%%%%%%%%%%%%%%%%%%%%%%%%%%%%%%%%%%%%%%%%%%%%%%%%%%%%%%%%%%%%%%%%%%%%%%%%%%

\useoutertheme{infolines}
\makeatletter
% based on
% /usr/share/texmf-dist/tex/latex/beamer/beamerouterthemeinfolines.sty
\setbeamertemplate{footline}
{%
	\leavevmode%
	\hbox{%
	\begin{beamercolorbox}[wd=.333333\paperwidth,ht=2.25ex,dp=1ex,center]{author in head/foot}%
		\usebeamerfont{author in head/foot}\insertshortauthor\expandafter\beamer@ifempty\expandafter{\beamer@shortinstitute}{}{~~(\insertshortinstitute)}
	\end{beamercolorbox}%
	\begin{beamercolorbox}[wd=.333333\paperwidth,ht=2.25ex,dp=1ex,center]{title in head/foot}%
		\usebeamerfont{title in head/foot}\insertshorttitle
	\end{beamercolorbox}%
	\begin{beamercolorbox}[wd=.333333\paperwidth,ht=2.25ex,dp=1ex,right]{date in head/foot}%
		\usebeamerfont{date in head/foot}%
		\hfill\insertshortdate\hfill\hfill%
		%\hspace*{2ex}%
		%\insertshortdate%
		%\hspace{0pt plus 1 filll}%
		%(\insertframenumber.\insertoverlaynumber{} / \insertmainframenumber)%
		%\hspace{0pt plus 1 filll}%
		\phantom{000}\llap{\insertpagenumber} / \insertpresentationendpage%
		\hspace*{2ex}%
	\end{beamercolorbox}}%
	\vskip0pt%
}
\makeatother
\useinnertheme{circles}
\beamertemplatenavigationsymbolsempty
\setbeamertemplate{bibliography item}{}
\setbeamertemplate{headline}[default]

\input{tudcolors.tex}
\setbeamercolor*{alerted text}{fg=HKS07K100}
\usecolortheme[named=HKS41K100]{structure}

\setbeamercolor*{palette primary}{use=structure,fg=white,bg=structure.fg}
\setbeamercolor*{palette secondary}{use=structure,fg=white,bg=structure.fg!80}
\setbeamercolor*{palette tertiary}{use=structure,fg=white,bg=structure.fg!60}
\setbeamercolor*{palette quaternary}{fg=white,bg=black}

\setbeamercolor*{sidebar}{use=structure,bg=structure.fg}

\setbeamercolor*{palette sidebar primary}{use=structure,fg=structure.fg!20}
\setbeamercolor*{palette sidebar secondary}{fg=white}
\setbeamercolor*{palette sidebar tertiary}{use=structure,fg=structure.fg!40}
\setbeamercolor*{palette sidebar quaternary}{fg=white}

\setbeamercolor*{titlelike}{parent=palette primary}

\setbeamercolor*{separation line}{}
\setbeamercolor*{fine separation line}{}

\setbeamercolor{block title}{use=structure,fg=white,bg=structure.fg}
\setbeamercolor{block title alerted}{use=alerted text,fg=white,bg=alerted text.fg!75!black}
\setbeamercolor{block title example}{use=example text,fg=white,bg=example text.fg!75!black}

\setbeamercolor{block body}{parent=normal text,use=block title,bg=block title.bg!10!bg}
\setbeamercolor{block body alerted}{parent=normal text,use=block title alerted,bg=block title alerted.bg!10!bg}
\setbeamercolor{block body example}{parent=normal text,use=block title example,bg=block title example.bg!10!bg}

% \setbeamertemplate{itemize items}[default]


%%%%%%%%%%%%%%%%%%%%%%%%%%%%%%%%%%%%%%%%%%%%%%%%%%%%%%%%%%%%%%%%%%%%%%%%%%%%%%
% TikZ
%%%%%%%%%%%%%%%%%%%%%%%%%%%%%%%%%%%%%%%%%%%%%%%%%%%%%%%%%%%%%%%%%%%%%%%%%%%%%%

\tikzset
	{ > = Stealth
	}


%%%%%%%%%%%%%%%%%%%%%%%%%%%%%%%%%%%%%%%%%%%%%%%%%%%%%%%%%%%%%%%%%%%%%%%%%%%%%%
% general commands and styles
%%%%%%%%%%%%%%%%%%%%%%%%%%%%%%%%%%%%%%%%%%%%%%%%%%%%%%%%%%%%%%%%%%%%%%%%%%%%%%

% \delegateStyle and \inheritStyle command
% usage: \delegateStyle{… \inheritStyle{…} …}
% example: \(X_{\delegateStyle{\fbox{\inheritStyle{X}}}}\)
% Save the current style and regain it in the argument.
% This works both for math and text mode, and can be nested.
% Acknowledgments: Based on \ThisStyle and \SavedStyle from scalerel package.
\makeatletter
\newcommand*{\@inheritStyle@D}[1]{\(\displaystyle      #1\)}
\newcommand*{\@inheritStyle@T}[1]{\(\textstyle         #1\)}
\newcommand*{\@inheritStyle@S}[1]{\(\scriptstyle       #1\)}
\newcommand*{\@inheritStyle@s}[1]{\(\scriptscriptstyle #1\)}
\newcommand*{\@inheritStyle@t}[1]{#1}
\newcommand*{\inheritStyle}{\csname @inheritStyle@\@inheritStyleSwitch\endcsname}
\newcommand*{\delegateStyle}[1]{%
	\ifmmode%
		\mathchoice%
		{\edef\@inheritStyleSwitch{D}#1}%
		{\edef\@inheritStyleSwitch{T}#1}%
		{\edef\@inheritStyleSwitch{S}#1}%
		{\edef\@inheritStyleSwitch{s}#1}%
	\else%
		\edef\@inheritStyleSwitch{t}#1%
	\fi%
}
\makeatother


% \oalt command
% requires: \delegateStyle and \inheritStyle command
% usage: \oalt<…>[…]{…}{…} (cf. \alt)
% Behaves like \alt, but reserves space according to largest overlays.
% The optional argument defines the alignment inside the reserved space;
% it is one of c, l, r, s (cf. \makebox); the default is c.
\makeatletter
\newlength{\oalt@dp}
\newlength{\oalt@ht}
\newlength{\oalt@wd}
\newbox{\oalt@a}
\newbox{\oalt@b}
\newbox{\oalt@empty}
\newcommand<>*{\oalt}[3][c]{%
	\delegateStyle{%
		% based on \setto… in /usr/share/texmf-dist/tex/latex/base/latex.ltx
		\setbox\oalt@a\hbox{\inheritStyle{#2}}%
		\setbox\oalt@b\hbox{\inheritStyle{#3}}%
		\pgfmathsetlength{\oalt@dp}{max(\dp\oalt@a,\dp\oalt@b)}%
		\pgfmathsetlength{\oalt@ht}{max(\ht\oalt@a,\ht\oalt@b)}%
		\pgfmathsetlength{\oalt@wd}{max(\wd\oalt@a,\wd\oalt@b)}%
		\raisebox{0pt}[\oalt@ht][\oalt@dp]{%
			\makebox[\oalt@wd][#1]{%
				\alt#4{\unhbox\oalt@a}{\unhbox\oalt@b}%
			}%
		}%
		\setbox\oalt@a\box\oalt@empty%
		\setbox\oalt@b\box\oalt@empty%
	}%
}
\makeatother


% \otemporal command
% requires: \delegateStyle and \inheritStyle command
% usage: \otemporal<…>[…]{…}{…}{…} (cf. \temporal)
% Behaves like \temporal, but reserves space according to largest overlays.
% The optional argument defines the alignment inside the reserved space;
% it is one of c, l, r, s (cf. \makebox); the default is c.
\makeatletter
\newlength{\ot@dp}
\newlength{\ot@ht}
\newlength{\ot@wd}
\newbox{\ot@a}
\newbox{\ot@b}
\newbox{\ot@c}
\newbox{\ot@empty}
\newcommand<>*{\otemporal}[4][c]{%
	\delegateStyle{%
		% based on \setto… in /usr/share/texmf-dist/tex/latex/base/latex.ltx
		\setbox\ot@a\hbox{\inheritStyle{#2}}%
		\setbox\ot@b\hbox{\inheritStyle{#3}}%
		\setbox\ot@c\hbox{\inheritStyle{#4}}%
		\pgfmathsetlength{\ot@dp}{max(\dp\ot@a,\dp\ot@b,\dp\ot@c)}%
		\pgfmathsetlength{\ot@ht}{max(\ht\ot@a,\ht\ot@b,\ht\ot@c)}%
		\pgfmathsetlength{\ot@wd}{max(\wd\ot@a,\wd\ot@b,\wd\ot@c)}%
		\raisebox{0pt}[\ot@ht][\ot@dp]{%
			\makebox[\ot@wd][#1]{%
				\temporal#5{\unhbox\ot@a}{\unhbox\ot@b}{\unhbox\ot@c}%
			}%
		}%
		\setbox\ot@a\box\ot@empty%
		\setbox\ot@b\box\ot@empty%
		\setbox\ot@c\box\ot@empty%
	}%
}
\makeatother


% Resize delimiters like braces, brackets, etc.
% Parameters: size, left delimiter, formula, right delimiter
% Example: \delim2({\frac{1}{2}})
\newcommand*{\delim}[4]{%
	\ifcase#1%
		#2#3#4%
	\or%
		\bigl#2#3\bigr#4%
	\or%
		\Bigl#2#3\Bigr#4%
	\or%
		\biggl#2#3\biggr#4%
	\or%
		\Biggl#2#3\Biggr#4%
	\else%
		\left#2#3\right#4%
	\fi%
}


% similar to \fullcite, but using the formatting of \printbibliography
\newcommand*{\printfullcite}[1]{%
	\begin{refsection}%
		\nocite{#1}%
		\DeclareNameAlias{author}{first-last}%
		\printbibliography[heading = none]%
	\end{refsection}%
}


\colorlet{light alert}{HKS07K60}
\tikzset{alert.bg/.style={rounded corners, fill=light alert}}
\tikzset{every picture/.style={line cap=round, semithick}}
% http://tex.stackexchange.com/questions/6135/how-to-make-beamer-overlays-with-tikz-node
\tikzset{onslide/.code args={<#1>#2}{\only<#1>{\pgfkeysalso{#2}}}}
\tikzset{invisible/.code args={<#1>}{\alt<#1>{\pgfkeysalso{transparent}}{\pgfkeysalso{opaque}}}}
\tikzset{uncover/.code args={<#1>}{\alt<#1>{\pgfkeysalso{opaque}}{\pgfkeysalso{opacity=0.25}}}}
\tikzset{visible/.code args={<#1>}{\alt<#1>{\pgfkeysalso{opaque}}{\pgfkeysalso{transparent}}}}
\tikzset{vuncover/.code args=%
	{<#1><#2>}%
	{\alt<#1>%
		{\alt<#2>%
			{\pgfkeysalso{opaque}}%
			{\pgfkeysalso{opacity=0.25}}%
		}{\pgfkeysalso{transparent}}%
	}%
}

\newcommand<%
	>{\tikzhighlight}[2][]{%
	\delegateStyle{\alt#3%
		{\tikz[baseline=0, anchor=base, inner sep=0.2em, text height=, text depth=]{\node[alert.bg, #1]{\inheritStyle{#2}};}}%
		{\tikz[baseline=0, anchor=base, inner sep=0.2em, text height=, text depth=]{\node[#1, fill=none]{\inheritStyle{#2}};}}%
	}%
}

\newcommand{\mathhighlight}{\tikzhighlight}

\newcommand<>{\mhl}[2][]{\mathhighlight#3[inner sep=0.2em, #1]{#2}}


\newcommand<>{\inlineblock}[2][]{{%
	\usebeamercolor*[fg]{block body}%
	\tikzhighlight#3[fill=block body.bg, #1]{#2}%
}}


% a small letter s for plurals of abbreviations
\newcommand*{\s}{{\scriptsize s}\xspace}


\newcommand<>*{\sout}[2][opacity=0.75, ultra thick]{%
	\delegateStyle{%
		\tikz[baseline=0, anchor=base, inner sep=0, outer sep=0]{
			\useasboundingbox node (n) {\inheritStyle{#2}};
			\only#3{
				\node (h) {\inheritStyle{\ifmmode\mathstrut\else\strut\fi}};
				\draw[#1] (n.west |- {$(h.south)!0.5!(h.north)$}) -- (n.east |- {$(h.south)!0.5!(h.north)$});
			}
		}%
	}%
}


% tight style
% Sets outer sep to default inner sep and inner sep to 0.
% Use this style for nodes that are neither drawn nor filled to prevent
% unwanted growth of the bounding box.
\tikzset{tight/.style={inner sep=0, outer sep=0.3333em}}


% rounded tree edges style
% usage: rounded tree edges={⟨direction⟩}{⟨looseness⟩}{⟨strength⟩}
\tikzset{
	rounded tree edges/.style n args={3}{
	edge from parent path={
	let
		\n{direction}={#1},
		\n{looseness}={#2},
		\n{strength}={#3},
		\p1=(\tikzparentnode),
		\p2=(\tikzchildnode),
		\p3=(\n{direction}:1pt),
		\p4=(\x2 - \x1, \y2 - \y1),
		\n{dist}={veclen(\p4)},
		\p4=(\x4 / \n{dist}, \y4 / \n{dist}),
		\n{angle}={atan2(\y4, \x4)},
		\n{delta}={Mod(\n{angle} - \n{direction}, 360)},
		\n{delta}={\n{delta} > 180 ? \n{delta} - 360  : \n{delta}},
		\n{delta}={\n{delta} >  90 ?  180 - \n{delta} : \n{delta}},
		\n{delta}={\n{delta} < -90 ? -180 - \n{delta} : \n{delta}}
	in (\tikzparentnode) .. controls
		+(    \n{angle}+\n{strength}*\n{delta}:\n{looseness}*0.3915*\n{dist}) and
		+(180+\n{angle}-\n{strength}*\n{delta}:\n{looseness}*0.3915*\n{dist}) ..
		(\tikzchildnode)
	}
	}
}


% Tear out snippets from PDFs.
% Usage: \tear[…]{file.pdf}
% The optional parameter is the same as for \includegraphics.
% Useful Arguments:
%   * page=‹pagenumber›
%   * trim=‹left› ‹bottom› ‹right› ‹top›
%   * width=0.98\linewidth
\newcommand*{\tear}[2][]{%
	\begin{tikzpicture}
		\node
			[ blur shadow
			, clip
			, decorate
			, decoration=random steps
			, draw
			, inner sep=0
			, preaction={fill=white}% hide the shadow if paper is transparent
			] {\includegraphics[#1]{#2}};
	\end{tikzpicture}%
}


\makeatletter
\newcommand*{\timeline}[3][0]{%
	\ifcsname timeline@cmd@#3\endcsname%
		\@timeline[#1]{#2}{#3}%
		\PackageWarning{timeline}{redefining timeline \@backslashchar\string#3}%
	\else%
		\ifcsname#3\endcsname%
			\errmessage{Command \@backslashchar\string#3 already defined}%
		\else%
			\@timeline[#1]{#2}{#3}%
		\fi%
	\fi%
}%
\newcommand*{\@timeline}[3][0]{%
	% mark command as timeline command – they can be overwritten
	\expandafter\def\csname timeline@cmd@#3\endcsname{}%
	\setcounter{@timeline}{#1}%
	\def\timeline@cmd{#3}%
	\timeline@reset%
	\timeline@append{0}%
	\@tfor\timeline@next:=#2\do{%
		\if\timeline@next+%
			\stepcounter{@timeline}%
			\timeline@append{,\the@timeline}%
		\else\if\timeline@next-%
			\stepcounter{@timeline}%
		\else%
			%\timeline@append{\timeline@next}%
			\GenericError{}{\protect\timeline: ignoring unknown character: \timeline@next}%
		\fi\fi%
	}%
}%
% \newcommand*{\tl}[1]{%
% 	\ifcsname timeline@cmd@#1\endcsname%
% 		\csname timeline@cmd@#1\endcsname%
% 	\else%
% 		0%
% 		%\GenericError{}{\protect\tl: timeline not defined: #1}%
% 	\fi%
% }%
\newcounter{@timeline}%
\def\timeline@reset{%
	\expandafter\def\csname\timeline@cmd\endcsname{}%
}%
\def\timeline@append#1{%
	\expandafter\edef\csname\timeline@cmd\endcsname{%
		\csname\timeline@cmd\endcsname#1%
	}%
}%
\makeatother


\newcommand*{\xminus}[1]{%
	\mathrel{\tikz[baseline={([yshift=-0.25em]n.south)}, inner sep=0, outer sep=0.2em]{%
		\node (n) {\(\scriptstyle #1\)};
		\draw (n.south west) -- (n.south east);
	}}%
}
\newcommand*{\tikzrightarrow}[1]{%
	\mathrel{\tikz[baseline={([yshift=-0.25em]n.south)}, inner sep=0, outer sep=0.2em]{%
		\node (n) {\(\scriptstyle #1\)};
		\draw[->, > = Computer Modern Rightarrow, line width = 0.4pt] (n.south west) -- (n.south east);
	}}%
}


%%%%%%%%%%%%%%%%%%%%%%%%%%%%%%%%%%%%%%%%%%%%%%%%%%%%%%%%%%%%%%%%%%%%%%%%%%%%%%
% document specific commands
%%%%%%%%%%%%%%%%%%%%%%%%%%%%%%%%%%%%%%%%%%%%%%%%%%%%%%%%%%%%%%%%%%%%%%%%%%%%%%

\newcommand<>*{\mycite}[1]{\uncover#2{{\color{HKS57K100}[\cite{#1}]}}}


\newcommand{\statetree}[1]{
	\tikz
	[ anchor=base
	, baseline=(current bounding box.center)
	, level distance=2em
	, sibling distance=2em
	]{
		\matrix
		[ draw=nt
		, edge from parent/.style={draw=black}
		, inner sep=0
		, nodes={inner sep=0.2em, rounded corners=0}
		, rounded corners
		] {#1\\}
	}
}


\newcommand*{\mylargeleaf}[1]{{\LARGE\color{HKS41K70}#1}}

\definecolor{state s}{named}{HKS57K80}
\definecolor{state t}{named}{HKS41K70}
\newcommand*{\stateS}[1]{{\color{state s}#1}}
\newcommand*{\stateT}[1]{{\color{state t}#1}}

\tikzset{
	subtree/.style =
		{ fill=lightgray
		, inner sep=0.02em
		, isosceles triangle apex angle=60
		, shape=isosceles triangle
		, shape border rotate=90
		}
	, state/.style = {circle, draw, inner sep=0.1em}
	, trans/.style = {rectangle, draw}
}

\newcommand*{\srBool}{\mathbb{B}}
\newcommand*{\srProb}{ℙ}


%%%%%%%%%%%%%%%%%%%%%%%%%%%%%%%%%%%%%%%%%%%%%%%%%%%%%%%%%%%%%%%%%%%%%%%%%%%%%%
% commands for specific notations
%%%%%%%%%%%%%%%%%%%%%%%%%%%%%%%%%%%%%%%%%%%%%%%%%%%%%%%%%%%%%%%%%%%%%%%%%%%%%%

\DeclareMathOperator*{\argmax}{argmax}

\newcommand*{\cardinality}[1]{\lvert#1\rvert}
\newcommand*{\corpussize}[1]{\lvert#1\rvert}

\DeclareMathOperator{\crispOp}{crisp}
\newcommand*        {\crisp}[2][0]{\crispOp\delim{#1}({#2})}

\DeclareMathOperator{\lhsOp}{lhs}
\newcommand*{\lhs}[1]{\lhsOp(#1)}

\DeclareMathOperator{\lklhdOp}{L}
\newcommand*{\lklhd}[2]{\lklhdOp(#1 ∣ #2)}

\DeclareMathOperator{\mleOp}{mle}
\newcommand*{\mle}[2][]{%
	\ifthenelse{\isempty{#1}}{%
		\mleOp(#2)%
	}{%
		\mleOp_{#1}(#2)%
	}%
}

\DeclareMathOperator{\mrg}{merge}

% CVD: color vision deficiencies
\definecolor{CVD light red}   {HTML}{FF8080}
\definecolor{CVD light yellow}{HTML}{FFFF80}
\definecolor{CVD light green} {HTML}{40FFC0}

\definecolor{nt}{named}{HKS41K70}
\newcommand*{\nt}[1]{{\color{nt}#1}}

% set of all probability distributions over #1
\DeclareMathOperator{\pdsOp}{Pd}
\newcommand*{\pds}[1]{\pdsOp(#1)}

\DeclareMathOperator{\positionsOp}{pos}
\newcommand*{\positions}[1]{\positionsOp(#1)}

\DeclareMathOperator{\rankOp}{rk}
\newcommand*{\rank}[1]{\rankOp(#1)}

\DeclareMathOperator{\runsOp}{run}
\newcommand*{\runs}[2][]{%
	\ifthenelse%
		{\isempty{#1}}%
		{\runsOp(#2)}%
		{\runsOp_{#1}(#2)}%
}

\newcommand*{\semantics}[1]{⟦#1⟧}

\DeclareMathOperator{\splt}{split}

\newcommand*{\subtree}[2]{#1|_{#2}}

\DeclareMathOperator{\supportOp}{supp}
\newcommand*{\support}[1]{\supportOp(#1)}

\newcommand*{\symId}{\textsc{\color{gray}Id}}
\newcommand*{\symCons}{\textsc{\color{gray}Cons}}
\newcommand*{\symFlip}{\textsc{\color{gray}Flip}}
\newcommand*{\symNull}{\textsc{\color{gray}Null}}
\newcommand*{\symNullR}{\textsc{\color{gray}N\(\overline{\textsc{ull}}\)}}
\newcommand*{\symSnoc}{\textsc{\color{gray}Snoc}}

\newcommand*{\transWTA}[4][]{#3 \xrightarrow{#1} #2(#4)}

\DeclareMathOperator{\uniqueRunOp}{r}
\newcommand*{\uniqueRun}[2][]{%
	\ifthenelse%
		{\isempty{#1}}%
		{\uniqueRunOp^{#2}}%
		{\uniqueRunOp_{\!#1}^{#2}}%
}

\DeclareMathOperator{\treesOp}{T}
\newcommand*{\trees}[2][]{%
	\ifthenelse%
		{\isempty{#1}}%
		{\treesOp_{\!#2}}%
		{\treesOp_{\!#2}(#1)}%
}
\DeclareMathOperator{\treesUOp}{U}
\newcommand*{\treesU}[2][]{%
	\ifthenelse%
		{\isempty{#1}}%
		{\treesUOp_{#2}}%
		{\treesUOp_{#2}(#1)}%
}


%%%%%%%%%%%%%%%%%%%%%%%%%%%%%%%%%%%%%%%%%%%%%%%%%%%%%%%%%%%%%%%%%%%%%%%%%%%%%%
% metadata
%%%%%%%%%%%%%%%%%%%%%%%%%%%%%%%%%%%%%%%%%%%%%%%%%%%%%%%%%%%%%%%%%%%%%%%%%%%%%%

\ifstandalonebeamer\else
	\title[Defense of Dissertation]{A Formal View on Training of Weighted Tree Automata by Likelihood-Driven State Splitting and Merging}
	\subtitle{Defense of Dissertation}
\fi
\author{Toni Dietze}
\institute[TU Dresden]{%
	\href{https://www.orchid.inf.tu-dresden.de/index.en/}{Chair for Foundations of Programming}
\\	\href{https://tu-dresden.de/ing/informatik/thi}{Institute of Theoretical Computer Science}
\\	\href{https://tu-dresden.de/ing/informatik}{Faculty of Computer Science}
\\	\href{https://tu-dresden.de/}{Technische Universität Dresden}
\\	01062 Dresden, Germany
}
\date[2018-09-27]{September 27, 2018}

\begin{document}
%         1 2 3 4 5 6 7
\timeline{+ - - - - - -}{tlUncover}
\timeline{- - + + - - -}{tlTrainingOfWTAs}
\timeline{- - - - + - -}{tlStateSplittingAndMerging}
\timeline{- - - - - + -}{tlCountBasedStateMerging}
\timeline{- - - - - - +}{tlBinarization}
\begin{standaloneframe}{\jobname}
	\setbeamercovered{transparent}
	\begin{columns}
	\column{0.44\linewidth}
		\begin{enumerate}
		\item
			Introduction
		\item<\tlUncover>
			Preliminaries
		\item<\tlUncover>
			Language Formalisms
		\item<1-|alert@\tlTrainingOfWTAs>
			Training of WTAs
		\item<1-|alert@\tlStateSplittingAndMerging>
			State Splitting and Merging
		\item<1-|alert@\tlCountBasedStateMerging>
			Count-Based State Merging
		\item<1-|alert@\tlBinarization>
			Binarization
		\end{enumerate}
	\end{columns}
	%\only<\tlStateSplittingAndMerging>{%
	%	\vspace{1em}
	%	\begin{block}{Chapter~5 rigorously formalizes the approach of}
	%		\printfullcite{2006PetrovBarrettThibauxKlein}%
	%	\end{block}
	%}%
	\only<\tlCountBasedStateMerging>{%
		\vspace{1em}
		\begin{block}{Chapter~6 is a substantially extended version of}
			\printfullcite{2015DietzeNederhof}%
		\end{block}
	}%
	\only<\tlBinarization>{%
		\vspace{0.5em}
		\begin{block}{Chapter~7 is a substantially extended version of}
			\printfullcite{2016Dietze}%
		\end{block}
	}%
\end{standaloneframe}
\end{document}

\end{frame}


\section{Training}

\begin{frame}[t]{\secname}{}
	\centering
	% SPDX-License-Identifier: CC-BY-4.0
% Copyright 2018 Toni Dietze
\documentclass[beamer]{standalone}
% \usepackage{amsmath}  % has to be loaded before unicode-math
% \usepackage[math-style=ISO]{unicode-math}
% \usepackage{tikz}
% \usetikzlibrary{calc}
% \usetikzlibrary{fit}
% \usetikzlibrary{positioning}
% SPDX-License-Identifier: CC-BY-4.0 OR MIT-0
% Copyright 2018 Toni Dietze
%
\usefonttheme{professionalfonts}

% LuaLaTeX specific packages
\usepackage{fontspec}
	\defaultfontfeatures{Ligatures=TeX}
\usepackage{polyglossia}
	\setdefaultlanguage{english}
\usepackage{amsmath}  % has to be loaded before unicode-math
\usepackage[math-style=ISO]{unicode-math}
	\setmathfont{Latin Modern Math}
% 	\setmathfont[range={\mathcal,\mathbfcal},StylisticSet=1]{xits-math.otf}
% 	\setmathfont[range={"029F5}]{XITS Math}  % ⧵
% 	\setmathfont[range={\mathscr,\mathbfscr},StylisticSet=1]{Latin Modern Math}  % make \mathscr use the correct font

\usepackage[noend]{algpseudocode}
	\algrenewcommand\algorithmicrequire{\textbf{Input:}}
	\algrenewcommand\algorithmicensure{\textbf{Output:}}
\usepackage[backend=biber, maxbibnames=42, maxcitenames=42, sorting=ynt, style=authoryear]{biblatex}
\usepackage{csquotes}
\usepackage{mathtools}
\usepackage{media9}
\usepackage{scalerel}
\usepackage{standalone}
\usepackage{tikz}
	\usetikzlibrary{arrows.meta}
	\usetikzlibrary{backgrounds}
	\usetikzlibrary{calc}
	\usetikzlibrary{decorations}
	\usetikzlibrary{decorations.pathmorphing}
	\usetikzlibrary{decorations.pathreplacing}
	\usetikzlibrary{fadings}
	\usetikzlibrary{fit}
	\usetikzlibrary{graphs}
	\usetikzlibrary{graphdrawing}
	\usetikzlibrary{intersections}
	\usetikzlibrary{positioning}
	\usetikzlibrary{quotes}
	\usetikzlibrary{shadows.blur}
	\usetikzlibrary{shapes.arrows}
	\usetikzlibrary{shapes.geometric}
	\usegdlibrary{trees}
\usepackage{xifthen}
\usepackage{xspace}

\usepackage{pgfplots}
	\pgfplotsset
		{ compat = 1.15
		, /pgf/number format/1000 sep = {\,}
		, /pgf/number format/assume math mode = true
		, every axis plot/.append style =
			{ mark options = {fill opacity = 0.25}
			}
		}
	\usepgfplotslibrary{groupplots}
\usepackage{pgfplotstable}

\hypersetup
	{ bookmarksopen
	, pdflang = en
	, unicode
	}


%%%%%%%%%%%%%%%%%%%%%%%%%%%%%%%%%%%%%%%%%%%%%%%%%%%%%%%%%%%%%%%%%%%%%%%%%%%%%%


% always show bad boxes
%\overfullrule=1em


%%%%%%%%%%%%%%%%%%%%%%%%%%%%%%%%%%%%%%%%%%%%%%%%%%%%%%%%%%%%%%%%%%%%%%%%%%%%%%
% biblatex
%%%%%%%%%%%%%%%%%%%%%%%%%%%%%%%%%%%%%%%%%%%%%%%%%%%%%%%%%%%%%%%%%%%%%%%%%%%%%%

\addbibresource{slides-dissertation-defense.bib}
% \renewcommand*{\finalnamedelim}{\addcomma\space}
% \setlength{\bibitemsep}{1em}
% 
\AtEveryBibitem{% Clean up the bibtex rather than editing it
 \clearlist{address}
 \clearfield{date}
 \clearfield{eprint}
 \clearfield{isbn}
 \clearfield{issn}
 \clearlist{language}
 \clearlist{location}
 \clearfield{month}
 \clearfield{series}
%  \clearfield{url}
%  \clearfield{doi}
 \clearfield{organization}

%  \ifentrytype{book}{}{% Remove stuff except for books
%   \clearfield{booktitle}
%   \clearfield{pages}
  \clearlist{publisher}
  \clearname{editor}
%  }
}
% do not print url if doi is present
% http://tex.stackexchange.com/questions/154864/biblatex-use-doi-only-if-there-is-no-url
\DeclareSourcemap{
	\maps[datatype=bibtex]{
		\map{
			\step[fieldsource=doi,final]
			\step[fieldset=url,null]
}	}	}
%
% remove qoutes around titles
\DeclareFieldFormat
	[article,inbook,incollection,inproceedings,patent,thesis,unpublished]
	{title}{#1\isdot}
% 
% \DeclareFieldFormat{url}{\mkbibacro{URL}\addcolon\addnbspace\url{#1}}
% 
% \DeclareNameAlias{sortname}{first-last}
% 
\renewbibmacro{in:}{\ifentrytype{article}{}{}}


%%%%%%%%%%%%%%%%%%%%%%%%%%%%%%%%%%%%%%%%%%%%%%%%%%%%%%%%%%%%%%%%%%%%%%%%%%%%%%
% beamer
%%%%%%%%%%%%%%%%%%%%%%%%%%%%%%%%%%%%%%%%%%%%%%%%%%%%%%%%%%%%%%%%%%%%%%%%%%%%%%

\useoutertheme{infolines}
\makeatletter
% based on
% /usr/share/texmf-dist/tex/latex/beamer/beamerouterthemeinfolines.sty
\setbeamertemplate{footline}
{%
	\leavevmode%
	\hbox{%
	\begin{beamercolorbox}[wd=.333333\paperwidth,ht=2.25ex,dp=1ex,center]{author in head/foot}%
		\usebeamerfont{author in head/foot}\insertshortauthor\expandafter\beamer@ifempty\expandafter{\beamer@shortinstitute}{}{~~(\insertshortinstitute)}
	\end{beamercolorbox}%
	\begin{beamercolorbox}[wd=.333333\paperwidth,ht=2.25ex,dp=1ex,center]{title in head/foot}%
		\usebeamerfont{title in head/foot}\insertshorttitle
	\end{beamercolorbox}%
	\begin{beamercolorbox}[wd=.333333\paperwidth,ht=2.25ex,dp=1ex,right]{date in head/foot}%
		\usebeamerfont{date in head/foot}%
		\hfill\insertshortdate\hfill\hfill%
		%\hspace*{2ex}%
		%\insertshortdate%
		%\hspace{0pt plus 1 filll}%
		%(\insertframenumber.\insertoverlaynumber{} / \insertmainframenumber)%
		%\hspace{0pt plus 1 filll}%
		\phantom{000}\llap{\insertpagenumber} / \insertpresentationendpage%
		\hspace*{2ex}%
	\end{beamercolorbox}}%
	\vskip0pt%
}
\makeatother
\useinnertheme{circles}
\beamertemplatenavigationsymbolsempty
\setbeamertemplate{bibliography item}{}
\setbeamertemplate{headline}[default]

\input{tudcolors.tex}
\setbeamercolor*{alerted text}{fg=HKS07K100}
\usecolortheme[named=HKS41K100]{structure}

\setbeamercolor*{palette primary}{use=structure,fg=white,bg=structure.fg}
\setbeamercolor*{palette secondary}{use=structure,fg=white,bg=structure.fg!80}
\setbeamercolor*{palette tertiary}{use=structure,fg=white,bg=structure.fg!60}
\setbeamercolor*{palette quaternary}{fg=white,bg=black}

\setbeamercolor*{sidebar}{use=structure,bg=structure.fg}

\setbeamercolor*{palette sidebar primary}{use=structure,fg=structure.fg!20}
\setbeamercolor*{palette sidebar secondary}{fg=white}
\setbeamercolor*{palette sidebar tertiary}{use=structure,fg=structure.fg!40}
\setbeamercolor*{palette sidebar quaternary}{fg=white}

\setbeamercolor*{titlelike}{parent=palette primary}

\setbeamercolor*{separation line}{}
\setbeamercolor*{fine separation line}{}

\setbeamercolor{block title}{use=structure,fg=white,bg=structure.fg}
\setbeamercolor{block title alerted}{use=alerted text,fg=white,bg=alerted text.fg!75!black}
\setbeamercolor{block title example}{use=example text,fg=white,bg=example text.fg!75!black}

\setbeamercolor{block body}{parent=normal text,use=block title,bg=block title.bg!10!bg}
\setbeamercolor{block body alerted}{parent=normal text,use=block title alerted,bg=block title alerted.bg!10!bg}
\setbeamercolor{block body example}{parent=normal text,use=block title example,bg=block title example.bg!10!bg}

% \setbeamertemplate{itemize items}[default]


%%%%%%%%%%%%%%%%%%%%%%%%%%%%%%%%%%%%%%%%%%%%%%%%%%%%%%%%%%%%%%%%%%%%%%%%%%%%%%
% TikZ
%%%%%%%%%%%%%%%%%%%%%%%%%%%%%%%%%%%%%%%%%%%%%%%%%%%%%%%%%%%%%%%%%%%%%%%%%%%%%%

\tikzset
	{ > = Stealth
	}


%%%%%%%%%%%%%%%%%%%%%%%%%%%%%%%%%%%%%%%%%%%%%%%%%%%%%%%%%%%%%%%%%%%%%%%%%%%%%%
% general commands and styles
%%%%%%%%%%%%%%%%%%%%%%%%%%%%%%%%%%%%%%%%%%%%%%%%%%%%%%%%%%%%%%%%%%%%%%%%%%%%%%

% \delegateStyle and \inheritStyle command
% usage: \delegateStyle{… \inheritStyle{…} …}
% example: \(X_{\delegateStyle{\fbox{\inheritStyle{X}}}}\)
% Save the current style and regain it in the argument.
% This works both for math and text mode, and can be nested.
% Acknowledgments: Based on \ThisStyle and \SavedStyle from scalerel package.
\makeatletter
\newcommand*{\@inheritStyle@D}[1]{\(\displaystyle      #1\)}
\newcommand*{\@inheritStyle@T}[1]{\(\textstyle         #1\)}
\newcommand*{\@inheritStyle@S}[1]{\(\scriptstyle       #1\)}
\newcommand*{\@inheritStyle@s}[1]{\(\scriptscriptstyle #1\)}
\newcommand*{\@inheritStyle@t}[1]{#1}
\newcommand*{\inheritStyle}{\csname @inheritStyle@\@inheritStyleSwitch\endcsname}
\newcommand*{\delegateStyle}[1]{%
	\ifmmode%
		\mathchoice%
		{\edef\@inheritStyleSwitch{D}#1}%
		{\edef\@inheritStyleSwitch{T}#1}%
		{\edef\@inheritStyleSwitch{S}#1}%
		{\edef\@inheritStyleSwitch{s}#1}%
	\else%
		\edef\@inheritStyleSwitch{t}#1%
	\fi%
}
\makeatother


% \oalt command
% requires: \delegateStyle and \inheritStyle command
% usage: \oalt<…>[…]{…}{…} (cf. \alt)
% Behaves like \alt, but reserves space according to largest overlays.
% The optional argument defines the alignment inside the reserved space;
% it is one of c, l, r, s (cf. \makebox); the default is c.
\makeatletter
\newlength{\oalt@dp}
\newlength{\oalt@ht}
\newlength{\oalt@wd}
\newbox{\oalt@a}
\newbox{\oalt@b}
\newbox{\oalt@empty}
\newcommand<>*{\oalt}[3][c]{%
	\delegateStyle{%
		% based on \setto… in /usr/share/texmf-dist/tex/latex/base/latex.ltx
		\setbox\oalt@a\hbox{\inheritStyle{#2}}%
		\setbox\oalt@b\hbox{\inheritStyle{#3}}%
		\pgfmathsetlength{\oalt@dp}{max(\dp\oalt@a,\dp\oalt@b)}%
		\pgfmathsetlength{\oalt@ht}{max(\ht\oalt@a,\ht\oalt@b)}%
		\pgfmathsetlength{\oalt@wd}{max(\wd\oalt@a,\wd\oalt@b)}%
		\raisebox{0pt}[\oalt@ht][\oalt@dp]{%
			\makebox[\oalt@wd][#1]{%
				\alt#4{\unhbox\oalt@a}{\unhbox\oalt@b}%
			}%
		}%
		\setbox\oalt@a\box\oalt@empty%
		\setbox\oalt@b\box\oalt@empty%
	}%
}
\makeatother


% \otemporal command
% requires: \delegateStyle and \inheritStyle command
% usage: \otemporal<…>[…]{…}{…}{…} (cf. \temporal)
% Behaves like \temporal, but reserves space according to largest overlays.
% The optional argument defines the alignment inside the reserved space;
% it is one of c, l, r, s (cf. \makebox); the default is c.
\makeatletter
\newlength{\ot@dp}
\newlength{\ot@ht}
\newlength{\ot@wd}
\newbox{\ot@a}
\newbox{\ot@b}
\newbox{\ot@c}
\newbox{\ot@empty}
\newcommand<>*{\otemporal}[4][c]{%
	\delegateStyle{%
		% based on \setto… in /usr/share/texmf-dist/tex/latex/base/latex.ltx
		\setbox\ot@a\hbox{\inheritStyle{#2}}%
		\setbox\ot@b\hbox{\inheritStyle{#3}}%
		\setbox\ot@c\hbox{\inheritStyle{#4}}%
		\pgfmathsetlength{\ot@dp}{max(\dp\ot@a,\dp\ot@b,\dp\ot@c)}%
		\pgfmathsetlength{\ot@ht}{max(\ht\ot@a,\ht\ot@b,\ht\ot@c)}%
		\pgfmathsetlength{\ot@wd}{max(\wd\ot@a,\wd\ot@b,\wd\ot@c)}%
		\raisebox{0pt}[\ot@ht][\ot@dp]{%
			\makebox[\ot@wd][#1]{%
				\temporal#5{\unhbox\ot@a}{\unhbox\ot@b}{\unhbox\ot@c}%
			}%
		}%
		\setbox\ot@a\box\ot@empty%
		\setbox\ot@b\box\ot@empty%
		\setbox\ot@c\box\ot@empty%
	}%
}
\makeatother


% Resize delimiters like braces, brackets, etc.
% Parameters: size, left delimiter, formula, right delimiter
% Example: \delim2({\frac{1}{2}})
\newcommand*{\delim}[4]{%
	\ifcase#1%
		#2#3#4%
	\or%
		\bigl#2#3\bigr#4%
	\or%
		\Bigl#2#3\Bigr#4%
	\or%
		\biggl#2#3\biggr#4%
	\or%
		\Biggl#2#3\Biggr#4%
	\else%
		\left#2#3\right#4%
	\fi%
}


% similar to \fullcite, but using the formatting of \printbibliography
\newcommand*{\printfullcite}[1]{%
	\begin{refsection}%
		\nocite{#1}%
		\DeclareNameAlias{author}{first-last}%
		\printbibliography[heading = none]%
	\end{refsection}%
}


\colorlet{light alert}{HKS07K60}
\tikzset{alert.bg/.style={rounded corners, fill=light alert}}
\tikzset{every picture/.style={line cap=round, semithick}}
% http://tex.stackexchange.com/questions/6135/how-to-make-beamer-overlays-with-tikz-node
\tikzset{onslide/.code args={<#1>#2}{\only<#1>{\pgfkeysalso{#2}}}}
\tikzset{invisible/.code args={<#1>}{\alt<#1>{\pgfkeysalso{transparent}}{\pgfkeysalso{opaque}}}}
\tikzset{uncover/.code args={<#1>}{\alt<#1>{\pgfkeysalso{opaque}}{\pgfkeysalso{opacity=0.25}}}}
\tikzset{visible/.code args={<#1>}{\alt<#1>{\pgfkeysalso{opaque}}{\pgfkeysalso{transparent}}}}
\tikzset{vuncover/.code args=%
	{<#1><#2>}%
	{\alt<#1>%
		{\alt<#2>%
			{\pgfkeysalso{opaque}}%
			{\pgfkeysalso{opacity=0.25}}%
		}{\pgfkeysalso{transparent}}%
	}%
}

\newcommand<%
	>{\tikzhighlight}[2][]{%
	\delegateStyle{\alt#3%
		{\tikz[baseline=0, anchor=base, inner sep=0.2em, text height=, text depth=]{\node[alert.bg, #1]{\inheritStyle{#2}};}}%
		{\tikz[baseline=0, anchor=base, inner sep=0.2em, text height=, text depth=]{\node[#1, fill=none]{\inheritStyle{#2}};}}%
	}%
}

\newcommand{\mathhighlight}{\tikzhighlight}

\newcommand<>{\mhl}[2][]{\mathhighlight#3[inner sep=0.2em, #1]{#2}}


\newcommand<>{\inlineblock}[2][]{{%
	\usebeamercolor*[fg]{block body}%
	\tikzhighlight#3[fill=block body.bg, #1]{#2}%
}}


% a small letter s for plurals of abbreviations
\newcommand*{\s}{{\scriptsize s}\xspace}


\newcommand<>*{\sout}[2][opacity=0.75, ultra thick]{%
	\delegateStyle{%
		\tikz[baseline=0, anchor=base, inner sep=0, outer sep=0]{
			\useasboundingbox node (n) {\inheritStyle{#2}};
			\only#3{
				\node (h) {\inheritStyle{\ifmmode\mathstrut\else\strut\fi}};
				\draw[#1] (n.west |- {$(h.south)!0.5!(h.north)$}) -- (n.east |- {$(h.south)!0.5!(h.north)$});
			}
		}%
	}%
}


% tight style
% Sets outer sep to default inner sep and inner sep to 0.
% Use this style for nodes that are neither drawn nor filled to prevent
% unwanted growth of the bounding box.
\tikzset{tight/.style={inner sep=0, outer sep=0.3333em}}


% rounded tree edges style
% usage: rounded tree edges={⟨direction⟩}{⟨looseness⟩}{⟨strength⟩}
\tikzset{
	rounded tree edges/.style n args={3}{
	edge from parent path={
	let
		\n{direction}={#1},
		\n{looseness}={#2},
		\n{strength}={#3},
		\p1=(\tikzparentnode),
		\p2=(\tikzchildnode),
		\p3=(\n{direction}:1pt),
		\p4=(\x2 - \x1, \y2 - \y1),
		\n{dist}={veclen(\p4)},
		\p4=(\x4 / \n{dist}, \y4 / \n{dist}),
		\n{angle}={atan2(\y4, \x4)},
		\n{delta}={Mod(\n{angle} - \n{direction}, 360)},
		\n{delta}={\n{delta} > 180 ? \n{delta} - 360  : \n{delta}},
		\n{delta}={\n{delta} >  90 ?  180 - \n{delta} : \n{delta}},
		\n{delta}={\n{delta} < -90 ? -180 - \n{delta} : \n{delta}}
	in (\tikzparentnode) .. controls
		+(    \n{angle}+\n{strength}*\n{delta}:\n{looseness}*0.3915*\n{dist}) and
		+(180+\n{angle}-\n{strength}*\n{delta}:\n{looseness}*0.3915*\n{dist}) ..
		(\tikzchildnode)
	}
	}
}


% Tear out snippets from PDFs.
% Usage: \tear[…]{file.pdf}
% The optional parameter is the same as for \includegraphics.
% Useful Arguments:
%   * page=‹pagenumber›
%   * trim=‹left› ‹bottom› ‹right› ‹top›
%   * width=0.98\linewidth
\newcommand*{\tear}[2][]{%
	\begin{tikzpicture}
		\node
			[ blur shadow
			, clip
			, decorate
			, decoration=random steps
			, draw
			, inner sep=0
			, preaction={fill=white}% hide the shadow if paper is transparent
			] {\includegraphics[#1]{#2}};
	\end{tikzpicture}%
}


\makeatletter
\newcommand*{\timeline}[3][0]{%
	\ifcsname timeline@cmd@#3\endcsname%
		\@timeline[#1]{#2}{#3}%
		\PackageWarning{timeline}{redefining timeline \@backslashchar\string#3}%
	\else%
		\ifcsname#3\endcsname%
			\errmessage{Command \@backslashchar\string#3 already defined}%
		\else%
			\@timeline[#1]{#2}{#3}%
		\fi%
	\fi%
}%
\newcommand*{\@timeline}[3][0]{%
	% mark command as timeline command – they can be overwritten
	\expandafter\def\csname timeline@cmd@#3\endcsname{}%
	\setcounter{@timeline}{#1}%
	\def\timeline@cmd{#3}%
	\timeline@reset%
	\timeline@append{0}%
	\@tfor\timeline@next:=#2\do{%
		\if\timeline@next+%
			\stepcounter{@timeline}%
			\timeline@append{,\the@timeline}%
		\else\if\timeline@next-%
			\stepcounter{@timeline}%
		\else%
			%\timeline@append{\timeline@next}%
			\GenericError{}{\protect\timeline: ignoring unknown character: \timeline@next}%
		\fi\fi%
	}%
}%
% \newcommand*{\tl}[1]{%
% 	\ifcsname timeline@cmd@#1\endcsname%
% 		\csname timeline@cmd@#1\endcsname%
% 	\else%
% 		0%
% 		%\GenericError{}{\protect\tl: timeline not defined: #1}%
% 	\fi%
% }%
\newcounter{@timeline}%
\def\timeline@reset{%
	\expandafter\def\csname\timeline@cmd\endcsname{}%
}%
\def\timeline@append#1{%
	\expandafter\edef\csname\timeline@cmd\endcsname{%
		\csname\timeline@cmd\endcsname#1%
	}%
}%
\makeatother


\newcommand*{\xminus}[1]{%
	\mathrel{\tikz[baseline={([yshift=-0.25em]n.south)}, inner sep=0, outer sep=0.2em]{%
		\node (n) {\(\scriptstyle #1\)};
		\draw (n.south west) -- (n.south east);
	}}%
}
\newcommand*{\tikzrightarrow}[1]{%
	\mathrel{\tikz[baseline={([yshift=-0.25em]n.south)}, inner sep=0, outer sep=0.2em]{%
		\node (n) {\(\scriptstyle #1\)};
		\draw[->, > = Computer Modern Rightarrow, line width = 0.4pt] (n.south west) -- (n.south east);
	}}%
}


%%%%%%%%%%%%%%%%%%%%%%%%%%%%%%%%%%%%%%%%%%%%%%%%%%%%%%%%%%%%%%%%%%%%%%%%%%%%%%
% document specific commands
%%%%%%%%%%%%%%%%%%%%%%%%%%%%%%%%%%%%%%%%%%%%%%%%%%%%%%%%%%%%%%%%%%%%%%%%%%%%%%

\newcommand<>*{\mycite}[1]{\uncover#2{{\color{HKS57K100}[\cite{#1}]}}}


\newcommand{\statetree}[1]{
	\tikz
	[ anchor=base
	, baseline=(current bounding box.center)
	, level distance=2em
	, sibling distance=2em
	]{
		\matrix
		[ draw=nt
		, edge from parent/.style={draw=black}
		, inner sep=0
		, nodes={inner sep=0.2em, rounded corners=0}
		, rounded corners
		] {#1\\}
	}
}


\newcommand*{\mylargeleaf}[1]{{\LARGE\color{HKS41K70}#1}}

\definecolor{state s}{named}{HKS57K80}
\definecolor{state t}{named}{HKS41K70}
\newcommand*{\stateS}[1]{{\color{state s}#1}}
\newcommand*{\stateT}[1]{{\color{state t}#1}}

\tikzset{
	subtree/.style =
		{ fill=lightgray
		, inner sep=0.02em
		, isosceles triangle apex angle=60
		, shape=isosceles triangle
		, shape border rotate=90
		}
	, state/.style = {circle, draw, inner sep=0.1em}
	, trans/.style = {rectangle, draw}
}

\newcommand*{\srBool}{\mathbb{B}}
\newcommand*{\srProb}{ℙ}


%%%%%%%%%%%%%%%%%%%%%%%%%%%%%%%%%%%%%%%%%%%%%%%%%%%%%%%%%%%%%%%%%%%%%%%%%%%%%%
% commands for specific notations
%%%%%%%%%%%%%%%%%%%%%%%%%%%%%%%%%%%%%%%%%%%%%%%%%%%%%%%%%%%%%%%%%%%%%%%%%%%%%%

\DeclareMathOperator*{\argmax}{argmax}

\newcommand*{\cardinality}[1]{\lvert#1\rvert}
\newcommand*{\corpussize}[1]{\lvert#1\rvert}

\DeclareMathOperator{\crispOp}{crisp}
\newcommand*        {\crisp}[2][0]{\crispOp\delim{#1}({#2})}

\DeclareMathOperator{\lhsOp}{lhs}
\newcommand*{\lhs}[1]{\lhsOp(#1)}

\DeclareMathOperator{\lklhdOp}{L}
\newcommand*{\lklhd}[2]{\lklhdOp(#1 ∣ #2)}

\DeclareMathOperator{\mleOp}{mle}
\newcommand*{\mle}[2][]{%
	\ifthenelse{\isempty{#1}}{%
		\mleOp(#2)%
	}{%
		\mleOp_{#1}(#2)%
	}%
}

\DeclareMathOperator{\mrg}{merge}

% CVD: color vision deficiencies
\definecolor{CVD light red}   {HTML}{FF8080}
\definecolor{CVD light yellow}{HTML}{FFFF80}
\definecolor{CVD light green} {HTML}{40FFC0}

\definecolor{nt}{named}{HKS41K70}
\newcommand*{\nt}[1]{{\color{nt}#1}}

% set of all probability distributions over #1
\DeclareMathOperator{\pdsOp}{Pd}
\newcommand*{\pds}[1]{\pdsOp(#1)}

\DeclareMathOperator{\positionsOp}{pos}
\newcommand*{\positions}[1]{\positionsOp(#1)}

\DeclareMathOperator{\rankOp}{rk}
\newcommand*{\rank}[1]{\rankOp(#1)}

\DeclareMathOperator{\runsOp}{run}
\newcommand*{\runs}[2][]{%
	\ifthenelse%
		{\isempty{#1}}%
		{\runsOp(#2)}%
		{\runsOp_{#1}(#2)}%
}

\newcommand*{\semantics}[1]{⟦#1⟧}

\DeclareMathOperator{\splt}{split}

\newcommand*{\subtree}[2]{#1|_{#2}}

\DeclareMathOperator{\supportOp}{supp}
\newcommand*{\support}[1]{\supportOp(#1)}

\newcommand*{\symId}{\textsc{\color{gray}Id}}
\newcommand*{\symCons}{\textsc{\color{gray}Cons}}
\newcommand*{\symFlip}{\textsc{\color{gray}Flip}}
\newcommand*{\symNull}{\textsc{\color{gray}Null}}
\newcommand*{\symNullR}{\textsc{\color{gray}N\(\overline{\textsc{ull}}\)}}
\newcommand*{\symSnoc}{\textsc{\color{gray}Snoc}}

\newcommand*{\transWTA}[4][]{#3 \xrightarrow{#1} #2(#4)}

\DeclareMathOperator{\uniqueRunOp}{r}
\newcommand*{\uniqueRun}[2][]{%
	\ifthenelse%
		{\isempty{#1}}%
		{\uniqueRunOp^{#2}}%
		{\uniqueRunOp_{\!#1}^{#2}}%
}

\DeclareMathOperator{\treesOp}{T}
\newcommand*{\trees}[2][]{%
	\ifthenelse%
		{\isempty{#1}}%
		{\treesOp_{\!#2}}%
		{\treesOp_{\!#2}(#1)}%
}
\DeclareMathOperator{\treesUOp}{U}
\newcommand*{\treesU}[2][]{%
	\ifthenelse%
		{\isempty{#1}}%
		{\treesUOp_{#2}}%
		{\treesUOp_{#2}(#1)}%
}


%%%%%%%%%%%%%%%%%%%%%%%%%%%%%%%%%%%%%%%%%%%%%%%%%%%%%%%%%%%%%%%%%%%%%%%%%%%%%%
% metadata
%%%%%%%%%%%%%%%%%%%%%%%%%%%%%%%%%%%%%%%%%%%%%%%%%%%%%%%%%%%%%%%%%%%%%%%%%%%%%%

\ifstandalonebeamer\else
	\title[Defense of Dissertation]{A Formal View on Training of Weighted Tree Automata by Likelihood-Driven State Splitting and Merging}
	\subtitle{Defense of Dissertation}
\fi
\author{Toni Dietze}
\institute[TU Dresden]{%
	\href{https://www.orchid.inf.tu-dresden.de/index.en/}{Chair for Foundations of Programming}
\\	\href{https://tu-dresden.de/ing/informatik/thi}{Institute of Theoretical Computer Science}
\\	\href{https://tu-dresden.de/ing/informatik}{Faculty of Computer Science}
\\	\href{https://tu-dresden.de/}{Technische Universität Dresden}
\\	01062 Dresden, Germany
}
\date[2018-09-27]{September 27, 2018}

\begin{document}
%         1 2 3 4 5 6 7 8 9
\timeline{- + + + - - - - -}{tlDefCorpus}
\timeline{- - + + - - - - -}{tlDefLklhd}
\timeline{- - - + - - - - -}{tlDefMle}
\timeline{- - - - + + + + +}{tlStack}
\timeline{- - - - + + + + +}{tlPlot}
\timeline{- - - - - + + + +}{tlPlotUnderFitting}
\timeline{- - - - - - + + +}{tlPlotOverFitting}
\timeline{- - - - - - - + +}{tlHoldOut}
\timeline{- - - - - - - - +}{tlPlotHoldOut}
\begin{standaloneframe}[t]{\jobname}
	\centering%
	\alt<\tlStack>{%
		%\newcommand{\stacklist}{8/0.390625, 7/0.78125, 6/1.5625, 5/3.125, 4/6.25, 3/12.5, 3/12.5, 2/25, 1/50, 0/100}%
		%
		% let n = 8; size = 1.25; dist = 10 in mapM_ (\ (a, b) -> putStrLn $ ", " ++ show a ++ "em/" ++ show b ++ "%") $ map (\i -> (size * (n + dist) * i / (n * (dist + i)), 100 * 2 ** (-i))) [n, n-1 .. 0]
		\newcommand{\stacklist}
		{ 1.25em/0.390625%
		, 1.1580882352941178em/0.78125%
		, 1.0546875em/1.5625%
		, 0.9375em/3.125%
		, 0.8035714285714286em/6.25%
		, 0.6490384615384616em/12.5%
		, 0.46875em/25.0%
		, 0.2556818181818182em/50.0%
		, 0.0em/100.0%
		}%
	}{%
		\newcommand{\stacklist}{0/100}%
	}%
	\begin{tikzpicture}
	[ data/.style = {align=center, draw, rounded corners}
	, func/.style = {align=center, draw}
	]
		\node[data] (training data) {\strut\color{HKS41K70} training data};
		\node[func, below = 1.5em of training data, align = center] (training) {training \\ (maximize {\color{HKS41K70}reward})};
		\foreach \s/\c in \stacklist {
			\node
				[ data
				, left = of training
				, align = center
				, color = black!\c
				, fill = white
				, xshift = \s
				, yshift = \s
				] (model) {model \(\mathbfit{M}\)\\ \((p_θ \mid θ ∈ Θ)\)};
			\node
				[ data
				, below = 1.5em of training
				, color = black!\c
				, fill = white
				, xshift = \s
				, yshift = \s
				] (param) {\strut \(\hat{θ} ∈ Θ\)};
		}

		\draw[->] (training data) -- (training);
		\draw[->] (model) -- (training);
		\draw[->] (training) -- (param);

		\visible<\tlHoldOut>{%
			\node[data, right = 0.3333em of training data] (hold-out data) {\strut\color{HKS57K100} hold-out data};
			\node[inner sep = 0] at ([yshift = 1.75em]{$(training data)!0.5!(hold-out data)$}) (data) {data};
			\node[data, fit=(data) (training data) (hold-out data)] {};
			\node[func, align = center] (validation) at (hold-out data |- param) {hold-out \\ validation};
			\foreach \s/\c in \stacklist
				\node
					[ data
					, right = of validation
					, color = black!\c
					, fill = white
					, xshift = \s
					, yshift = \s
					] (reward) {\strut\color{HKS57K100} reward};

			\draw[->] (hold-out data) -- (validation);
			\draw[->] (param) -- (validation);
			\draw[->] (validation) -- (reward);
		}%
		\path[use as bounding box] ([xshift = 1.25em]reward.east);
	\end{tikzpicture}%
	\only<\tlDefCorpus,\tlDefLklhd,\tlDefMle>{%
		\\[0.5em]
		\begin{columns}[t]
		\column<\tlDefCorpus>{0.19\linewidth}
			\centering
			training data
			\begin{block}{\strut Corpus}
				\(c\colon \trees{Σ} → ℝ_{≥ 0}\)
				\\[0.5em] \(\support{c}\) finite
			\end{block}
		\column<\tlDefLklhd>{0.35\linewidth}
			\centering
			reward\vphantom{g}
			\begin{block}{\strut Likelihood}
			\vspace{0.65em}%
			\(\displaystyle
				\lklhd{c}{p} = ∏_{t ∈ \support{c}} p(t)^{c(t)}
			\)
			\end{block}
		\column<\tlDefMle>{0.39\linewidth}
			\centering
			training
			\begin{block}{Maximum Likelihood Estimation (mle)}
			\(\displaystyle
				\mle[\mathbfit{M}]{c} = \argmax_{θ ∈ Θ} \lklhd{c}{p_θ}
			\)
			\end{block}
		\end{columns}
	}%
	\only<\tlPlot>{%
		\\[1.5em]
		\begin{tikzpicture}[anchor = base]
			\only<\tlPlotUnderFitting>{%
				\fill[fill=lightgray, path fading=east] (0, 0) rectangle (9em, 5em);
				\node[gray, anchor=base east] at ( 9em, 0.3em) {underfitting};
			}%
			\only<\tlPlotOverFitting>{%
				\fill[fill=lightgray, path fading=west] (11em, 0) rectangle (16em, 5em);
				\node[gray, anchor=base west] at (11em, 0.3em) {overfitting};
			}%

			% axes
			\draw[<->] (18em, 0)
				-- (0, 0  ) node[pos = 0, below right, xshift = -2em] (xlabel) {model complexity}
				-- (0, 5em) node[midway, above, rotate = 90] (ylabel) {reward};

			\draw[HKS41K70] (0, 1em)
				.. controls +(4em, 3em) and +(0, 0) .. (16em, 4.5em)
				node[right] {reward for training data};

			\visible<\tlPlotHoldOut>{%
				\draw[HKS57K100] (0, 0.5em)
					.. controls +(0  , 0) and +(-4em, 0) .. (10em, 3em)
					.. controls +(3em, 0) and +( 0  , 0) .. (16em, 2.5em)
					node[right] {reward for hold-out data};

				\draw[dashed] (10em, -0.05) -- ++(0, 5em);
				\node at ({10em, 0} |- xlabel.base) {chosen model};
			}%
		\end{tikzpicture}
	}%
\end{standaloneframe}
\end{document}

\end{frame}


% \section{Literature}
%
% \begin{frame}{\secname{}}
% 	\printfullcite{2006PetrovBarrettThibauxKlein}
% 	\printfullcite{2001CarrascoOncinaCalera-Rubio}
% \end{frame}


\againframe<4>{toc}


% \section{(Weighted) Tree automata ((w)ta)}
%
% \begin{frame}{\secname}
% 	% SPDX-License-Identifier: CC-BY-4.0
% Copyright 2018 Toni Dietze
\documentclass[beamer]{standalone}
% SPDX-License-Identifier: CC-BY-4.0 OR MIT-0
% Copyright 2018 Toni Dietze
%
\usefonttheme{professionalfonts}

% LuaLaTeX specific packages
\usepackage{fontspec}
	\defaultfontfeatures{Ligatures=TeX}
\usepackage{polyglossia}
	\setdefaultlanguage{english}
\usepackage{amsmath}  % has to be loaded before unicode-math
\usepackage[math-style=ISO]{unicode-math}
	\setmathfont{Latin Modern Math}
% 	\setmathfont[range={\mathcal,\mathbfcal},StylisticSet=1]{xits-math.otf}
% 	\setmathfont[range={"029F5}]{XITS Math}  % ⧵
% 	\setmathfont[range={\mathscr,\mathbfscr},StylisticSet=1]{Latin Modern Math}  % make \mathscr use the correct font

\usepackage[noend]{algpseudocode}
	\algrenewcommand\algorithmicrequire{\textbf{Input:}}
	\algrenewcommand\algorithmicensure{\textbf{Output:}}
\usepackage[backend=biber, maxbibnames=42, maxcitenames=42, sorting=ynt, style=authoryear]{biblatex}
\usepackage{csquotes}
\usepackage{mathtools}
\usepackage{media9}
\usepackage{scalerel}
\usepackage{standalone}
\usepackage{tikz}
	\usetikzlibrary{arrows.meta}
	\usetikzlibrary{backgrounds}
	\usetikzlibrary{calc}
	\usetikzlibrary{decorations}
	\usetikzlibrary{decorations.pathmorphing}
	\usetikzlibrary{decorations.pathreplacing}
	\usetikzlibrary{fadings}
	\usetikzlibrary{fit}
	\usetikzlibrary{graphs}
	\usetikzlibrary{graphdrawing}
	\usetikzlibrary{intersections}
	\usetikzlibrary{positioning}
	\usetikzlibrary{quotes}
	\usetikzlibrary{shadows.blur}
	\usetikzlibrary{shapes.arrows}
	\usetikzlibrary{shapes.geometric}
	\usegdlibrary{trees}
\usepackage{xifthen}
\usepackage{xspace}

\usepackage{pgfplots}
	\pgfplotsset
		{ compat = 1.15
		, /pgf/number format/1000 sep = {\,}
		, /pgf/number format/assume math mode = true
		, every axis plot/.append style =
			{ mark options = {fill opacity = 0.25}
			}
		}
	\usepgfplotslibrary{groupplots}
\usepackage{pgfplotstable}

\hypersetup
	{ bookmarksopen
	, pdflang = en
	, unicode
	}


%%%%%%%%%%%%%%%%%%%%%%%%%%%%%%%%%%%%%%%%%%%%%%%%%%%%%%%%%%%%%%%%%%%%%%%%%%%%%%


% always show bad boxes
%\overfullrule=1em


%%%%%%%%%%%%%%%%%%%%%%%%%%%%%%%%%%%%%%%%%%%%%%%%%%%%%%%%%%%%%%%%%%%%%%%%%%%%%%
% biblatex
%%%%%%%%%%%%%%%%%%%%%%%%%%%%%%%%%%%%%%%%%%%%%%%%%%%%%%%%%%%%%%%%%%%%%%%%%%%%%%

\addbibresource{slides-dissertation-defense.bib}
% \renewcommand*{\finalnamedelim}{\addcomma\space}
% \setlength{\bibitemsep}{1em}
% 
\AtEveryBibitem{% Clean up the bibtex rather than editing it
 \clearlist{address}
 \clearfield{date}
 \clearfield{eprint}
 \clearfield{isbn}
 \clearfield{issn}
 \clearlist{language}
 \clearlist{location}
 \clearfield{month}
 \clearfield{series}
%  \clearfield{url}
%  \clearfield{doi}
 \clearfield{organization}

%  \ifentrytype{book}{}{% Remove stuff except for books
%   \clearfield{booktitle}
%   \clearfield{pages}
  \clearlist{publisher}
  \clearname{editor}
%  }
}
% do not print url if doi is present
% http://tex.stackexchange.com/questions/154864/biblatex-use-doi-only-if-there-is-no-url
\DeclareSourcemap{
	\maps[datatype=bibtex]{
		\map{
			\step[fieldsource=doi,final]
			\step[fieldset=url,null]
}	}	}
%
% remove qoutes around titles
\DeclareFieldFormat
	[article,inbook,incollection,inproceedings,patent,thesis,unpublished]
	{title}{#1\isdot}
% 
% \DeclareFieldFormat{url}{\mkbibacro{URL}\addcolon\addnbspace\url{#1}}
% 
% \DeclareNameAlias{sortname}{first-last}
% 
\renewbibmacro{in:}{\ifentrytype{article}{}{}}


%%%%%%%%%%%%%%%%%%%%%%%%%%%%%%%%%%%%%%%%%%%%%%%%%%%%%%%%%%%%%%%%%%%%%%%%%%%%%%
% beamer
%%%%%%%%%%%%%%%%%%%%%%%%%%%%%%%%%%%%%%%%%%%%%%%%%%%%%%%%%%%%%%%%%%%%%%%%%%%%%%

\useoutertheme{infolines}
\makeatletter
% based on
% /usr/share/texmf-dist/tex/latex/beamer/beamerouterthemeinfolines.sty
\setbeamertemplate{footline}
{%
	\leavevmode%
	\hbox{%
	\begin{beamercolorbox}[wd=.333333\paperwidth,ht=2.25ex,dp=1ex,center]{author in head/foot}%
		\usebeamerfont{author in head/foot}\insertshortauthor\expandafter\beamer@ifempty\expandafter{\beamer@shortinstitute}{}{~~(\insertshortinstitute)}
	\end{beamercolorbox}%
	\begin{beamercolorbox}[wd=.333333\paperwidth,ht=2.25ex,dp=1ex,center]{title in head/foot}%
		\usebeamerfont{title in head/foot}\insertshorttitle
	\end{beamercolorbox}%
	\begin{beamercolorbox}[wd=.333333\paperwidth,ht=2.25ex,dp=1ex,right]{date in head/foot}%
		\usebeamerfont{date in head/foot}%
		\hfill\insertshortdate\hfill\hfill%
		%\hspace*{2ex}%
		%\insertshortdate%
		%\hspace{0pt plus 1 filll}%
		%(\insertframenumber.\insertoverlaynumber{} / \insertmainframenumber)%
		%\hspace{0pt plus 1 filll}%
		\phantom{000}\llap{\insertpagenumber} / \insertpresentationendpage%
		\hspace*{2ex}%
	\end{beamercolorbox}}%
	\vskip0pt%
}
\makeatother
\useinnertheme{circles}
\beamertemplatenavigationsymbolsempty
\setbeamertemplate{bibliography item}{}
\setbeamertemplate{headline}[default]

\input{tudcolors.tex}
\setbeamercolor*{alerted text}{fg=HKS07K100}
\usecolortheme[named=HKS41K100]{structure}

\setbeamercolor*{palette primary}{use=structure,fg=white,bg=structure.fg}
\setbeamercolor*{palette secondary}{use=structure,fg=white,bg=structure.fg!80}
\setbeamercolor*{palette tertiary}{use=structure,fg=white,bg=structure.fg!60}
\setbeamercolor*{palette quaternary}{fg=white,bg=black}

\setbeamercolor*{sidebar}{use=structure,bg=structure.fg}

\setbeamercolor*{palette sidebar primary}{use=structure,fg=structure.fg!20}
\setbeamercolor*{palette sidebar secondary}{fg=white}
\setbeamercolor*{palette sidebar tertiary}{use=structure,fg=structure.fg!40}
\setbeamercolor*{palette sidebar quaternary}{fg=white}

\setbeamercolor*{titlelike}{parent=palette primary}

\setbeamercolor*{separation line}{}
\setbeamercolor*{fine separation line}{}

\setbeamercolor{block title}{use=structure,fg=white,bg=structure.fg}
\setbeamercolor{block title alerted}{use=alerted text,fg=white,bg=alerted text.fg!75!black}
\setbeamercolor{block title example}{use=example text,fg=white,bg=example text.fg!75!black}

\setbeamercolor{block body}{parent=normal text,use=block title,bg=block title.bg!10!bg}
\setbeamercolor{block body alerted}{parent=normal text,use=block title alerted,bg=block title alerted.bg!10!bg}
\setbeamercolor{block body example}{parent=normal text,use=block title example,bg=block title example.bg!10!bg}

% \setbeamertemplate{itemize items}[default]


%%%%%%%%%%%%%%%%%%%%%%%%%%%%%%%%%%%%%%%%%%%%%%%%%%%%%%%%%%%%%%%%%%%%%%%%%%%%%%
% TikZ
%%%%%%%%%%%%%%%%%%%%%%%%%%%%%%%%%%%%%%%%%%%%%%%%%%%%%%%%%%%%%%%%%%%%%%%%%%%%%%

\tikzset
	{ > = Stealth
	}


%%%%%%%%%%%%%%%%%%%%%%%%%%%%%%%%%%%%%%%%%%%%%%%%%%%%%%%%%%%%%%%%%%%%%%%%%%%%%%
% general commands and styles
%%%%%%%%%%%%%%%%%%%%%%%%%%%%%%%%%%%%%%%%%%%%%%%%%%%%%%%%%%%%%%%%%%%%%%%%%%%%%%

% \delegateStyle and \inheritStyle command
% usage: \delegateStyle{… \inheritStyle{…} …}
% example: \(X_{\delegateStyle{\fbox{\inheritStyle{X}}}}\)
% Save the current style and regain it in the argument.
% This works both for math and text mode, and can be nested.
% Acknowledgments: Based on \ThisStyle and \SavedStyle from scalerel package.
\makeatletter
\newcommand*{\@inheritStyle@D}[1]{\(\displaystyle      #1\)}
\newcommand*{\@inheritStyle@T}[1]{\(\textstyle         #1\)}
\newcommand*{\@inheritStyle@S}[1]{\(\scriptstyle       #1\)}
\newcommand*{\@inheritStyle@s}[1]{\(\scriptscriptstyle #1\)}
\newcommand*{\@inheritStyle@t}[1]{#1}
\newcommand*{\inheritStyle}{\csname @inheritStyle@\@inheritStyleSwitch\endcsname}
\newcommand*{\delegateStyle}[1]{%
	\ifmmode%
		\mathchoice%
		{\edef\@inheritStyleSwitch{D}#1}%
		{\edef\@inheritStyleSwitch{T}#1}%
		{\edef\@inheritStyleSwitch{S}#1}%
		{\edef\@inheritStyleSwitch{s}#1}%
	\else%
		\edef\@inheritStyleSwitch{t}#1%
	\fi%
}
\makeatother


% \oalt command
% requires: \delegateStyle and \inheritStyle command
% usage: \oalt<…>[…]{…}{…} (cf. \alt)
% Behaves like \alt, but reserves space according to largest overlays.
% The optional argument defines the alignment inside the reserved space;
% it is one of c, l, r, s (cf. \makebox); the default is c.
\makeatletter
\newlength{\oalt@dp}
\newlength{\oalt@ht}
\newlength{\oalt@wd}
\newbox{\oalt@a}
\newbox{\oalt@b}
\newbox{\oalt@empty}
\newcommand<>*{\oalt}[3][c]{%
	\delegateStyle{%
		% based on \setto… in /usr/share/texmf-dist/tex/latex/base/latex.ltx
		\setbox\oalt@a\hbox{\inheritStyle{#2}}%
		\setbox\oalt@b\hbox{\inheritStyle{#3}}%
		\pgfmathsetlength{\oalt@dp}{max(\dp\oalt@a,\dp\oalt@b)}%
		\pgfmathsetlength{\oalt@ht}{max(\ht\oalt@a,\ht\oalt@b)}%
		\pgfmathsetlength{\oalt@wd}{max(\wd\oalt@a,\wd\oalt@b)}%
		\raisebox{0pt}[\oalt@ht][\oalt@dp]{%
			\makebox[\oalt@wd][#1]{%
				\alt#4{\unhbox\oalt@a}{\unhbox\oalt@b}%
			}%
		}%
		\setbox\oalt@a\box\oalt@empty%
		\setbox\oalt@b\box\oalt@empty%
	}%
}
\makeatother


% \otemporal command
% requires: \delegateStyle and \inheritStyle command
% usage: \otemporal<…>[…]{…}{…}{…} (cf. \temporal)
% Behaves like \temporal, but reserves space according to largest overlays.
% The optional argument defines the alignment inside the reserved space;
% it is one of c, l, r, s (cf. \makebox); the default is c.
\makeatletter
\newlength{\ot@dp}
\newlength{\ot@ht}
\newlength{\ot@wd}
\newbox{\ot@a}
\newbox{\ot@b}
\newbox{\ot@c}
\newbox{\ot@empty}
\newcommand<>*{\otemporal}[4][c]{%
	\delegateStyle{%
		% based on \setto… in /usr/share/texmf-dist/tex/latex/base/latex.ltx
		\setbox\ot@a\hbox{\inheritStyle{#2}}%
		\setbox\ot@b\hbox{\inheritStyle{#3}}%
		\setbox\ot@c\hbox{\inheritStyle{#4}}%
		\pgfmathsetlength{\ot@dp}{max(\dp\ot@a,\dp\ot@b,\dp\ot@c)}%
		\pgfmathsetlength{\ot@ht}{max(\ht\ot@a,\ht\ot@b,\ht\ot@c)}%
		\pgfmathsetlength{\ot@wd}{max(\wd\ot@a,\wd\ot@b,\wd\ot@c)}%
		\raisebox{0pt}[\ot@ht][\ot@dp]{%
			\makebox[\ot@wd][#1]{%
				\temporal#5{\unhbox\ot@a}{\unhbox\ot@b}{\unhbox\ot@c}%
			}%
		}%
		\setbox\ot@a\box\ot@empty%
		\setbox\ot@b\box\ot@empty%
		\setbox\ot@c\box\ot@empty%
	}%
}
\makeatother


% Resize delimiters like braces, brackets, etc.
% Parameters: size, left delimiter, formula, right delimiter
% Example: \delim2({\frac{1}{2}})
\newcommand*{\delim}[4]{%
	\ifcase#1%
		#2#3#4%
	\or%
		\bigl#2#3\bigr#4%
	\or%
		\Bigl#2#3\Bigr#4%
	\or%
		\biggl#2#3\biggr#4%
	\or%
		\Biggl#2#3\Biggr#4%
	\else%
		\left#2#3\right#4%
	\fi%
}


% similar to \fullcite, but using the formatting of \printbibliography
\newcommand*{\printfullcite}[1]{%
	\begin{refsection}%
		\nocite{#1}%
		\DeclareNameAlias{author}{first-last}%
		\printbibliography[heading = none]%
	\end{refsection}%
}


\colorlet{light alert}{HKS07K60}
\tikzset{alert.bg/.style={rounded corners, fill=light alert}}
\tikzset{every picture/.style={line cap=round, semithick}}
% http://tex.stackexchange.com/questions/6135/how-to-make-beamer-overlays-with-tikz-node
\tikzset{onslide/.code args={<#1>#2}{\only<#1>{\pgfkeysalso{#2}}}}
\tikzset{invisible/.code args={<#1>}{\alt<#1>{\pgfkeysalso{transparent}}{\pgfkeysalso{opaque}}}}
\tikzset{uncover/.code args={<#1>}{\alt<#1>{\pgfkeysalso{opaque}}{\pgfkeysalso{opacity=0.25}}}}
\tikzset{visible/.code args={<#1>}{\alt<#1>{\pgfkeysalso{opaque}}{\pgfkeysalso{transparent}}}}
\tikzset{vuncover/.code args=%
	{<#1><#2>}%
	{\alt<#1>%
		{\alt<#2>%
			{\pgfkeysalso{opaque}}%
			{\pgfkeysalso{opacity=0.25}}%
		}{\pgfkeysalso{transparent}}%
	}%
}

\newcommand<%
	>{\tikzhighlight}[2][]{%
	\delegateStyle{\alt#3%
		{\tikz[baseline=0, anchor=base, inner sep=0.2em, text height=, text depth=]{\node[alert.bg, #1]{\inheritStyle{#2}};}}%
		{\tikz[baseline=0, anchor=base, inner sep=0.2em, text height=, text depth=]{\node[#1, fill=none]{\inheritStyle{#2}};}}%
	}%
}

\newcommand{\mathhighlight}{\tikzhighlight}

\newcommand<>{\mhl}[2][]{\mathhighlight#3[inner sep=0.2em, #1]{#2}}


\newcommand<>{\inlineblock}[2][]{{%
	\usebeamercolor*[fg]{block body}%
	\tikzhighlight#3[fill=block body.bg, #1]{#2}%
}}


% a small letter s for plurals of abbreviations
\newcommand*{\s}{{\scriptsize s}\xspace}


\newcommand<>*{\sout}[2][opacity=0.75, ultra thick]{%
	\delegateStyle{%
		\tikz[baseline=0, anchor=base, inner sep=0, outer sep=0]{
			\useasboundingbox node (n) {\inheritStyle{#2}};
			\only#3{
				\node (h) {\inheritStyle{\ifmmode\mathstrut\else\strut\fi}};
				\draw[#1] (n.west |- {$(h.south)!0.5!(h.north)$}) -- (n.east |- {$(h.south)!0.5!(h.north)$});
			}
		}%
	}%
}


% tight style
% Sets outer sep to default inner sep and inner sep to 0.
% Use this style for nodes that are neither drawn nor filled to prevent
% unwanted growth of the bounding box.
\tikzset{tight/.style={inner sep=0, outer sep=0.3333em}}


% rounded tree edges style
% usage: rounded tree edges={⟨direction⟩}{⟨looseness⟩}{⟨strength⟩}
\tikzset{
	rounded tree edges/.style n args={3}{
	edge from parent path={
	let
		\n{direction}={#1},
		\n{looseness}={#2},
		\n{strength}={#3},
		\p1=(\tikzparentnode),
		\p2=(\tikzchildnode),
		\p3=(\n{direction}:1pt),
		\p4=(\x2 - \x1, \y2 - \y1),
		\n{dist}={veclen(\p4)},
		\p4=(\x4 / \n{dist}, \y4 / \n{dist}),
		\n{angle}={atan2(\y4, \x4)},
		\n{delta}={Mod(\n{angle} - \n{direction}, 360)},
		\n{delta}={\n{delta} > 180 ? \n{delta} - 360  : \n{delta}},
		\n{delta}={\n{delta} >  90 ?  180 - \n{delta} : \n{delta}},
		\n{delta}={\n{delta} < -90 ? -180 - \n{delta} : \n{delta}}
	in (\tikzparentnode) .. controls
		+(    \n{angle}+\n{strength}*\n{delta}:\n{looseness}*0.3915*\n{dist}) and
		+(180+\n{angle}-\n{strength}*\n{delta}:\n{looseness}*0.3915*\n{dist}) ..
		(\tikzchildnode)
	}
	}
}


% Tear out snippets from PDFs.
% Usage: \tear[…]{file.pdf}
% The optional parameter is the same as for \includegraphics.
% Useful Arguments:
%   * page=‹pagenumber›
%   * trim=‹left› ‹bottom› ‹right› ‹top›
%   * width=0.98\linewidth
\newcommand*{\tear}[2][]{%
	\begin{tikzpicture}
		\node
			[ blur shadow
			, clip
			, decorate
			, decoration=random steps
			, draw
			, inner sep=0
			, preaction={fill=white}% hide the shadow if paper is transparent
			] {\includegraphics[#1]{#2}};
	\end{tikzpicture}%
}


\makeatletter
\newcommand*{\timeline}[3][0]{%
	\ifcsname timeline@cmd@#3\endcsname%
		\@timeline[#1]{#2}{#3}%
		\PackageWarning{timeline}{redefining timeline \@backslashchar\string#3}%
	\else%
		\ifcsname#3\endcsname%
			\errmessage{Command \@backslashchar\string#3 already defined}%
		\else%
			\@timeline[#1]{#2}{#3}%
		\fi%
	\fi%
}%
\newcommand*{\@timeline}[3][0]{%
	% mark command as timeline command – they can be overwritten
	\expandafter\def\csname timeline@cmd@#3\endcsname{}%
	\setcounter{@timeline}{#1}%
	\def\timeline@cmd{#3}%
	\timeline@reset%
	\timeline@append{0}%
	\@tfor\timeline@next:=#2\do{%
		\if\timeline@next+%
			\stepcounter{@timeline}%
			\timeline@append{,\the@timeline}%
		\else\if\timeline@next-%
			\stepcounter{@timeline}%
		\else%
			%\timeline@append{\timeline@next}%
			\GenericError{}{\protect\timeline: ignoring unknown character: \timeline@next}%
		\fi\fi%
	}%
}%
% \newcommand*{\tl}[1]{%
% 	\ifcsname timeline@cmd@#1\endcsname%
% 		\csname timeline@cmd@#1\endcsname%
% 	\else%
% 		0%
% 		%\GenericError{}{\protect\tl: timeline not defined: #1}%
% 	\fi%
% }%
\newcounter{@timeline}%
\def\timeline@reset{%
	\expandafter\def\csname\timeline@cmd\endcsname{}%
}%
\def\timeline@append#1{%
	\expandafter\edef\csname\timeline@cmd\endcsname{%
		\csname\timeline@cmd\endcsname#1%
	}%
}%
\makeatother


\newcommand*{\xminus}[1]{%
	\mathrel{\tikz[baseline={([yshift=-0.25em]n.south)}, inner sep=0, outer sep=0.2em]{%
		\node (n) {\(\scriptstyle #1\)};
		\draw (n.south west) -- (n.south east);
	}}%
}
\newcommand*{\tikzrightarrow}[1]{%
	\mathrel{\tikz[baseline={([yshift=-0.25em]n.south)}, inner sep=0, outer sep=0.2em]{%
		\node (n) {\(\scriptstyle #1\)};
		\draw[->, > = Computer Modern Rightarrow, line width = 0.4pt] (n.south west) -- (n.south east);
	}}%
}


%%%%%%%%%%%%%%%%%%%%%%%%%%%%%%%%%%%%%%%%%%%%%%%%%%%%%%%%%%%%%%%%%%%%%%%%%%%%%%
% document specific commands
%%%%%%%%%%%%%%%%%%%%%%%%%%%%%%%%%%%%%%%%%%%%%%%%%%%%%%%%%%%%%%%%%%%%%%%%%%%%%%

\newcommand<>*{\mycite}[1]{\uncover#2{{\color{HKS57K100}[\cite{#1}]}}}


\newcommand{\statetree}[1]{
	\tikz
	[ anchor=base
	, baseline=(current bounding box.center)
	, level distance=2em
	, sibling distance=2em
	]{
		\matrix
		[ draw=nt
		, edge from parent/.style={draw=black}
		, inner sep=0
		, nodes={inner sep=0.2em, rounded corners=0}
		, rounded corners
		] {#1\\}
	}
}


\newcommand*{\mylargeleaf}[1]{{\LARGE\color{HKS41K70}#1}}

\definecolor{state s}{named}{HKS57K80}
\definecolor{state t}{named}{HKS41K70}
\newcommand*{\stateS}[1]{{\color{state s}#1}}
\newcommand*{\stateT}[1]{{\color{state t}#1}}

\tikzset{
	subtree/.style =
		{ fill=lightgray
		, inner sep=0.02em
		, isosceles triangle apex angle=60
		, shape=isosceles triangle
		, shape border rotate=90
		}
	, state/.style = {circle, draw, inner sep=0.1em}
	, trans/.style = {rectangle, draw}
}

\newcommand*{\srBool}{\mathbb{B}}
\newcommand*{\srProb}{ℙ}


%%%%%%%%%%%%%%%%%%%%%%%%%%%%%%%%%%%%%%%%%%%%%%%%%%%%%%%%%%%%%%%%%%%%%%%%%%%%%%
% commands for specific notations
%%%%%%%%%%%%%%%%%%%%%%%%%%%%%%%%%%%%%%%%%%%%%%%%%%%%%%%%%%%%%%%%%%%%%%%%%%%%%%

\DeclareMathOperator*{\argmax}{argmax}

\newcommand*{\cardinality}[1]{\lvert#1\rvert}
\newcommand*{\corpussize}[1]{\lvert#1\rvert}

\DeclareMathOperator{\crispOp}{crisp}
\newcommand*        {\crisp}[2][0]{\crispOp\delim{#1}({#2})}

\DeclareMathOperator{\lhsOp}{lhs}
\newcommand*{\lhs}[1]{\lhsOp(#1)}

\DeclareMathOperator{\lklhdOp}{L}
\newcommand*{\lklhd}[2]{\lklhdOp(#1 ∣ #2)}

\DeclareMathOperator{\mleOp}{mle}
\newcommand*{\mle}[2][]{%
	\ifthenelse{\isempty{#1}}{%
		\mleOp(#2)%
	}{%
		\mleOp_{#1}(#2)%
	}%
}

\DeclareMathOperator{\mrg}{merge}

% CVD: color vision deficiencies
\definecolor{CVD light red}   {HTML}{FF8080}
\definecolor{CVD light yellow}{HTML}{FFFF80}
\definecolor{CVD light green} {HTML}{40FFC0}

\definecolor{nt}{named}{HKS41K70}
\newcommand*{\nt}[1]{{\color{nt}#1}}

% set of all probability distributions over #1
\DeclareMathOperator{\pdsOp}{Pd}
\newcommand*{\pds}[1]{\pdsOp(#1)}

\DeclareMathOperator{\positionsOp}{pos}
\newcommand*{\positions}[1]{\positionsOp(#1)}

\DeclareMathOperator{\rankOp}{rk}
\newcommand*{\rank}[1]{\rankOp(#1)}

\DeclareMathOperator{\runsOp}{run}
\newcommand*{\runs}[2][]{%
	\ifthenelse%
		{\isempty{#1}}%
		{\runsOp(#2)}%
		{\runsOp_{#1}(#2)}%
}

\newcommand*{\semantics}[1]{⟦#1⟧}

\DeclareMathOperator{\splt}{split}

\newcommand*{\subtree}[2]{#1|_{#2}}

\DeclareMathOperator{\supportOp}{supp}
\newcommand*{\support}[1]{\supportOp(#1)}

\newcommand*{\symId}{\textsc{\color{gray}Id}}
\newcommand*{\symCons}{\textsc{\color{gray}Cons}}
\newcommand*{\symFlip}{\textsc{\color{gray}Flip}}
\newcommand*{\symNull}{\textsc{\color{gray}Null}}
\newcommand*{\symNullR}{\textsc{\color{gray}N\(\overline{\textsc{ull}}\)}}
\newcommand*{\symSnoc}{\textsc{\color{gray}Snoc}}

\newcommand*{\transWTA}[4][]{#3 \xrightarrow{#1} #2(#4)}

\DeclareMathOperator{\uniqueRunOp}{r}
\newcommand*{\uniqueRun}[2][]{%
	\ifthenelse%
		{\isempty{#1}}%
		{\uniqueRunOp^{#2}}%
		{\uniqueRunOp_{\!#1}^{#2}}%
}

\DeclareMathOperator{\treesOp}{T}
\newcommand*{\trees}[2][]{%
	\ifthenelse%
		{\isempty{#1}}%
		{\treesOp_{\!#2}}%
		{\treesOp_{\!#2}(#1)}%
}
\DeclareMathOperator{\treesUOp}{U}
\newcommand*{\treesU}[2][]{%
	\ifthenelse%
		{\isempty{#1}}%
		{\treesUOp_{#2}}%
		{\treesUOp_{#2}(#1)}%
}


%%%%%%%%%%%%%%%%%%%%%%%%%%%%%%%%%%%%%%%%%%%%%%%%%%%%%%%%%%%%%%%%%%%%%%%%%%%%%%
% metadata
%%%%%%%%%%%%%%%%%%%%%%%%%%%%%%%%%%%%%%%%%%%%%%%%%%%%%%%%%%%%%%%%%%%%%%%%%%%%%%

\ifstandalonebeamer\else
	\title[Defense of Dissertation]{A Formal View on Training of Weighted Tree Automata by Likelihood-Driven State Splitting and Merging}
	\subtitle{Defense of Dissertation}
\fi
\author{Toni Dietze}
\institute[TU Dresden]{%
	\href{https://www.orchid.inf.tu-dresden.de/index.en/}{Chair for Foundations of Programming}
\\	\href{https://tu-dresden.de/ing/informatik/thi}{Institute of Theoretical Computer Science}
\\	\href{https://tu-dresden.de/ing/informatik}{Faculty of Computer Science}
\\	\href{https://tu-dresden.de/}{Technische Universität Dresden}
\\	01062 Dresden, Germany
}
\date[2018-09-27]{September 27, 2018}

\title{\jobname}
\begin{document}
\begin{standaloneframe}{\jobname}
\alt<-13>
	{\frametitle{Tree Automata (ta)}}
	{\frametitle{Weighted Tree Automata (wta)}}
\begin{columns}[T]
\column{0.45\linewidth}
	\begin{overprint}
	\onslide<1->
		\begin{align*}
			Q & = \{\nt{S}, \nt{A}\}
		\\
			Σ & = \{σ^{(2)}, α^{(0)}\}
		\\
			I & = \{\nt{S}\}
		\end{align*}
		\uncover<14->{\[
			ι\colon \nt{S} ↦ 1
		\]}
		\begin{alignat*}{2}
			\otemporal<14->[r]{Δ}{δ}{}\colon
			\nt{S} & → σ(\nt{S}, \nt{A}) && \uncover<14->{↦ 0.25}
		\\
			\nt{S} & → σ(\nt{A}, \nt{S}) && \uncover<14->{↦ 0.25}
		\\
			\nt{S} & → α && \uncover<14->{↦ 0.5}
		\\
			\nt{A} & → α && \uncover<14->{↦ 1}
		\end{alignat*}
	\end{overprint}
\column{0.54\linewidth}
	\begin{overprint}
	\onslide<1>
		\begin{block}{tree automaton (ta)}
			tuple \((Q, Σ, I, Δ)\) where
			\begin{itemize}
			\item
				\(Q\) alphabet \hfill (\emph{states})
			\item
				\(Σ\) ranked alphabet \hfill (\emph{terminals})
			\item
				\(I ⊆ Q\) alphabet \hfill (\emph{root states})
			\item
				\(Δ\) is a finite set of \emph{transitions} of form \(A_0 → σ(A_1, \dots, A_k)\) where \(k ∈ ℕ\), \(σ ∈ Σ^{(k)}\), \(A_i ∈ Q\).
			\end{itemize}
		\end{block}
	\onslide<2-12>\setcounter{beamerpauses}{2}
		\begin{center}
		\begin{tikzpicture}[anchor=base, level distance=3em]
			\node (t) {\(σ\)}
				child[uncover={<-.(2),.(5)->}] { node {\(σ\)}
					child[uncover={<-.(2),.(6)->}] { node {\(α\)}
						edge from parent node[left, vuncover={<.(2)-><.(2),.(5)-.(6)>}] {\(\nt{A}\)}
					}
					child[uncover={<-.(2),.(7)->}] { node {\(σ\)}
						child[uncover={<-.(2),.(8)->}] { node {\(α\)}
							edge from parent node[left, vuncover={<.(2)-><.(2),.(7)-.(8),.(10)->}] {\(\nt{\alt<.(11)->AS}\)}
						}
						child[uncover={<-.(2),.(9)->}] { node {\(α\)}
							edge from parent node[right, vuncover={<.(2)-><.(2),.(7),.(9),.(10)->}] {\(\nt{\alt<.(11)->SA}\)}
						}
						edge from parent node[right, vuncover={<.(2)-><.(2),.(5),.(7),.(10)->}] {\(\nt{S}\)}
					}
					edge from parent node[left, vuncover={<.(2)-><.(2)-.(3),.(5)>}] {\(\nt{S}\)}
				}
				child[uncover={<-.(2),.(4)->}] { node {\(α\)}
					edge from parent node[right, vuncover={<.(2)-><.(2)-.(4)>}] {\(\nt{A}\)}
				};
			\node[above right=0 of t.north, vuncover={<.(2)-><.(2)-.(3)>}] {\(\nt{S}\)};
			\node[left=1em of t] {\(t\colon\)};
		\end{tikzpicture}
		\end{center}
	\onslide<13>
		\begin{block}{bottom-up deterministic ta}
			\centering
			\((Q, Σ, I, Δ)\) is bottom-up deterministic

			\emph{if}

			for every \(σ(A_1, \dots, A_k)\) there is at most one \(A_0\) such that \(A_0 → σ(A_1, \dots, A_k) ∈ Δ\)
		\end{block}
		\begin{flushright}
			\(\implies\) there is at most one valid run for a tree
		\end{flushright}
	\onslide<14>
		\begin{block}{weighted tree automaton (wta)}
			tuple \(ℳ = (\mathscr{A}, ι, δ)\) where
			\begin{itemize}
			\item
				\(\mathscr{A} = (Q, Σ, I, Δ)\) is a ta
			\item
				\(ι\colon I → [0, 1]\) \hfill (\emph{root weights})
			\item
				\(δ\colon Δ → [0, 1]\) \hfill (\emph{transition weights})
			\end{itemize}
		\end{block}
	\onslide<15-17>\setcounter{beamerpauses}{15}\centering
		\begin{tikzpicture}
			\node (t) {\(σ\)}
				child { node {\(α\)}
					edge from parent node[left] {\(\nt{S}\)}
				}
				child { node {\(α\)}
					edge from parent node[right] {\(\nt{A}\)}
				};
			\node[above right=0 of t.north] {\(\nt{S}\)};
			\node[left=1em of t] {\(t\), \(r_1\):};
		\end{tikzpicture}
		\quad
		\uncover<.(2)->{%
		\begin{tikzpicture}
			\node (t) {\(σ\)}
				child { node {\(α\)}
					edge from parent node[left] {\(\nt{A}\)}
				}
				child { node {\(α\)}
					edge from parent node[right] {\(\nt{S}\)}
				};
			\node[above right=0 of t.north] {\(\nt{S}\)};
			\node[left=1em of t] {\(t\), \(r_2\):};
		\end{tikzpicture}
		}%
		\begin{align*}
			\otemporal<.(3)->[r]{⟦ℳ⟧(t, r_1)}{⟦ℳ⟧(t)}{} = {}& ι(\nt{S})
		\\	{} ⋅ {}& δ(\nt{S} → σ(\nt{S}, \nt{A}))
		\\	{} ⋅ {}& δ(\nt{S} → α)
		\\	{} ⋅ {}& δ(\nt{A} → α)
		\only<.(2)->{
		\\	\otemporal<.(3)->[r]{⟦ℳ⟧(t, r_2) = {}}{{} + {}}{} & ι(\nt{S})
		\\	{} ⋅ {}& δ(\nt{S} → σ(\nt{A}, \nt{S}))
		\\	{} ⋅ {}& δ(\nt{A} → α)
		\\	{} ⋅ {}& δ(\nt{S} → α)
		%\\	{} = {}& 0.125
		}
		\end{align*}
	\onslide<18->
		\begin{block}{semi-probabilistic wta}
			\begin{itemize}
			\item
				root weights sum up to \(1\)
			\item
				weights of transitions with same left-hand-side sum up to \(1\)
			\end{itemize}
		\end{block}
		\begin{block}{probabilistic wta}
			\begin{itemize}
			\item
				semi-probabilistic
			\item
				weights of all trees sum up to \(1\)
			\end{itemize}
		\end{block}
	\end{overprint}
\end{columns}%
\end{standaloneframe}
\end{document}

% \end{frame}

\section{Automata}

\begin{frame}<-7>[t, label = frame:automata]{\secname}
	\documentclass[beamer]{standalone}
% SPDX-License-Identifier: CC-BY-4.0 OR MIT-0
% Copyright 2018 Toni Dietze
%
\usefonttheme{professionalfonts}

% LuaLaTeX specific packages
\usepackage{fontspec}
	\defaultfontfeatures{Ligatures=TeX}
\usepackage{polyglossia}
	\setdefaultlanguage{english}
\usepackage{amsmath}  % has to be loaded before unicode-math
\usepackage[math-style=ISO]{unicode-math}
	\setmathfont{Latin Modern Math}
% 	\setmathfont[range={\mathcal,\mathbfcal},StylisticSet=1]{xits-math.otf}
% 	\setmathfont[range={"029F5}]{XITS Math}  % ⧵
% 	\setmathfont[range={\mathscr,\mathbfscr},StylisticSet=1]{Latin Modern Math}  % make \mathscr use the correct font

\usepackage[noend]{algpseudocode}
	\algrenewcommand\algorithmicrequire{\textbf{Input:}}
	\algrenewcommand\algorithmicensure{\textbf{Output:}}
\usepackage[backend=biber, maxbibnames=42, maxcitenames=42, sorting=ynt, style=authoryear]{biblatex}
\usepackage{csquotes}
\usepackage{mathtools}
\usepackage{media9}
\usepackage{scalerel}
\usepackage{standalone}
\usepackage{tikz}
	\usetikzlibrary{arrows.meta}
	\usetikzlibrary{backgrounds}
	\usetikzlibrary{calc}
	\usetikzlibrary{decorations}
	\usetikzlibrary{decorations.pathmorphing}
	\usetikzlibrary{decorations.pathreplacing}
	\usetikzlibrary{fadings}
	\usetikzlibrary{fit}
	\usetikzlibrary{graphs}
	\usetikzlibrary{graphdrawing}
	\usetikzlibrary{intersections}
	\usetikzlibrary{positioning}
	\usetikzlibrary{quotes}
	\usetikzlibrary{shadows.blur}
	\usetikzlibrary{shapes.arrows}
	\usetikzlibrary{shapes.geometric}
	\usegdlibrary{trees}
\usepackage{xifthen}
\usepackage{xspace}

\usepackage{pgfplots}
	\pgfplotsset
		{ compat = 1.15
		, /pgf/number format/1000 sep = {\,}
		, /pgf/number format/assume math mode = true
		, every axis plot/.append style =
			{ mark options = {fill opacity = 0.25}
			}
		}
	\usepgfplotslibrary{groupplots}
\usepackage{pgfplotstable}

\hypersetup
	{ bookmarksopen
	, pdflang = en
	, unicode
	}


%%%%%%%%%%%%%%%%%%%%%%%%%%%%%%%%%%%%%%%%%%%%%%%%%%%%%%%%%%%%%%%%%%%%%%%%%%%%%%


% always show bad boxes
%\overfullrule=1em


%%%%%%%%%%%%%%%%%%%%%%%%%%%%%%%%%%%%%%%%%%%%%%%%%%%%%%%%%%%%%%%%%%%%%%%%%%%%%%
% biblatex
%%%%%%%%%%%%%%%%%%%%%%%%%%%%%%%%%%%%%%%%%%%%%%%%%%%%%%%%%%%%%%%%%%%%%%%%%%%%%%

\addbibresource{slides-dissertation-defense.bib}
% \renewcommand*{\finalnamedelim}{\addcomma\space}
% \setlength{\bibitemsep}{1em}
% 
\AtEveryBibitem{% Clean up the bibtex rather than editing it
 \clearlist{address}
 \clearfield{date}
 \clearfield{eprint}
 \clearfield{isbn}
 \clearfield{issn}
 \clearlist{language}
 \clearlist{location}
 \clearfield{month}
 \clearfield{series}
%  \clearfield{url}
%  \clearfield{doi}
 \clearfield{organization}

%  \ifentrytype{book}{}{% Remove stuff except for books
%   \clearfield{booktitle}
%   \clearfield{pages}
  \clearlist{publisher}
  \clearname{editor}
%  }
}
% do not print url if doi is present
% http://tex.stackexchange.com/questions/154864/biblatex-use-doi-only-if-there-is-no-url
\DeclareSourcemap{
	\maps[datatype=bibtex]{
		\map{
			\step[fieldsource=doi,final]
			\step[fieldset=url,null]
}	}	}
%
% remove qoutes around titles
\DeclareFieldFormat
	[article,inbook,incollection,inproceedings,patent,thesis,unpublished]
	{title}{#1\isdot}
% 
% \DeclareFieldFormat{url}{\mkbibacro{URL}\addcolon\addnbspace\url{#1}}
% 
% \DeclareNameAlias{sortname}{first-last}
% 
\renewbibmacro{in:}{\ifentrytype{article}{}{}}


%%%%%%%%%%%%%%%%%%%%%%%%%%%%%%%%%%%%%%%%%%%%%%%%%%%%%%%%%%%%%%%%%%%%%%%%%%%%%%
% beamer
%%%%%%%%%%%%%%%%%%%%%%%%%%%%%%%%%%%%%%%%%%%%%%%%%%%%%%%%%%%%%%%%%%%%%%%%%%%%%%

\useoutertheme{infolines}
\makeatletter
% based on
% /usr/share/texmf-dist/tex/latex/beamer/beamerouterthemeinfolines.sty
\setbeamertemplate{footline}
{%
	\leavevmode%
	\hbox{%
	\begin{beamercolorbox}[wd=.333333\paperwidth,ht=2.25ex,dp=1ex,center]{author in head/foot}%
		\usebeamerfont{author in head/foot}\insertshortauthor\expandafter\beamer@ifempty\expandafter{\beamer@shortinstitute}{}{~~(\insertshortinstitute)}
	\end{beamercolorbox}%
	\begin{beamercolorbox}[wd=.333333\paperwidth,ht=2.25ex,dp=1ex,center]{title in head/foot}%
		\usebeamerfont{title in head/foot}\insertshorttitle
	\end{beamercolorbox}%
	\begin{beamercolorbox}[wd=.333333\paperwidth,ht=2.25ex,dp=1ex,right]{date in head/foot}%
		\usebeamerfont{date in head/foot}%
		\hfill\insertshortdate\hfill\hfill%
		%\hspace*{2ex}%
		%\insertshortdate%
		%\hspace{0pt plus 1 filll}%
		%(\insertframenumber.\insertoverlaynumber{} / \insertmainframenumber)%
		%\hspace{0pt plus 1 filll}%
		\phantom{000}\llap{\insertpagenumber} / \insertpresentationendpage%
		\hspace*{2ex}%
	\end{beamercolorbox}}%
	\vskip0pt%
}
\makeatother
\useinnertheme{circles}
\beamertemplatenavigationsymbolsempty
\setbeamertemplate{bibliography item}{}
\setbeamertemplate{headline}[default]

\input{tudcolors.tex}
\setbeamercolor*{alerted text}{fg=HKS07K100}
\usecolortheme[named=HKS41K100]{structure}

\setbeamercolor*{palette primary}{use=structure,fg=white,bg=structure.fg}
\setbeamercolor*{palette secondary}{use=structure,fg=white,bg=structure.fg!80}
\setbeamercolor*{palette tertiary}{use=structure,fg=white,bg=structure.fg!60}
\setbeamercolor*{palette quaternary}{fg=white,bg=black}

\setbeamercolor*{sidebar}{use=structure,bg=structure.fg}

\setbeamercolor*{palette sidebar primary}{use=structure,fg=structure.fg!20}
\setbeamercolor*{palette sidebar secondary}{fg=white}
\setbeamercolor*{palette sidebar tertiary}{use=structure,fg=structure.fg!40}
\setbeamercolor*{palette sidebar quaternary}{fg=white}

\setbeamercolor*{titlelike}{parent=palette primary}

\setbeamercolor*{separation line}{}
\setbeamercolor*{fine separation line}{}

\setbeamercolor{block title}{use=structure,fg=white,bg=structure.fg}
\setbeamercolor{block title alerted}{use=alerted text,fg=white,bg=alerted text.fg!75!black}
\setbeamercolor{block title example}{use=example text,fg=white,bg=example text.fg!75!black}

\setbeamercolor{block body}{parent=normal text,use=block title,bg=block title.bg!10!bg}
\setbeamercolor{block body alerted}{parent=normal text,use=block title alerted,bg=block title alerted.bg!10!bg}
\setbeamercolor{block body example}{parent=normal text,use=block title example,bg=block title example.bg!10!bg}

% \setbeamertemplate{itemize items}[default]


%%%%%%%%%%%%%%%%%%%%%%%%%%%%%%%%%%%%%%%%%%%%%%%%%%%%%%%%%%%%%%%%%%%%%%%%%%%%%%
% TikZ
%%%%%%%%%%%%%%%%%%%%%%%%%%%%%%%%%%%%%%%%%%%%%%%%%%%%%%%%%%%%%%%%%%%%%%%%%%%%%%

\tikzset
	{ > = Stealth
	}


%%%%%%%%%%%%%%%%%%%%%%%%%%%%%%%%%%%%%%%%%%%%%%%%%%%%%%%%%%%%%%%%%%%%%%%%%%%%%%
% general commands and styles
%%%%%%%%%%%%%%%%%%%%%%%%%%%%%%%%%%%%%%%%%%%%%%%%%%%%%%%%%%%%%%%%%%%%%%%%%%%%%%

% \delegateStyle and \inheritStyle command
% usage: \delegateStyle{… \inheritStyle{…} …}
% example: \(X_{\delegateStyle{\fbox{\inheritStyle{X}}}}\)
% Save the current style and regain it in the argument.
% This works both for math and text mode, and can be nested.
% Acknowledgments: Based on \ThisStyle and \SavedStyle from scalerel package.
\makeatletter
\newcommand*{\@inheritStyle@D}[1]{\(\displaystyle      #1\)}
\newcommand*{\@inheritStyle@T}[1]{\(\textstyle         #1\)}
\newcommand*{\@inheritStyle@S}[1]{\(\scriptstyle       #1\)}
\newcommand*{\@inheritStyle@s}[1]{\(\scriptscriptstyle #1\)}
\newcommand*{\@inheritStyle@t}[1]{#1}
\newcommand*{\inheritStyle}{\csname @inheritStyle@\@inheritStyleSwitch\endcsname}
\newcommand*{\delegateStyle}[1]{%
	\ifmmode%
		\mathchoice%
		{\edef\@inheritStyleSwitch{D}#1}%
		{\edef\@inheritStyleSwitch{T}#1}%
		{\edef\@inheritStyleSwitch{S}#1}%
		{\edef\@inheritStyleSwitch{s}#1}%
	\else%
		\edef\@inheritStyleSwitch{t}#1%
	\fi%
}
\makeatother


% \oalt command
% requires: \delegateStyle and \inheritStyle command
% usage: \oalt<…>[…]{…}{…} (cf. \alt)
% Behaves like \alt, but reserves space according to largest overlays.
% The optional argument defines the alignment inside the reserved space;
% it is one of c, l, r, s (cf. \makebox); the default is c.
\makeatletter
\newlength{\oalt@dp}
\newlength{\oalt@ht}
\newlength{\oalt@wd}
\newbox{\oalt@a}
\newbox{\oalt@b}
\newbox{\oalt@empty}
\newcommand<>*{\oalt}[3][c]{%
	\delegateStyle{%
		% based on \setto… in /usr/share/texmf-dist/tex/latex/base/latex.ltx
		\setbox\oalt@a\hbox{\inheritStyle{#2}}%
		\setbox\oalt@b\hbox{\inheritStyle{#3}}%
		\pgfmathsetlength{\oalt@dp}{max(\dp\oalt@a,\dp\oalt@b)}%
		\pgfmathsetlength{\oalt@ht}{max(\ht\oalt@a,\ht\oalt@b)}%
		\pgfmathsetlength{\oalt@wd}{max(\wd\oalt@a,\wd\oalt@b)}%
		\raisebox{0pt}[\oalt@ht][\oalt@dp]{%
			\makebox[\oalt@wd][#1]{%
				\alt#4{\unhbox\oalt@a}{\unhbox\oalt@b}%
			}%
		}%
		\setbox\oalt@a\box\oalt@empty%
		\setbox\oalt@b\box\oalt@empty%
	}%
}
\makeatother


% \otemporal command
% requires: \delegateStyle and \inheritStyle command
% usage: \otemporal<…>[…]{…}{…}{…} (cf. \temporal)
% Behaves like \temporal, but reserves space according to largest overlays.
% The optional argument defines the alignment inside the reserved space;
% it is one of c, l, r, s (cf. \makebox); the default is c.
\makeatletter
\newlength{\ot@dp}
\newlength{\ot@ht}
\newlength{\ot@wd}
\newbox{\ot@a}
\newbox{\ot@b}
\newbox{\ot@c}
\newbox{\ot@empty}
\newcommand<>*{\otemporal}[4][c]{%
	\delegateStyle{%
		% based on \setto… in /usr/share/texmf-dist/tex/latex/base/latex.ltx
		\setbox\ot@a\hbox{\inheritStyle{#2}}%
		\setbox\ot@b\hbox{\inheritStyle{#3}}%
		\setbox\ot@c\hbox{\inheritStyle{#4}}%
		\pgfmathsetlength{\ot@dp}{max(\dp\ot@a,\dp\ot@b,\dp\ot@c)}%
		\pgfmathsetlength{\ot@ht}{max(\ht\ot@a,\ht\ot@b,\ht\ot@c)}%
		\pgfmathsetlength{\ot@wd}{max(\wd\ot@a,\wd\ot@b,\wd\ot@c)}%
		\raisebox{0pt}[\ot@ht][\ot@dp]{%
			\makebox[\ot@wd][#1]{%
				\temporal#5{\unhbox\ot@a}{\unhbox\ot@b}{\unhbox\ot@c}%
			}%
		}%
		\setbox\ot@a\box\ot@empty%
		\setbox\ot@b\box\ot@empty%
		\setbox\ot@c\box\ot@empty%
	}%
}
\makeatother


% Resize delimiters like braces, brackets, etc.
% Parameters: size, left delimiter, formula, right delimiter
% Example: \delim2({\frac{1}{2}})
\newcommand*{\delim}[4]{%
	\ifcase#1%
		#2#3#4%
	\or%
		\bigl#2#3\bigr#4%
	\or%
		\Bigl#2#3\Bigr#4%
	\or%
		\biggl#2#3\biggr#4%
	\or%
		\Biggl#2#3\Biggr#4%
	\else%
		\left#2#3\right#4%
	\fi%
}


% similar to \fullcite, but using the formatting of \printbibliography
\newcommand*{\printfullcite}[1]{%
	\begin{refsection}%
		\nocite{#1}%
		\DeclareNameAlias{author}{first-last}%
		\printbibliography[heading = none]%
	\end{refsection}%
}


\colorlet{light alert}{HKS07K60}
\tikzset{alert.bg/.style={rounded corners, fill=light alert}}
\tikzset{every picture/.style={line cap=round, semithick}}
% http://tex.stackexchange.com/questions/6135/how-to-make-beamer-overlays-with-tikz-node
\tikzset{onslide/.code args={<#1>#2}{\only<#1>{\pgfkeysalso{#2}}}}
\tikzset{invisible/.code args={<#1>}{\alt<#1>{\pgfkeysalso{transparent}}{\pgfkeysalso{opaque}}}}
\tikzset{uncover/.code args={<#1>}{\alt<#1>{\pgfkeysalso{opaque}}{\pgfkeysalso{opacity=0.25}}}}
\tikzset{visible/.code args={<#1>}{\alt<#1>{\pgfkeysalso{opaque}}{\pgfkeysalso{transparent}}}}
\tikzset{vuncover/.code args=%
	{<#1><#2>}%
	{\alt<#1>%
		{\alt<#2>%
			{\pgfkeysalso{opaque}}%
			{\pgfkeysalso{opacity=0.25}}%
		}{\pgfkeysalso{transparent}}%
	}%
}

\newcommand<%
	>{\tikzhighlight}[2][]{%
	\delegateStyle{\alt#3%
		{\tikz[baseline=0, anchor=base, inner sep=0.2em, text height=, text depth=]{\node[alert.bg, #1]{\inheritStyle{#2}};}}%
		{\tikz[baseline=0, anchor=base, inner sep=0.2em, text height=, text depth=]{\node[#1, fill=none]{\inheritStyle{#2}};}}%
	}%
}

\newcommand{\mathhighlight}{\tikzhighlight}

\newcommand<>{\mhl}[2][]{\mathhighlight#3[inner sep=0.2em, #1]{#2}}


\newcommand<>{\inlineblock}[2][]{{%
	\usebeamercolor*[fg]{block body}%
	\tikzhighlight#3[fill=block body.bg, #1]{#2}%
}}


% a small letter s for plurals of abbreviations
\newcommand*{\s}{{\scriptsize s}\xspace}


\newcommand<>*{\sout}[2][opacity=0.75, ultra thick]{%
	\delegateStyle{%
		\tikz[baseline=0, anchor=base, inner sep=0, outer sep=0]{
			\useasboundingbox node (n) {\inheritStyle{#2}};
			\only#3{
				\node (h) {\inheritStyle{\ifmmode\mathstrut\else\strut\fi}};
				\draw[#1] (n.west |- {$(h.south)!0.5!(h.north)$}) -- (n.east |- {$(h.south)!0.5!(h.north)$});
			}
		}%
	}%
}


% tight style
% Sets outer sep to default inner sep and inner sep to 0.
% Use this style for nodes that are neither drawn nor filled to prevent
% unwanted growth of the bounding box.
\tikzset{tight/.style={inner sep=0, outer sep=0.3333em}}


% rounded tree edges style
% usage: rounded tree edges={⟨direction⟩}{⟨looseness⟩}{⟨strength⟩}
\tikzset{
	rounded tree edges/.style n args={3}{
	edge from parent path={
	let
		\n{direction}={#1},
		\n{looseness}={#2},
		\n{strength}={#3},
		\p1=(\tikzparentnode),
		\p2=(\tikzchildnode),
		\p3=(\n{direction}:1pt),
		\p4=(\x2 - \x1, \y2 - \y1),
		\n{dist}={veclen(\p4)},
		\p4=(\x4 / \n{dist}, \y4 / \n{dist}),
		\n{angle}={atan2(\y4, \x4)},
		\n{delta}={Mod(\n{angle} - \n{direction}, 360)},
		\n{delta}={\n{delta} > 180 ? \n{delta} - 360  : \n{delta}},
		\n{delta}={\n{delta} >  90 ?  180 - \n{delta} : \n{delta}},
		\n{delta}={\n{delta} < -90 ? -180 - \n{delta} : \n{delta}}
	in (\tikzparentnode) .. controls
		+(    \n{angle}+\n{strength}*\n{delta}:\n{looseness}*0.3915*\n{dist}) and
		+(180+\n{angle}-\n{strength}*\n{delta}:\n{looseness}*0.3915*\n{dist}) ..
		(\tikzchildnode)
	}
	}
}


% Tear out snippets from PDFs.
% Usage: \tear[…]{file.pdf}
% The optional parameter is the same as for \includegraphics.
% Useful Arguments:
%   * page=‹pagenumber›
%   * trim=‹left› ‹bottom› ‹right› ‹top›
%   * width=0.98\linewidth
\newcommand*{\tear}[2][]{%
	\begin{tikzpicture}
		\node
			[ blur shadow
			, clip
			, decorate
			, decoration=random steps
			, draw
			, inner sep=0
			, preaction={fill=white}% hide the shadow if paper is transparent
			] {\includegraphics[#1]{#2}};
	\end{tikzpicture}%
}


\makeatletter
\newcommand*{\timeline}[3][0]{%
	\ifcsname timeline@cmd@#3\endcsname%
		\@timeline[#1]{#2}{#3}%
		\PackageWarning{timeline}{redefining timeline \@backslashchar\string#3}%
	\else%
		\ifcsname#3\endcsname%
			\errmessage{Command \@backslashchar\string#3 already defined}%
		\else%
			\@timeline[#1]{#2}{#3}%
		\fi%
	\fi%
}%
\newcommand*{\@timeline}[3][0]{%
	% mark command as timeline command – they can be overwritten
	\expandafter\def\csname timeline@cmd@#3\endcsname{}%
	\setcounter{@timeline}{#1}%
	\def\timeline@cmd{#3}%
	\timeline@reset%
	\timeline@append{0}%
	\@tfor\timeline@next:=#2\do{%
		\if\timeline@next+%
			\stepcounter{@timeline}%
			\timeline@append{,\the@timeline}%
		\else\if\timeline@next-%
			\stepcounter{@timeline}%
		\else%
			%\timeline@append{\timeline@next}%
			\GenericError{}{\protect\timeline: ignoring unknown character: \timeline@next}%
		\fi\fi%
	}%
}%
% \newcommand*{\tl}[1]{%
% 	\ifcsname timeline@cmd@#1\endcsname%
% 		\csname timeline@cmd@#1\endcsname%
% 	\else%
% 		0%
% 		%\GenericError{}{\protect\tl: timeline not defined: #1}%
% 	\fi%
% }%
\newcounter{@timeline}%
\def\timeline@reset{%
	\expandafter\def\csname\timeline@cmd\endcsname{}%
}%
\def\timeline@append#1{%
	\expandafter\edef\csname\timeline@cmd\endcsname{%
		\csname\timeline@cmd\endcsname#1%
	}%
}%
\makeatother


\newcommand*{\xminus}[1]{%
	\mathrel{\tikz[baseline={([yshift=-0.25em]n.south)}, inner sep=0, outer sep=0.2em]{%
		\node (n) {\(\scriptstyle #1\)};
		\draw (n.south west) -- (n.south east);
	}}%
}
\newcommand*{\tikzrightarrow}[1]{%
	\mathrel{\tikz[baseline={([yshift=-0.25em]n.south)}, inner sep=0, outer sep=0.2em]{%
		\node (n) {\(\scriptstyle #1\)};
		\draw[->, > = Computer Modern Rightarrow, line width = 0.4pt] (n.south west) -- (n.south east);
	}}%
}


%%%%%%%%%%%%%%%%%%%%%%%%%%%%%%%%%%%%%%%%%%%%%%%%%%%%%%%%%%%%%%%%%%%%%%%%%%%%%%
% document specific commands
%%%%%%%%%%%%%%%%%%%%%%%%%%%%%%%%%%%%%%%%%%%%%%%%%%%%%%%%%%%%%%%%%%%%%%%%%%%%%%

\newcommand<>*{\mycite}[1]{\uncover#2{{\color{HKS57K100}[\cite{#1}]}}}


\newcommand{\statetree}[1]{
	\tikz
	[ anchor=base
	, baseline=(current bounding box.center)
	, level distance=2em
	, sibling distance=2em
	]{
		\matrix
		[ draw=nt
		, edge from parent/.style={draw=black}
		, inner sep=0
		, nodes={inner sep=0.2em, rounded corners=0}
		, rounded corners
		] {#1\\}
	}
}


\newcommand*{\mylargeleaf}[1]{{\LARGE\color{HKS41K70}#1}}

\definecolor{state s}{named}{HKS57K80}
\definecolor{state t}{named}{HKS41K70}
\newcommand*{\stateS}[1]{{\color{state s}#1}}
\newcommand*{\stateT}[1]{{\color{state t}#1}}

\tikzset{
	subtree/.style =
		{ fill=lightgray
		, inner sep=0.02em
		, isosceles triangle apex angle=60
		, shape=isosceles triangle
		, shape border rotate=90
		}
	, state/.style = {circle, draw, inner sep=0.1em}
	, trans/.style = {rectangle, draw}
}

\newcommand*{\srBool}{\mathbb{B}}
\newcommand*{\srProb}{ℙ}


%%%%%%%%%%%%%%%%%%%%%%%%%%%%%%%%%%%%%%%%%%%%%%%%%%%%%%%%%%%%%%%%%%%%%%%%%%%%%%
% commands for specific notations
%%%%%%%%%%%%%%%%%%%%%%%%%%%%%%%%%%%%%%%%%%%%%%%%%%%%%%%%%%%%%%%%%%%%%%%%%%%%%%

\DeclareMathOperator*{\argmax}{argmax}

\newcommand*{\cardinality}[1]{\lvert#1\rvert}
\newcommand*{\corpussize}[1]{\lvert#1\rvert}

\DeclareMathOperator{\crispOp}{crisp}
\newcommand*        {\crisp}[2][0]{\crispOp\delim{#1}({#2})}

\DeclareMathOperator{\lhsOp}{lhs}
\newcommand*{\lhs}[1]{\lhsOp(#1)}

\DeclareMathOperator{\lklhdOp}{L}
\newcommand*{\lklhd}[2]{\lklhdOp(#1 ∣ #2)}

\DeclareMathOperator{\mleOp}{mle}
\newcommand*{\mle}[2][]{%
	\ifthenelse{\isempty{#1}}{%
		\mleOp(#2)%
	}{%
		\mleOp_{#1}(#2)%
	}%
}

\DeclareMathOperator{\mrg}{merge}

% CVD: color vision deficiencies
\definecolor{CVD light red}   {HTML}{FF8080}
\definecolor{CVD light yellow}{HTML}{FFFF80}
\definecolor{CVD light green} {HTML}{40FFC0}

\definecolor{nt}{named}{HKS41K70}
\newcommand*{\nt}[1]{{\color{nt}#1}}

% set of all probability distributions over #1
\DeclareMathOperator{\pdsOp}{Pd}
\newcommand*{\pds}[1]{\pdsOp(#1)}

\DeclareMathOperator{\positionsOp}{pos}
\newcommand*{\positions}[1]{\positionsOp(#1)}

\DeclareMathOperator{\rankOp}{rk}
\newcommand*{\rank}[1]{\rankOp(#1)}

\DeclareMathOperator{\runsOp}{run}
\newcommand*{\runs}[2][]{%
	\ifthenelse%
		{\isempty{#1}}%
		{\runsOp(#2)}%
		{\runsOp_{#1}(#2)}%
}

\newcommand*{\semantics}[1]{⟦#1⟧}

\DeclareMathOperator{\splt}{split}

\newcommand*{\subtree}[2]{#1|_{#2}}

\DeclareMathOperator{\supportOp}{supp}
\newcommand*{\support}[1]{\supportOp(#1)}

\newcommand*{\symId}{\textsc{\color{gray}Id}}
\newcommand*{\symCons}{\textsc{\color{gray}Cons}}
\newcommand*{\symFlip}{\textsc{\color{gray}Flip}}
\newcommand*{\symNull}{\textsc{\color{gray}Null}}
\newcommand*{\symNullR}{\textsc{\color{gray}N\(\overline{\textsc{ull}}\)}}
\newcommand*{\symSnoc}{\textsc{\color{gray}Snoc}}

\newcommand*{\transWTA}[4][]{#3 \xrightarrow{#1} #2(#4)}

\DeclareMathOperator{\uniqueRunOp}{r}
\newcommand*{\uniqueRun}[2][]{%
	\ifthenelse%
		{\isempty{#1}}%
		{\uniqueRunOp^{#2}}%
		{\uniqueRunOp_{\!#1}^{#2}}%
}

\DeclareMathOperator{\treesOp}{T}
\newcommand*{\trees}[2][]{%
	\ifthenelse%
		{\isempty{#1}}%
		{\treesOp_{\!#2}}%
		{\treesOp_{\!#2}(#1)}%
}
\DeclareMathOperator{\treesUOp}{U}
\newcommand*{\treesU}[2][]{%
	\ifthenelse%
		{\isempty{#1}}%
		{\treesUOp_{#2}}%
		{\treesUOp_{#2}(#1)}%
}


%%%%%%%%%%%%%%%%%%%%%%%%%%%%%%%%%%%%%%%%%%%%%%%%%%%%%%%%%%%%%%%%%%%%%%%%%%%%%%
% metadata
%%%%%%%%%%%%%%%%%%%%%%%%%%%%%%%%%%%%%%%%%%%%%%%%%%%%%%%%%%%%%%%%%%%%%%%%%%%%%%

\ifstandalonebeamer\else
	\title[Defense of Dissertation]{A Formal View on Training of Weighted Tree Automata by Likelihood-Driven State Splitting and Merging}
	\subtitle{Defense of Dissertation}
\fi
\author{Toni Dietze}
\institute[TU Dresden]{%
	\href{https://www.orchid.inf.tu-dresden.de/index.en/}{Chair for Foundations of Programming}
\\	\href{https://tu-dresden.de/ing/informatik/thi}{Institute of Theoretical Computer Science}
\\	\href{https://tu-dresden.de/ing/informatik}{Faculty of Computer Science}
\\	\href{https://tu-dresden.de/}{Technische Universität Dresden}
\\	01062 Dresden, Germany
}
\date[2018-09-27]{September 27, 2018}

\begin{document}
%         1 2 3 4 5 6 7 8 9 0 1 2 3 4 5 6 7
\timeline{+ + + + + + + + + + + + + + + - -}{tlFSAandTA}
\timeline{- - - - - - - + + + + + + + + + +}{tlTAandUTA}
\timeline{+ + + + + + + + - - - - - - - - -}{tlString}
\timeline{+ + + + + + + - - - - - - - - - -}{tlFSA}
\timeline{- + + + + + + - - - - - - - - - -}{fsaRunA}
\timeline{- - + + + + + - - - - - - - - - -}{fsaRunB}
\timeline{- - - + + + + - - - - - - - - - -}{fsaRunC}
\timeline{- - - - + + + - - - - - - - - - -}{fsaRunD}
\timeline{- - - - - + + - - - - - - - - - -}{fsaRunE}
\timeline{- - - - - - + - - - - - - - - - -}{fsaRunF}
\timeline{- - - - - - - + - - - - - - - - -}{tlRaise}
\timeline{- - - - - - - + + + + + + + + - -}{tlTA}
\timeline{- - - - - - - - + + + + + + + - -}{tlTAb}
\timeline{- - - - - - - - + - - - - - - - -}{tlTreeA}
\timeline{- - - - - - - - - + + + + + + + -}{tlTreeB}
\timeline{- - - - - - - - - - - - - - - - +}{tlTreeC}
\timeline{- - - - - - - - - - - - - - - + +}{tlUTA}
\timeline{- - - - - - - - - - - - - - - + -}{tlUTAa}
\timeline{- - - - - - - - - - + + + + - - -}{tlFormalDefs}
\timeline{- - - - - - - - - - + - - - - - -}{tlFormalTA}
\timeline{- - - - - - - - - - - + - - - - -}{tlFormalWTA}
\timeline{- - - - - - - - - - - - + - - - -}{tlRunWeighted}
\timeline{- - - - - - - - - - - - - + - - -}{tlDefProbabilistic}
\timeline{- - - - - - - - - - - + + + - - -}{tlWeights}
\begin{standaloneframe}[t]{\jobname}
\begin{columns}[T]
\column{0.45\linewidth}
	\only<\tlFSA>{%
		\begin{tikzpicture}[x=1em, y=1em, overlay, shift={(4em, -15em)}]
			\node[state] (1) at (0, 0) {\(\nt{A}\)};
			\node[state] (2) at (6, 0) {\(\nt{B}\)};
			\draw[<-] (1) -- ++(-3,  0);
			\draw[->] (1) edge[out= 60, in=120] node[above] {\(a\)} (2);
			\draw[->] (2) edge[out=240, in=300] node[below] {\(b\)} (1);
			\draw[->] (2) -- ++(3, 0);
		\end{tikzpicture}%
	}%
	\only<\tlTA>{%
		\begin{tikzpicture}[x=1em, y=1em, overlay, shift={(4em, -15em)}]
			\node[state] (1) at ( 0,  0) {\(\nt{A}\)};
			\node[state] (2) at ( 6,  0) {\(\nt{B}\)};
			\node[trans] (a) at ( 3,  2) {};
			\node[trans] (b) at ( 3, -2) {};
			\node[trans] (f) at ( 9,  0) {};
			\node[above=0 of a] {\(a\)};
			\node[below=0 of b] {\(b\)};
			\node[above=0 of f] {\(⊥\)};
			\draw[<-] (1) -- ++(-3,  0);
			\draw     (1) edge[out= 60, in=180] (a);
			\draw[->] (a) edge[out=  0, in=120] (2);
			\draw<\tlTAb>
			     [->] (a) edge[bend right] (2);
			\draw     (2) edge[out=240, in=  0] (b);
			\draw[->] (b) edge[out=180, in=300] (1);
			\draw     (2) -- (f);
		\end{tikzpicture}%
	}%
	\setlength\abovedisplayskip{0pt}%
	\begin{align*}
		\oalt<\tlFSA>[r]%
			{\text{initial: }}%
			{\oalt[r]<\tlFormalDefs>%
				{\oalt[r]<\tlFormalTA>%
					{I\colon}%
					{ι\colon}%
				}%
				{\text{root: }}%
			}%
		\nt{A}
		&&& \only<\tlWeights>{\hspace{-4em} ↦ 1}
	\\[1em]
		\oalt[r]<\tlFormalDefs>%
			{\oalt[r]<\tlFormalTA>%
				{Δ\colon}%
				{δ\colon}%
			}%
			{\text{trans.: }}
		\nt{A} & \xrightarrow{\invisible<\tlFormalDefs>{a}}
			\otemporal<\tlUTAa>[l]%
			{\oalt<\tlTAb>[l]%
				{\oalt<\tlFormalDefs>[l]%
					{a(\nt{B}, \nt{B})}%
					{\nt{B}\,\nt{B}}%
				}%
				{\nt{B}}%
			}%
			{\tikz[baseline = (current bounding box.center)]{\node[draw, rounded corners]{\tikz{
				\node[state, inner sep = 0.2em] (p0) at (  0, 0) {};
				\node[state, inner sep = 0.2em] (p1) at (2em, 0) {};
				\node[state, inner sep = 0.2em] (p2) at (4em, 0) {};
				\draw[<-] (p0) -- ++(-1em, 0);
				\draw[->] (p2) -- ++( 1em, 0);
				\draw[->] (p0) -- node[above, tight] {\(\nt{B}\)} (p1);
				\draw[->] (p1) -- node[above, tight] {\(\nt{B}\)} (p2);
			}};}}%
			{\tikz[baseline = (current bounding box.center)]{\node[draw, rounded corners]{\tikz{
				\node[state, inner sep = 0.2em] (p) {};
				\draw[<-] (p) -- ++(-1em, 0);
				\draw[->] (p) -- ++( 1em, 0);
				\draw[->] (p) edge[loop above] node[right, tight] {\(\nt{B}\)} ();
			}};}}
		&& \only<\tlWeights>{\hspace{-4em} ↦ 1}
	\\
		\nt{B} & \xrightarrow{\invisible<\tlFormalDefs>{b}}
			\oalt<\tlFSAandTA>[l]%
			{\oalt<\tlFormalDefs>[l]%
				{b(\nt{A})}%
				{\nt{A}}%
			}%
			{\tikz[baseline = (current bounding box.center)]{\node[draw, rounded corners]{\tikz{
				\node[state, inner sep = 0.2em] (p1) {};
				\node[state, inner sep = 0.2em] (p2) at (2em, 0) {};
				\draw[<-] (p1) -- ++(-1em, 0);
				\draw[->] (p2) -- ++( 1em, 0);
				\draw[->] (p1) -- (p2) node[midway, above, tight] {\(\nt{A}\)};
			}};}}
		&& \only<\tlWeights>{\hspace{-4em} ↦ 0.5}
	\\
		\visible<\tlFSA>{\text{final: }} \nt{B}
	&
		\visible<\tlTAandUTA>{\xrightarrow{\invisible<\tlFormalDefs>{⊥}}
			\oalt<\tlTA>[l]%
			{\oalt<\tlFormalDefs>[l]%
				{⊥()}%
				{ε}%
			}%
			{\tikz[baseline = (current bounding box.center)]{\node[draw, rounded corners]{\tikz{
				\node[state, inner sep = 0.2em] (p) {};
				\draw[<-] (p) -- ++(-1em, 0);
				\draw[->] (p) -- ++( 1em, 0);
			}};}}
		&& \only<\tlWeights>{\hspace{-4em} ↦ 0.5}
		}
	\end{align*}
\column{0.55\linewidth}
	\alt<\tlFormalDefs>{%
		\only<\tlFormalTA>{%
			\begin{block}{tree automaton (ta)}
				tuple \(\mathscr{A} = (Q, Σ, I, Δ)\) where
				\begin{itemize}
				\item
					\(Q\) alphabet \hfill (\emph{states})
				\item
					\(Σ\) ranked alphabet \hfill (\emph{terminals})
				\item
					\(I ⊆ Q\) alphabet \hfill (\emph{root states})
				\item
					\(Δ\) is a finite set of \emph{transitions} of form \(A_0 → σ(A_1, \dots, A_k)\) where \(k ∈ ℕ\), \(σ ∈ Σ^{(k)}\), \(A_i ∈ Q\).
				\end{itemize}
			\end{block}
		}%
		\only<\tlFormalWTA>{%
			\begin{block}{weighted tree automaton (wta)}
				tuple \(ℳ = (\mathscr{A}, ι, δ)\) where
				\begin{itemize}
				\item
					\(\mathscr{A} = (Q, Σ, I, Δ)\) is a ta
				\item
					\(ι\colon I → ℝ_{≥ 0}\) \hfill (\emph{root weights})
				\item
					\(δ\colon Δ → ℝ_{≥ 0}\) \hfill (\emph{transition weights})
				\end{itemize}
			\end{block}
		}%
		\only<\tlRunWeighted>{%
			\centering
			\begin{tikzpicture}
				\node (t) {\(a\)}
					child { node {\(⊥\)}
						edge from parent node[left] {\(\nt{B}\)}
					}
					child { node {\(⊥\)}
						edge from parent node[right] {\(\nt{B}\)}
					};
				\node[above right=0 of t.north] {\(\nt{A}\)};
				\node[left=1em of t] {\(t\), \(r\):};
			\end{tikzpicture}
			\begin{align*}
				⟦ℳ⟧(t, r) = {}& ι(\nt{A})
			\\	{} ⋅ {}& δ(\nt{A} → a(\nt{B}, \nt{B}))
			\\	{} ⋅ {}& δ(\nt{B} → ⊥)
			\\	{} ⋅ {}& δ(\nt{B} → ⊥)
			\end{align*}
			\[
				⟦ℳ⟧(t) = ∑_{r} ⟦ℳ⟧(t, r)
			\]
		}%
		\only<\tlDefProbabilistic>{%
			\begin{block}{probabilistic wta (pta)}
				\begin{itemize}
				\item
					root weights sum up to \(1\)
				\item
					weights of transitions with same left-hand-side sum up to \(1\)
				\item
					weights of all trees sum up to \(1\)
				\end{itemize}
			\end{block}
			\begin{block}{Observation}
				If \(ℳ\) is a probabilistic wta,
				\\ then \(⟦ℳ⟧\) is a probability distribution.
			\end{block}
		}%
	}{%
		\hfill%
		\begin{tikzpicture}
		[ anchor = base
		, ampersand replacement = \&
		, row sep = -0.75em
		, graphs/tree graph/.style =
			{ number nodes
			, branch right = 2em
			, grow down
			, math nodes
			, edge quotes = {auto, inner sep = 0.15em, node font = \small}
			}
		]
			\matrix[anchor=base west, visible=<\tlString>] at (0, -1em) {
			\&   \node (s1) {\(a\)};
			\&\& \node (s2) {\(b\)};
			\&\& \node (s3) {\(a\)};
			\&\& \node (s4) {\(b\)};
			\&\& \node (s5) {\(a\)};
			\\   \node[visible={<\fsaRunA>}] (0) {\(\nt{A}\)};
			\&\& \node[visible={<\fsaRunB>}] (1) {\(\nt{B}\)};
			\&\& \node[visible={<\fsaRunC>}] (2) {\(\nt{A}\)};
			\&\& \node[visible={<\fsaRunD>}] (3) {\(\nt{B}\)};
			\&\& \node[visible={<\fsaRunE>}] (4) {\(\nt{A}\)};
			\&\& \node[visible={<\fsaRunF>}] (5) {\(\nt{B}\)};
			\\};

			\begin{scope}[->, > = Computer Modern Rightarrow, line width = 0.4pt]
				\draw <\fsaRunB> (0) -- (1);
				\draw <\fsaRunC> (1) -- (2);
				\draw <\fsaRunD> (2) -- (3);
				\draw <\fsaRunE> (3) -- (4);
				\draw <\fsaRunF> (4) -- (5);
			\end{scope}

			\begin{scope}[visible={<\tlRaise>}]
				% \node (t) {}
				% 	child { node (t1) {\(a\)}
				% 	child { node (t2) {\(b\)}
				% 	child { node (t3) {\(a\)}
				% 	child { node (t4) {\(b\)}
				% 	child { node (t5) {\(a\)}
				% 	child { node (t6) {\(⊥\)}
				% 	  edge from parent node[left] {\(\nt{B}\)}
				% 	} edge from parent node[left] {\(\nt{A}\)}
				% 	} edge from parent node[left] {\(\nt{B}\)}
				% 	} edge from parent node[left] {\(\nt{A}\)}
				% 	} edge from parent node[left] {\(\nt{B}\)}
				% 	} edge from parent[draw=none] node[left] {\(\nt{A}\)}
				% 	};
				\graph[tree graph] {
					"" --[draw=none] a [>"\(\nt{A}\)"']
					-- b [>"\(\nt{B}\)"']
					-- a [>"\(\nt{A}\)"']
					-- b [>"\(\nt{B}\)"']
					-- a [>"\(\nt{A}\)"']
					-- ⊥ [>"\(\nt{B}\)"']
				};
			\end{scope}

			\begin{scope}[gray, visible={<\tlRaise>}]
				\draw (s1) edge[bend left] (a 2);
				\draw (s2) edge[bend left] (b 3);
				\draw (s3) edge[bend left] (a 4);
				\draw (s4) edge[bend left] (b 5);
				\draw (s5) edge[bend left] (a 6);
			\end{scope}

			\begin{scope}[visible={<\tlTreeA>}]
				\graph[tree graph] {
					"" --[draw=none] a [>"\(\nt{A}\)"'] --
						{ b [>"\(\nt{B}\)"'] -- a [>"\(\nt{A}\)"'] --
							{ b [>"\(\nt{B}\)"'] -- a [>"\(\nt{A}\)"'] --
								{ ⊥ [>"\(\nt{B}\)"']
								, ⊥ [>"\(\nt{B}\)"]
								}
							, ⊥ [>"\(\nt{B}\)"']
							}
						, ⊥ [>"\(\nt{B}\)"]
						}
				};
			\end{scope}
			\begin{scope}[visible={<\tlTreeB>}]
				\graph[tree graph] {
					"" --[draw=none] a [>"\(\nt{A}\)"'] --
						{ b [>"\(\nt{B}\)"'] -- a [>"\(\nt{A}\)"'] --
							{ b [>"\(\nt{B}\)"'] -- a [>"\(\nt{A}\)"'] --
								{ ⊥ [>"\(\nt{B}\)"']
								, ⊥ [>"\(\nt{B}\)"]
								}
							, b [>"\(\nt{B}\)"'] -- a [>"\(\nt{A}\)"'] --
								{ ⊥ [>"\(\nt{B}\)"']
								, ⊥ [>"\(\nt{B}\)"]
								}
							}
						, ⊥ [>"\(\nt{B}\)"]
						}
				};
			\end{scope}
			\begin{scope}[visible={<\tlTreeC>}]
				\graph[tree graph] {
					"" --[draw=none] a [>"\(\nt{A}\)"'] --
						{ b [>"\(\nt{B}\)"'] -- a [>"\(\nt{A}\)"'] --
							{ b [>"\(\nt{B}\)"'] -- a [>"\(\nt{A}\)"'] --
								{ ⊥ [>"\(\nt{B}\)"']
								, ⊥ [>"\(\nt{B}\)"]
								}
							, b [>"\(\nt{B}\)"'] -- a [>"\(\nt{A}\)"'] --
								{ ⊥ [>"\(\nt{B}\)"']
								, ⊥ [>"\(\nt{B}\)"]
								}
							, b [>"\(\nt{B}\)"] -- a [>"\(\nt{A}\)"'] --
								{ ⊥ [>"\(\nt{B}\)"']
								, ⊥ [>"\(\nt{B}\)"]
								}
							}
						, ⊥ [>"\(\nt{B}\)"]
						}
				};
			\end{scope}
		\end{tikzpicture}
	}
\end{columns}
\end{standaloneframe}
\end{document}

\end{frame}


\section{Tree Automata (ta)}

\againframe<8-11>[t]{frame:automata}


\section{Weighted Tree Automata (wta)}

\againframe<12-14>[t]{frame:automata}


\section{Read-Off}

\begin{frame}{\secname}
	\centering
	% SPDX-License-Identifier: CC-BY-4.0
% Copyright 2018 Toni Dietze
\documentclass[beamer]{standalone}
% SPDX-License-Identifier: CC-BY-4.0 OR MIT-0
% Copyright 2018 Toni Dietze
%
\usefonttheme{professionalfonts}

% LuaLaTeX specific packages
\usepackage{fontspec}
	\defaultfontfeatures{Ligatures=TeX}
\usepackage{polyglossia}
	\setdefaultlanguage{english}
\usepackage{amsmath}  % has to be loaded before unicode-math
\usepackage[math-style=ISO]{unicode-math}
	\setmathfont{Latin Modern Math}
% 	\setmathfont[range={\mathcal,\mathbfcal},StylisticSet=1]{xits-math.otf}
% 	\setmathfont[range={"029F5}]{XITS Math}  % ⧵
% 	\setmathfont[range={\mathscr,\mathbfscr},StylisticSet=1]{Latin Modern Math}  % make \mathscr use the correct font

\usepackage[noend]{algpseudocode}
	\algrenewcommand\algorithmicrequire{\textbf{Input:}}
	\algrenewcommand\algorithmicensure{\textbf{Output:}}
\usepackage[backend=biber, maxbibnames=42, maxcitenames=42, sorting=ynt, style=authoryear]{biblatex}
\usepackage{csquotes}
\usepackage{mathtools}
\usepackage{media9}
\usepackage{scalerel}
\usepackage{standalone}
\usepackage{tikz}
	\usetikzlibrary{arrows.meta}
	\usetikzlibrary{backgrounds}
	\usetikzlibrary{calc}
	\usetikzlibrary{decorations}
	\usetikzlibrary{decorations.pathmorphing}
	\usetikzlibrary{decorations.pathreplacing}
	\usetikzlibrary{fadings}
	\usetikzlibrary{fit}
	\usetikzlibrary{graphs}
	\usetikzlibrary{graphdrawing}
	\usetikzlibrary{intersections}
	\usetikzlibrary{positioning}
	\usetikzlibrary{quotes}
	\usetikzlibrary{shadows.blur}
	\usetikzlibrary{shapes.arrows}
	\usetikzlibrary{shapes.geometric}
	\usegdlibrary{trees}
\usepackage{xifthen}
\usepackage{xspace}

\usepackage{pgfplots}
	\pgfplotsset
		{ compat = 1.15
		, /pgf/number format/1000 sep = {\,}
		, /pgf/number format/assume math mode = true
		, every axis plot/.append style =
			{ mark options = {fill opacity = 0.25}
			}
		}
	\usepgfplotslibrary{groupplots}
\usepackage{pgfplotstable}

\hypersetup
	{ bookmarksopen
	, pdflang = en
	, unicode
	}


%%%%%%%%%%%%%%%%%%%%%%%%%%%%%%%%%%%%%%%%%%%%%%%%%%%%%%%%%%%%%%%%%%%%%%%%%%%%%%


% always show bad boxes
%\overfullrule=1em


%%%%%%%%%%%%%%%%%%%%%%%%%%%%%%%%%%%%%%%%%%%%%%%%%%%%%%%%%%%%%%%%%%%%%%%%%%%%%%
% biblatex
%%%%%%%%%%%%%%%%%%%%%%%%%%%%%%%%%%%%%%%%%%%%%%%%%%%%%%%%%%%%%%%%%%%%%%%%%%%%%%

\addbibresource{slides-dissertation-defense.bib}
% \renewcommand*{\finalnamedelim}{\addcomma\space}
% \setlength{\bibitemsep}{1em}
% 
\AtEveryBibitem{% Clean up the bibtex rather than editing it
 \clearlist{address}
 \clearfield{date}
 \clearfield{eprint}
 \clearfield{isbn}
 \clearfield{issn}
 \clearlist{language}
 \clearlist{location}
 \clearfield{month}
 \clearfield{series}
%  \clearfield{url}
%  \clearfield{doi}
 \clearfield{organization}

%  \ifentrytype{book}{}{% Remove stuff except for books
%   \clearfield{booktitle}
%   \clearfield{pages}
  \clearlist{publisher}
  \clearname{editor}
%  }
}
% do not print url if doi is present
% http://tex.stackexchange.com/questions/154864/biblatex-use-doi-only-if-there-is-no-url
\DeclareSourcemap{
	\maps[datatype=bibtex]{
		\map{
			\step[fieldsource=doi,final]
			\step[fieldset=url,null]
}	}	}
%
% remove qoutes around titles
\DeclareFieldFormat
	[article,inbook,incollection,inproceedings,patent,thesis,unpublished]
	{title}{#1\isdot}
% 
% \DeclareFieldFormat{url}{\mkbibacro{URL}\addcolon\addnbspace\url{#1}}
% 
% \DeclareNameAlias{sortname}{first-last}
% 
\renewbibmacro{in:}{\ifentrytype{article}{}{}}


%%%%%%%%%%%%%%%%%%%%%%%%%%%%%%%%%%%%%%%%%%%%%%%%%%%%%%%%%%%%%%%%%%%%%%%%%%%%%%
% beamer
%%%%%%%%%%%%%%%%%%%%%%%%%%%%%%%%%%%%%%%%%%%%%%%%%%%%%%%%%%%%%%%%%%%%%%%%%%%%%%

\useoutertheme{infolines}
\makeatletter
% based on
% /usr/share/texmf-dist/tex/latex/beamer/beamerouterthemeinfolines.sty
\setbeamertemplate{footline}
{%
	\leavevmode%
	\hbox{%
	\begin{beamercolorbox}[wd=.333333\paperwidth,ht=2.25ex,dp=1ex,center]{author in head/foot}%
		\usebeamerfont{author in head/foot}\insertshortauthor\expandafter\beamer@ifempty\expandafter{\beamer@shortinstitute}{}{~~(\insertshortinstitute)}
	\end{beamercolorbox}%
	\begin{beamercolorbox}[wd=.333333\paperwidth,ht=2.25ex,dp=1ex,center]{title in head/foot}%
		\usebeamerfont{title in head/foot}\insertshorttitle
	\end{beamercolorbox}%
	\begin{beamercolorbox}[wd=.333333\paperwidth,ht=2.25ex,dp=1ex,right]{date in head/foot}%
		\usebeamerfont{date in head/foot}%
		\hfill\insertshortdate\hfill\hfill%
		%\hspace*{2ex}%
		%\insertshortdate%
		%\hspace{0pt plus 1 filll}%
		%(\insertframenumber.\insertoverlaynumber{} / \insertmainframenumber)%
		%\hspace{0pt plus 1 filll}%
		\phantom{000}\llap{\insertpagenumber} / \insertpresentationendpage%
		\hspace*{2ex}%
	\end{beamercolorbox}}%
	\vskip0pt%
}
\makeatother
\useinnertheme{circles}
\beamertemplatenavigationsymbolsempty
\setbeamertemplate{bibliography item}{}
\setbeamertemplate{headline}[default]

\input{tudcolors.tex}
\setbeamercolor*{alerted text}{fg=HKS07K100}
\usecolortheme[named=HKS41K100]{structure}

\setbeamercolor*{palette primary}{use=structure,fg=white,bg=structure.fg}
\setbeamercolor*{palette secondary}{use=structure,fg=white,bg=structure.fg!80}
\setbeamercolor*{palette tertiary}{use=structure,fg=white,bg=structure.fg!60}
\setbeamercolor*{palette quaternary}{fg=white,bg=black}

\setbeamercolor*{sidebar}{use=structure,bg=structure.fg}

\setbeamercolor*{palette sidebar primary}{use=structure,fg=structure.fg!20}
\setbeamercolor*{palette sidebar secondary}{fg=white}
\setbeamercolor*{palette sidebar tertiary}{use=structure,fg=structure.fg!40}
\setbeamercolor*{palette sidebar quaternary}{fg=white}

\setbeamercolor*{titlelike}{parent=palette primary}

\setbeamercolor*{separation line}{}
\setbeamercolor*{fine separation line}{}

\setbeamercolor{block title}{use=structure,fg=white,bg=structure.fg}
\setbeamercolor{block title alerted}{use=alerted text,fg=white,bg=alerted text.fg!75!black}
\setbeamercolor{block title example}{use=example text,fg=white,bg=example text.fg!75!black}

\setbeamercolor{block body}{parent=normal text,use=block title,bg=block title.bg!10!bg}
\setbeamercolor{block body alerted}{parent=normal text,use=block title alerted,bg=block title alerted.bg!10!bg}
\setbeamercolor{block body example}{parent=normal text,use=block title example,bg=block title example.bg!10!bg}

% \setbeamertemplate{itemize items}[default]


%%%%%%%%%%%%%%%%%%%%%%%%%%%%%%%%%%%%%%%%%%%%%%%%%%%%%%%%%%%%%%%%%%%%%%%%%%%%%%
% TikZ
%%%%%%%%%%%%%%%%%%%%%%%%%%%%%%%%%%%%%%%%%%%%%%%%%%%%%%%%%%%%%%%%%%%%%%%%%%%%%%

\tikzset
	{ > = Stealth
	}


%%%%%%%%%%%%%%%%%%%%%%%%%%%%%%%%%%%%%%%%%%%%%%%%%%%%%%%%%%%%%%%%%%%%%%%%%%%%%%
% general commands and styles
%%%%%%%%%%%%%%%%%%%%%%%%%%%%%%%%%%%%%%%%%%%%%%%%%%%%%%%%%%%%%%%%%%%%%%%%%%%%%%

% \delegateStyle and \inheritStyle command
% usage: \delegateStyle{… \inheritStyle{…} …}
% example: \(X_{\delegateStyle{\fbox{\inheritStyle{X}}}}\)
% Save the current style and regain it in the argument.
% This works both for math and text mode, and can be nested.
% Acknowledgments: Based on \ThisStyle and \SavedStyle from scalerel package.
\makeatletter
\newcommand*{\@inheritStyle@D}[1]{\(\displaystyle      #1\)}
\newcommand*{\@inheritStyle@T}[1]{\(\textstyle         #1\)}
\newcommand*{\@inheritStyle@S}[1]{\(\scriptstyle       #1\)}
\newcommand*{\@inheritStyle@s}[1]{\(\scriptscriptstyle #1\)}
\newcommand*{\@inheritStyle@t}[1]{#1}
\newcommand*{\inheritStyle}{\csname @inheritStyle@\@inheritStyleSwitch\endcsname}
\newcommand*{\delegateStyle}[1]{%
	\ifmmode%
		\mathchoice%
		{\edef\@inheritStyleSwitch{D}#1}%
		{\edef\@inheritStyleSwitch{T}#1}%
		{\edef\@inheritStyleSwitch{S}#1}%
		{\edef\@inheritStyleSwitch{s}#1}%
	\else%
		\edef\@inheritStyleSwitch{t}#1%
	\fi%
}
\makeatother


% \oalt command
% requires: \delegateStyle and \inheritStyle command
% usage: \oalt<…>[…]{…}{…} (cf. \alt)
% Behaves like \alt, but reserves space according to largest overlays.
% The optional argument defines the alignment inside the reserved space;
% it is one of c, l, r, s (cf. \makebox); the default is c.
\makeatletter
\newlength{\oalt@dp}
\newlength{\oalt@ht}
\newlength{\oalt@wd}
\newbox{\oalt@a}
\newbox{\oalt@b}
\newbox{\oalt@empty}
\newcommand<>*{\oalt}[3][c]{%
	\delegateStyle{%
		% based on \setto… in /usr/share/texmf-dist/tex/latex/base/latex.ltx
		\setbox\oalt@a\hbox{\inheritStyle{#2}}%
		\setbox\oalt@b\hbox{\inheritStyle{#3}}%
		\pgfmathsetlength{\oalt@dp}{max(\dp\oalt@a,\dp\oalt@b)}%
		\pgfmathsetlength{\oalt@ht}{max(\ht\oalt@a,\ht\oalt@b)}%
		\pgfmathsetlength{\oalt@wd}{max(\wd\oalt@a,\wd\oalt@b)}%
		\raisebox{0pt}[\oalt@ht][\oalt@dp]{%
			\makebox[\oalt@wd][#1]{%
				\alt#4{\unhbox\oalt@a}{\unhbox\oalt@b}%
			}%
		}%
		\setbox\oalt@a\box\oalt@empty%
		\setbox\oalt@b\box\oalt@empty%
	}%
}
\makeatother


% \otemporal command
% requires: \delegateStyle and \inheritStyle command
% usage: \otemporal<…>[…]{…}{…}{…} (cf. \temporal)
% Behaves like \temporal, but reserves space according to largest overlays.
% The optional argument defines the alignment inside the reserved space;
% it is one of c, l, r, s (cf. \makebox); the default is c.
\makeatletter
\newlength{\ot@dp}
\newlength{\ot@ht}
\newlength{\ot@wd}
\newbox{\ot@a}
\newbox{\ot@b}
\newbox{\ot@c}
\newbox{\ot@empty}
\newcommand<>*{\otemporal}[4][c]{%
	\delegateStyle{%
		% based on \setto… in /usr/share/texmf-dist/tex/latex/base/latex.ltx
		\setbox\ot@a\hbox{\inheritStyle{#2}}%
		\setbox\ot@b\hbox{\inheritStyle{#3}}%
		\setbox\ot@c\hbox{\inheritStyle{#4}}%
		\pgfmathsetlength{\ot@dp}{max(\dp\ot@a,\dp\ot@b,\dp\ot@c)}%
		\pgfmathsetlength{\ot@ht}{max(\ht\ot@a,\ht\ot@b,\ht\ot@c)}%
		\pgfmathsetlength{\ot@wd}{max(\wd\ot@a,\wd\ot@b,\wd\ot@c)}%
		\raisebox{0pt}[\ot@ht][\ot@dp]{%
			\makebox[\ot@wd][#1]{%
				\temporal#5{\unhbox\ot@a}{\unhbox\ot@b}{\unhbox\ot@c}%
			}%
		}%
		\setbox\ot@a\box\ot@empty%
		\setbox\ot@b\box\ot@empty%
		\setbox\ot@c\box\ot@empty%
	}%
}
\makeatother


% Resize delimiters like braces, brackets, etc.
% Parameters: size, left delimiter, formula, right delimiter
% Example: \delim2({\frac{1}{2}})
\newcommand*{\delim}[4]{%
	\ifcase#1%
		#2#3#4%
	\or%
		\bigl#2#3\bigr#4%
	\or%
		\Bigl#2#3\Bigr#4%
	\or%
		\biggl#2#3\biggr#4%
	\or%
		\Biggl#2#3\Biggr#4%
	\else%
		\left#2#3\right#4%
	\fi%
}


% similar to \fullcite, but using the formatting of \printbibliography
\newcommand*{\printfullcite}[1]{%
	\begin{refsection}%
		\nocite{#1}%
		\DeclareNameAlias{author}{first-last}%
		\printbibliography[heading = none]%
	\end{refsection}%
}


\colorlet{light alert}{HKS07K60}
\tikzset{alert.bg/.style={rounded corners, fill=light alert}}
\tikzset{every picture/.style={line cap=round, semithick}}
% http://tex.stackexchange.com/questions/6135/how-to-make-beamer-overlays-with-tikz-node
\tikzset{onslide/.code args={<#1>#2}{\only<#1>{\pgfkeysalso{#2}}}}
\tikzset{invisible/.code args={<#1>}{\alt<#1>{\pgfkeysalso{transparent}}{\pgfkeysalso{opaque}}}}
\tikzset{uncover/.code args={<#1>}{\alt<#1>{\pgfkeysalso{opaque}}{\pgfkeysalso{opacity=0.25}}}}
\tikzset{visible/.code args={<#1>}{\alt<#1>{\pgfkeysalso{opaque}}{\pgfkeysalso{transparent}}}}
\tikzset{vuncover/.code args=%
	{<#1><#2>}%
	{\alt<#1>%
		{\alt<#2>%
			{\pgfkeysalso{opaque}}%
			{\pgfkeysalso{opacity=0.25}}%
		}{\pgfkeysalso{transparent}}%
	}%
}

\newcommand<%
	>{\tikzhighlight}[2][]{%
	\delegateStyle{\alt#3%
		{\tikz[baseline=0, anchor=base, inner sep=0.2em, text height=, text depth=]{\node[alert.bg, #1]{\inheritStyle{#2}};}}%
		{\tikz[baseline=0, anchor=base, inner sep=0.2em, text height=, text depth=]{\node[#1, fill=none]{\inheritStyle{#2}};}}%
	}%
}

\newcommand{\mathhighlight}{\tikzhighlight}

\newcommand<>{\mhl}[2][]{\mathhighlight#3[inner sep=0.2em, #1]{#2}}


\newcommand<>{\inlineblock}[2][]{{%
	\usebeamercolor*[fg]{block body}%
	\tikzhighlight#3[fill=block body.bg, #1]{#2}%
}}


% a small letter s for plurals of abbreviations
\newcommand*{\s}{{\scriptsize s}\xspace}


\newcommand<>*{\sout}[2][opacity=0.75, ultra thick]{%
	\delegateStyle{%
		\tikz[baseline=0, anchor=base, inner sep=0, outer sep=0]{
			\useasboundingbox node (n) {\inheritStyle{#2}};
			\only#3{
				\node (h) {\inheritStyle{\ifmmode\mathstrut\else\strut\fi}};
				\draw[#1] (n.west |- {$(h.south)!0.5!(h.north)$}) -- (n.east |- {$(h.south)!0.5!(h.north)$});
			}
		}%
	}%
}


% tight style
% Sets outer sep to default inner sep and inner sep to 0.
% Use this style for nodes that are neither drawn nor filled to prevent
% unwanted growth of the bounding box.
\tikzset{tight/.style={inner sep=0, outer sep=0.3333em}}


% rounded tree edges style
% usage: rounded tree edges={⟨direction⟩}{⟨looseness⟩}{⟨strength⟩}
\tikzset{
	rounded tree edges/.style n args={3}{
	edge from parent path={
	let
		\n{direction}={#1},
		\n{looseness}={#2},
		\n{strength}={#3},
		\p1=(\tikzparentnode),
		\p2=(\tikzchildnode),
		\p3=(\n{direction}:1pt),
		\p4=(\x2 - \x1, \y2 - \y1),
		\n{dist}={veclen(\p4)},
		\p4=(\x4 / \n{dist}, \y4 / \n{dist}),
		\n{angle}={atan2(\y4, \x4)},
		\n{delta}={Mod(\n{angle} - \n{direction}, 360)},
		\n{delta}={\n{delta} > 180 ? \n{delta} - 360  : \n{delta}},
		\n{delta}={\n{delta} >  90 ?  180 - \n{delta} : \n{delta}},
		\n{delta}={\n{delta} < -90 ? -180 - \n{delta} : \n{delta}}
	in (\tikzparentnode) .. controls
		+(    \n{angle}+\n{strength}*\n{delta}:\n{looseness}*0.3915*\n{dist}) and
		+(180+\n{angle}-\n{strength}*\n{delta}:\n{looseness}*0.3915*\n{dist}) ..
		(\tikzchildnode)
	}
	}
}


% Tear out snippets from PDFs.
% Usage: \tear[…]{file.pdf}
% The optional parameter is the same as for \includegraphics.
% Useful Arguments:
%   * page=‹pagenumber›
%   * trim=‹left› ‹bottom› ‹right› ‹top›
%   * width=0.98\linewidth
\newcommand*{\tear}[2][]{%
	\begin{tikzpicture}
		\node
			[ blur shadow
			, clip
			, decorate
			, decoration=random steps
			, draw
			, inner sep=0
			, preaction={fill=white}% hide the shadow if paper is transparent
			] {\includegraphics[#1]{#2}};
	\end{tikzpicture}%
}


\makeatletter
\newcommand*{\timeline}[3][0]{%
	\ifcsname timeline@cmd@#3\endcsname%
		\@timeline[#1]{#2}{#3}%
		\PackageWarning{timeline}{redefining timeline \@backslashchar\string#3}%
	\else%
		\ifcsname#3\endcsname%
			\errmessage{Command \@backslashchar\string#3 already defined}%
		\else%
			\@timeline[#1]{#2}{#3}%
		\fi%
	\fi%
}%
\newcommand*{\@timeline}[3][0]{%
	% mark command as timeline command – they can be overwritten
	\expandafter\def\csname timeline@cmd@#3\endcsname{}%
	\setcounter{@timeline}{#1}%
	\def\timeline@cmd{#3}%
	\timeline@reset%
	\timeline@append{0}%
	\@tfor\timeline@next:=#2\do{%
		\if\timeline@next+%
			\stepcounter{@timeline}%
			\timeline@append{,\the@timeline}%
		\else\if\timeline@next-%
			\stepcounter{@timeline}%
		\else%
			%\timeline@append{\timeline@next}%
			\GenericError{}{\protect\timeline: ignoring unknown character: \timeline@next}%
		\fi\fi%
	}%
}%
% \newcommand*{\tl}[1]{%
% 	\ifcsname timeline@cmd@#1\endcsname%
% 		\csname timeline@cmd@#1\endcsname%
% 	\else%
% 		0%
% 		%\GenericError{}{\protect\tl: timeline not defined: #1}%
% 	\fi%
% }%
\newcounter{@timeline}%
\def\timeline@reset{%
	\expandafter\def\csname\timeline@cmd\endcsname{}%
}%
\def\timeline@append#1{%
	\expandafter\edef\csname\timeline@cmd\endcsname{%
		\csname\timeline@cmd\endcsname#1%
	}%
}%
\makeatother


\newcommand*{\xminus}[1]{%
	\mathrel{\tikz[baseline={([yshift=-0.25em]n.south)}, inner sep=0, outer sep=0.2em]{%
		\node (n) {\(\scriptstyle #1\)};
		\draw (n.south west) -- (n.south east);
	}}%
}
\newcommand*{\tikzrightarrow}[1]{%
	\mathrel{\tikz[baseline={([yshift=-0.25em]n.south)}, inner sep=0, outer sep=0.2em]{%
		\node (n) {\(\scriptstyle #1\)};
		\draw[->, > = Computer Modern Rightarrow, line width = 0.4pt] (n.south west) -- (n.south east);
	}}%
}


%%%%%%%%%%%%%%%%%%%%%%%%%%%%%%%%%%%%%%%%%%%%%%%%%%%%%%%%%%%%%%%%%%%%%%%%%%%%%%
% document specific commands
%%%%%%%%%%%%%%%%%%%%%%%%%%%%%%%%%%%%%%%%%%%%%%%%%%%%%%%%%%%%%%%%%%%%%%%%%%%%%%

\newcommand<>*{\mycite}[1]{\uncover#2{{\color{HKS57K100}[\cite{#1}]}}}


\newcommand{\statetree}[1]{
	\tikz
	[ anchor=base
	, baseline=(current bounding box.center)
	, level distance=2em
	, sibling distance=2em
	]{
		\matrix
		[ draw=nt
		, edge from parent/.style={draw=black}
		, inner sep=0
		, nodes={inner sep=0.2em, rounded corners=0}
		, rounded corners
		] {#1\\}
	}
}


\newcommand*{\mylargeleaf}[1]{{\LARGE\color{HKS41K70}#1}}

\definecolor{state s}{named}{HKS57K80}
\definecolor{state t}{named}{HKS41K70}
\newcommand*{\stateS}[1]{{\color{state s}#1}}
\newcommand*{\stateT}[1]{{\color{state t}#1}}

\tikzset{
	subtree/.style =
		{ fill=lightgray
		, inner sep=0.02em
		, isosceles triangle apex angle=60
		, shape=isosceles triangle
		, shape border rotate=90
		}
	, state/.style = {circle, draw, inner sep=0.1em}
	, trans/.style = {rectangle, draw}
}

\newcommand*{\srBool}{\mathbb{B}}
\newcommand*{\srProb}{ℙ}


%%%%%%%%%%%%%%%%%%%%%%%%%%%%%%%%%%%%%%%%%%%%%%%%%%%%%%%%%%%%%%%%%%%%%%%%%%%%%%
% commands for specific notations
%%%%%%%%%%%%%%%%%%%%%%%%%%%%%%%%%%%%%%%%%%%%%%%%%%%%%%%%%%%%%%%%%%%%%%%%%%%%%%

\DeclareMathOperator*{\argmax}{argmax}

\newcommand*{\cardinality}[1]{\lvert#1\rvert}
\newcommand*{\corpussize}[1]{\lvert#1\rvert}

\DeclareMathOperator{\crispOp}{crisp}
\newcommand*        {\crisp}[2][0]{\crispOp\delim{#1}({#2})}

\DeclareMathOperator{\lhsOp}{lhs}
\newcommand*{\lhs}[1]{\lhsOp(#1)}

\DeclareMathOperator{\lklhdOp}{L}
\newcommand*{\lklhd}[2]{\lklhdOp(#1 ∣ #2)}

\DeclareMathOperator{\mleOp}{mle}
\newcommand*{\mle}[2][]{%
	\ifthenelse{\isempty{#1}}{%
		\mleOp(#2)%
	}{%
		\mleOp_{#1}(#2)%
	}%
}

\DeclareMathOperator{\mrg}{merge}

% CVD: color vision deficiencies
\definecolor{CVD light red}   {HTML}{FF8080}
\definecolor{CVD light yellow}{HTML}{FFFF80}
\definecolor{CVD light green} {HTML}{40FFC0}

\definecolor{nt}{named}{HKS41K70}
\newcommand*{\nt}[1]{{\color{nt}#1}}

% set of all probability distributions over #1
\DeclareMathOperator{\pdsOp}{Pd}
\newcommand*{\pds}[1]{\pdsOp(#1)}

\DeclareMathOperator{\positionsOp}{pos}
\newcommand*{\positions}[1]{\positionsOp(#1)}

\DeclareMathOperator{\rankOp}{rk}
\newcommand*{\rank}[1]{\rankOp(#1)}

\DeclareMathOperator{\runsOp}{run}
\newcommand*{\runs}[2][]{%
	\ifthenelse%
		{\isempty{#1}}%
		{\runsOp(#2)}%
		{\runsOp_{#1}(#2)}%
}

\newcommand*{\semantics}[1]{⟦#1⟧}

\DeclareMathOperator{\splt}{split}

\newcommand*{\subtree}[2]{#1|_{#2}}

\DeclareMathOperator{\supportOp}{supp}
\newcommand*{\support}[1]{\supportOp(#1)}

\newcommand*{\symId}{\textsc{\color{gray}Id}}
\newcommand*{\symCons}{\textsc{\color{gray}Cons}}
\newcommand*{\symFlip}{\textsc{\color{gray}Flip}}
\newcommand*{\symNull}{\textsc{\color{gray}Null}}
\newcommand*{\symNullR}{\textsc{\color{gray}N\(\overline{\textsc{ull}}\)}}
\newcommand*{\symSnoc}{\textsc{\color{gray}Snoc}}

\newcommand*{\transWTA}[4][]{#3 \xrightarrow{#1} #2(#4)}

\DeclareMathOperator{\uniqueRunOp}{r}
\newcommand*{\uniqueRun}[2][]{%
	\ifthenelse%
		{\isempty{#1}}%
		{\uniqueRunOp^{#2}}%
		{\uniqueRunOp_{\!#1}^{#2}}%
}

\DeclareMathOperator{\treesOp}{T}
\newcommand*{\trees}[2][]{%
	\ifthenelse%
		{\isempty{#1}}%
		{\treesOp_{\!#2}}%
		{\treesOp_{\!#2}(#1)}%
}
\DeclareMathOperator{\treesUOp}{U}
\newcommand*{\treesU}[2][]{%
	\ifthenelse%
		{\isempty{#1}}%
		{\treesUOp_{#2}}%
		{\treesUOp_{#2}(#1)}%
}


%%%%%%%%%%%%%%%%%%%%%%%%%%%%%%%%%%%%%%%%%%%%%%%%%%%%%%%%%%%%%%%%%%%%%%%%%%%%%%
% metadata
%%%%%%%%%%%%%%%%%%%%%%%%%%%%%%%%%%%%%%%%%%%%%%%%%%%%%%%%%%%%%%%%%%%%%%%%%%%%%%

\ifstandalonebeamer\else
	\title[Defense of Dissertation]{A Formal View on Training of Weighted Tree Automata by Likelihood-Driven State Splitting and Merging}
	\subtitle{Defense of Dissertation}
\fi
\author{Toni Dietze}
\institute[TU Dresden]{%
	\href{https://www.orchid.inf.tu-dresden.de/index.en/}{Chair for Foundations of Programming}
\\	\href{https://tu-dresden.de/ing/informatik/thi}{Institute of Theoretical Computer Science}
\\	\href{https://tu-dresden.de/ing/informatik}{Faculty of Computer Science}
\\	\href{https://tu-dresden.de/}{Technische Universität Dresden}
\\	01062 Dresden, Germany
}
\date[2018-09-27]{September 27, 2018}

\begin{document}
%         1 2 3 4 5 6 7 8 9 0 1 2
\timeline{- + + + + + + + + + + -}{tlRuns}
\timeline{- - + - - - - - - - - -}{tlRuleSsAB}
\timeline{- - - + - - - - - - - -}{tlRuleAa}
\timeline{- - - - + - - - - - - -}{tlRuleBbB}
\timeline{- - - - - + - - - - - -}{tlRuleBbA}
\timeline{- - - - - - + - - - - -}{tlRootState}
\timeline{- - - - - - - + + + + +}{tlMLE}
\timeline{- - - - - - - - + + + +}{tlEM}
\timeline{- - - - - - - - - + + +}{tlWeights}
\timeline{- - - - - - - - - - + -}{tlWeightsInTrees}
\timeline{- - - - - - - - - - + +}{tlLklhd}
\timeline{- - - - - - - - - - - +}{tlOdd}
\begin{standaloneframe}{\jobname}
\centering
\begin{tikzpicture}
[ text height=0.75em
, text depth=0.25em
]
	\node at (2em, 3em) {corpus \(c\):};
	\node<\tlRuleSsAB-> at (15em, 3em) {automaton \(\oalt<\tlWeights>{ℳ}{\mathcal{A}}\):};
	\begin{scope}
	[ level distance=2.25em
	, sibling distance=1.5em
	, anchor=base
	, inner sep=0.1em
	, node distance=0.4em
	]
		\node[onslide={<\tlRuleSsAB>alert.bg}] (t1) at (0, 0) {\(σ\)}
			child { node[onslide={<\tlRuleAa>alert.bg}] {\(α\)}
			}
			child { node[onslide={<\tlRuleBbA>alert.bg}] {\(β\)}
				child { node[onslide={<\tlRuleAa>alert.bg}] {\(α\)}
				}
			};
		\node[onslide={<\tlRuleSsAB>alert.bg}] (t2) at (6em, 0) {\(σ\)}
			child { node[onslide={<\tlRuleAa>alert.bg}] {\(α\)}
			}
			child { node[onslide={<\tlRuleBbB>alert.bg}] {\(β\)}
				child { node[onslide={<\tlRuleBbB>alert.bg}] {\(β\)}
					child { node[onslide={<\tlRuleBbA>alert.bg}] {\(β\)}
						child { node[onslide={<\tlRuleAa>alert.bg}] {\(α\)}
						}
					}
				}
			};
		\node[base left = 1em of t1] {$t_1\colon$};
		%\node at (0, -9em) {$c(t_1) = 1$};
		\node[base left = 1em of t2] {$t_2\colon$};
		%\node[anchor=base] at (5em, -9em) {$c(t_2) = 1$};
	\end{scope}

	\matrix
	[ matrix anchor=base.base west
	, every node/.style={anchor=base west}
	, row sep=0.1em
	, ampersand replacement=\&
	] at (13em, 0) {
		\uncover<\tlRuleSsAB->{\node[onslide=<\tlRuleSsAB>alert.bg] (base) {$\nt{S} → σ(\nt{A}, \nt{B})$};}
	\&
		\uncover<\tlWeights>{\node {${} ↦ 1$};}
	\\
		\uncover<\tlRuleAa->{\node[onslide=<\tlRuleAa>alert.bg] {$\nt{A} → α$};}
	\&
		\uncover<\tlWeights>{\node {${} ↦ 1$};}
	\\
		\uncover<\tlRuleBbB->{\node[onslide=<\tlRuleBbB>alert.bg] {$\nt{B} → β(\nt{B})$};}
	\&
		\uncover<\tlWeights>{\node {${} ↦ 0.5$};}
	\\
		\uncover<\tlRuleBbA->{\node[onslide=<\tlRuleBbA>alert.bg] {$\nt{B} → β(\nt{A})$};}
	\&
		\uncover<\tlWeights>{\node {${} ↦ 0.5$};}
	\\[0.25em]
		\uncover<\tlRootState->{\node[onslide={<\tlRootState>alert.bg}] (iota) {\(\nt{S}\)};}
	\&
		\uncover<\tlWeights>{\node {${} ↦ 1$};}
	\\};

	\node<\tlRuleSsAB-> [left=0.4em of base, inner sep=0] {\(\alt<\tlWeights>{δ}{Δ}\colon\)};
	\node<\tlRootState->[left=0.4em of iota, inner sep=0] {\(\alt<\tlWeights>{ι}{I}\colon\)};

	\uncover<\tlLklhd>{
		\node[anchor=base west] at (13em, -9.5em) {$\lklhd{c}{⟦ℳ⟧} = 0.5^4 = 0.0625$};
	}
	\only<\tlRuns>{
		\begin{scope}[node distance = -0.3em and -0.6em, text depth = 0]
			\node[above right = of t1, onslide={<\tlRootState>alert.bg}] {\(\nt{S}\)};
			\node[above left  = of t1-1      ] {\(\nt{A}\)};
			\node[above right = of t1-2      ] {\(\nt{B}\)};
			\node[above right = of t1-2-1    ] {\(\nt{A}\)};

			\node[above right = of t2, onslide={<\tlRootState>alert.bg}] {\(\nt{S}\)};
			\node[above left  = of t2-1      ] {\(\nt{A}\)};
			\node[above right = of t2-2      ] {\(\nt{B}\)};
			\node[above right = of t2-2-1    ] {\(\nt{B}\)};
			\node[above right = of t2-2-1-1  ] {\(\nt{B}\)};
			\node[above right = of t2-2-1-1-1] {\(\nt{A}\)};
		\end{scope}
	}
	\only<\tlWeightsInTrees>{
		\begin{scope}[node distance=0em, every node/.style=alert.bg, inner sep=0.2em, text depth = 0]
			\scriptsize

			\node[base right=of t1] {1};
			\node[base left=of t1-1] {1};
			\node[base right=of t1-2] {0.5};
			\node[base right=of t1-2-1] {1};

			\node[base right=of t2] {1};
			\node[base left=of t2-1] {1};
			\node[base right=of t2-2] {0.5};
			\node[base right=of t2-2-1] {0.5};
			\node[base right=of t2-2-1-1] {0.5};
			\node[base right=of t2-2-1-1-1] {1};
		\end{scope}
	}

	\draw<\tlOdd>[decorate, decoration=brace] (t1-2.north east) -- node[right, alert.bg, inner sep=0.1em, xshift=0.3em] {odd} (t1-2.south east);
	\draw<\tlOdd>[decorate, decoration=brace] (t2-2.north east) -- node[right, alert.bg, inner sep=0.1em, xshift=0.3em] {odd} (t2-2-1-1.south east);
\end{tikzpicture}
\begin{columns}
\column<\tlMLE>{0.59\linewidth}
	\begin{block}{mle for pta\s with fixed ta \(\mathcal{A}\)}
		\centering
		\(\displaystyle
			%\mle[\mathbfit{M}]{c} =
			\argmax_{(ι, δ) ∈ Θ}\ \lklhd{c}{⟦(\mathcal{A}, ι, δ)⟧}
		\)
		\\[0.5em]
		\(Θ = \{(ι, δ) \mid \text{\((\mathcal{A}, ι, δ)\) is a probabilistic wta}\}\)
	\end{block}
\column<\tlEM>{0.41\linewidth}
	\flushright
	\(⟹\) \emph{EM algorithm} [\cite{1977DempsterLairdRubin}] can approximate local optimum
\end{columns}
\end{standaloneframe}
\end{document}

\end{frame}


\iffalse
\section{Expectation Maximization Algorithm (EM Algorithm)}

\begin{frame}{\secname}
	\centering
	\documentclass[beamer]{standalone}
% SPDX-License-Identifier: CC-BY-4.0 OR MIT-0
% Copyright 2018 Toni Dietze
%
\usefonttheme{professionalfonts}

% LuaLaTeX specific packages
\usepackage{fontspec}
	\defaultfontfeatures{Ligatures=TeX}
\usepackage{polyglossia}
	\setdefaultlanguage{english}
\usepackage{amsmath}  % has to be loaded before unicode-math
\usepackage[math-style=ISO]{unicode-math}
	\setmathfont{Latin Modern Math}
% 	\setmathfont[range={\mathcal,\mathbfcal},StylisticSet=1]{xits-math.otf}
% 	\setmathfont[range={"029F5}]{XITS Math}  % ⧵
% 	\setmathfont[range={\mathscr,\mathbfscr},StylisticSet=1]{Latin Modern Math}  % make \mathscr use the correct font

\usepackage[noend]{algpseudocode}
	\algrenewcommand\algorithmicrequire{\textbf{Input:}}
	\algrenewcommand\algorithmicensure{\textbf{Output:}}
\usepackage[backend=biber, maxbibnames=42, maxcitenames=42, sorting=ynt, style=authoryear]{biblatex}
\usepackage{csquotes}
\usepackage{mathtools}
\usepackage{media9}
\usepackage{scalerel}
\usepackage{standalone}
\usepackage{tikz}
	\usetikzlibrary{arrows.meta}
	\usetikzlibrary{backgrounds}
	\usetikzlibrary{calc}
	\usetikzlibrary{decorations}
	\usetikzlibrary{decorations.pathmorphing}
	\usetikzlibrary{decorations.pathreplacing}
	\usetikzlibrary{fadings}
	\usetikzlibrary{fit}
	\usetikzlibrary{graphs}
	\usetikzlibrary{graphdrawing}
	\usetikzlibrary{intersections}
	\usetikzlibrary{positioning}
	\usetikzlibrary{quotes}
	\usetikzlibrary{shadows.blur}
	\usetikzlibrary{shapes.arrows}
	\usetikzlibrary{shapes.geometric}
	\usegdlibrary{trees}
\usepackage{xifthen}
\usepackage{xspace}

\usepackage{pgfplots}
	\pgfplotsset
		{ compat = 1.15
		, /pgf/number format/1000 sep = {\,}
		, /pgf/number format/assume math mode = true
		, every axis plot/.append style =
			{ mark options = {fill opacity = 0.25}
			}
		}
	\usepgfplotslibrary{groupplots}
\usepackage{pgfplotstable}

\hypersetup
	{ bookmarksopen
	, pdflang = en
	, unicode
	}


%%%%%%%%%%%%%%%%%%%%%%%%%%%%%%%%%%%%%%%%%%%%%%%%%%%%%%%%%%%%%%%%%%%%%%%%%%%%%%


% always show bad boxes
%\overfullrule=1em


%%%%%%%%%%%%%%%%%%%%%%%%%%%%%%%%%%%%%%%%%%%%%%%%%%%%%%%%%%%%%%%%%%%%%%%%%%%%%%
% biblatex
%%%%%%%%%%%%%%%%%%%%%%%%%%%%%%%%%%%%%%%%%%%%%%%%%%%%%%%%%%%%%%%%%%%%%%%%%%%%%%

\addbibresource{slides-dissertation-defense.bib}
% \renewcommand*{\finalnamedelim}{\addcomma\space}
% \setlength{\bibitemsep}{1em}
% 
\AtEveryBibitem{% Clean up the bibtex rather than editing it
 \clearlist{address}
 \clearfield{date}
 \clearfield{eprint}
 \clearfield{isbn}
 \clearfield{issn}
 \clearlist{language}
 \clearlist{location}
 \clearfield{month}
 \clearfield{series}
%  \clearfield{url}
%  \clearfield{doi}
 \clearfield{organization}

%  \ifentrytype{book}{}{% Remove stuff except for books
%   \clearfield{booktitle}
%   \clearfield{pages}
  \clearlist{publisher}
  \clearname{editor}
%  }
}
% do not print url if doi is present
% http://tex.stackexchange.com/questions/154864/biblatex-use-doi-only-if-there-is-no-url
\DeclareSourcemap{
	\maps[datatype=bibtex]{
		\map{
			\step[fieldsource=doi,final]
			\step[fieldset=url,null]
}	}	}
%
% remove qoutes around titles
\DeclareFieldFormat
	[article,inbook,incollection,inproceedings,patent,thesis,unpublished]
	{title}{#1\isdot}
% 
% \DeclareFieldFormat{url}{\mkbibacro{URL}\addcolon\addnbspace\url{#1}}
% 
% \DeclareNameAlias{sortname}{first-last}
% 
\renewbibmacro{in:}{\ifentrytype{article}{}{}}


%%%%%%%%%%%%%%%%%%%%%%%%%%%%%%%%%%%%%%%%%%%%%%%%%%%%%%%%%%%%%%%%%%%%%%%%%%%%%%
% beamer
%%%%%%%%%%%%%%%%%%%%%%%%%%%%%%%%%%%%%%%%%%%%%%%%%%%%%%%%%%%%%%%%%%%%%%%%%%%%%%

\useoutertheme{infolines}
\makeatletter
% based on
% /usr/share/texmf-dist/tex/latex/beamer/beamerouterthemeinfolines.sty
\setbeamertemplate{footline}
{%
	\leavevmode%
	\hbox{%
	\begin{beamercolorbox}[wd=.333333\paperwidth,ht=2.25ex,dp=1ex,center]{author in head/foot}%
		\usebeamerfont{author in head/foot}\insertshortauthor\expandafter\beamer@ifempty\expandafter{\beamer@shortinstitute}{}{~~(\insertshortinstitute)}
	\end{beamercolorbox}%
	\begin{beamercolorbox}[wd=.333333\paperwidth,ht=2.25ex,dp=1ex,center]{title in head/foot}%
		\usebeamerfont{title in head/foot}\insertshorttitle
	\end{beamercolorbox}%
	\begin{beamercolorbox}[wd=.333333\paperwidth,ht=2.25ex,dp=1ex,right]{date in head/foot}%
		\usebeamerfont{date in head/foot}%
		\hfill\insertshortdate\hfill\hfill%
		%\hspace*{2ex}%
		%\insertshortdate%
		%\hspace{0pt plus 1 filll}%
		%(\insertframenumber.\insertoverlaynumber{} / \insertmainframenumber)%
		%\hspace{0pt plus 1 filll}%
		\phantom{000}\llap{\insertpagenumber} / \insertpresentationendpage%
		\hspace*{2ex}%
	\end{beamercolorbox}}%
	\vskip0pt%
}
\makeatother
\useinnertheme{circles}
\beamertemplatenavigationsymbolsempty
\setbeamertemplate{bibliography item}{}
\setbeamertemplate{headline}[default]

\input{tudcolors.tex}
\setbeamercolor*{alerted text}{fg=HKS07K100}
\usecolortheme[named=HKS41K100]{structure}

\setbeamercolor*{palette primary}{use=structure,fg=white,bg=structure.fg}
\setbeamercolor*{palette secondary}{use=structure,fg=white,bg=structure.fg!80}
\setbeamercolor*{palette tertiary}{use=structure,fg=white,bg=structure.fg!60}
\setbeamercolor*{palette quaternary}{fg=white,bg=black}

\setbeamercolor*{sidebar}{use=structure,bg=structure.fg}

\setbeamercolor*{palette sidebar primary}{use=structure,fg=structure.fg!20}
\setbeamercolor*{palette sidebar secondary}{fg=white}
\setbeamercolor*{palette sidebar tertiary}{use=structure,fg=structure.fg!40}
\setbeamercolor*{palette sidebar quaternary}{fg=white}

\setbeamercolor*{titlelike}{parent=palette primary}

\setbeamercolor*{separation line}{}
\setbeamercolor*{fine separation line}{}

\setbeamercolor{block title}{use=structure,fg=white,bg=structure.fg}
\setbeamercolor{block title alerted}{use=alerted text,fg=white,bg=alerted text.fg!75!black}
\setbeamercolor{block title example}{use=example text,fg=white,bg=example text.fg!75!black}

\setbeamercolor{block body}{parent=normal text,use=block title,bg=block title.bg!10!bg}
\setbeamercolor{block body alerted}{parent=normal text,use=block title alerted,bg=block title alerted.bg!10!bg}
\setbeamercolor{block body example}{parent=normal text,use=block title example,bg=block title example.bg!10!bg}

% \setbeamertemplate{itemize items}[default]


%%%%%%%%%%%%%%%%%%%%%%%%%%%%%%%%%%%%%%%%%%%%%%%%%%%%%%%%%%%%%%%%%%%%%%%%%%%%%%
% TikZ
%%%%%%%%%%%%%%%%%%%%%%%%%%%%%%%%%%%%%%%%%%%%%%%%%%%%%%%%%%%%%%%%%%%%%%%%%%%%%%

\tikzset
	{ > = Stealth
	}


%%%%%%%%%%%%%%%%%%%%%%%%%%%%%%%%%%%%%%%%%%%%%%%%%%%%%%%%%%%%%%%%%%%%%%%%%%%%%%
% general commands and styles
%%%%%%%%%%%%%%%%%%%%%%%%%%%%%%%%%%%%%%%%%%%%%%%%%%%%%%%%%%%%%%%%%%%%%%%%%%%%%%

% \delegateStyle and \inheritStyle command
% usage: \delegateStyle{… \inheritStyle{…} …}
% example: \(X_{\delegateStyle{\fbox{\inheritStyle{X}}}}\)
% Save the current style and regain it in the argument.
% This works both for math and text mode, and can be nested.
% Acknowledgments: Based on \ThisStyle and \SavedStyle from scalerel package.
\makeatletter
\newcommand*{\@inheritStyle@D}[1]{\(\displaystyle      #1\)}
\newcommand*{\@inheritStyle@T}[1]{\(\textstyle         #1\)}
\newcommand*{\@inheritStyle@S}[1]{\(\scriptstyle       #1\)}
\newcommand*{\@inheritStyle@s}[1]{\(\scriptscriptstyle #1\)}
\newcommand*{\@inheritStyle@t}[1]{#1}
\newcommand*{\inheritStyle}{\csname @inheritStyle@\@inheritStyleSwitch\endcsname}
\newcommand*{\delegateStyle}[1]{%
	\ifmmode%
		\mathchoice%
		{\edef\@inheritStyleSwitch{D}#1}%
		{\edef\@inheritStyleSwitch{T}#1}%
		{\edef\@inheritStyleSwitch{S}#1}%
		{\edef\@inheritStyleSwitch{s}#1}%
	\else%
		\edef\@inheritStyleSwitch{t}#1%
	\fi%
}
\makeatother


% \oalt command
% requires: \delegateStyle and \inheritStyle command
% usage: \oalt<…>[…]{…}{…} (cf. \alt)
% Behaves like \alt, but reserves space according to largest overlays.
% The optional argument defines the alignment inside the reserved space;
% it is one of c, l, r, s (cf. \makebox); the default is c.
\makeatletter
\newlength{\oalt@dp}
\newlength{\oalt@ht}
\newlength{\oalt@wd}
\newbox{\oalt@a}
\newbox{\oalt@b}
\newbox{\oalt@empty}
\newcommand<>*{\oalt}[3][c]{%
	\delegateStyle{%
		% based on \setto… in /usr/share/texmf-dist/tex/latex/base/latex.ltx
		\setbox\oalt@a\hbox{\inheritStyle{#2}}%
		\setbox\oalt@b\hbox{\inheritStyle{#3}}%
		\pgfmathsetlength{\oalt@dp}{max(\dp\oalt@a,\dp\oalt@b)}%
		\pgfmathsetlength{\oalt@ht}{max(\ht\oalt@a,\ht\oalt@b)}%
		\pgfmathsetlength{\oalt@wd}{max(\wd\oalt@a,\wd\oalt@b)}%
		\raisebox{0pt}[\oalt@ht][\oalt@dp]{%
			\makebox[\oalt@wd][#1]{%
				\alt#4{\unhbox\oalt@a}{\unhbox\oalt@b}%
			}%
		}%
		\setbox\oalt@a\box\oalt@empty%
		\setbox\oalt@b\box\oalt@empty%
	}%
}
\makeatother


% \otemporal command
% requires: \delegateStyle and \inheritStyle command
% usage: \otemporal<…>[…]{…}{…}{…} (cf. \temporal)
% Behaves like \temporal, but reserves space according to largest overlays.
% The optional argument defines the alignment inside the reserved space;
% it is one of c, l, r, s (cf. \makebox); the default is c.
\makeatletter
\newlength{\ot@dp}
\newlength{\ot@ht}
\newlength{\ot@wd}
\newbox{\ot@a}
\newbox{\ot@b}
\newbox{\ot@c}
\newbox{\ot@empty}
\newcommand<>*{\otemporal}[4][c]{%
	\delegateStyle{%
		% based on \setto… in /usr/share/texmf-dist/tex/latex/base/latex.ltx
		\setbox\ot@a\hbox{\inheritStyle{#2}}%
		\setbox\ot@b\hbox{\inheritStyle{#3}}%
		\setbox\ot@c\hbox{\inheritStyle{#4}}%
		\pgfmathsetlength{\ot@dp}{max(\dp\ot@a,\dp\ot@b,\dp\ot@c)}%
		\pgfmathsetlength{\ot@ht}{max(\ht\ot@a,\ht\ot@b,\ht\ot@c)}%
		\pgfmathsetlength{\ot@wd}{max(\wd\ot@a,\wd\ot@b,\wd\ot@c)}%
		\raisebox{0pt}[\ot@ht][\ot@dp]{%
			\makebox[\ot@wd][#1]{%
				\temporal#5{\unhbox\ot@a}{\unhbox\ot@b}{\unhbox\ot@c}%
			}%
		}%
		\setbox\ot@a\box\ot@empty%
		\setbox\ot@b\box\ot@empty%
		\setbox\ot@c\box\ot@empty%
	}%
}
\makeatother


% Resize delimiters like braces, brackets, etc.
% Parameters: size, left delimiter, formula, right delimiter
% Example: \delim2({\frac{1}{2}})
\newcommand*{\delim}[4]{%
	\ifcase#1%
		#2#3#4%
	\or%
		\bigl#2#3\bigr#4%
	\or%
		\Bigl#2#3\Bigr#4%
	\or%
		\biggl#2#3\biggr#4%
	\or%
		\Biggl#2#3\Biggr#4%
	\else%
		\left#2#3\right#4%
	\fi%
}


% similar to \fullcite, but using the formatting of \printbibliography
\newcommand*{\printfullcite}[1]{%
	\begin{refsection}%
		\nocite{#1}%
		\DeclareNameAlias{author}{first-last}%
		\printbibliography[heading = none]%
	\end{refsection}%
}


\colorlet{light alert}{HKS07K60}
\tikzset{alert.bg/.style={rounded corners, fill=light alert}}
\tikzset{every picture/.style={line cap=round, semithick}}
% http://tex.stackexchange.com/questions/6135/how-to-make-beamer-overlays-with-tikz-node
\tikzset{onslide/.code args={<#1>#2}{\only<#1>{\pgfkeysalso{#2}}}}
\tikzset{invisible/.code args={<#1>}{\alt<#1>{\pgfkeysalso{transparent}}{\pgfkeysalso{opaque}}}}
\tikzset{uncover/.code args={<#1>}{\alt<#1>{\pgfkeysalso{opaque}}{\pgfkeysalso{opacity=0.25}}}}
\tikzset{visible/.code args={<#1>}{\alt<#1>{\pgfkeysalso{opaque}}{\pgfkeysalso{transparent}}}}
\tikzset{vuncover/.code args=%
	{<#1><#2>}%
	{\alt<#1>%
		{\alt<#2>%
			{\pgfkeysalso{opaque}}%
			{\pgfkeysalso{opacity=0.25}}%
		}{\pgfkeysalso{transparent}}%
	}%
}

\newcommand<%
	>{\tikzhighlight}[2][]{%
	\delegateStyle{\alt#3%
		{\tikz[baseline=0, anchor=base, inner sep=0.2em, text height=, text depth=]{\node[alert.bg, #1]{\inheritStyle{#2}};}}%
		{\tikz[baseline=0, anchor=base, inner sep=0.2em, text height=, text depth=]{\node[#1, fill=none]{\inheritStyle{#2}};}}%
	}%
}

\newcommand{\mathhighlight}{\tikzhighlight}

\newcommand<>{\mhl}[2][]{\mathhighlight#3[inner sep=0.2em, #1]{#2}}


\newcommand<>{\inlineblock}[2][]{{%
	\usebeamercolor*[fg]{block body}%
	\tikzhighlight#3[fill=block body.bg, #1]{#2}%
}}


% a small letter s for plurals of abbreviations
\newcommand*{\s}{{\scriptsize s}\xspace}


\newcommand<>*{\sout}[2][opacity=0.75, ultra thick]{%
	\delegateStyle{%
		\tikz[baseline=0, anchor=base, inner sep=0, outer sep=0]{
			\useasboundingbox node (n) {\inheritStyle{#2}};
			\only#3{
				\node (h) {\inheritStyle{\ifmmode\mathstrut\else\strut\fi}};
				\draw[#1] (n.west |- {$(h.south)!0.5!(h.north)$}) -- (n.east |- {$(h.south)!0.5!(h.north)$});
			}
		}%
	}%
}


% tight style
% Sets outer sep to default inner sep and inner sep to 0.
% Use this style for nodes that are neither drawn nor filled to prevent
% unwanted growth of the bounding box.
\tikzset{tight/.style={inner sep=0, outer sep=0.3333em}}


% rounded tree edges style
% usage: rounded tree edges={⟨direction⟩}{⟨looseness⟩}{⟨strength⟩}
\tikzset{
	rounded tree edges/.style n args={3}{
	edge from parent path={
	let
		\n{direction}={#1},
		\n{looseness}={#2},
		\n{strength}={#3},
		\p1=(\tikzparentnode),
		\p2=(\tikzchildnode),
		\p3=(\n{direction}:1pt),
		\p4=(\x2 - \x1, \y2 - \y1),
		\n{dist}={veclen(\p4)},
		\p4=(\x4 / \n{dist}, \y4 / \n{dist}),
		\n{angle}={atan2(\y4, \x4)},
		\n{delta}={Mod(\n{angle} - \n{direction}, 360)},
		\n{delta}={\n{delta} > 180 ? \n{delta} - 360  : \n{delta}},
		\n{delta}={\n{delta} >  90 ?  180 - \n{delta} : \n{delta}},
		\n{delta}={\n{delta} < -90 ? -180 - \n{delta} : \n{delta}}
	in (\tikzparentnode) .. controls
		+(    \n{angle}+\n{strength}*\n{delta}:\n{looseness}*0.3915*\n{dist}) and
		+(180+\n{angle}-\n{strength}*\n{delta}:\n{looseness}*0.3915*\n{dist}) ..
		(\tikzchildnode)
	}
	}
}


% Tear out snippets from PDFs.
% Usage: \tear[…]{file.pdf}
% The optional parameter is the same as for \includegraphics.
% Useful Arguments:
%   * page=‹pagenumber›
%   * trim=‹left› ‹bottom› ‹right› ‹top›
%   * width=0.98\linewidth
\newcommand*{\tear}[2][]{%
	\begin{tikzpicture}
		\node
			[ blur shadow
			, clip
			, decorate
			, decoration=random steps
			, draw
			, inner sep=0
			, preaction={fill=white}% hide the shadow if paper is transparent
			] {\includegraphics[#1]{#2}};
	\end{tikzpicture}%
}


\makeatletter
\newcommand*{\timeline}[3][0]{%
	\ifcsname timeline@cmd@#3\endcsname%
		\@timeline[#1]{#2}{#3}%
		\PackageWarning{timeline}{redefining timeline \@backslashchar\string#3}%
	\else%
		\ifcsname#3\endcsname%
			\errmessage{Command \@backslashchar\string#3 already defined}%
		\else%
			\@timeline[#1]{#2}{#3}%
		\fi%
	\fi%
}%
\newcommand*{\@timeline}[3][0]{%
	% mark command as timeline command – they can be overwritten
	\expandafter\def\csname timeline@cmd@#3\endcsname{}%
	\setcounter{@timeline}{#1}%
	\def\timeline@cmd{#3}%
	\timeline@reset%
	\timeline@append{0}%
	\@tfor\timeline@next:=#2\do{%
		\if\timeline@next+%
			\stepcounter{@timeline}%
			\timeline@append{,\the@timeline}%
		\else\if\timeline@next-%
			\stepcounter{@timeline}%
		\else%
			%\timeline@append{\timeline@next}%
			\GenericError{}{\protect\timeline: ignoring unknown character: \timeline@next}%
		\fi\fi%
	}%
}%
% \newcommand*{\tl}[1]{%
% 	\ifcsname timeline@cmd@#1\endcsname%
% 		\csname timeline@cmd@#1\endcsname%
% 	\else%
% 		0%
% 		%\GenericError{}{\protect\tl: timeline not defined: #1}%
% 	\fi%
% }%
\newcounter{@timeline}%
\def\timeline@reset{%
	\expandafter\def\csname\timeline@cmd\endcsname{}%
}%
\def\timeline@append#1{%
	\expandafter\edef\csname\timeline@cmd\endcsname{%
		\csname\timeline@cmd\endcsname#1%
	}%
}%
\makeatother


\newcommand*{\xminus}[1]{%
	\mathrel{\tikz[baseline={([yshift=-0.25em]n.south)}, inner sep=0, outer sep=0.2em]{%
		\node (n) {\(\scriptstyle #1\)};
		\draw (n.south west) -- (n.south east);
	}}%
}
\newcommand*{\tikzrightarrow}[1]{%
	\mathrel{\tikz[baseline={([yshift=-0.25em]n.south)}, inner sep=0, outer sep=0.2em]{%
		\node (n) {\(\scriptstyle #1\)};
		\draw[->, > = Computer Modern Rightarrow, line width = 0.4pt] (n.south west) -- (n.south east);
	}}%
}


%%%%%%%%%%%%%%%%%%%%%%%%%%%%%%%%%%%%%%%%%%%%%%%%%%%%%%%%%%%%%%%%%%%%%%%%%%%%%%
% document specific commands
%%%%%%%%%%%%%%%%%%%%%%%%%%%%%%%%%%%%%%%%%%%%%%%%%%%%%%%%%%%%%%%%%%%%%%%%%%%%%%

\newcommand<>*{\mycite}[1]{\uncover#2{{\color{HKS57K100}[\cite{#1}]}}}


\newcommand{\statetree}[1]{
	\tikz
	[ anchor=base
	, baseline=(current bounding box.center)
	, level distance=2em
	, sibling distance=2em
	]{
		\matrix
		[ draw=nt
		, edge from parent/.style={draw=black}
		, inner sep=0
		, nodes={inner sep=0.2em, rounded corners=0}
		, rounded corners
		] {#1\\}
	}
}


\newcommand*{\mylargeleaf}[1]{{\LARGE\color{HKS41K70}#1}}

\definecolor{state s}{named}{HKS57K80}
\definecolor{state t}{named}{HKS41K70}
\newcommand*{\stateS}[1]{{\color{state s}#1}}
\newcommand*{\stateT}[1]{{\color{state t}#1}}

\tikzset{
	subtree/.style =
		{ fill=lightgray
		, inner sep=0.02em
		, isosceles triangle apex angle=60
		, shape=isosceles triangle
		, shape border rotate=90
		}
	, state/.style = {circle, draw, inner sep=0.1em}
	, trans/.style = {rectangle, draw}
}

\newcommand*{\srBool}{\mathbb{B}}
\newcommand*{\srProb}{ℙ}


%%%%%%%%%%%%%%%%%%%%%%%%%%%%%%%%%%%%%%%%%%%%%%%%%%%%%%%%%%%%%%%%%%%%%%%%%%%%%%
% commands for specific notations
%%%%%%%%%%%%%%%%%%%%%%%%%%%%%%%%%%%%%%%%%%%%%%%%%%%%%%%%%%%%%%%%%%%%%%%%%%%%%%

\DeclareMathOperator*{\argmax}{argmax}

\newcommand*{\cardinality}[1]{\lvert#1\rvert}
\newcommand*{\corpussize}[1]{\lvert#1\rvert}

\DeclareMathOperator{\crispOp}{crisp}
\newcommand*        {\crisp}[2][0]{\crispOp\delim{#1}({#2})}

\DeclareMathOperator{\lhsOp}{lhs}
\newcommand*{\lhs}[1]{\lhsOp(#1)}

\DeclareMathOperator{\lklhdOp}{L}
\newcommand*{\lklhd}[2]{\lklhdOp(#1 ∣ #2)}

\DeclareMathOperator{\mleOp}{mle}
\newcommand*{\mle}[2][]{%
	\ifthenelse{\isempty{#1}}{%
		\mleOp(#2)%
	}{%
		\mleOp_{#1}(#2)%
	}%
}

\DeclareMathOperator{\mrg}{merge}

% CVD: color vision deficiencies
\definecolor{CVD light red}   {HTML}{FF8080}
\definecolor{CVD light yellow}{HTML}{FFFF80}
\definecolor{CVD light green} {HTML}{40FFC0}

\definecolor{nt}{named}{HKS41K70}
\newcommand*{\nt}[1]{{\color{nt}#1}}

% set of all probability distributions over #1
\DeclareMathOperator{\pdsOp}{Pd}
\newcommand*{\pds}[1]{\pdsOp(#1)}

\DeclareMathOperator{\positionsOp}{pos}
\newcommand*{\positions}[1]{\positionsOp(#1)}

\DeclareMathOperator{\rankOp}{rk}
\newcommand*{\rank}[1]{\rankOp(#1)}

\DeclareMathOperator{\runsOp}{run}
\newcommand*{\runs}[2][]{%
	\ifthenelse%
		{\isempty{#1}}%
		{\runsOp(#2)}%
		{\runsOp_{#1}(#2)}%
}

\newcommand*{\semantics}[1]{⟦#1⟧}

\DeclareMathOperator{\splt}{split}

\newcommand*{\subtree}[2]{#1|_{#2}}

\DeclareMathOperator{\supportOp}{supp}
\newcommand*{\support}[1]{\supportOp(#1)}

\newcommand*{\symId}{\textsc{\color{gray}Id}}
\newcommand*{\symCons}{\textsc{\color{gray}Cons}}
\newcommand*{\symFlip}{\textsc{\color{gray}Flip}}
\newcommand*{\symNull}{\textsc{\color{gray}Null}}
\newcommand*{\symNullR}{\textsc{\color{gray}N\(\overline{\textsc{ull}}\)}}
\newcommand*{\symSnoc}{\textsc{\color{gray}Snoc}}

\newcommand*{\transWTA}[4][]{#3 \xrightarrow{#1} #2(#4)}

\DeclareMathOperator{\uniqueRunOp}{r}
\newcommand*{\uniqueRun}[2][]{%
	\ifthenelse%
		{\isempty{#1}}%
		{\uniqueRunOp^{#2}}%
		{\uniqueRunOp_{\!#1}^{#2}}%
}

\DeclareMathOperator{\treesOp}{T}
\newcommand*{\trees}[2][]{%
	\ifthenelse%
		{\isempty{#1}}%
		{\treesOp_{\!#2}}%
		{\treesOp_{\!#2}(#1)}%
}
\DeclareMathOperator{\treesUOp}{U}
\newcommand*{\treesU}[2][]{%
	\ifthenelse%
		{\isempty{#1}}%
		{\treesUOp_{#2}}%
		{\treesUOp_{#2}(#1)}%
}


%%%%%%%%%%%%%%%%%%%%%%%%%%%%%%%%%%%%%%%%%%%%%%%%%%%%%%%%%%%%%%%%%%%%%%%%%%%%%%
% metadata
%%%%%%%%%%%%%%%%%%%%%%%%%%%%%%%%%%%%%%%%%%%%%%%%%%%%%%%%%%%%%%%%%%%%%%%%%%%%%%

\ifstandalonebeamer\else
	\title[Defense of Dissertation]{A Formal View on Training of Weighted Tree Automata by Likelihood-Driven State Splitting and Merging}
	\subtitle{Defense of Dissertation}
\fi
\author{Toni Dietze}
\institute[TU Dresden]{%
	\href{https://www.orchid.inf.tu-dresden.de/index.en/}{Chair for Foundations of Programming}
\\	\href{https://tu-dresden.de/ing/informatik/thi}{Institute of Theoretical Computer Science}
\\	\href{https://tu-dresden.de/ing/informatik}{Faculty of Computer Science}
\\	\href{https://tu-dresden.de/}{Technische Universität Dresden}
\\	01062 Dresden, Germany
}
\date[2018-09-27]{September 27, 2018}

\begin{document}
\begin{standaloneframe}{\jobname}
\centering
\begin{tikzpicture}[anchor=base, level distance=2.5em]
	\node (t) {\(σ\)}
		child { node {\(α\)}
			edge from parent node[left] {\only<2->{\(\nt{?}\)}}
		}
		child { node {\(σ\)}
			child { node {\(β\)}
				edge from parent node[left] {\only<2->{\(\nt{?}\)}}
			}
			child { node {\(β\)}
				edge from parent node[right] {\only<2->{\(\nt{?}\)}}
			}
			edge from parent node[right] {\only<2->{\(\nt{?}\)}}
		};
	\node[above=0 of t.north] {\only<2->{\(\nt{?}\)}};

	% coordinates of tree triangle
	\path coordinate (top)   at ([yshift=2em]t.north)
			coordinate (right) at ([xshift=2em]t-2-2.south east)
			let \p1 = (top), \p2 = (right), \p3 = (2 * \x1 - \x2, \y2) in
			coordinate (left) at (\p3);

	% draw tree triangle
	\draw (top) -- (right) -- (left) -- cycle;

	% draw subsequent tree triangles
	\coordinate (s) at (0, 0);
	\foreach \s in {1, 2, ..., 4}
		\draw
			coordinate (sOld) at (s)
			coordinate (s) at (\s*0.5em, \s*0.125em)
			(intersection cs: first line = {([shift={(sOld)}]top) -- ([shift={(sOld)}]right)}
							, second line = {([shift={(s)}]top)    -- ([shift={(s)}]left)})
			-- ([shift={(s)}]top)
			-- ([shift={(s)}]right)
			-- (intersection cs: first line = {([shift={(sOld)}]top) -- ([shift={(sOld)}]right)}
								, second line = {([shift={(s)}]right)  -- ([shift={(s)}]left)});
\end{tikzpicture}

\begin{overprint}
\onslide<1-3>
	\begin{description}
	\item[input]
		corpus $c\colon \trees{Σ} → ℝ_{≥ 0}$, ta $\mathscr{A}$ over $Σ$\strut
	\item[output]
		$\displaystyle \argmax_{\substack{\text{prob.\ wta} \\ ℳ = (\mathscr{A}, ι, δ)}} \lklhd{c}{⟦ℳ⟧}$ \only<3->{\qquad \alert{$→$ difficult!}}
	\end{description}
\onslide<4->
	\begin{description}
	\item[input]
		corpus $c\colon \trees{Σ} → ℝ_{≥ 0}$, prob.\ wta $ℳ_0 = (\mathscr{A}, ι_0, δ_0)$ over $Σ$\strut
	\item[output]
		sequence $ℳ_0, ℳ_1, …$, such~that~$\lklhd{c}{⟦ℳ_0⟧} ≤ \lklhd{c}{⟦ℳ_1⟧} ≤ …$
	\item[algorithm]
% 		\begin{spacing}{1.6}
			\begin{algorithmic}
				\For{$i \gets 0, 1, …$}
					\hfill \mycite{1977DempsterLairdRubin}
					\State
						{\usebeamercolor[fg]{enumerate item} E-Step:} $\displaystyle \forall d \in \operatorname{D}_G\colon c'_i(t, r) \gets c(t) \cdot \frac{⟦ℳ_i⟧(t, r)}{⟦ℳ_i⟧(t)}$
					\State
						{\usebeamercolor[fg]{enumerate item} M-Step:} $\displaystyle ℳ_{i + 1} \gets \argmax_{\substack{\text{prob.\ wta} \\ ℳ = (\mathscr{A}, ι, δ)}} \lklhd{c_i'}{⟦ℳ⟧}$
						\vphantom{\({A^A}^A\)}
				\EndFor
			\end{algorithmic}
% 		\end{spacing}
	\end{description}
\end{overprint}
\end{standaloneframe}
\end{document}

\end{frame}
\fi


\againframe<5>{toc}


\section{State Splitting and Merging Algorithm}

\begin{frame}<1-2>[label=frame:algorithm-split-merge]{\secname}
	% SPDX-License-Identifier: CC-BY-4.0
% Copyright 2018 Toni Dietze
\documentclass[beamer]{standalone}
% SPDX-License-Identifier: CC-BY-4.0 OR MIT-0
% Copyright 2018 Toni Dietze
%
\usefonttheme{professionalfonts}

% LuaLaTeX specific packages
\usepackage{fontspec}
	\defaultfontfeatures{Ligatures=TeX}
\usepackage{polyglossia}
	\setdefaultlanguage{english}
\usepackage{amsmath}  % has to be loaded before unicode-math
\usepackage[math-style=ISO]{unicode-math}
	\setmathfont{Latin Modern Math}
% 	\setmathfont[range={\mathcal,\mathbfcal},StylisticSet=1]{xits-math.otf}
% 	\setmathfont[range={"029F5}]{XITS Math}  % ⧵
% 	\setmathfont[range={\mathscr,\mathbfscr},StylisticSet=1]{Latin Modern Math}  % make \mathscr use the correct font

\usepackage[noend]{algpseudocode}
	\algrenewcommand\algorithmicrequire{\textbf{Input:}}
	\algrenewcommand\algorithmicensure{\textbf{Output:}}
\usepackage[backend=biber, maxbibnames=42, maxcitenames=42, sorting=ynt, style=authoryear]{biblatex}
\usepackage{csquotes}
\usepackage{mathtools}
\usepackage{media9}
\usepackage{scalerel}
\usepackage{standalone}
\usepackage{tikz}
	\usetikzlibrary{arrows.meta}
	\usetikzlibrary{backgrounds}
	\usetikzlibrary{calc}
	\usetikzlibrary{decorations}
	\usetikzlibrary{decorations.pathmorphing}
	\usetikzlibrary{decorations.pathreplacing}
	\usetikzlibrary{fadings}
	\usetikzlibrary{fit}
	\usetikzlibrary{graphs}
	\usetikzlibrary{graphdrawing}
	\usetikzlibrary{intersections}
	\usetikzlibrary{positioning}
	\usetikzlibrary{quotes}
	\usetikzlibrary{shadows.blur}
	\usetikzlibrary{shapes.arrows}
	\usetikzlibrary{shapes.geometric}
	\usegdlibrary{trees}
\usepackage{xifthen}
\usepackage{xspace}

\usepackage{pgfplots}
	\pgfplotsset
		{ compat = 1.15
		, /pgf/number format/1000 sep = {\,}
		, /pgf/number format/assume math mode = true
		, every axis plot/.append style =
			{ mark options = {fill opacity = 0.25}
			}
		}
	\usepgfplotslibrary{groupplots}
\usepackage{pgfplotstable}

\hypersetup
	{ bookmarksopen
	, pdflang = en
	, unicode
	}


%%%%%%%%%%%%%%%%%%%%%%%%%%%%%%%%%%%%%%%%%%%%%%%%%%%%%%%%%%%%%%%%%%%%%%%%%%%%%%


% always show bad boxes
%\overfullrule=1em


%%%%%%%%%%%%%%%%%%%%%%%%%%%%%%%%%%%%%%%%%%%%%%%%%%%%%%%%%%%%%%%%%%%%%%%%%%%%%%
% biblatex
%%%%%%%%%%%%%%%%%%%%%%%%%%%%%%%%%%%%%%%%%%%%%%%%%%%%%%%%%%%%%%%%%%%%%%%%%%%%%%

\addbibresource{slides-dissertation-defense.bib}
% \renewcommand*{\finalnamedelim}{\addcomma\space}
% \setlength{\bibitemsep}{1em}
% 
\AtEveryBibitem{% Clean up the bibtex rather than editing it
 \clearlist{address}
 \clearfield{date}
 \clearfield{eprint}
 \clearfield{isbn}
 \clearfield{issn}
 \clearlist{language}
 \clearlist{location}
 \clearfield{month}
 \clearfield{series}
%  \clearfield{url}
%  \clearfield{doi}
 \clearfield{organization}

%  \ifentrytype{book}{}{% Remove stuff except for books
%   \clearfield{booktitle}
%   \clearfield{pages}
  \clearlist{publisher}
  \clearname{editor}
%  }
}
% do not print url if doi is present
% http://tex.stackexchange.com/questions/154864/biblatex-use-doi-only-if-there-is-no-url
\DeclareSourcemap{
	\maps[datatype=bibtex]{
		\map{
			\step[fieldsource=doi,final]
			\step[fieldset=url,null]
}	}	}
%
% remove qoutes around titles
\DeclareFieldFormat
	[article,inbook,incollection,inproceedings,patent,thesis,unpublished]
	{title}{#1\isdot}
% 
% \DeclareFieldFormat{url}{\mkbibacro{URL}\addcolon\addnbspace\url{#1}}
% 
% \DeclareNameAlias{sortname}{first-last}
% 
\renewbibmacro{in:}{\ifentrytype{article}{}{}}


%%%%%%%%%%%%%%%%%%%%%%%%%%%%%%%%%%%%%%%%%%%%%%%%%%%%%%%%%%%%%%%%%%%%%%%%%%%%%%
% beamer
%%%%%%%%%%%%%%%%%%%%%%%%%%%%%%%%%%%%%%%%%%%%%%%%%%%%%%%%%%%%%%%%%%%%%%%%%%%%%%

\useoutertheme{infolines}
\makeatletter
% based on
% /usr/share/texmf-dist/tex/latex/beamer/beamerouterthemeinfolines.sty
\setbeamertemplate{footline}
{%
	\leavevmode%
	\hbox{%
	\begin{beamercolorbox}[wd=.333333\paperwidth,ht=2.25ex,dp=1ex,center]{author in head/foot}%
		\usebeamerfont{author in head/foot}\insertshortauthor\expandafter\beamer@ifempty\expandafter{\beamer@shortinstitute}{}{~~(\insertshortinstitute)}
	\end{beamercolorbox}%
	\begin{beamercolorbox}[wd=.333333\paperwidth,ht=2.25ex,dp=1ex,center]{title in head/foot}%
		\usebeamerfont{title in head/foot}\insertshorttitle
	\end{beamercolorbox}%
	\begin{beamercolorbox}[wd=.333333\paperwidth,ht=2.25ex,dp=1ex,right]{date in head/foot}%
		\usebeamerfont{date in head/foot}%
		\hfill\insertshortdate\hfill\hfill%
		%\hspace*{2ex}%
		%\insertshortdate%
		%\hspace{0pt plus 1 filll}%
		%(\insertframenumber.\insertoverlaynumber{} / \insertmainframenumber)%
		%\hspace{0pt plus 1 filll}%
		\phantom{000}\llap{\insertpagenumber} / \insertpresentationendpage%
		\hspace*{2ex}%
	\end{beamercolorbox}}%
	\vskip0pt%
}
\makeatother
\useinnertheme{circles}
\beamertemplatenavigationsymbolsempty
\setbeamertemplate{bibliography item}{}
\setbeamertemplate{headline}[default]

\input{tudcolors.tex}
\setbeamercolor*{alerted text}{fg=HKS07K100}
\usecolortheme[named=HKS41K100]{structure}

\setbeamercolor*{palette primary}{use=structure,fg=white,bg=structure.fg}
\setbeamercolor*{palette secondary}{use=structure,fg=white,bg=structure.fg!80}
\setbeamercolor*{palette tertiary}{use=structure,fg=white,bg=structure.fg!60}
\setbeamercolor*{palette quaternary}{fg=white,bg=black}

\setbeamercolor*{sidebar}{use=structure,bg=structure.fg}

\setbeamercolor*{palette sidebar primary}{use=structure,fg=structure.fg!20}
\setbeamercolor*{palette sidebar secondary}{fg=white}
\setbeamercolor*{palette sidebar tertiary}{use=structure,fg=structure.fg!40}
\setbeamercolor*{palette sidebar quaternary}{fg=white}

\setbeamercolor*{titlelike}{parent=palette primary}

\setbeamercolor*{separation line}{}
\setbeamercolor*{fine separation line}{}

\setbeamercolor{block title}{use=structure,fg=white,bg=structure.fg}
\setbeamercolor{block title alerted}{use=alerted text,fg=white,bg=alerted text.fg!75!black}
\setbeamercolor{block title example}{use=example text,fg=white,bg=example text.fg!75!black}

\setbeamercolor{block body}{parent=normal text,use=block title,bg=block title.bg!10!bg}
\setbeamercolor{block body alerted}{parent=normal text,use=block title alerted,bg=block title alerted.bg!10!bg}
\setbeamercolor{block body example}{parent=normal text,use=block title example,bg=block title example.bg!10!bg}

% \setbeamertemplate{itemize items}[default]


%%%%%%%%%%%%%%%%%%%%%%%%%%%%%%%%%%%%%%%%%%%%%%%%%%%%%%%%%%%%%%%%%%%%%%%%%%%%%%
% TikZ
%%%%%%%%%%%%%%%%%%%%%%%%%%%%%%%%%%%%%%%%%%%%%%%%%%%%%%%%%%%%%%%%%%%%%%%%%%%%%%

\tikzset
	{ > = Stealth
	}


%%%%%%%%%%%%%%%%%%%%%%%%%%%%%%%%%%%%%%%%%%%%%%%%%%%%%%%%%%%%%%%%%%%%%%%%%%%%%%
% general commands and styles
%%%%%%%%%%%%%%%%%%%%%%%%%%%%%%%%%%%%%%%%%%%%%%%%%%%%%%%%%%%%%%%%%%%%%%%%%%%%%%

% \delegateStyle and \inheritStyle command
% usage: \delegateStyle{… \inheritStyle{…} …}
% example: \(X_{\delegateStyle{\fbox{\inheritStyle{X}}}}\)
% Save the current style and regain it in the argument.
% This works both for math and text mode, and can be nested.
% Acknowledgments: Based on \ThisStyle and \SavedStyle from scalerel package.
\makeatletter
\newcommand*{\@inheritStyle@D}[1]{\(\displaystyle      #1\)}
\newcommand*{\@inheritStyle@T}[1]{\(\textstyle         #1\)}
\newcommand*{\@inheritStyle@S}[1]{\(\scriptstyle       #1\)}
\newcommand*{\@inheritStyle@s}[1]{\(\scriptscriptstyle #1\)}
\newcommand*{\@inheritStyle@t}[1]{#1}
\newcommand*{\inheritStyle}{\csname @inheritStyle@\@inheritStyleSwitch\endcsname}
\newcommand*{\delegateStyle}[1]{%
	\ifmmode%
		\mathchoice%
		{\edef\@inheritStyleSwitch{D}#1}%
		{\edef\@inheritStyleSwitch{T}#1}%
		{\edef\@inheritStyleSwitch{S}#1}%
		{\edef\@inheritStyleSwitch{s}#1}%
	\else%
		\edef\@inheritStyleSwitch{t}#1%
	\fi%
}
\makeatother


% \oalt command
% requires: \delegateStyle and \inheritStyle command
% usage: \oalt<…>[…]{…}{…} (cf. \alt)
% Behaves like \alt, but reserves space according to largest overlays.
% The optional argument defines the alignment inside the reserved space;
% it is one of c, l, r, s (cf. \makebox); the default is c.
\makeatletter
\newlength{\oalt@dp}
\newlength{\oalt@ht}
\newlength{\oalt@wd}
\newbox{\oalt@a}
\newbox{\oalt@b}
\newbox{\oalt@empty}
\newcommand<>*{\oalt}[3][c]{%
	\delegateStyle{%
		% based on \setto… in /usr/share/texmf-dist/tex/latex/base/latex.ltx
		\setbox\oalt@a\hbox{\inheritStyle{#2}}%
		\setbox\oalt@b\hbox{\inheritStyle{#3}}%
		\pgfmathsetlength{\oalt@dp}{max(\dp\oalt@a,\dp\oalt@b)}%
		\pgfmathsetlength{\oalt@ht}{max(\ht\oalt@a,\ht\oalt@b)}%
		\pgfmathsetlength{\oalt@wd}{max(\wd\oalt@a,\wd\oalt@b)}%
		\raisebox{0pt}[\oalt@ht][\oalt@dp]{%
			\makebox[\oalt@wd][#1]{%
				\alt#4{\unhbox\oalt@a}{\unhbox\oalt@b}%
			}%
		}%
		\setbox\oalt@a\box\oalt@empty%
		\setbox\oalt@b\box\oalt@empty%
	}%
}
\makeatother


% \otemporal command
% requires: \delegateStyle and \inheritStyle command
% usage: \otemporal<…>[…]{…}{…}{…} (cf. \temporal)
% Behaves like \temporal, but reserves space according to largest overlays.
% The optional argument defines the alignment inside the reserved space;
% it is one of c, l, r, s (cf. \makebox); the default is c.
\makeatletter
\newlength{\ot@dp}
\newlength{\ot@ht}
\newlength{\ot@wd}
\newbox{\ot@a}
\newbox{\ot@b}
\newbox{\ot@c}
\newbox{\ot@empty}
\newcommand<>*{\otemporal}[4][c]{%
	\delegateStyle{%
		% based on \setto… in /usr/share/texmf-dist/tex/latex/base/latex.ltx
		\setbox\ot@a\hbox{\inheritStyle{#2}}%
		\setbox\ot@b\hbox{\inheritStyle{#3}}%
		\setbox\ot@c\hbox{\inheritStyle{#4}}%
		\pgfmathsetlength{\ot@dp}{max(\dp\ot@a,\dp\ot@b,\dp\ot@c)}%
		\pgfmathsetlength{\ot@ht}{max(\ht\ot@a,\ht\ot@b,\ht\ot@c)}%
		\pgfmathsetlength{\ot@wd}{max(\wd\ot@a,\wd\ot@b,\wd\ot@c)}%
		\raisebox{0pt}[\ot@ht][\ot@dp]{%
			\makebox[\ot@wd][#1]{%
				\temporal#5{\unhbox\ot@a}{\unhbox\ot@b}{\unhbox\ot@c}%
			}%
		}%
		\setbox\ot@a\box\ot@empty%
		\setbox\ot@b\box\ot@empty%
		\setbox\ot@c\box\ot@empty%
	}%
}
\makeatother


% Resize delimiters like braces, brackets, etc.
% Parameters: size, left delimiter, formula, right delimiter
% Example: \delim2({\frac{1}{2}})
\newcommand*{\delim}[4]{%
	\ifcase#1%
		#2#3#4%
	\or%
		\bigl#2#3\bigr#4%
	\or%
		\Bigl#2#3\Bigr#4%
	\or%
		\biggl#2#3\biggr#4%
	\or%
		\Biggl#2#3\Biggr#4%
	\else%
		\left#2#3\right#4%
	\fi%
}


% similar to \fullcite, but using the formatting of \printbibliography
\newcommand*{\printfullcite}[1]{%
	\begin{refsection}%
		\nocite{#1}%
		\DeclareNameAlias{author}{first-last}%
		\printbibliography[heading = none]%
	\end{refsection}%
}


\colorlet{light alert}{HKS07K60}
\tikzset{alert.bg/.style={rounded corners, fill=light alert}}
\tikzset{every picture/.style={line cap=round, semithick}}
% http://tex.stackexchange.com/questions/6135/how-to-make-beamer-overlays-with-tikz-node
\tikzset{onslide/.code args={<#1>#2}{\only<#1>{\pgfkeysalso{#2}}}}
\tikzset{invisible/.code args={<#1>}{\alt<#1>{\pgfkeysalso{transparent}}{\pgfkeysalso{opaque}}}}
\tikzset{uncover/.code args={<#1>}{\alt<#1>{\pgfkeysalso{opaque}}{\pgfkeysalso{opacity=0.25}}}}
\tikzset{visible/.code args={<#1>}{\alt<#1>{\pgfkeysalso{opaque}}{\pgfkeysalso{transparent}}}}
\tikzset{vuncover/.code args=%
	{<#1><#2>}%
	{\alt<#1>%
		{\alt<#2>%
			{\pgfkeysalso{opaque}}%
			{\pgfkeysalso{opacity=0.25}}%
		}{\pgfkeysalso{transparent}}%
	}%
}

\newcommand<%
	>{\tikzhighlight}[2][]{%
	\delegateStyle{\alt#3%
		{\tikz[baseline=0, anchor=base, inner sep=0.2em, text height=, text depth=]{\node[alert.bg, #1]{\inheritStyle{#2}};}}%
		{\tikz[baseline=0, anchor=base, inner sep=0.2em, text height=, text depth=]{\node[#1, fill=none]{\inheritStyle{#2}};}}%
	}%
}

\newcommand{\mathhighlight}{\tikzhighlight}

\newcommand<>{\mhl}[2][]{\mathhighlight#3[inner sep=0.2em, #1]{#2}}


\newcommand<>{\inlineblock}[2][]{{%
	\usebeamercolor*[fg]{block body}%
	\tikzhighlight#3[fill=block body.bg, #1]{#2}%
}}


% a small letter s for plurals of abbreviations
\newcommand*{\s}{{\scriptsize s}\xspace}


\newcommand<>*{\sout}[2][opacity=0.75, ultra thick]{%
	\delegateStyle{%
		\tikz[baseline=0, anchor=base, inner sep=0, outer sep=0]{
			\useasboundingbox node (n) {\inheritStyle{#2}};
			\only#3{
				\node (h) {\inheritStyle{\ifmmode\mathstrut\else\strut\fi}};
				\draw[#1] (n.west |- {$(h.south)!0.5!(h.north)$}) -- (n.east |- {$(h.south)!0.5!(h.north)$});
			}
		}%
	}%
}


% tight style
% Sets outer sep to default inner sep and inner sep to 0.
% Use this style for nodes that are neither drawn nor filled to prevent
% unwanted growth of the bounding box.
\tikzset{tight/.style={inner sep=0, outer sep=0.3333em}}


% rounded tree edges style
% usage: rounded tree edges={⟨direction⟩}{⟨looseness⟩}{⟨strength⟩}
\tikzset{
	rounded tree edges/.style n args={3}{
	edge from parent path={
	let
		\n{direction}={#1},
		\n{looseness}={#2},
		\n{strength}={#3},
		\p1=(\tikzparentnode),
		\p2=(\tikzchildnode),
		\p3=(\n{direction}:1pt),
		\p4=(\x2 - \x1, \y2 - \y1),
		\n{dist}={veclen(\p4)},
		\p4=(\x4 / \n{dist}, \y4 / \n{dist}),
		\n{angle}={atan2(\y4, \x4)},
		\n{delta}={Mod(\n{angle} - \n{direction}, 360)},
		\n{delta}={\n{delta} > 180 ? \n{delta} - 360  : \n{delta}},
		\n{delta}={\n{delta} >  90 ?  180 - \n{delta} : \n{delta}},
		\n{delta}={\n{delta} < -90 ? -180 - \n{delta} : \n{delta}}
	in (\tikzparentnode) .. controls
		+(    \n{angle}+\n{strength}*\n{delta}:\n{looseness}*0.3915*\n{dist}) and
		+(180+\n{angle}-\n{strength}*\n{delta}:\n{looseness}*0.3915*\n{dist}) ..
		(\tikzchildnode)
	}
	}
}


% Tear out snippets from PDFs.
% Usage: \tear[…]{file.pdf}
% The optional parameter is the same as for \includegraphics.
% Useful Arguments:
%   * page=‹pagenumber›
%   * trim=‹left› ‹bottom› ‹right› ‹top›
%   * width=0.98\linewidth
\newcommand*{\tear}[2][]{%
	\begin{tikzpicture}
		\node
			[ blur shadow
			, clip
			, decorate
			, decoration=random steps
			, draw
			, inner sep=0
			, preaction={fill=white}% hide the shadow if paper is transparent
			] {\includegraphics[#1]{#2}};
	\end{tikzpicture}%
}


\makeatletter
\newcommand*{\timeline}[3][0]{%
	\ifcsname timeline@cmd@#3\endcsname%
		\@timeline[#1]{#2}{#3}%
		\PackageWarning{timeline}{redefining timeline \@backslashchar\string#3}%
	\else%
		\ifcsname#3\endcsname%
			\errmessage{Command \@backslashchar\string#3 already defined}%
		\else%
			\@timeline[#1]{#2}{#3}%
		\fi%
	\fi%
}%
\newcommand*{\@timeline}[3][0]{%
	% mark command as timeline command – they can be overwritten
	\expandafter\def\csname timeline@cmd@#3\endcsname{}%
	\setcounter{@timeline}{#1}%
	\def\timeline@cmd{#3}%
	\timeline@reset%
	\timeline@append{0}%
	\@tfor\timeline@next:=#2\do{%
		\if\timeline@next+%
			\stepcounter{@timeline}%
			\timeline@append{,\the@timeline}%
		\else\if\timeline@next-%
			\stepcounter{@timeline}%
		\else%
			%\timeline@append{\timeline@next}%
			\GenericError{}{\protect\timeline: ignoring unknown character: \timeline@next}%
		\fi\fi%
	}%
}%
% \newcommand*{\tl}[1]{%
% 	\ifcsname timeline@cmd@#1\endcsname%
% 		\csname timeline@cmd@#1\endcsname%
% 	\else%
% 		0%
% 		%\GenericError{}{\protect\tl: timeline not defined: #1}%
% 	\fi%
% }%
\newcounter{@timeline}%
\def\timeline@reset{%
	\expandafter\def\csname\timeline@cmd\endcsname{}%
}%
\def\timeline@append#1{%
	\expandafter\edef\csname\timeline@cmd\endcsname{%
		\csname\timeline@cmd\endcsname#1%
	}%
}%
\makeatother


\newcommand*{\xminus}[1]{%
	\mathrel{\tikz[baseline={([yshift=-0.25em]n.south)}, inner sep=0, outer sep=0.2em]{%
		\node (n) {\(\scriptstyle #1\)};
		\draw (n.south west) -- (n.south east);
	}}%
}
\newcommand*{\tikzrightarrow}[1]{%
	\mathrel{\tikz[baseline={([yshift=-0.25em]n.south)}, inner sep=0, outer sep=0.2em]{%
		\node (n) {\(\scriptstyle #1\)};
		\draw[->, > = Computer Modern Rightarrow, line width = 0.4pt] (n.south west) -- (n.south east);
	}}%
}


%%%%%%%%%%%%%%%%%%%%%%%%%%%%%%%%%%%%%%%%%%%%%%%%%%%%%%%%%%%%%%%%%%%%%%%%%%%%%%
% document specific commands
%%%%%%%%%%%%%%%%%%%%%%%%%%%%%%%%%%%%%%%%%%%%%%%%%%%%%%%%%%%%%%%%%%%%%%%%%%%%%%

\newcommand<>*{\mycite}[1]{\uncover#2{{\color{HKS57K100}[\cite{#1}]}}}


\newcommand{\statetree}[1]{
	\tikz
	[ anchor=base
	, baseline=(current bounding box.center)
	, level distance=2em
	, sibling distance=2em
	]{
		\matrix
		[ draw=nt
		, edge from parent/.style={draw=black}
		, inner sep=0
		, nodes={inner sep=0.2em, rounded corners=0}
		, rounded corners
		] {#1\\}
	}
}


\newcommand*{\mylargeleaf}[1]{{\LARGE\color{HKS41K70}#1}}

\definecolor{state s}{named}{HKS57K80}
\definecolor{state t}{named}{HKS41K70}
\newcommand*{\stateS}[1]{{\color{state s}#1}}
\newcommand*{\stateT}[1]{{\color{state t}#1}}

\tikzset{
	subtree/.style =
		{ fill=lightgray
		, inner sep=0.02em
		, isosceles triangle apex angle=60
		, shape=isosceles triangle
		, shape border rotate=90
		}
	, state/.style = {circle, draw, inner sep=0.1em}
	, trans/.style = {rectangle, draw}
}

\newcommand*{\srBool}{\mathbb{B}}
\newcommand*{\srProb}{ℙ}


%%%%%%%%%%%%%%%%%%%%%%%%%%%%%%%%%%%%%%%%%%%%%%%%%%%%%%%%%%%%%%%%%%%%%%%%%%%%%%
% commands for specific notations
%%%%%%%%%%%%%%%%%%%%%%%%%%%%%%%%%%%%%%%%%%%%%%%%%%%%%%%%%%%%%%%%%%%%%%%%%%%%%%

\DeclareMathOperator*{\argmax}{argmax}

\newcommand*{\cardinality}[1]{\lvert#1\rvert}
\newcommand*{\corpussize}[1]{\lvert#1\rvert}

\DeclareMathOperator{\crispOp}{crisp}
\newcommand*        {\crisp}[2][0]{\crispOp\delim{#1}({#2})}

\DeclareMathOperator{\lhsOp}{lhs}
\newcommand*{\lhs}[1]{\lhsOp(#1)}

\DeclareMathOperator{\lklhdOp}{L}
\newcommand*{\lklhd}[2]{\lklhdOp(#1 ∣ #2)}

\DeclareMathOperator{\mleOp}{mle}
\newcommand*{\mle}[2][]{%
	\ifthenelse{\isempty{#1}}{%
		\mleOp(#2)%
	}{%
		\mleOp_{#1}(#2)%
	}%
}

\DeclareMathOperator{\mrg}{merge}

% CVD: color vision deficiencies
\definecolor{CVD light red}   {HTML}{FF8080}
\definecolor{CVD light yellow}{HTML}{FFFF80}
\definecolor{CVD light green} {HTML}{40FFC0}

\definecolor{nt}{named}{HKS41K70}
\newcommand*{\nt}[1]{{\color{nt}#1}}

% set of all probability distributions over #1
\DeclareMathOperator{\pdsOp}{Pd}
\newcommand*{\pds}[1]{\pdsOp(#1)}

\DeclareMathOperator{\positionsOp}{pos}
\newcommand*{\positions}[1]{\positionsOp(#1)}

\DeclareMathOperator{\rankOp}{rk}
\newcommand*{\rank}[1]{\rankOp(#1)}

\DeclareMathOperator{\runsOp}{run}
\newcommand*{\runs}[2][]{%
	\ifthenelse%
		{\isempty{#1}}%
		{\runsOp(#2)}%
		{\runsOp_{#1}(#2)}%
}

\newcommand*{\semantics}[1]{⟦#1⟧}

\DeclareMathOperator{\splt}{split}

\newcommand*{\subtree}[2]{#1|_{#2}}

\DeclareMathOperator{\supportOp}{supp}
\newcommand*{\support}[1]{\supportOp(#1)}

\newcommand*{\symId}{\textsc{\color{gray}Id}}
\newcommand*{\symCons}{\textsc{\color{gray}Cons}}
\newcommand*{\symFlip}{\textsc{\color{gray}Flip}}
\newcommand*{\symNull}{\textsc{\color{gray}Null}}
\newcommand*{\symNullR}{\textsc{\color{gray}N\(\overline{\textsc{ull}}\)}}
\newcommand*{\symSnoc}{\textsc{\color{gray}Snoc}}

\newcommand*{\transWTA}[4][]{#3 \xrightarrow{#1} #2(#4)}

\DeclareMathOperator{\uniqueRunOp}{r}
\newcommand*{\uniqueRun}[2][]{%
	\ifthenelse%
		{\isempty{#1}}%
		{\uniqueRunOp^{#2}}%
		{\uniqueRunOp_{\!#1}^{#2}}%
}

\DeclareMathOperator{\treesOp}{T}
\newcommand*{\trees}[2][]{%
	\ifthenelse%
		{\isempty{#1}}%
		{\treesOp_{\!#2}}%
		{\treesOp_{\!#2}(#1)}%
}
\DeclareMathOperator{\treesUOp}{U}
\newcommand*{\treesU}[2][]{%
	\ifthenelse%
		{\isempty{#1}}%
		{\treesUOp_{#2}}%
		{\treesUOp_{#2}(#1)}%
}


%%%%%%%%%%%%%%%%%%%%%%%%%%%%%%%%%%%%%%%%%%%%%%%%%%%%%%%%%%%%%%%%%%%%%%%%%%%%%%
% metadata
%%%%%%%%%%%%%%%%%%%%%%%%%%%%%%%%%%%%%%%%%%%%%%%%%%%%%%%%%%%%%%%%%%%%%%%%%%%%%%

\ifstandalonebeamer\else
	\title[Defense of Dissertation]{A Formal View on Training of Weighted Tree Automata by Likelihood-Driven State Splitting and Merging}
	\subtitle{Defense of Dissertation}
\fi
\author{Toni Dietze}
\institute[TU Dresden]{%
	\href{https://www.orchid.inf.tu-dresden.de/index.en/}{Chair for Foundations of Programming}
\\	\href{https://tu-dresden.de/ing/informatik/thi}{Institute of Theoretical Computer Science}
\\	\href{https://tu-dresden.de/ing/informatik}{Faculty of Computer Science}
\\	\href{https://tu-dresden.de/}{Technische Universität Dresden}
\\	01062 Dresden, Germany
}
\date[2018-09-27]{September 27, 2018}

\begin{document}
%         1 2 3 4 5 6 7
\timeline{+ - - - - - -}{tlBPRef}
\timeline{- + + + + + +}{tlPseudocode}
\timeline{+ + - + + + +}{tlFlowchart}
\timeline{- - + - - - -}{tlTheorem}
\timeline{- - - - + + -}{tlSoutSplit}
\timeline{- - - - - + -}{tlSoutEM}
\begin{standaloneframe}{\jobname}
	\transfade<\tlSoutSplit,\tlSoutEM>%
	\begin{overprint}
	\onslide<\tlBPRef>
		\begin{block}{Berkeley Parser}
			\printfullcite{2006PetrovBarrettThibauxKlein}
		\end{block}
	\onslide<\tlPseudocode>
		\begin{description}
		\item[\textbf{Input:} \usebeamertemplate{itemize item}]
			corpus \(c\)
		\item[\usebeamertemplate{itemize item}]
			pta $ℳ_0$ such that $\lklhd{c}{⟦ℳ_0⟧} > 0$
		%\item[\usebeamertemplate{itemize item}]
		%	\(μ ∈ [0, 1]\) and \(ε ∈ {]0, 1[}\)
		\item[\textbf{Output:} \usebeamertemplate{itemize item}]
			sequence of pta\s
		\end{description}
		\begin{algorithmic}
			\For{\(i ← 1, 2, \dots\)}
				\State
					\sout<\tlSoutSplit>{\makebox[18.5em]{%
						\(ℳ'_1 ← \Call{split}{ℳ_{i-1}}\);%
						\hfill%
						\(ℳ'_2 ← \Call{EM}{ℳ'_1, c}\)%
					}}
				\State
					\makebox[18.5em]{%
						\(ℳ'_3 ← \Call{merge}{ℳ'_2, c}\);%
						\hfill%
						\sout<\tlSoutEM>{\(ℳ_i ← \Call{EM}{ℳ'_3, c}\)}%
					}
			\EndFor
		\end{algorithmic}
	\end{overprint}
	\begin{overprint}
	\onslide<\tlFlowchart>
		\centering
		\vspace{1.25em}
		\footnotesize
		\begin{tikzpicture}
		[ data/.style={align=center, draw, rounded corners}
		, func/.style={align=center, draw}
		]
		\matrix[column sep=3em, row sep=2em, ampersand replacement = \&]{
			\coordinate (init);
		\&
			\node (Mi)      [data                              ] {\strut \(ℳ_i\)};
		\&
			\node (split)   [func, invisible = {<\tlSoutSplit>}] {\strut \Call{split}{}};
		\&
			\node (M1)      [data, invisible = {<\tlSoutSplit>}] {\strut \(ℳ'_1\)};
		\\\&
			\node (EMmerge) [func, invisible = {<\tlSoutEM>}   ] {\strut \Call{EM}{}};
		\&
			\node (corpus)  [data                              ] {\strut corpus};
		\&
			\node (EMsplit) [func, invisible = {<\tlSoutSplit>}] {\strut \Call{EM}{}};
		\\\&
			\node (M3)      [data, invisible = {<\tlSoutEM>}   ] {\strut \(ℳ'_3\)};
		\&
			\node (merge)   [func                              ] {\strut \Call{merge}{}};
		\&
			\node (M2)      [data, invisible = {<\tlSoutSplit>}] {\strut \(ℳ'_2\)};
		\\};

		\begin{scope}[->, line cap = rect, rounded corners]
			\draw (init)    -- (Mi) node[midway, above] {\(i ≔ 0\)};
			\path (EMmerge) -- (Mi) node[midway, left ] {\(i ≔ i + 1\)};
			\draw (corpus) to[bend left=20] (merge);
			\alt<\tlSoutSplit>{%
				\draw[rounded corners] (Mi) -- (M1.center) |- (merge);
			}{%
				\draw (Mi)      -- (split);
				\draw (split)   -- (M1);
				\draw (M1)      -- (EMsplit);
				\draw (EMsplit) -- (M2);
				\draw (M2)      -- (merge);
				\draw (corpus) to[bend left=20] (EMsplit);
			}%
			\alt<\tlSoutEM>{%
				\draw[rounded corners] (merge) -| (Mi);
			}{%
				\draw (merge)   -- (M3);
				\draw (M3)      -- (EMmerge);
				\draw (EMmerge) -- (Mi);
				\draw (corpus) to[bend left=20] (EMmerge);
			}%
		\end{scope}
		\end{tikzpicture}
	\onslide<\tlTheorem>
		\begin{block}{Theorem \hfill [TD]}\it
			Let \(i ≥ 1\) be an iteration of the state splitting and merging algorithm.
			Under the assumptions that
			\begin{itemize}
			\item
				\(\Call{EM}{ℳ, c} = \mle[c]{ℳ}\) for every pta \(ℳ\), and
			\item
				\(\crisp{ℳ_{i-1}}\) is a sub-ta of \(\crisp{ℳ'_3}\) up to isomorphism,
			\end{itemize}
			we have that
			\setlength{\abovedisplayskip}     {0pt}%
			\setlength{\belowdisplayskip}     {0pt}%
			\setlength{\abovedisplayshortskip}{0pt}%
			\setlength{\belowdisplayshortskip}{0pt}%
			\[
				\lklhd{c}{⟦ℳ_{i-1}⟧}
				≤ \lklhd{c}{⟦ℳ_i⟧}
				\text{.}
			\]
		\end{block}
	\end{overprint}
\end{standaloneframe}
\end{document}

\end{frame}


\section{State Splitting and Merging Algorithm – Example}

\begin{frame}
	% SPDX-License-Identifier: CC-BY-4.0
% Copyright 2018 Toni Dietze
\documentclass[beamer]{standalone}
% SPDX-License-Identifier: CC-BY-4.0 OR MIT-0
% Copyright 2018 Toni Dietze
%
\usefonttheme{professionalfonts}

% LuaLaTeX specific packages
\usepackage{fontspec}
	\defaultfontfeatures{Ligatures=TeX}
\usepackage{polyglossia}
	\setdefaultlanguage{english}
\usepackage{amsmath}  % has to be loaded before unicode-math
\usepackage[math-style=ISO]{unicode-math}
	\setmathfont{Latin Modern Math}
% 	\setmathfont[range={\mathcal,\mathbfcal},StylisticSet=1]{xits-math.otf}
% 	\setmathfont[range={"029F5}]{XITS Math}  % ⧵
% 	\setmathfont[range={\mathscr,\mathbfscr},StylisticSet=1]{Latin Modern Math}  % make \mathscr use the correct font

\usepackage[noend]{algpseudocode}
	\algrenewcommand\algorithmicrequire{\textbf{Input:}}
	\algrenewcommand\algorithmicensure{\textbf{Output:}}
\usepackage[backend=biber, maxbibnames=42, maxcitenames=42, sorting=ynt, style=authoryear]{biblatex}
\usepackage{csquotes}
\usepackage{mathtools}
\usepackage{media9}
\usepackage{scalerel}
\usepackage{standalone}
\usepackage{tikz}
	\usetikzlibrary{arrows.meta}
	\usetikzlibrary{backgrounds}
	\usetikzlibrary{calc}
	\usetikzlibrary{decorations}
	\usetikzlibrary{decorations.pathmorphing}
	\usetikzlibrary{decorations.pathreplacing}
	\usetikzlibrary{fadings}
	\usetikzlibrary{fit}
	\usetikzlibrary{graphs}
	\usetikzlibrary{graphdrawing}
	\usetikzlibrary{intersections}
	\usetikzlibrary{positioning}
	\usetikzlibrary{quotes}
	\usetikzlibrary{shadows.blur}
	\usetikzlibrary{shapes.arrows}
	\usetikzlibrary{shapes.geometric}
	\usegdlibrary{trees}
\usepackage{xifthen}
\usepackage{xspace}

\usepackage{pgfplots}
	\pgfplotsset
		{ compat = 1.15
		, /pgf/number format/1000 sep = {\,}
		, /pgf/number format/assume math mode = true
		, every axis plot/.append style =
			{ mark options = {fill opacity = 0.25}
			}
		}
	\usepgfplotslibrary{groupplots}
\usepackage{pgfplotstable}

\hypersetup
	{ bookmarksopen
	, pdflang = en
	, unicode
	}


%%%%%%%%%%%%%%%%%%%%%%%%%%%%%%%%%%%%%%%%%%%%%%%%%%%%%%%%%%%%%%%%%%%%%%%%%%%%%%


% always show bad boxes
%\overfullrule=1em


%%%%%%%%%%%%%%%%%%%%%%%%%%%%%%%%%%%%%%%%%%%%%%%%%%%%%%%%%%%%%%%%%%%%%%%%%%%%%%
% biblatex
%%%%%%%%%%%%%%%%%%%%%%%%%%%%%%%%%%%%%%%%%%%%%%%%%%%%%%%%%%%%%%%%%%%%%%%%%%%%%%

\addbibresource{slides-dissertation-defense.bib}
% \renewcommand*{\finalnamedelim}{\addcomma\space}
% \setlength{\bibitemsep}{1em}
% 
\AtEveryBibitem{% Clean up the bibtex rather than editing it
 \clearlist{address}
 \clearfield{date}
 \clearfield{eprint}
 \clearfield{isbn}
 \clearfield{issn}
 \clearlist{language}
 \clearlist{location}
 \clearfield{month}
 \clearfield{series}
%  \clearfield{url}
%  \clearfield{doi}
 \clearfield{organization}

%  \ifentrytype{book}{}{% Remove stuff except for books
%   \clearfield{booktitle}
%   \clearfield{pages}
  \clearlist{publisher}
  \clearname{editor}
%  }
}
% do not print url if doi is present
% http://tex.stackexchange.com/questions/154864/biblatex-use-doi-only-if-there-is-no-url
\DeclareSourcemap{
	\maps[datatype=bibtex]{
		\map{
			\step[fieldsource=doi,final]
			\step[fieldset=url,null]
}	}	}
%
% remove qoutes around titles
\DeclareFieldFormat
	[article,inbook,incollection,inproceedings,patent,thesis,unpublished]
	{title}{#1\isdot}
% 
% \DeclareFieldFormat{url}{\mkbibacro{URL}\addcolon\addnbspace\url{#1}}
% 
% \DeclareNameAlias{sortname}{first-last}
% 
\renewbibmacro{in:}{\ifentrytype{article}{}{}}


%%%%%%%%%%%%%%%%%%%%%%%%%%%%%%%%%%%%%%%%%%%%%%%%%%%%%%%%%%%%%%%%%%%%%%%%%%%%%%
% beamer
%%%%%%%%%%%%%%%%%%%%%%%%%%%%%%%%%%%%%%%%%%%%%%%%%%%%%%%%%%%%%%%%%%%%%%%%%%%%%%

\useoutertheme{infolines}
\makeatletter
% based on
% /usr/share/texmf-dist/tex/latex/beamer/beamerouterthemeinfolines.sty
\setbeamertemplate{footline}
{%
	\leavevmode%
	\hbox{%
	\begin{beamercolorbox}[wd=.333333\paperwidth,ht=2.25ex,dp=1ex,center]{author in head/foot}%
		\usebeamerfont{author in head/foot}\insertshortauthor\expandafter\beamer@ifempty\expandafter{\beamer@shortinstitute}{}{~~(\insertshortinstitute)}
	\end{beamercolorbox}%
	\begin{beamercolorbox}[wd=.333333\paperwidth,ht=2.25ex,dp=1ex,center]{title in head/foot}%
		\usebeamerfont{title in head/foot}\insertshorttitle
	\end{beamercolorbox}%
	\begin{beamercolorbox}[wd=.333333\paperwidth,ht=2.25ex,dp=1ex,right]{date in head/foot}%
		\usebeamerfont{date in head/foot}%
		\hfill\insertshortdate\hfill\hfill%
		%\hspace*{2ex}%
		%\insertshortdate%
		%\hspace{0pt plus 1 filll}%
		%(\insertframenumber.\insertoverlaynumber{} / \insertmainframenumber)%
		%\hspace{0pt plus 1 filll}%
		\phantom{000}\llap{\insertpagenumber} / \insertpresentationendpage%
		\hspace*{2ex}%
	\end{beamercolorbox}}%
	\vskip0pt%
}
\makeatother
\useinnertheme{circles}
\beamertemplatenavigationsymbolsempty
\setbeamertemplate{bibliography item}{}
\setbeamertemplate{headline}[default]

\input{tudcolors.tex}
\setbeamercolor*{alerted text}{fg=HKS07K100}
\usecolortheme[named=HKS41K100]{structure}

\setbeamercolor*{palette primary}{use=structure,fg=white,bg=structure.fg}
\setbeamercolor*{palette secondary}{use=structure,fg=white,bg=structure.fg!80}
\setbeamercolor*{palette tertiary}{use=structure,fg=white,bg=structure.fg!60}
\setbeamercolor*{palette quaternary}{fg=white,bg=black}

\setbeamercolor*{sidebar}{use=structure,bg=structure.fg}

\setbeamercolor*{palette sidebar primary}{use=structure,fg=structure.fg!20}
\setbeamercolor*{palette sidebar secondary}{fg=white}
\setbeamercolor*{palette sidebar tertiary}{use=structure,fg=structure.fg!40}
\setbeamercolor*{palette sidebar quaternary}{fg=white}

\setbeamercolor*{titlelike}{parent=palette primary}

\setbeamercolor*{separation line}{}
\setbeamercolor*{fine separation line}{}

\setbeamercolor{block title}{use=structure,fg=white,bg=structure.fg}
\setbeamercolor{block title alerted}{use=alerted text,fg=white,bg=alerted text.fg!75!black}
\setbeamercolor{block title example}{use=example text,fg=white,bg=example text.fg!75!black}

\setbeamercolor{block body}{parent=normal text,use=block title,bg=block title.bg!10!bg}
\setbeamercolor{block body alerted}{parent=normal text,use=block title alerted,bg=block title alerted.bg!10!bg}
\setbeamercolor{block body example}{parent=normal text,use=block title example,bg=block title example.bg!10!bg}

% \setbeamertemplate{itemize items}[default]


%%%%%%%%%%%%%%%%%%%%%%%%%%%%%%%%%%%%%%%%%%%%%%%%%%%%%%%%%%%%%%%%%%%%%%%%%%%%%%
% TikZ
%%%%%%%%%%%%%%%%%%%%%%%%%%%%%%%%%%%%%%%%%%%%%%%%%%%%%%%%%%%%%%%%%%%%%%%%%%%%%%

\tikzset
	{ > = Stealth
	}


%%%%%%%%%%%%%%%%%%%%%%%%%%%%%%%%%%%%%%%%%%%%%%%%%%%%%%%%%%%%%%%%%%%%%%%%%%%%%%
% general commands and styles
%%%%%%%%%%%%%%%%%%%%%%%%%%%%%%%%%%%%%%%%%%%%%%%%%%%%%%%%%%%%%%%%%%%%%%%%%%%%%%

% \delegateStyle and \inheritStyle command
% usage: \delegateStyle{… \inheritStyle{…} …}
% example: \(X_{\delegateStyle{\fbox{\inheritStyle{X}}}}\)
% Save the current style and regain it in the argument.
% This works both for math and text mode, and can be nested.
% Acknowledgments: Based on \ThisStyle and \SavedStyle from scalerel package.
\makeatletter
\newcommand*{\@inheritStyle@D}[1]{\(\displaystyle      #1\)}
\newcommand*{\@inheritStyle@T}[1]{\(\textstyle         #1\)}
\newcommand*{\@inheritStyle@S}[1]{\(\scriptstyle       #1\)}
\newcommand*{\@inheritStyle@s}[1]{\(\scriptscriptstyle #1\)}
\newcommand*{\@inheritStyle@t}[1]{#1}
\newcommand*{\inheritStyle}{\csname @inheritStyle@\@inheritStyleSwitch\endcsname}
\newcommand*{\delegateStyle}[1]{%
	\ifmmode%
		\mathchoice%
		{\edef\@inheritStyleSwitch{D}#1}%
		{\edef\@inheritStyleSwitch{T}#1}%
		{\edef\@inheritStyleSwitch{S}#1}%
		{\edef\@inheritStyleSwitch{s}#1}%
	\else%
		\edef\@inheritStyleSwitch{t}#1%
	\fi%
}
\makeatother


% \oalt command
% requires: \delegateStyle and \inheritStyle command
% usage: \oalt<…>[…]{…}{…} (cf. \alt)
% Behaves like \alt, but reserves space according to largest overlays.
% The optional argument defines the alignment inside the reserved space;
% it is one of c, l, r, s (cf. \makebox); the default is c.
\makeatletter
\newlength{\oalt@dp}
\newlength{\oalt@ht}
\newlength{\oalt@wd}
\newbox{\oalt@a}
\newbox{\oalt@b}
\newbox{\oalt@empty}
\newcommand<>*{\oalt}[3][c]{%
	\delegateStyle{%
		% based on \setto… in /usr/share/texmf-dist/tex/latex/base/latex.ltx
		\setbox\oalt@a\hbox{\inheritStyle{#2}}%
		\setbox\oalt@b\hbox{\inheritStyle{#3}}%
		\pgfmathsetlength{\oalt@dp}{max(\dp\oalt@a,\dp\oalt@b)}%
		\pgfmathsetlength{\oalt@ht}{max(\ht\oalt@a,\ht\oalt@b)}%
		\pgfmathsetlength{\oalt@wd}{max(\wd\oalt@a,\wd\oalt@b)}%
		\raisebox{0pt}[\oalt@ht][\oalt@dp]{%
			\makebox[\oalt@wd][#1]{%
				\alt#4{\unhbox\oalt@a}{\unhbox\oalt@b}%
			}%
		}%
		\setbox\oalt@a\box\oalt@empty%
		\setbox\oalt@b\box\oalt@empty%
	}%
}
\makeatother


% \otemporal command
% requires: \delegateStyle and \inheritStyle command
% usage: \otemporal<…>[…]{…}{…}{…} (cf. \temporal)
% Behaves like \temporal, but reserves space according to largest overlays.
% The optional argument defines the alignment inside the reserved space;
% it is one of c, l, r, s (cf. \makebox); the default is c.
\makeatletter
\newlength{\ot@dp}
\newlength{\ot@ht}
\newlength{\ot@wd}
\newbox{\ot@a}
\newbox{\ot@b}
\newbox{\ot@c}
\newbox{\ot@empty}
\newcommand<>*{\otemporal}[4][c]{%
	\delegateStyle{%
		% based on \setto… in /usr/share/texmf-dist/tex/latex/base/latex.ltx
		\setbox\ot@a\hbox{\inheritStyle{#2}}%
		\setbox\ot@b\hbox{\inheritStyle{#3}}%
		\setbox\ot@c\hbox{\inheritStyle{#4}}%
		\pgfmathsetlength{\ot@dp}{max(\dp\ot@a,\dp\ot@b,\dp\ot@c)}%
		\pgfmathsetlength{\ot@ht}{max(\ht\ot@a,\ht\ot@b,\ht\ot@c)}%
		\pgfmathsetlength{\ot@wd}{max(\wd\ot@a,\wd\ot@b,\wd\ot@c)}%
		\raisebox{0pt}[\ot@ht][\ot@dp]{%
			\makebox[\ot@wd][#1]{%
				\temporal#5{\unhbox\ot@a}{\unhbox\ot@b}{\unhbox\ot@c}%
			}%
		}%
		\setbox\ot@a\box\ot@empty%
		\setbox\ot@b\box\ot@empty%
		\setbox\ot@c\box\ot@empty%
	}%
}
\makeatother


% Resize delimiters like braces, brackets, etc.
% Parameters: size, left delimiter, formula, right delimiter
% Example: \delim2({\frac{1}{2}})
\newcommand*{\delim}[4]{%
	\ifcase#1%
		#2#3#4%
	\or%
		\bigl#2#3\bigr#4%
	\or%
		\Bigl#2#3\Bigr#4%
	\or%
		\biggl#2#3\biggr#4%
	\or%
		\Biggl#2#3\Biggr#4%
	\else%
		\left#2#3\right#4%
	\fi%
}


% similar to \fullcite, but using the formatting of \printbibliography
\newcommand*{\printfullcite}[1]{%
	\begin{refsection}%
		\nocite{#1}%
		\DeclareNameAlias{author}{first-last}%
		\printbibliography[heading = none]%
	\end{refsection}%
}


\colorlet{light alert}{HKS07K60}
\tikzset{alert.bg/.style={rounded corners, fill=light alert}}
\tikzset{every picture/.style={line cap=round, semithick}}
% http://tex.stackexchange.com/questions/6135/how-to-make-beamer-overlays-with-tikz-node
\tikzset{onslide/.code args={<#1>#2}{\only<#1>{\pgfkeysalso{#2}}}}
\tikzset{invisible/.code args={<#1>}{\alt<#1>{\pgfkeysalso{transparent}}{\pgfkeysalso{opaque}}}}
\tikzset{uncover/.code args={<#1>}{\alt<#1>{\pgfkeysalso{opaque}}{\pgfkeysalso{opacity=0.25}}}}
\tikzset{visible/.code args={<#1>}{\alt<#1>{\pgfkeysalso{opaque}}{\pgfkeysalso{transparent}}}}
\tikzset{vuncover/.code args=%
	{<#1><#2>}%
	{\alt<#1>%
		{\alt<#2>%
			{\pgfkeysalso{opaque}}%
			{\pgfkeysalso{opacity=0.25}}%
		}{\pgfkeysalso{transparent}}%
	}%
}

\newcommand<%
	>{\tikzhighlight}[2][]{%
	\delegateStyle{\alt#3%
		{\tikz[baseline=0, anchor=base, inner sep=0.2em, text height=, text depth=]{\node[alert.bg, #1]{\inheritStyle{#2}};}}%
		{\tikz[baseline=0, anchor=base, inner sep=0.2em, text height=, text depth=]{\node[#1, fill=none]{\inheritStyle{#2}};}}%
	}%
}

\newcommand{\mathhighlight}{\tikzhighlight}

\newcommand<>{\mhl}[2][]{\mathhighlight#3[inner sep=0.2em, #1]{#2}}


\newcommand<>{\inlineblock}[2][]{{%
	\usebeamercolor*[fg]{block body}%
	\tikzhighlight#3[fill=block body.bg, #1]{#2}%
}}


% a small letter s for plurals of abbreviations
\newcommand*{\s}{{\scriptsize s}\xspace}


\newcommand<>*{\sout}[2][opacity=0.75, ultra thick]{%
	\delegateStyle{%
		\tikz[baseline=0, anchor=base, inner sep=0, outer sep=0]{
			\useasboundingbox node (n) {\inheritStyle{#2}};
			\only#3{
				\node (h) {\inheritStyle{\ifmmode\mathstrut\else\strut\fi}};
				\draw[#1] (n.west |- {$(h.south)!0.5!(h.north)$}) -- (n.east |- {$(h.south)!0.5!(h.north)$});
			}
		}%
	}%
}


% tight style
% Sets outer sep to default inner sep and inner sep to 0.
% Use this style for nodes that are neither drawn nor filled to prevent
% unwanted growth of the bounding box.
\tikzset{tight/.style={inner sep=0, outer sep=0.3333em}}


% rounded tree edges style
% usage: rounded tree edges={⟨direction⟩}{⟨looseness⟩}{⟨strength⟩}
\tikzset{
	rounded tree edges/.style n args={3}{
	edge from parent path={
	let
		\n{direction}={#1},
		\n{looseness}={#2},
		\n{strength}={#3},
		\p1=(\tikzparentnode),
		\p2=(\tikzchildnode),
		\p3=(\n{direction}:1pt),
		\p4=(\x2 - \x1, \y2 - \y1),
		\n{dist}={veclen(\p4)},
		\p4=(\x4 / \n{dist}, \y4 / \n{dist}),
		\n{angle}={atan2(\y4, \x4)},
		\n{delta}={Mod(\n{angle} - \n{direction}, 360)},
		\n{delta}={\n{delta} > 180 ? \n{delta} - 360  : \n{delta}},
		\n{delta}={\n{delta} >  90 ?  180 - \n{delta} : \n{delta}},
		\n{delta}={\n{delta} < -90 ? -180 - \n{delta} : \n{delta}}
	in (\tikzparentnode) .. controls
		+(    \n{angle}+\n{strength}*\n{delta}:\n{looseness}*0.3915*\n{dist}) and
		+(180+\n{angle}-\n{strength}*\n{delta}:\n{looseness}*0.3915*\n{dist}) ..
		(\tikzchildnode)
	}
	}
}


% Tear out snippets from PDFs.
% Usage: \tear[…]{file.pdf}
% The optional parameter is the same as for \includegraphics.
% Useful Arguments:
%   * page=‹pagenumber›
%   * trim=‹left› ‹bottom› ‹right› ‹top›
%   * width=0.98\linewidth
\newcommand*{\tear}[2][]{%
	\begin{tikzpicture}
		\node
			[ blur shadow
			, clip
			, decorate
			, decoration=random steps
			, draw
			, inner sep=0
			, preaction={fill=white}% hide the shadow if paper is transparent
			] {\includegraphics[#1]{#2}};
	\end{tikzpicture}%
}


\makeatletter
\newcommand*{\timeline}[3][0]{%
	\ifcsname timeline@cmd@#3\endcsname%
		\@timeline[#1]{#2}{#3}%
		\PackageWarning{timeline}{redefining timeline \@backslashchar\string#3}%
	\else%
		\ifcsname#3\endcsname%
			\errmessage{Command \@backslashchar\string#3 already defined}%
		\else%
			\@timeline[#1]{#2}{#3}%
		\fi%
	\fi%
}%
\newcommand*{\@timeline}[3][0]{%
	% mark command as timeline command – they can be overwritten
	\expandafter\def\csname timeline@cmd@#3\endcsname{}%
	\setcounter{@timeline}{#1}%
	\def\timeline@cmd{#3}%
	\timeline@reset%
	\timeline@append{0}%
	\@tfor\timeline@next:=#2\do{%
		\if\timeline@next+%
			\stepcounter{@timeline}%
			\timeline@append{,\the@timeline}%
		\else\if\timeline@next-%
			\stepcounter{@timeline}%
		\else%
			%\timeline@append{\timeline@next}%
			\GenericError{}{\protect\timeline: ignoring unknown character: \timeline@next}%
		\fi\fi%
	}%
}%
% \newcommand*{\tl}[1]{%
% 	\ifcsname timeline@cmd@#1\endcsname%
% 		\csname timeline@cmd@#1\endcsname%
% 	\else%
% 		0%
% 		%\GenericError{}{\protect\tl: timeline not defined: #1}%
% 	\fi%
% }%
\newcounter{@timeline}%
\def\timeline@reset{%
	\expandafter\def\csname\timeline@cmd\endcsname{}%
}%
\def\timeline@append#1{%
	\expandafter\edef\csname\timeline@cmd\endcsname{%
		\csname\timeline@cmd\endcsname#1%
	}%
}%
\makeatother


\newcommand*{\xminus}[1]{%
	\mathrel{\tikz[baseline={([yshift=-0.25em]n.south)}, inner sep=0, outer sep=0.2em]{%
		\node (n) {\(\scriptstyle #1\)};
		\draw (n.south west) -- (n.south east);
	}}%
}
\newcommand*{\tikzrightarrow}[1]{%
	\mathrel{\tikz[baseline={([yshift=-0.25em]n.south)}, inner sep=0, outer sep=0.2em]{%
		\node (n) {\(\scriptstyle #1\)};
		\draw[->, > = Computer Modern Rightarrow, line width = 0.4pt] (n.south west) -- (n.south east);
	}}%
}


%%%%%%%%%%%%%%%%%%%%%%%%%%%%%%%%%%%%%%%%%%%%%%%%%%%%%%%%%%%%%%%%%%%%%%%%%%%%%%
% document specific commands
%%%%%%%%%%%%%%%%%%%%%%%%%%%%%%%%%%%%%%%%%%%%%%%%%%%%%%%%%%%%%%%%%%%%%%%%%%%%%%

\newcommand<>*{\mycite}[1]{\uncover#2{{\color{HKS57K100}[\cite{#1}]}}}


\newcommand{\statetree}[1]{
	\tikz
	[ anchor=base
	, baseline=(current bounding box.center)
	, level distance=2em
	, sibling distance=2em
	]{
		\matrix
		[ draw=nt
		, edge from parent/.style={draw=black}
		, inner sep=0
		, nodes={inner sep=0.2em, rounded corners=0}
		, rounded corners
		] {#1\\}
	}
}


\newcommand*{\mylargeleaf}[1]{{\LARGE\color{HKS41K70}#1}}

\definecolor{state s}{named}{HKS57K80}
\definecolor{state t}{named}{HKS41K70}
\newcommand*{\stateS}[1]{{\color{state s}#1}}
\newcommand*{\stateT}[1]{{\color{state t}#1}}

\tikzset{
	subtree/.style =
		{ fill=lightgray
		, inner sep=0.02em
		, isosceles triangle apex angle=60
		, shape=isosceles triangle
		, shape border rotate=90
		}
	, state/.style = {circle, draw, inner sep=0.1em}
	, trans/.style = {rectangle, draw}
}

\newcommand*{\srBool}{\mathbb{B}}
\newcommand*{\srProb}{ℙ}


%%%%%%%%%%%%%%%%%%%%%%%%%%%%%%%%%%%%%%%%%%%%%%%%%%%%%%%%%%%%%%%%%%%%%%%%%%%%%%
% commands for specific notations
%%%%%%%%%%%%%%%%%%%%%%%%%%%%%%%%%%%%%%%%%%%%%%%%%%%%%%%%%%%%%%%%%%%%%%%%%%%%%%

\DeclareMathOperator*{\argmax}{argmax}

\newcommand*{\cardinality}[1]{\lvert#1\rvert}
\newcommand*{\corpussize}[1]{\lvert#1\rvert}

\DeclareMathOperator{\crispOp}{crisp}
\newcommand*        {\crisp}[2][0]{\crispOp\delim{#1}({#2})}

\DeclareMathOperator{\lhsOp}{lhs}
\newcommand*{\lhs}[1]{\lhsOp(#1)}

\DeclareMathOperator{\lklhdOp}{L}
\newcommand*{\lklhd}[2]{\lklhdOp(#1 ∣ #2)}

\DeclareMathOperator{\mleOp}{mle}
\newcommand*{\mle}[2][]{%
	\ifthenelse{\isempty{#1}}{%
		\mleOp(#2)%
	}{%
		\mleOp_{#1}(#2)%
	}%
}

\DeclareMathOperator{\mrg}{merge}

% CVD: color vision deficiencies
\definecolor{CVD light red}   {HTML}{FF8080}
\definecolor{CVD light yellow}{HTML}{FFFF80}
\definecolor{CVD light green} {HTML}{40FFC0}

\definecolor{nt}{named}{HKS41K70}
\newcommand*{\nt}[1]{{\color{nt}#1}}

% set of all probability distributions over #1
\DeclareMathOperator{\pdsOp}{Pd}
\newcommand*{\pds}[1]{\pdsOp(#1)}

\DeclareMathOperator{\positionsOp}{pos}
\newcommand*{\positions}[1]{\positionsOp(#1)}

\DeclareMathOperator{\rankOp}{rk}
\newcommand*{\rank}[1]{\rankOp(#1)}

\DeclareMathOperator{\runsOp}{run}
\newcommand*{\runs}[2][]{%
	\ifthenelse%
		{\isempty{#1}}%
		{\runsOp(#2)}%
		{\runsOp_{#1}(#2)}%
}

\newcommand*{\semantics}[1]{⟦#1⟧}

\DeclareMathOperator{\splt}{split}

\newcommand*{\subtree}[2]{#1|_{#2}}

\DeclareMathOperator{\supportOp}{supp}
\newcommand*{\support}[1]{\supportOp(#1)}

\newcommand*{\symId}{\textsc{\color{gray}Id}}
\newcommand*{\symCons}{\textsc{\color{gray}Cons}}
\newcommand*{\symFlip}{\textsc{\color{gray}Flip}}
\newcommand*{\symNull}{\textsc{\color{gray}Null}}
\newcommand*{\symNullR}{\textsc{\color{gray}N\(\overline{\textsc{ull}}\)}}
\newcommand*{\symSnoc}{\textsc{\color{gray}Snoc}}

\newcommand*{\transWTA}[4][]{#3 \xrightarrow{#1} #2(#4)}

\DeclareMathOperator{\uniqueRunOp}{r}
\newcommand*{\uniqueRun}[2][]{%
	\ifthenelse%
		{\isempty{#1}}%
		{\uniqueRunOp^{#2}}%
		{\uniqueRunOp_{\!#1}^{#2}}%
}

\DeclareMathOperator{\treesOp}{T}
\newcommand*{\trees}[2][]{%
	\ifthenelse%
		{\isempty{#1}}%
		{\treesOp_{\!#2}}%
		{\treesOp_{\!#2}(#1)}%
}
\DeclareMathOperator{\treesUOp}{U}
\newcommand*{\treesU}[2][]{%
	\ifthenelse%
		{\isempty{#1}}%
		{\treesUOp_{#2}}%
		{\treesUOp_{#2}(#1)}%
}


%%%%%%%%%%%%%%%%%%%%%%%%%%%%%%%%%%%%%%%%%%%%%%%%%%%%%%%%%%%%%%%%%%%%%%%%%%%%%%
% metadata
%%%%%%%%%%%%%%%%%%%%%%%%%%%%%%%%%%%%%%%%%%%%%%%%%%%%%%%%%%%%%%%%%%%%%%%%%%%%%%

\ifstandalonebeamer\else
	\title[Defense of Dissertation]{A Formal View on Training of Weighted Tree Automata by Likelihood-Driven State Splitting and Merging}
	\subtitle{Defense of Dissertation}
\fi
\author{Toni Dietze}
\institute[TU Dresden]{%
	\href{https://www.orchid.inf.tu-dresden.de/index.en/}{Chair for Foundations of Programming}
\\	\href{https://tu-dresden.de/ing/informatik/thi}{Institute of Theoretical Computer Science}
\\	\href{https://tu-dresden.de/ing/informatik}{Faculty of Computer Science}
\\	\href{https://tu-dresden.de/}{Technische Universität Dresden}
\\	01062 Dresden, Germany
}
\date[2018-09-27]{September 27, 2018}

\begin{document}
\begin{standaloneframe}{\jobname}
\centering%
% \frametitle{\temporal<8-9>%
% 	{Splitting: $\Call{split}{(G_0, p_0)}$}%
% 	{EM Algorithm: $\Call{EM}{}(\Call{split}{(G_0, p_0)}, c)$}%
% 	{Merging: $\Call{merge}{(G', p')}$}%
% 	}%
\frametitle{\temporal<8-9>%
	{Splitting}%
	{EM Algorithm}%
	{Merging}%
	}%
\begin{columns}
\column{\linewidth}\centering
\begin{tikzpicture}
[ ampersand replacement=\&
, node distance=0.5em
, cm arrow/.style = {> = Computer Modern Rightarrow, line width = 0.4pt}
]
	\node[visible=<1-7>] (splt) {$\splt\colon \nt{S} \mapsto \{\nt{S}\},\ \nt{A} \mapsto \{\nt{A^1}, \nt{A^2}\},\ \nt{B} \mapsto \{\nt{B^1}, \nt{B^2}\}$};
	\matrix
	[ anchor=base
	, column sep=0.7em
	, row sep=0.1em
	, below=of splt
	] (m) {
		\node[visible=<1-7>, onslide=<3>alert.bg] (S->A-B) {$\nt{S} \tikzrightarrow{1} σ(\nt{A}, \nt{B})$};
	\&
		\node[visible=<1-7>, onslide=<2>alert.bg] {$\nt{A} \tikzrightarrow{1} α$};
	\&
		\node[visible=<1-7>, onslide=<4>alert.bg] {$\nt{B} \tikzrightarrow{0.5} β(\nt{B})$};
	\&
		\node[visible=<1-7>] {$\nt{B} \tikzrightarrow{0.5} β(\nt{A})$};
	\\
		\draw[visible=<3-7>, |->, cm arrow] (0, 0) -- ++(0, -3em) node[midway, sloped, above] {$\splt$};
	\&
		\draw[visible=<2-7>, |->, cm arrow] (0, 0) -- ++(0, -3em) node[midway, sloped, above] {$\splt$};
	\&
		\draw[visible=<4-7>, |->, cm arrow] (0, 0) -- ++(0, -3em) node[midway, sloped, above] {$\splt$};
	\&
		\draw[visible=<5-7>, |->, cm arrow] (0, 0) -- ++(0, -3em) node[midway, sloped, above] {$\splt$};
	\\ \uncover<3->{
		\node (S->A1-B1) {$\nt{S} \tikzrightarrow{\otemporal<7-8>{0.25}{0.23}{\otemporal<-13>{}{0.49}{1.00}}} σ(\nt{A^{\uncover<-13>{1}}}, \nt{B^1})$};
	} \& \uncover<2->{
		\node (A1->) {$\nt{A^{\uncover<-15>{1}}} \tikzrightarrow{1} α$};
	} \& \uncover<4->{
		\node[onslide=<9->{opacity=0.25}] (B1->B1) {$\nt{B^1} \tikzrightarrow{\otemporal<7-8>{0.25}{0.24}{0.00}} β(\nt{B^1})$};
	} \& \uncover<5->{
		\node (B1->A1) {$\nt{B^1} \tikzrightarrow{\otemporal<7-8>{0.25}{0.28}{\otemporal<-11>{}{0.33}{0.66}}} β(\nt{A^{\uncover<-11>{1}}})$};
	} \\ \uncover<3-13>{
		\node (S->A2-B1) {$\nt{S} \tikzrightarrow{\otemporal<7-8>{0.25}{0.26}{0.51}} σ(\nt{A^2}, \nt{B^1})$};
	} \& \uncover<2-15>{
		\node (A2->) {$\nt{A^2} \tikzrightarrow{1} α$};
	} \& \uncover<4->{
		\node (B1->B2) {$\nt{B^1} \tikzrightarrow{\otemporal<7-8>{0.25}{0.26}{0.33}} β(\nt{B^2})$};
	} \& \uncover<5-11>{
		\node (B1->A2) {$\nt{B^1} \tikzrightarrow{\otemporal<7-8>{0.25}{0.22}{0.33}} β(\nt{A^2})$};
	} \\ \uncover<3->{
		\node[onslide=<9->{opacity=0.25}] (S->A1-B2) {$\nt{S} \tikzrightarrow{\otemporal<7-8>{0.25}{0.24}{0.00}} σ(\nt{A^{\uncover<-14>{1}}}, \nt{B^2})$};
	} \&
	\& \uncover<4->{
		\node (B2->B1) {$\nt{B^2} \tikzrightarrow{\otemporal<7-8>{0.25}{0.27}{1.00}} β(\nt{B^1})$};
	} \& \uncover<5->{
		\node[onslide=<9->{opacity=0.25}] (B2->A1) {$\nt{B^2} \tikzrightarrow{\otemporal<7-8>{0.25}{0.24}{0.00}} β(\nt{A^{\uncover<-12>{1}}})$};
	} \\ \uncover<3-14>{
		\node[onslide=<9->{opacity=0.25}] (S->A2-B2) {$\nt{S} \tikzrightarrow{\otemporal<7-8>{0.25}{0.27}{0.00}} σ(\nt{A^2}, \nt{B^2})$};
	} \&
	\& \uncover<4->{
		\node[onslide=<9->{opacity=0.25}] (B2->B2) {$\nt{B^2} \tikzrightarrow{\otemporal<7-8>{0.25}{0.23}{0.00}} β(\nt{B^2})$};
	} \& \uncover<5-12>{
		\node[onslide=<9->{opacity=0.25}] (B2->A2) {$\nt{B^2} \tikzrightarrow{\otemporal<7-8>{0.25}{0.26}{0.00}} β(\nt{A^2})$};
	} \\
	};
	\begin{scope}[every node/.style={draw, color=gray, rounded corners, inner sep=-0.1em}]
		\node<3-7>[fit=(S->A1-B1) (S->A1-B2) (S->A2-B1) (S->A2-B2)] {};
		\node<2-7>[fit=(A1->)] {};
		\node<2-7>[fit=(A2->)] {};
		\node<4-7>[fit=(B1->B1) (B1->B2)] {};
		\node<4-7>[fit=(B2->B1) (B2->B2)] {};
		\node<5-7>[fit=(B1->A1) (B1->A2)] {};
		\node<5-7>[fit=(B2->A1) (B2->A2)] {};
	\end{scope}
	%\node[visible=<1-7>, base left=-0.5em of S->A-B] {$(G_0, p_0)\colon$};
	%\node[visible=<2-8>, above left=-0.2em and -1.5em of S->A1-B1] {$(G_\mathsf{s}, p_\mathsf{s})\colon$};
	%\node[visible=<9-11>, above left=-0.2em and -1.5em of S->A1-B1] {$(G', p')\colon$};
	%\node[visible=<16->, above left=-0.2em and -1.5em of S->A1-B1] {$(G_1, p_1)\colon$};
	\node[visible=<5-7>, below=of m] {$\mathhighlight<5>{\text{weights of trees unchanged}} \uncover<6->{\implies \mathhighlight<6>{\text{likelihood unchanged}}}$};
	% \lklhd(c \mid (G', p')) = \lklhd(c \mid (G, p))
	\node[visible=<8-9>, above=-1.5em of m, xshift=-0.5em] {Find weights that maximize the corpus’ likelihood.};
	\matrix[visible=<8-9>, below=of m, xshift=-0.5em, every node/.style={anchor=base, inner sep=0}] {
		\node {Likelihood of corpus before EM: $\approx 0.06$};
	\&
		\node[visible=<9-9>] {; after EM: $\approx 0.15$};
	\\
	};

	\only<10->{
		\matrix
		[ every node/.style={anchor=base west, text width=1.25em, align=left}
		, row sep=-0.85em
		, column sep=4em
		, text height=0.75em
		, text depth=0em
		, above=-5em of m
		] (m) {
			\node (sS) {$\nt{S}$};
		\&
			\node (mS) {$\nt{S}$};
		\\
			\node {};
		\\
			\node (sA1) {$\nt{A^1}$};
		\\\&
			\node (mA) {$\nt{A}$};
		\\
			\node (sA2) {$\nt{A^2}$};
		\\
			\node {};
		\\
			\node (sB1) {$\nt{B^1}$};
		\&
			\node (mB1) {$\nt{B^1}$};
		\\
			\node {};
		\\
			\node (sB2) {$\nt{B^2}$};
		\&
			\node (mB2) {$\nt{B^2}$};
		\\
		};
		\begin{scope}
		[ ->
		, cm arrow
		, onslide = <2->{|->}
		, out = 0
		, in = 180
		]
			\draw (sS) to node[visible=<2->, above=-0.2em, text height=0.75em, text depth=0.25em] {$\mrg$} (mS);
			\draw (sA1) to[in=165] (mA);
			\draw (sA2) to[in=195] (mA);
			\draw (sB1) to (mB1);
			\draw (sB2) to (mB2);
		\end{scope}
	}

	\only<11->{
		\begin{scope}[every node/.style={draw, color=gray, rounded corners, inner sep=-0.1em}, highlight/.style={ultra thick, color=light alert}]
			\node[fit=(S->A1-B1) (S->A2-B1), onslide=<13>highlight] {};
			\node[fit=(S->A1-B2) (S->A2-B2), onslide=<14>highlight] {};
			\node[fit=(A1->) (A2->), onslide=<15>highlight] {};
			\node[fit=(B1->B1)] {};
			\node[fit=(B1->B2)] {};
			\node[fit=(B2->B1)] {};
			\node[fit=(B2->B2)] {};
			\node[fit=(B1->A1) (B1->A2), onslide=<11>highlight] {};
			\node[fit=(B2->A1) (B2->A2), onslide=<12>highlight] {};
		\end{scope}
	}
\only<16>{};
% \fill[black, opacity=0.2] (current bounding box.north west) rectangle (current bounding box.south east);
\end{tikzpicture}
\end{columns}
\end{standaloneframe}
\end{document}

\end{frame}


\section{State Splitting and Merging Algorithm – Main Result}

\againframe<3>{frame:algorithm-split-merge}


\iffalse
\section{State Splitting and Merging Algorithm – Results}

\begin{frame}{\secname}
	% SPDX-License-Identifier: CC-BY-4.0
% Copyright 2018 Toni Dietze
\documentclass[beamer]{standalone}
% SPDX-License-Identifier: CC-BY-4.0 OR MIT-0
% Copyright 2018 Toni Dietze
%
\usefonttheme{professionalfonts}

% LuaLaTeX specific packages
\usepackage{fontspec}
	\defaultfontfeatures{Ligatures=TeX}
\usepackage{polyglossia}
	\setdefaultlanguage{english}
\usepackage{amsmath}  % has to be loaded before unicode-math
\usepackage[math-style=ISO]{unicode-math}
	\setmathfont{Latin Modern Math}
% 	\setmathfont[range={\mathcal,\mathbfcal},StylisticSet=1]{xits-math.otf}
% 	\setmathfont[range={"029F5}]{XITS Math}  % ⧵
% 	\setmathfont[range={\mathscr,\mathbfscr},StylisticSet=1]{Latin Modern Math}  % make \mathscr use the correct font

\usepackage[noend]{algpseudocode}
	\algrenewcommand\algorithmicrequire{\textbf{Input:}}
	\algrenewcommand\algorithmicensure{\textbf{Output:}}
\usepackage[backend=biber, maxbibnames=42, maxcitenames=42, sorting=ynt, style=authoryear]{biblatex}
\usepackage{csquotes}
\usepackage{mathtools}
\usepackage{media9}
\usepackage{scalerel}
\usepackage{standalone}
\usepackage{tikz}
	\usetikzlibrary{arrows.meta}
	\usetikzlibrary{backgrounds}
	\usetikzlibrary{calc}
	\usetikzlibrary{decorations}
	\usetikzlibrary{decorations.pathmorphing}
	\usetikzlibrary{decorations.pathreplacing}
	\usetikzlibrary{fadings}
	\usetikzlibrary{fit}
	\usetikzlibrary{graphs}
	\usetikzlibrary{graphdrawing}
	\usetikzlibrary{intersections}
	\usetikzlibrary{positioning}
	\usetikzlibrary{quotes}
	\usetikzlibrary{shadows.blur}
	\usetikzlibrary{shapes.arrows}
	\usetikzlibrary{shapes.geometric}
	\usegdlibrary{trees}
\usepackage{xifthen}
\usepackage{xspace}

\usepackage{pgfplots}
	\pgfplotsset
		{ compat = 1.15
		, /pgf/number format/1000 sep = {\,}
		, /pgf/number format/assume math mode = true
		, every axis plot/.append style =
			{ mark options = {fill opacity = 0.25}
			}
		}
	\usepgfplotslibrary{groupplots}
\usepackage{pgfplotstable}

\hypersetup
	{ bookmarksopen
	, pdflang = en
	, unicode
	}


%%%%%%%%%%%%%%%%%%%%%%%%%%%%%%%%%%%%%%%%%%%%%%%%%%%%%%%%%%%%%%%%%%%%%%%%%%%%%%


% always show bad boxes
%\overfullrule=1em


%%%%%%%%%%%%%%%%%%%%%%%%%%%%%%%%%%%%%%%%%%%%%%%%%%%%%%%%%%%%%%%%%%%%%%%%%%%%%%
% biblatex
%%%%%%%%%%%%%%%%%%%%%%%%%%%%%%%%%%%%%%%%%%%%%%%%%%%%%%%%%%%%%%%%%%%%%%%%%%%%%%

\addbibresource{slides-dissertation-defense.bib}
% \renewcommand*{\finalnamedelim}{\addcomma\space}
% \setlength{\bibitemsep}{1em}
% 
\AtEveryBibitem{% Clean up the bibtex rather than editing it
 \clearlist{address}
 \clearfield{date}
 \clearfield{eprint}
 \clearfield{isbn}
 \clearfield{issn}
 \clearlist{language}
 \clearlist{location}
 \clearfield{month}
 \clearfield{series}
%  \clearfield{url}
%  \clearfield{doi}
 \clearfield{organization}

%  \ifentrytype{book}{}{% Remove stuff except for books
%   \clearfield{booktitle}
%   \clearfield{pages}
  \clearlist{publisher}
  \clearname{editor}
%  }
}
% do not print url if doi is present
% http://tex.stackexchange.com/questions/154864/biblatex-use-doi-only-if-there-is-no-url
\DeclareSourcemap{
	\maps[datatype=bibtex]{
		\map{
			\step[fieldsource=doi,final]
			\step[fieldset=url,null]
}	}	}
%
% remove qoutes around titles
\DeclareFieldFormat
	[article,inbook,incollection,inproceedings,patent,thesis,unpublished]
	{title}{#1\isdot}
% 
% \DeclareFieldFormat{url}{\mkbibacro{URL}\addcolon\addnbspace\url{#1}}
% 
% \DeclareNameAlias{sortname}{first-last}
% 
\renewbibmacro{in:}{\ifentrytype{article}{}{}}


%%%%%%%%%%%%%%%%%%%%%%%%%%%%%%%%%%%%%%%%%%%%%%%%%%%%%%%%%%%%%%%%%%%%%%%%%%%%%%
% beamer
%%%%%%%%%%%%%%%%%%%%%%%%%%%%%%%%%%%%%%%%%%%%%%%%%%%%%%%%%%%%%%%%%%%%%%%%%%%%%%

\useoutertheme{infolines}
\makeatletter
% based on
% /usr/share/texmf-dist/tex/latex/beamer/beamerouterthemeinfolines.sty
\setbeamertemplate{footline}
{%
	\leavevmode%
	\hbox{%
	\begin{beamercolorbox}[wd=.333333\paperwidth,ht=2.25ex,dp=1ex,center]{author in head/foot}%
		\usebeamerfont{author in head/foot}\insertshortauthor\expandafter\beamer@ifempty\expandafter{\beamer@shortinstitute}{}{~~(\insertshortinstitute)}
	\end{beamercolorbox}%
	\begin{beamercolorbox}[wd=.333333\paperwidth,ht=2.25ex,dp=1ex,center]{title in head/foot}%
		\usebeamerfont{title in head/foot}\insertshorttitle
	\end{beamercolorbox}%
	\begin{beamercolorbox}[wd=.333333\paperwidth,ht=2.25ex,dp=1ex,right]{date in head/foot}%
		\usebeamerfont{date in head/foot}%
		\hfill\insertshortdate\hfill\hfill%
		%\hspace*{2ex}%
		%\insertshortdate%
		%\hspace{0pt plus 1 filll}%
		%(\insertframenumber.\insertoverlaynumber{} / \insertmainframenumber)%
		%\hspace{0pt plus 1 filll}%
		\phantom{000}\llap{\insertpagenumber} / \insertpresentationendpage%
		\hspace*{2ex}%
	\end{beamercolorbox}}%
	\vskip0pt%
}
\makeatother
\useinnertheme{circles}
\beamertemplatenavigationsymbolsempty
\setbeamertemplate{bibliography item}{}
\setbeamertemplate{headline}[default]

\input{tudcolors.tex}
\setbeamercolor*{alerted text}{fg=HKS07K100}
\usecolortheme[named=HKS41K100]{structure}

\setbeamercolor*{palette primary}{use=structure,fg=white,bg=structure.fg}
\setbeamercolor*{palette secondary}{use=structure,fg=white,bg=structure.fg!80}
\setbeamercolor*{palette tertiary}{use=structure,fg=white,bg=structure.fg!60}
\setbeamercolor*{palette quaternary}{fg=white,bg=black}

\setbeamercolor*{sidebar}{use=structure,bg=structure.fg}

\setbeamercolor*{palette sidebar primary}{use=structure,fg=structure.fg!20}
\setbeamercolor*{palette sidebar secondary}{fg=white}
\setbeamercolor*{palette sidebar tertiary}{use=structure,fg=structure.fg!40}
\setbeamercolor*{palette sidebar quaternary}{fg=white}

\setbeamercolor*{titlelike}{parent=palette primary}

\setbeamercolor*{separation line}{}
\setbeamercolor*{fine separation line}{}

\setbeamercolor{block title}{use=structure,fg=white,bg=structure.fg}
\setbeamercolor{block title alerted}{use=alerted text,fg=white,bg=alerted text.fg!75!black}
\setbeamercolor{block title example}{use=example text,fg=white,bg=example text.fg!75!black}

\setbeamercolor{block body}{parent=normal text,use=block title,bg=block title.bg!10!bg}
\setbeamercolor{block body alerted}{parent=normal text,use=block title alerted,bg=block title alerted.bg!10!bg}
\setbeamercolor{block body example}{parent=normal text,use=block title example,bg=block title example.bg!10!bg}

% \setbeamertemplate{itemize items}[default]


%%%%%%%%%%%%%%%%%%%%%%%%%%%%%%%%%%%%%%%%%%%%%%%%%%%%%%%%%%%%%%%%%%%%%%%%%%%%%%
% TikZ
%%%%%%%%%%%%%%%%%%%%%%%%%%%%%%%%%%%%%%%%%%%%%%%%%%%%%%%%%%%%%%%%%%%%%%%%%%%%%%

\tikzset
	{ > = Stealth
	}


%%%%%%%%%%%%%%%%%%%%%%%%%%%%%%%%%%%%%%%%%%%%%%%%%%%%%%%%%%%%%%%%%%%%%%%%%%%%%%
% general commands and styles
%%%%%%%%%%%%%%%%%%%%%%%%%%%%%%%%%%%%%%%%%%%%%%%%%%%%%%%%%%%%%%%%%%%%%%%%%%%%%%

% \delegateStyle and \inheritStyle command
% usage: \delegateStyle{… \inheritStyle{…} …}
% example: \(X_{\delegateStyle{\fbox{\inheritStyle{X}}}}\)
% Save the current style and regain it in the argument.
% This works both for math and text mode, and can be nested.
% Acknowledgments: Based on \ThisStyle and \SavedStyle from scalerel package.
\makeatletter
\newcommand*{\@inheritStyle@D}[1]{\(\displaystyle      #1\)}
\newcommand*{\@inheritStyle@T}[1]{\(\textstyle         #1\)}
\newcommand*{\@inheritStyle@S}[1]{\(\scriptstyle       #1\)}
\newcommand*{\@inheritStyle@s}[1]{\(\scriptscriptstyle #1\)}
\newcommand*{\@inheritStyle@t}[1]{#1}
\newcommand*{\inheritStyle}{\csname @inheritStyle@\@inheritStyleSwitch\endcsname}
\newcommand*{\delegateStyle}[1]{%
	\ifmmode%
		\mathchoice%
		{\edef\@inheritStyleSwitch{D}#1}%
		{\edef\@inheritStyleSwitch{T}#1}%
		{\edef\@inheritStyleSwitch{S}#1}%
		{\edef\@inheritStyleSwitch{s}#1}%
	\else%
		\edef\@inheritStyleSwitch{t}#1%
	\fi%
}
\makeatother


% \oalt command
% requires: \delegateStyle and \inheritStyle command
% usage: \oalt<…>[…]{…}{…} (cf. \alt)
% Behaves like \alt, but reserves space according to largest overlays.
% The optional argument defines the alignment inside the reserved space;
% it is one of c, l, r, s (cf. \makebox); the default is c.
\makeatletter
\newlength{\oalt@dp}
\newlength{\oalt@ht}
\newlength{\oalt@wd}
\newbox{\oalt@a}
\newbox{\oalt@b}
\newbox{\oalt@empty}
\newcommand<>*{\oalt}[3][c]{%
	\delegateStyle{%
		% based on \setto… in /usr/share/texmf-dist/tex/latex/base/latex.ltx
		\setbox\oalt@a\hbox{\inheritStyle{#2}}%
		\setbox\oalt@b\hbox{\inheritStyle{#3}}%
		\pgfmathsetlength{\oalt@dp}{max(\dp\oalt@a,\dp\oalt@b)}%
		\pgfmathsetlength{\oalt@ht}{max(\ht\oalt@a,\ht\oalt@b)}%
		\pgfmathsetlength{\oalt@wd}{max(\wd\oalt@a,\wd\oalt@b)}%
		\raisebox{0pt}[\oalt@ht][\oalt@dp]{%
			\makebox[\oalt@wd][#1]{%
				\alt#4{\unhbox\oalt@a}{\unhbox\oalt@b}%
			}%
		}%
		\setbox\oalt@a\box\oalt@empty%
		\setbox\oalt@b\box\oalt@empty%
	}%
}
\makeatother


% \otemporal command
% requires: \delegateStyle and \inheritStyle command
% usage: \otemporal<…>[…]{…}{…}{…} (cf. \temporal)
% Behaves like \temporal, but reserves space according to largest overlays.
% The optional argument defines the alignment inside the reserved space;
% it is one of c, l, r, s (cf. \makebox); the default is c.
\makeatletter
\newlength{\ot@dp}
\newlength{\ot@ht}
\newlength{\ot@wd}
\newbox{\ot@a}
\newbox{\ot@b}
\newbox{\ot@c}
\newbox{\ot@empty}
\newcommand<>*{\otemporal}[4][c]{%
	\delegateStyle{%
		% based on \setto… in /usr/share/texmf-dist/tex/latex/base/latex.ltx
		\setbox\ot@a\hbox{\inheritStyle{#2}}%
		\setbox\ot@b\hbox{\inheritStyle{#3}}%
		\setbox\ot@c\hbox{\inheritStyle{#4}}%
		\pgfmathsetlength{\ot@dp}{max(\dp\ot@a,\dp\ot@b,\dp\ot@c)}%
		\pgfmathsetlength{\ot@ht}{max(\ht\ot@a,\ht\ot@b,\ht\ot@c)}%
		\pgfmathsetlength{\ot@wd}{max(\wd\ot@a,\wd\ot@b,\wd\ot@c)}%
		\raisebox{0pt}[\ot@ht][\ot@dp]{%
			\makebox[\ot@wd][#1]{%
				\temporal#5{\unhbox\ot@a}{\unhbox\ot@b}{\unhbox\ot@c}%
			}%
		}%
		\setbox\ot@a\box\ot@empty%
		\setbox\ot@b\box\ot@empty%
		\setbox\ot@c\box\ot@empty%
	}%
}
\makeatother


% Resize delimiters like braces, brackets, etc.
% Parameters: size, left delimiter, formula, right delimiter
% Example: \delim2({\frac{1}{2}})
\newcommand*{\delim}[4]{%
	\ifcase#1%
		#2#3#4%
	\or%
		\bigl#2#3\bigr#4%
	\or%
		\Bigl#2#3\Bigr#4%
	\or%
		\biggl#2#3\biggr#4%
	\or%
		\Biggl#2#3\Biggr#4%
	\else%
		\left#2#3\right#4%
	\fi%
}


% similar to \fullcite, but using the formatting of \printbibliography
\newcommand*{\printfullcite}[1]{%
	\begin{refsection}%
		\nocite{#1}%
		\DeclareNameAlias{author}{first-last}%
		\printbibliography[heading = none]%
	\end{refsection}%
}


\colorlet{light alert}{HKS07K60}
\tikzset{alert.bg/.style={rounded corners, fill=light alert}}
\tikzset{every picture/.style={line cap=round, semithick}}
% http://tex.stackexchange.com/questions/6135/how-to-make-beamer-overlays-with-tikz-node
\tikzset{onslide/.code args={<#1>#2}{\only<#1>{\pgfkeysalso{#2}}}}
\tikzset{invisible/.code args={<#1>}{\alt<#1>{\pgfkeysalso{transparent}}{\pgfkeysalso{opaque}}}}
\tikzset{uncover/.code args={<#1>}{\alt<#1>{\pgfkeysalso{opaque}}{\pgfkeysalso{opacity=0.25}}}}
\tikzset{visible/.code args={<#1>}{\alt<#1>{\pgfkeysalso{opaque}}{\pgfkeysalso{transparent}}}}
\tikzset{vuncover/.code args=%
	{<#1><#2>}%
	{\alt<#1>%
		{\alt<#2>%
			{\pgfkeysalso{opaque}}%
			{\pgfkeysalso{opacity=0.25}}%
		}{\pgfkeysalso{transparent}}%
	}%
}

\newcommand<%
	>{\tikzhighlight}[2][]{%
	\delegateStyle{\alt#3%
		{\tikz[baseline=0, anchor=base, inner sep=0.2em, text height=, text depth=]{\node[alert.bg, #1]{\inheritStyle{#2}};}}%
		{\tikz[baseline=0, anchor=base, inner sep=0.2em, text height=, text depth=]{\node[#1, fill=none]{\inheritStyle{#2}};}}%
	}%
}

\newcommand{\mathhighlight}{\tikzhighlight}

\newcommand<>{\mhl}[2][]{\mathhighlight#3[inner sep=0.2em, #1]{#2}}


\newcommand<>{\inlineblock}[2][]{{%
	\usebeamercolor*[fg]{block body}%
	\tikzhighlight#3[fill=block body.bg, #1]{#2}%
}}


% a small letter s for plurals of abbreviations
\newcommand*{\s}{{\scriptsize s}\xspace}


\newcommand<>*{\sout}[2][opacity=0.75, ultra thick]{%
	\delegateStyle{%
		\tikz[baseline=0, anchor=base, inner sep=0, outer sep=0]{
			\useasboundingbox node (n) {\inheritStyle{#2}};
			\only#3{
				\node (h) {\inheritStyle{\ifmmode\mathstrut\else\strut\fi}};
				\draw[#1] (n.west |- {$(h.south)!0.5!(h.north)$}) -- (n.east |- {$(h.south)!0.5!(h.north)$});
			}
		}%
	}%
}


% tight style
% Sets outer sep to default inner sep and inner sep to 0.
% Use this style for nodes that are neither drawn nor filled to prevent
% unwanted growth of the bounding box.
\tikzset{tight/.style={inner sep=0, outer sep=0.3333em}}


% rounded tree edges style
% usage: rounded tree edges={⟨direction⟩}{⟨looseness⟩}{⟨strength⟩}
\tikzset{
	rounded tree edges/.style n args={3}{
	edge from parent path={
	let
		\n{direction}={#1},
		\n{looseness}={#2},
		\n{strength}={#3},
		\p1=(\tikzparentnode),
		\p2=(\tikzchildnode),
		\p3=(\n{direction}:1pt),
		\p4=(\x2 - \x1, \y2 - \y1),
		\n{dist}={veclen(\p4)},
		\p4=(\x4 / \n{dist}, \y4 / \n{dist}),
		\n{angle}={atan2(\y4, \x4)},
		\n{delta}={Mod(\n{angle} - \n{direction}, 360)},
		\n{delta}={\n{delta} > 180 ? \n{delta} - 360  : \n{delta}},
		\n{delta}={\n{delta} >  90 ?  180 - \n{delta} : \n{delta}},
		\n{delta}={\n{delta} < -90 ? -180 - \n{delta} : \n{delta}}
	in (\tikzparentnode) .. controls
		+(    \n{angle}+\n{strength}*\n{delta}:\n{looseness}*0.3915*\n{dist}) and
		+(180+\n{angle}-\n{strength}*\n{delta}:\n{looseness}*0.3915*\n{dist}) ..
		(\tikzchildnode)
	}
	}
}


% Tear out snippets from PDFs.
% Usage: \tear[…]{file.pdf}
% The optional parameter is the same as for \includegraphics.
% Useful Arguments:
%   * page=‹pagenumber›
%   * trim=‹left› ‹bottom› ‹right› ‹top›
%   * width=0.98\linewidth
\newcommand*{\tear}[2][]{%
	\begin{tikzpicture}
		\node
			[ blur shadow
			, clip
			, decorate
			, decoration=random steps
			, draw
			, inner sep=0
			, preaction={fill=white}% hide the shadow if paper is transparent
			] {\includegraphics[#1]{#2}};
	\end{tikzpicture}%
}


\makeatletter
\newcommand*{\timeline}[3][0]{%
	\ifcsname timeline@cmd@#3\endcsname%
		\@timeline[#1]{#2}{#3}%
		\PackageWarning{timeline}{redefining timeline \@backslashchar\string#3}%
	\else%
		\ifcsname#3\endcsname%
			\errmessage{Command \@backslashchar\string#3 already defined}%
		\else%
			\@timeline[#1]{#2}{#3}%
		\fi%
	\fi%
}%
\newcommand*{\@timeline}[3][0]{%
	% mark command as timeline command – they can be overwritten
	\expandafter\def\csname timeline@cmd@#3\endcsname{}%
	\setcounter{@timeline}{#1}%
	\def\timeline@cmd{#3}%
	\timeline@reset%
	\timeline@append{0}%
	\@tfor\timeline@next:=#2\do{%
		\if\timeline@next+%
			\stepcounter{@timeline}%
			\timeline@append{,\the@timeline}%
		\else\if\timeline@next-%
			\stepcounter{@timeline}%
		\else%
			%\timeline@append{\timeline@next}%
			\GenericError{}{\protect\timeline: ignoring unknown character: \timeline@next}%
		\fi\fi%
	}%
}%
% \newcommand*{\tl}[1]{%
% 	\ifcsname timeline@cmd@#1\endcsname%
% 		\csname timeline@cmd@#1\endcsname%
% 	\else%
% 		0%
% 		%\GenericError{}{\protect\tl: timeline not defined: #1}%
% 	\fi%
% }%
\newcounter{@timeline}%
\def\timeline@reset{%
	\expandafter\def\csname\timeline@cmd\endcsname{}%
}%
\def\timeline@append#1{%
	\expandafter\edef\csname\timeline@cmd\endcsname{%
		\csname\timeline@cmd\endcsname#1%
	}%
}%
\makeatother


\newcommand*{\xminus}[1]{%
	\mathrel{\tikz[baseline={([yshift=-0.25em]n.south)}, inner sep=0, outer sep=0.2em]{%
		\node (n) {\(\scriptstyle #1\)};
		\draw (n.south west) -- (n.south east);
	}}%
}
\newcommand*{\tikzrightarrow}[1]{%
	\mathrel{\tikz[baseline={([yshift=-0.25em]n.south)}, inner sep=0, outer sep=0.2em]{%
		\node (n) {\(\scriptstyle #1\)};
		\draw[->, > = Computer Modern Rightarrow, line width = 0.4pt] (n.south west) -- (n.south east);
	}}%
}


%%%%%%%%%%%%%%%%%%%%%%%%%%%%%%%%%%%%%%%%%%%%%%%%%%%%%%%%%%%%%%%%%%%%%%%%%%%%%%
% document specific commands
%%%%%%%%%%%%%%%%%%%%%%%%%%%%%%%%%%%%%%%%%%%%%%%%%%%%%%%%%%%%%%%%%%%%%%%%%%%%%%

\newcommand<>*{\mycite}[1]{\uncover#2{{\color{HKS57K100}[\cite{#1}]}}}


\newcommand{\statetree}[1]{
	\tikz
	[ anchor=base
	, baseline=(current bounding box.center)
	, level distance=2em
	, sibling distance=2em
	]{
		\matrix
		[ draw=nt
		, edge from parent/.style={draw=black}
		, inner sep=0
		, nodes={inner sep=0.2em, rounded corners=0}
		, rounded corners
		] {#1\\}
	}
}


\newcommand*{\mylargeleaf}[1]{{\LARGE\color{HKS41K70}#1}}

\definecolor{state s}{named}{HKS57K80}
\definecolor{state t}{named}{HKS41K70}
\newcommand*{\stateS}[1]{{\color{state s}#1}}
\newcommand*{\stateT}[1]{{\color{state t}#1}}

\tikzset{
	subtree/.style =
		{ fill=lightgray
		, inner sep=0.02em
		, isosceles triangle apex angle=60
		, shape=isosceles triangle
		, shape border rotate=90
		}
	, state/.style = {circle, draw, inner sep=0.1em}
	, trans/.style = {rectangle, draw}
}

\newcommand*{\srBool}{\mathbb{B}}
\newcommand*{\srProb}{ℙ}


%%%%%%%%%%%%%%%%%%%%%%%%%%%%%%%%%%%%%%%%%%%%%%%%%%%%%%%%%%%%%%%%%%%%%%%%%%%%%%
% commands for specific notations
%%%%%%%%%%%%%%%%%%%%%%%%%%%%%%%%%%%%%%%%%%%%%%%%%%%%%%%%%%%%%%%%%%%%%%%%%%%%%%

\DeclareMathOperator*{\argmax}{argmax}

\newcommand*{\cardinality}[1]{\lvert#1\rvert}
\newcommand*{\corpussize}[1]{\lvert#1\rvert}

\DeclareMathOperator{\crispOp}{crisp}
\newcommand*        {\crisp}[2][0]{\crispOp\delim{#1}({#2})}

\DeclareMathOperator{\lhsOp}{lhs}
\newcommand*{\lhs}[1]{\lhsOp(#1)}

\DeclareMathOperator{\lklhdOp}{L}
\newcommand*{\lklhd}[2]{\lklhdOp(#1 ∣ #2)}

\DeclareMathOperator{\mleOp}{mle}
\newcommand*{\mle}[2][]{%
	\ifthenelse{\isempty{#1}}{%
		\mleOp(#2)%
	}{%
		\mleOp_{#1}(#2)%
	}%
}

\DeclareMathOperator{\mrg}{merge}

% CVD: color vision deficiencies
\definecolor{CVD light red}   {HTML}{FF8080}
\definecolor{CVD light yellow}{HTML}{FFFF80}
\definecolor{CVD light green} {HTML}{40FFC0}

\definecolor{nt}{named}{HKS41K70}
\newcommand*{\nt}[1]{{\color{nt}#1}}

% set of all probability distributions over #1
\DeclareMathOperator{\pdsOp}{Pd}
\newcommand*{\pds}[1]{\pdsOp(#1)}

\DeclareMathOperator{\positionsOp}{pos}
\newcommand*{\positions}[1]{\positionsOp(#1)}

\DeclareMathOperator{\rankOp}{rk}
\newcommand*{\rank}[1]{\rankOp(#1)}

\DeclareMathOperator{\runsOp}{run}
\newcommand*{\runs}[2][]{%
	\ifthenelse%
		{\isempty{#1}}%
		{\runsOp(#2)}%
		{\runsOp_{#1}(#2)}%
}

\newcommand*{\semantics}[1]{⟦#1⟧}

\DeclareMathOperator{\splt}{split}

\newcommand*{\subtree}[2]{#1|_{#2}}

\DeclareMathOperator{\supportOp}{supp}
\newcommand*{\support}[1]{\supportOp(#1)}

\newcommand*{\symId}{\textsc{\color{gray}Id}}
\newcommand*{\symCons}{\textsc{\color{gray}Cons}}
\newcommand*{\symFlip}{\textsc{\color{gray}Flip}}
\newcommand*{\symNull}{\textsc{\color{gray}Null}}
\newcommand*{\symNullR}{\textsc{\color{gray}N\(\overline{\textsc{ull}}\)}}
\newcommand*{\symSnoc}{\textsc{\color{gray}Snoc}}

\newcommand*{\transWTA}[4][]{#3 \xrightarrow{#1} #2(#4)}

\DeclareMathOperator{\uniqueRunOp}{r}
\newcommand*{\uniqueRun}[2][]{%
	\ifthenelse%
		{\isempty{#1}}%
		{\uniqueRunOp^{#2}}%
		{\uniqueRunOp_{\!#1}^{#2}}%
}

\DeclareMathOperator{\treesOp}{T}
\newcommand*{\trees}[2][]{%
	\ifthenelse%
		{\isempty{#1}}%
		{\treesOp_{\!#2}}%
		{\treesOp_{\!#2}(#1)}%
}
\DeclareMathOperator{\treesUOp}{U}
\newcommand*{\treesU}[2][]{%
	\ifthenelse%
		{\isempty{#1}}%
		{\treesUOp_{#2}}%
		{\treesUOp_{#2}(#1)}%
}


%%%%%%%%%%%%%%%%%%%%%%%%%%%%%%%%%%%%%%%%%%%%%%%%%%%%%%%%%%%%%%%%%%%%%%%%%%%%%%
% metadata
%%%%%%%%%%%%%%%%%%%%%%%%%%%%%%%%%%%%%%%%%%%%%%%%%%%%%%%%%%%%%%%%%%%%%%%%%%%%%%

\ifstandalonebeamer\else
	\title[Defense of Dissertation]{A Formal View on Training of Weighted Tree Automata by Likelihood-Driven State Splitting and Merging}
	\subtitle{Defense of Dissertation}
\fi
\author{Toni Dietze}
\institute[TU Dresden]{%
	\href{https://www.orchid.inf.tu-dresden.de/index.en/}{Chair for Foundations of Programming}
\\	\href{https://tu-dresden.de/ing/informatik/thi}{Institute of Theoretical Computer Science}
\\	\href{https://tu-dresden.de/ing/informatik}{Faculty of Computer Science}
\\	\href{https://tu-dresden.de/}{Technische Universität Dresden}
\\	01062 Dresden, Germany
}
\date[2018-09-27]{September 27, 2018}

\title{\jobname}
\begin{document}
\begin{standaloneframe}{\jobname}
	\setlength{\topsep}{-\parskip}%
	\setlength{\partopsep}{0pt}%
	\begin{lemma}[{{\small proper splitting preserves weights of trees}}]\label{lemma:proper-split}
		\begin{tabbing}
			\textbf{let} \= $ℳ$, $ℳ'$ be semi-probabilistic wta over $\varSigma$,
		%\\\>
			$\pi$ be a $ℳ$-splitter;
		\\
			\textbf{if} $ℳ'$ is a proper $\pi$-split of $ℳ$, \textbf{then} $\forall t \in \trees{\varSigma}\colon ⟦ℳ⟧(t) = ⟦ℳ'⟧(t)$
		\end{tabbing}
	\end{lemma}
	\begin{lemma}[{{\small merging preserves semi-probabilistic}}]
		\begin{tabbing}
			\textbf{let} \= $ℳ'$ be a semi-probabilistic wta,
		\\\>
			$\pi$ be a $ℳ'$-merger,
		\\\>
			$\lambda$ be a $\pi$-distributor;
		\\
			\textbf{then} $\mrg_\pi^\lambda(ℳ')$ is semi-probabilistic
		\end{tabbing}
	\end{lemma}
	\begin{lemma}[{{\small composition of merges}}]
		\begin{tabbing}
			$\mrg_{\pi_2}^{\lambda_2}(\mrg_{\pi_1}^{\lambda_1}(ℳ)) = \mrg_{\pi}^\lambda(ℳ)$,
		\\
			\quad \textbf{where} \= $\pi = \pi_2 \circ \pi_1$,
		\\
			\> $\lambda(q) = \lambda_1(q) \cdot \lambda_2(\pi_1(q))$ for every $q \in Q$
		\end{tabbing}
	\end{lemma}
\end{standaloneframe}
\end{document}

\end{frame}
\fi


\iffalse
\section{Open Questions}

\begin{frame}{\secname}
	\setbeamercovered{transparent}
	\begin{itemize}[<+->]
	\item
		What happens to the corpus’ likelihood while merging?
	\item
		Is the EM training after merging really needed?
	%\item
	%	When do we stop the state splitting and merging algorithm? Is cross-validation the best we can do?
	\item
		How do we choose the free parameters ($μ$, $λ$)?
	\item
		Can we avoid merging by cleverly splitting only selected states?
	\end{itemize}
\end{frame}
\fi


\section{State Splitting and Merging Algorithm – Revisited}

\againframe<4-6>{frame:algorithm-split-merge}


\againframe<6>{toc}


\section{Count-Based State Merging Algorithm}

\begin{frame}[t]{\secname}{}
	\centering
	\documentclass[beamer]{standalone}
% SPDX-License-Identifier: CC-BY-4.0 OR MIT-0
% Copyright 2018 Toni Dietze
%
\usefonttheme{professionalfonts}

% LuaLaTeX specific packages
\usepackage{fontspec}
	\defaultfontfeatures{Ligatures=TeX}
\usepackage{polyglossia}
	\setdefaultlanguage{english}
\usepackage{amsmath}  % has to be loaded before unicode-math
\usepackage[math-style=ISO]{unicode-math}
	\setmathfont{Latin Modern Math}
% 	\setmathfont[range={\mathcal,\mathbfcal},StylisticSet=1]{xits-math.otf}
% 	\setmathfont[range={"029F5}]{XITS Math}  % ⧵
% 	\setmathfont[range={\mathscr,\mathbfscr},StylisticSet=1]{Latin Modern Math}  % make \mathscr use the correct font

\usepackage[noend]{algpseudocode}
	\algrenewcommand\algorithmicrequire{\textbf{Input:}}
	\algrenewcommand\algorithmicensure{\textbf{Output:}}
\usepackage[backend=biber, maxbibnames=42, maxcitenames=42, sorting=ynt, style=authoryear]{biblatex}
\usepackage{csquotes}
\usepackage{mathtools}
\usepackage{media9}
\usepackage{scalerel}
\usepackage{standalone}
\usepackage{tikz}
	\usetikzlibrary{arrows.meta}
	\usetikzlibrary{backgrounds}
	\usetikzlibrary{calc}
	\usetikzlibrary{decorations}
	\usetikzlibrary{decorations.pathmorphing}
	\usetikzlibrary{decorations.pathreplacing}
	\usetikzlibrary{fadings}
	\usetikzlibrary{fit}
	\usetikzlibrary{graphs}
	\usetikzlibrary{graphdrawing}
	\usetikzlibrary{intersections}
	\usetikzlibrary{positioning}
	\usetikzlibrary{quotes}
	\usetikzlibrary{shadows.blur}
	\usetikzlibrary{shapes.arrows}
	\usetikzlibrary{shapes.geometric}
	\usegdlibrary{trees}
\usepackage{xifthen}
\usepackage{xspace}

\usepackage{pgfplots}
	\pgfplotsset
		{ compat = 1.15
		, /pgf/number format/1000 sep = {\,}
		, /pgf/number format/assume math mode = true
		, every axis plot/.append style =
			{ mark options = {fill opacity = 0.25}
			}
		}
	\usepgfplotslibrary{groupplots}
\usepackage{pgfplotstable}

\hypersetup
	{ bookmarksopen
	, pdflang = en
	, unicode
	}


%%%%%%%%%%%%%%%%%%%%%%%%%%%%%%%%%%%%%%%%%%%%%%%%%%%%%%%%%%%%%%%%%%%%%%%%%%%%%%


% always show bad boxes
%\overfullrule=1em


%%%%%%%%%%%%%%%%%%%%%%%%%%%%%%%%%%%%%%%%%%%%%%%%%%%%%%%%%%%%%%%%%%%%%%%%%%%%%%
% biblatex
%%%%%%%%%%%%%%%%%%%%%%%%%%%%%%%%%%%%%%%%%%%%%%%%%%%%%%%%%%%%%%%%%%%%%%%%%%%%%%

\addbibresource{slides-dissertation-defense.bib}
% \renewcommand*{\finalnamedelim}{\addcomma\space}
% \setlength{\bibitemsep}{1em}
% 
\AtEveryBibitem{% Clean up the bibtex rather than editing it
 \clearlist{address}
 \clearfield{date}
 \clearfield{eprint}
 \clearfield{isbn}
 \clearfield{issn}
 \clearlist{language}
 \clearlist{location}
 \clearfield{month}
 \clearfield{series}
%  \clearfield{url}
%  \clearfield{doi}
 \clearfield{organization}

%  \ifentrytype{book}{}{% Remove stuff except for books
%   \clearfield{booktitle}
%   \clearfield{pages}
  \clearlist{publisher}
  \clearname{editor}
%  }
}
% do not print url if doi is present
% http://tex.stackexchange.com/questions/154864/biblatex-use-doi-only-if-there-is-no-url
\DeclareSourcemap{
	\maps[datatype=bibtex]{
		\map{
			\step[fieldsource=doi,final]
			\step[fieldset=url,null]
}	}	}
%
% remove qoutes around titles
\DeclareFieldFormat
	[article,inbook,incollection,inproceedings,patent,thesis,unpublished]
	{title}{#1\isdot}
% 
% \DeclareFieldFormat{url}{\mkbibacro{URL}\addcolon\addnbspace\url{#1}}
% 
% \DeclareNameAlias{sortname}{first-last}
% 
\renewbibmacro{in:}{\ifentrytype{article}{}{}}


%%%%%%%%%%%%%%%%%%%%%%%%%%%%%%%%%%%%%%%%%%%%%%%%%%%%%%%%%%%%%%%%%%%%%%%%%%%%%%
% beamer
%%%%%%%%%%%%%%%%%%%%%%%%%%%%%%%%%%%%%%%%%%%%%%%%%%%%%%%%%%%%%%%%%%%%%%%%%%%%%%

\useoutertheme{infolines}
\makeatletter
% based on
% /usr/share/texmf-dist/tex/latex/beamer/beamerouterthemeinfolines.sty
\setbeamertemplate{footline}
{%
	\leavevmode%
	\hbox{%
	\begin{beamercolorbox}[wd=.333333\paperwidth,ht=2.25ex,dp=1ex,center]{author in head/foot}%
		\usebeamerfont{author in head/foot}\insertshortauthor\expandafter\beamer@ifempty\expandafter{\beamer@shortinstitute}{}{~~(\insertshortinstitute)}
	\end{beamercolorbox}%
	\begin{beamercolorbox}[wd=.333333\paperwidth,ht=2.25ex,dp=1ex,center]{title in head/foot}%
		\usebeamerfont{title in head/foot}\insertshorttitle
	\end{beamercolorbox}%
	\begin{beamercolorbox}[wd=.333333\paperwidth,ht=2.25ex,dp=1ex,right]{date in head/foot}%
		\usebeamerfont{date in head/foot}%
		\hfill\insertshortdate\hfill\hfill%
		%\hspace*{2ex}%
		%\insertshortdate%
		%\hspace{0pt plus 1 filll}%
		%(\insertframenumber.\insertoverlaynumber{} / \insertmainframenumber)%
		%\hspace{0pt plus 1 filll}%
		\phantom{000}\llap{\insertpagenumber} / \insertpresentationendpage%
		\hspace*{2ex}%
	\end{beamercolorbox}}%
	\vskip0pt%
}
\makeatother
\useinnertheme{circles}
\beamertemplatenavigationsymbolsempty
\setbeamertemplate{bibliography item}{}
\setbeamertemplate{headline}[default]

\input{tudcolors.tex}
\setbeamercolor*{alerted text}{fg=HKS07K100}
\usecolortheme[named=HKS41K100]{structure}

\setbeamercolor*{palette primary}{use=structure,fg=white,bg=structure.fg}
\setbeamercolor*{palette secondary}{use=structure,fg=white,bg=structure.fg!80}
\setbeamercolor*{palette tertiary}{use=structure,fg=white,bg=structure.fg!60}
\setbeamercolor*{palette quaternary}{fg=white,bg=black}

\setbeamercolor*{sidebar}{use=structure,bg=structure.fg}

\setbeamercolor*{palette sidebar primary}{use=structure,fg=structure.fg!20}
\setbeamercolor*{palette sidebar secondary}{fg=white}
\setbeamercolor*{palette sidebar tertiary}{use=structure,fg=structure.fg!40}
\setbeamercolor*{palette sidebar quaternary}{fg=white}

\setbeamercolor*{titlelike}{parent=palette primary}

\setbeamercolor*{separation line}{}
\setbeamercolor*{fine separation line}{}

\setbeamercolor{block title}{use=structure,fg=white,bg=structure.fg}
\setbeamercolor{block title alerted}{use=alerted text,fg=white,bg=alerted text.fg!75!black}
\setbeamercolor{block title example}{use=example text,fg=white,bg=example text.fg!75!black}

\setbeamercolor{block body}{parent=normal text,use=block title,bg=block title.bg!10!bg}
\setbeamercolor{block body alerted}{parent=normal text,use=block title alerted,bg=block title alerted.bg!10!bg}
\setbeamercolor{block body example}{parent=normal text,use=block title example,bg=block title example.bg!10!bg}

% \setbeamertemplate{itemize items}[default]


%%%%%%%%%%%%%%%%%%%%%%%%%%%%%%%%%%%%%%%%%%%%%%%%%%%%%%%%%%%%%%%%%%%%%%%%%%%%%%
% TikZ
%%%%%%%%%%%%%%%%%%%%%%%%%%%%%%%%%%%%%%%%%%%%%%%%%%%%%%%%%%%%%%%%%%%%%%%%%%%%%%

\tikzset
	{ > = Stealth
	}


%%%%%%%%%%%%%%%%%%%%%%%%%%%%%%%%%%%%%%%%%%%%%%%%%%%%%%%%%%%%%%%%%%%%%%%%%%%%%%
% general commands and styles
%%%%%%%%%%%%%%%%%%%%%%%%%%%%%%%%%%%%%%%%%%%%%%%%%%%%%%%%%%%%%%%%%%%%%%%%%%%%%%

% \delegateStyle and \inheritStyle command
% usage: \delegateStyle{… \inheritStyle{…} …}
% example: \(X_{\delegateStyle{\fbox{\inheritStyle{X}}}}\)
% Save the current style and regain it in the argument.
% This works both for math and text mode, and can be nested.
% Acknowledgments: Based on \ThisStyle and \SavedStyle from scalerel package.
\makeatletter
\newcommand*{\@inheritStyle@D}[1]{\(\displaystyle      #1\)}
\newcommand*{\@inheritStyle@T}[1]{\(\textstyle         #1\)}
\newcommand*{\@inheritStyle@S}[1]{\(\scriptstyle       #1\)}
\newcommand*{\@inheritStyle@s}[1]{\(\scriptscriptstyle #1\)}
\newcommand*{\@inheritStyle@t}[1]{#1}
\newcommand*{\inheritStyle}{\csname @inheritStyle@\@inheritStyleSwitch\endcsname}
\newcommand*{\delegateStyle}[1]{%
	\ifmmode%
		\mathchoice%
		{\edef\@inheritStyleSwitch{D}#1}%
		{\edef\@inheritStyleSwitch{T}#1}%
		{\edef\@inheritStyleSwitch{S}#1}%
		{\edef\@inheritStyleSwitch{s}#1}%
	\else%
		\edef\@inheritStyleSwitch{t}#1%
	\fi%
}
\makeatother


% \oalt command
% requires: \delegateStyle and \inheritStyle command
% usage: \oalt<…>[…]{…}{…} (cf. \alt)
% Behaves like \alt, but reserves space according to largest overlays.
% The optional argument defines the alignment inside the reserved space;
% it is one of c, l, r, s (cf. \makebox); the default is c.
\makeatletter
\newlength{\oalt@dp}
\newlength{\oalt@ht}
\newlength{\oalt@wd}
\newbox{\oalt@a}
\newbox{\oalt@b}
\newbox{\oalt@empty}
\newcommand<>*{\oalt}[3][c]{%
	\delegateStyle{%
		% based on \setto… in /usr/share/texmf-dist/tex/latex/base/latex.ltx
		\setbox\oalt@a\hbox{\inheritStyle{#2}}%
		\setbox\oalt@b\hbox{\inheritStyle{#3}}%
		\pgfmathsetlength{\oalt@dp}{max(\dp\oalt@a,\dp\oalt@b)}%
		\pgfmathsetlength{\oalt@ht}{max(\ht\oalt@a,\ht\oalt@b)}%
		\pgfmathsetlength{\oalt@wd}{max(\wd\oalt@a,\wd\oalt@b)}%
		\raisebox{0pt}[\oalt@ht][\oalt@dp]{%
			\makebox[\oalt@wd][#1]{%
				\alt#4{\unhbox\oalt@a}{\unhbox\oalt@b}%
			}%
		}%
		\setbox\oalt@a\box\oalt@empty%
		\setbox\oalt@b\box\oalt@empty%
	}%
}
\makeatother


% \otemporal command
% requires: \delegateStyle and \inheritStyle command
% usage: \otemporal<…>[…]{…}{…}{…} (cf. \temporal)
% Behaves like \temporal, but reserves space according to largest overlays.
% The optional argument defines the alignment inside the reserved space;
% it is one of c, l, r, s (cf. \makebox); the default is c.
\makeatletter
\newlength{\ot@dp}
\newlength{\ot@ht}
\newlength{\ot@wd}
\newbox{\ot@a}
\newbox{\ot@b}
\newbox{\ot@c}
\newbox{\ot@empty}
\newcommand<>*{\otemporal}[4][c]{%
	\delegateStyle{%
		% based on \setto… in /usr/share/texmf-dist/tex/latex/base/latex.ltx
		\setbox\ot@a\hbox{\inheritStyle{#2}}%
		\setbox\ot@b\hbox{\inheritStyle{#3}}%
		\setbox\ot@c\hbox{\inheritStyle{#4}}%
		\pgfmathsetlength{\ot@dp}{max(\dp\ot@a,\dp\ot@b,\dp\ot@c)}%
		\pgfmathsetlength{\ot@ht}{max(\ht\ot@a,\ht\ot@b,\ht\ot@c)}%
		\pgfmathsetlength{\ot@wd}{max(\wd\ot@a,\wd\ot@b,\wd\ot@c)}%
		\raisebox{0pt}[\ot@ht][\ot@dp]{%
			\makebox[\ot@wd][#1]{%
				\temporal#5{\unhbox\ot@a}{\unhbox\ot@b}{\unhbox\ot@c}%
			}%
		}%
		\setbox\ot@a\box\ot@empty%
		\setbox\ot@b\box\ot@empty%
		\setbox\ot@c\box\ot@empty%
	}%
}
\makeatother


% Resize delimiters like braces, brackets, etc.
% Parameters: size, left delimiter, formula, right delimiter
% Example: \delim2({\frac{1}{2}})
\newcommand*{\delim}[4]{%
	\ifcase#1%
		#2#3#4%
	\or%
		\bigl#2#3\bigr#4%
	\or%
		\Bigl#2#3\Bigr#4%
	\or%
		\biggl#2#3\biggr#4%
	\or%
		\Biggl#2#3\Biggr#4%
	\else%
		\left#2#3\right#4%
	\fi%
}


% similar to \fullcite, but using the formatting of \printbibliography
\newcommand*{\printfullcite}[1]{%
	\begin{refsection}%
		\nocite{#1}%
		\DeclareNameAlias{author}{first-last}%
		\printbibliography[heading = none]%
	\end{refsection}%
}


\colorlet{light alert}{HKS07K60}
\tikzset{alert.bg/.style={rounded corners, fill=light alert}}
\tikzset{every picture/.style={line cap=round, semithick}}
% http://tex.stackexchange.com/questions/6135/how-to-make-beamer-overlays-with-tikz-node
\tikzset{onslide/.code args={<#1>#2}{\only<#1>{\pgfkeysalso{#2}}}}
\tikzset{invisible/.code args={<#1>}{\alt<#1>{\pgfkeysalso{transparent}}{\pgfkeysalso{opaque}}}}
\tikzset{uncover/.code args={<#1>}{\alt<#1>{\pgfkeysalso{opaque}}{\pgfkeysalso{opacity=0.25}}}}
\tikzset{visible/.code args={<#1>}{\alt<#1>{\pgfkeysalso{opaque}}{\pgfkeysalso{transparent}}}}
\tikzset{vuncover/.code args=%
	{<#1><#2>}%
	{\alt<#1>%
		{\alt<#2>%
			{\pgfkeysalso{opaque}}%
			{\pgfkeysalso{opacity=0.25}}%
		}{\pgfkeysalso{transparent}}%
	}%
}

\newcommand<%
	>{\tikzhighlight}[2][]{%
	\delegateStyle{\alt#3%
		{\tikz[baseline=0, anchor=base, inner sep=0.2em, text height=, text depth=]{\node[alert.bg, #1]{\inheritStyle{#2}};}}%
		{\tikz[baseline=0, anchor=base, inner sep=0.2em, text height=, text depth=]{\node[#1, fill=none]{\inheritStyle{#2}};}}%
	}%
}

\newcommand{\mathhighlight}{\tikzhighlight}

\newcommand<>{\mhl}[2][]{\mathhighlight#3[inner sep=0.2em, #1]{#2}}


\newcommand<>{\inlineblock}[2][]{{%
	\usebeamercolor*[fg]{block body}%
	\tikzhighlight#3[fill=block body.bg, #1]{#2}%
}}


% a small letter s for plurals of abbreviations
\newcommand*{\s}{{\scriptsize s}\xspace}


\newcommand<>*{\sout}[2][opacity=0.75, ultra thick]{%
	\delegateStyle{%
		\tikz[baseline=0, anchor=base, inner sep=0, outer sep=0]{
			\useasboundingbox node (n) {\inheritStyle{#2}};
			\only#3{
				\node (h) {\inheritStyle{\ifmmode\mathstrut\else\strut\fi}};
				\draw[#1] (n.west |- {$(h.south)!0.5!(h.north)$}) -- (n.east |- {$(h.south)!0.5!(h.north)$});
			}
		}%
	}%
}


% tight style
% Sets outer sep to default inner sep and inner sep to 0.
% Use this style for nodes that are neither drawn nor filled to prevent
% unwanted growth of the bounding box.
\tikzset{tight/.style={inner sep=0, outer sep=0.3333em}}


% rounded tree edges style
% usage: rounded tree edges={⟨direction⟩}{⟨looseness⟩}{⟨strength⟩}
\tikzset{
	rounded tree edges/.style n args={3}{
	edge from parent path={
	let
		\n{direction}={#1},
		\n{looseness}={#2},
		\n{strength}={#3},
		\p1=(\tikzparentnode),
		\p2=(\tikzchildnode),
		\p3=(\n{direction}:1pt),
		\p4=(\x2 - \x1, \y2 - \y1),
		\n{dist}={veclen(\p4)},
		\p4=(\x4 / \n{dist}, \y4 / \n{dist}),
		\n{angle}={atan2(\y4, \x4)},
		\n{delta}={Mod(\n{angle} - \n{direction}, 360)},
		\n{delta}={\n{delta} > 180 ? \n{delta} - 360  : \n{delta}},
		\n{delta}={\n{delta} >  90 ?  180 - \n{delta} : \n{delta}},
		\n{delta}={\n{delta} < -90 ? -180 - \n{delta} : \n{delta}}
	in (\tikzparentnode) .. controls
		+(    \n{angle}+\n{strength}*\n{delta}:\n{looseness}*0.3915*\n{dist}) and
		+(180+\n{angle}-\n{strength}*\n{delta}:\n{looseness}*0.3915*\n{dist}) ..
		(\tikzchildnode)
	}
	}
}


% Tear out snippets from PDFs.
% Usage: \tear[…]{file.pdf}
% The optional parameter is the same as for \includegraphics.
% Useful Arguments:
%   * page=‹pagenumber›
%   * trim=‹left› ‹bottom› ‹right› ‹top›
%   * width=0.98\linewidth
\newcommand*{\tear}[2][]{%
	\begin{tikzpicture}
		\node
			[ blur shadow
			, clip
			, decorate
			, decoration=random steps
			, draw
			, inner sep=0
			, preaction={fill=white}% hide the shadow if paper is transparent
			] {\includegraphics[#1]{#2}};
	\end{tikzpicture}%
}


\makeatletter
\newcommand*{\timeline}[3][0]{%
	\ifcsname timeline@cmd@#3\endcsname%
		\@timeline[#1]{#2}{#3}%
		\PackageWarning{timeline}{redefining timeline \@backslashchar\string#3}%
	\else%
		\ifcsname#3\endcsname%
			\errmessage{Command \@backslashchar\string#3 already defined}%
		\else%
			\@timeline[#1]{#2}{#3}%
		\fi%
	\fi%
}%
\newcommand*{\@timeline}[3][0]{%
	% mark command as timeline command – they can be overwritten
	\expandafter\def\csname timeline@cmd@#3\endcsname{}%
	\setcounter{@timeline}{#1}%
	\def\timeline@cmd{#3}%
	\timeline@reset%
	\timeline@append{0}%
	\@tfor\timeline@next:=#2\do{%
		\if\timeline@next+%
			\stepcounter{@timeline}%
			\timeline@append{,\the@timeline}%
		\else\if\timeline@next-%
			\stepcounter{@timeline}%
		\else%
			%\timeline@append{\timeline@next}%
			\GenericError{}{\protect\timeline: ignoring unknown character: \timeline@next}%
		\fi\fi%
	}%
}%
% \newcommand*{\tl}[1]{%
% 	\ifcsname timeline@cmd@#1\endcsname%
% 		\csname timeline@cmd@#1\endcsname%
% 	\else%
% 		0%
% 		%\GenericError{}{\protect\tl: timeline not defined: #1}%
% 	\fi%
% }%
\newcounter{@timeline}%
\def\timeline@reset{%
	\expandafter\def\csname\timeline@cmd\endcsname{}%
}%
\def\timeline@append#1{%
	\expandafter\edef\csname\timeline@cmd\endcsname{%
		\csname\timeline@cmd\endcsname#1%
	}%
}%
\makeatother


\newcommand*{\xminus}[1]{%
	\mathrel{\tikz[baseline={([yshift=-0.25em]n.south)}, inner sep=0, outer sep=0.2em]{%
		\node (n) {\(\scriptstyle #1\)};
		\draw (n.south west) -- (n.south east);
	}}%
}
\newcommand*{\tikzrightarrow}[1]{%
	\mathrel{\tikz[baseline={([yshift=-0.25em]n.south)}, inner sep=0, outer sep=0.2em]{%
		\node (n) {\(\scriptstyle #1\)};
		\draw[->, > = Computer Modern Rightarrow, line width = 0.4pt] (n.south west) -- (n.south east);
	}}%
}


%%%%%%%%%%%%%%%%%%%%%%%%%%%%%%%%%%%%%%%%%%%%%%%%%%%%%%%%%%%%%%%%%%%%%%%%%%%%%%
% document specific commands
%%%%%%%%%%%%%%%%%%%%%%%%%%%%%%%%%%%%%%%%%%%%%%%%%%%%%%%%%%%%%%%%%%%%%%%%%%%%%%

\newcommand<>*{\mycite}[1]{\uncover#2{{\color{HKS57K100}[\cite{#1}]}}}


\newcommand{\statetree}[1]{
	\tikz
	[ anchor=base
	, baseline=(current bounding box.center)
	, level distance=2em
	, sibling distance=2em
	]{
		\matrix
		[ draw=nt
		, edge from parent/.style={draw=black}
		, inner sep=0
		, nodes={inner sep=0.2em, rounded corners=0}
		, rounded corners
		] {#1\\}
	}
}


\newcommand*{\mylargeleaf}[1]{{\LARGE\color{HKS41K70}#1}}

\definecolor{state s}{named}{HKS57K80}
\definecolor{state t}{named}{HKS41K70}
\newcommand*{\stateS}[1]{{\color{state s}#1}}
\newcommand*{\stateT}[1]{{\color{state t}#1}}

\tikzset{
	subtree/.style =
		{ fill=lightgray
		, inner sep=0.02em
		, isosceles triangle apex angle=60
		, shape=isosceles triangle
		, shape border rotate=90
		}
	, state/.style = {circle, draw, inner sep=0.1em}
	, trans/.style = {rectangle, draw}
}

\newcommand*{\srBool}{\mathbb{B}}
\newcommand*{\srProb}{ℙ}


%%%%%%%%%%%%%%%%%%%%%%%%%%%%%%%%%%%%%%%%%%%%%%%%%%%%%%%%%%%%%%%%%%%%%%%%%%%%%%
% commands for specific notations
%%%%%%%%%%%%%%%%%%%%%%%%%%%%%%%%%%%%%%%%%%%%%%%%%%%%%%%%%%%%%%%%%%%%%%%%%%%%%%

\DeclareMathOperator*{\argmax}{argmax}

\newcommand*{\cardinality}[1]{\lvert#1\rvert}
\newcommand*{\corpussize}[1]{\lvert#1\rvert}

\DeclareMathOperator{\crispOp}{crisp}
\newcommand*        {\crisp}[2][0]{\crispOp\delim{#1}({#2})}

\DeclareMathOperator{\lhsOp}{lhs}
\newcommand*{\lhs}[1]{\lhsOp(#1)}

\DeclareMathOperator{\lklhdOp}{L}
\newcommand*{\lklhd}[2]{\lklhdOp(#1 ∣ #2)}

\DeclareMathOperator{\mleOp}{mle}
\newcommand*{\mle}[2][]{%
	\ifthenelse{\isempty{#1}}{%
		\mleOp(#2)%
	}{%
		\mleOp_{#1}(#2)%
	}%
}

\DeclareMathOperator{\mrg}{merge}

% CVD: color vision deficiencies
\definecolor{CVD light red}   {HTML}{FF8080}
\definecolor{CVD light yellow}{HTML}{FFFF80}
\definecolor{CVD light green} {HTML}{40FFC0}

\definecolor{nt}{named}{HKS41K70}
\newcommand*{\nt}[1]{{\color{nt}#1}}

% set of all probability distributions over #1
\DeclareMathOperator{\pdsOp}{Pd}
\newcommand*{\pds}[1]{\pdsOp(#1)}

\DeclareMathOperator{\positionsOp}{pos}
\newcommand*{\positions}[1]{\positionsOp(#1)}

\DeclareMathOperator{\rankOp}{rk}
\newcommand*{\rank}[1]{\rankOp(#1)}

\DeclareMathOperator{\runsOp}{run}
\newcommand*{\runs}[2][]{%
	\ifthenelse%
		{\isempty{#1}}%
		{\runsOp(#2)}%
		{\runsOp_{#1}(#2)}%
}

\newcommand*{\semantics}[1]{⟦#1⟧}

\DeclareMathOperator{\splt}{split}

\newcommand*{\subtree}[2]{#1|_{#2}}

\DeclareMathOperator{\supportOp}{supp}
\newcommand*{\support}[1]{\supportOp(#1)}

\newcommand*{\symId}{\textsc{\color{gray}Id}}
\newcommand*{\symCons}{\textsc{\color{gray}Cons}}
\newcommand*{\symFlip}{\textsc{\color{gray}Flip}}
\newcommand*{\symNull}{\textsc{\color{gray}Null}}
\newcommand*{\symNullR}{\textsc{\color{gray}N\(\overline{\textsc{ull}}\)}}
\newcommand*{\symSnoc}{\textsc{\color{gray}Snoc}}

\newcommand*{\transWTA}[4][]{#3 \xrightarrow{#1} #2(#4)}

\DeclareMathOperator{\uniqueRunOp}{r}
\newcommand*{\uniqueRun}[2][]{%
	\ifthenelse%
		{\isempty{#1}}%
		{\uniqueRunOp^{#2}}%
		{\uniqueRunOp_{\!#1}^{#2}}%
}

\DeclareMathOperator{\treesOp}{T}
\newcommand*{\trees}[2][]{%
	\ifthenelse%
		{\isempty{#1}}%
		{\treesOp_{\!#2}}%
		{\treesOp_{\!#2}(#1)}%
}
\DeclareMathOperator{\treesUOp}{U}
\newcommand*{\treesU}[2][]{%
	\ifthenelse%
		{\isempty{#1}}%
		{\treesUOp_{#2}}%
		{\treesUOp_{#2}(#1)}%
}


%%%%%%%%%%%%%%%%%%%%%%%%%%%%%%%%%%%%%%%%%%%%%%%%%%%%%%%%%%%%%%%%%%%%%%%%%%%%%%
% metadata
%%%%%%%%%%%%%%%%%%%%%%%%%%%%%%%%%%%%%%%%%%%%%%%%%%%%%%%%%%%%%%%%%%%%%%%%%%%%%%

\ifstandalonebeamer\else
	\title[Defense of Dissertation]{A Formal View on Training of Weighted Tree Automata by Likelihood-Driven State Splitting and Merging}
	\subtitle{Defense of Dissertation}
\fi
\author{Toni Dietze}
\institute[TU Dresden]{%
	\href{https://www.orchid.inf.tu-dresden.de/index.en/}{Chair for Foundations of Programming}
\\	\href{https://tu-dresden.de/ing/informatik/thi}{Institute of Theoretical Computer Science}
\\	\href{https://tu-dresden.de/ing/informatik}{Faculty of Computer Science}
\\	\href{https://tu-dresden.de/}{Technische Universität Dresden}
\\	01062 Dresden, Germany
}
\date[2018-09-27]{September 27, 2018}

\begin{document}
%         1 2 3 4 5 6 7 8 9 0 1 2
\timeline{- + - - - - -}{tlCanonical}
\timeline{- - + - - - -}{tlThCanonicalMerges}
\timeline{- - - + + + +}{tlAssA}
\timeline{- - - - + + +}{tlAssB}
\timeline{- - - - - + +}{tlAssC}
\timeline{- - + + - - -}{tlCanonicalMerges}
\timeline{- - - - - - +}{tlAlgorithm}
\begin{standaloneframe}[t]{\jobname}
	\centering%
	\begin{columns}[t]
	\column{0.25\linewidth}
		\textbf{\usebeamercolor*[fg]{description item}Input:} corpus \(c\)
	\column{0.75\linewidth}
		\hfill
		\llap{\textbf{\usebeamercolor*[fg]{description item}Output: }}%
		\(ℳ_n,\ ℳ_{n-1},\ …,\ ℳ_2,\ ℳ_1\)
		\hfill\null
		\[
			ℳ_i = \argmax_{ℳ ∈ \symbf{M\/odel}_i} \lklhd{c}{⟦ℳ⟧}
		\]
		\[
			\symbf{M\/odel}_i = \text{set of all pta\s with at most \(i\) states}
		\]
	\end{columns}
	\vfill
	\only<\tlCanonical>{%
		\begin{block}{canonical tree automaton}
			\vspace{0.25em}
			\begin{tikzpicture}
			[ anchor = base
			, level distance = 3em
			, node distance = 0.5em
			, sibling distance = 2em
			, tri/.style =
				{ inner sep = 0.12em
				, outer sep = 0
				, shape border rotate = 90
				, isosceles triangle
				, isosceles triangle apex angle = 60
				, draw = black
				}
			]
				\node (t) at (0, -4em) {\(σ\)}
					child { node[tri] {\(t_1\)} edge from parent[child anchor = apex] }
					child { node {\(\dots\)} edge from parent[draw = none] }
					child { node[tri] {\(t_k\)} edge from parent[child anchor = apex] };
				\draw (t-1.right corner) -- (t-3.left corner);
				\draw (t-1.left corner)
					-- ++(-1em, 0) coordinate (left)
					-- (0.5em, 0) coordinate (top)
					-- ($(t-3.right corner) + (2em, 0)$) coordinate (right)
					-- (t-3.right corner);
				\draw[decorate, decoration = {snake, amplitude = 0.05em, segment length = 1em}]
					(t) .. controls ++(0, 1em) and ++(0, -1em) .. (0.5em, 0);


				% draw subsequent tree triangles
				\coordinate (s) at (0, 0);
				\foreach \s in {1, 2, ..., 4}
					\draw
						coordinate (sOld) at (s)
						coordinate (s) at (\s*0.5em, \s*0.125em)
						(intersection cs: first line = {([shift = {(sOld)}]top) -- ([shift = {(sOld)}]right)}
										, second line = {([shift = {(s)}]top)    -- ([shift = {(s)}]left)})
						-- ([shift = {(s)}]top)
						-- ([shift = {(s)}]right)
						-- (intersection cs: first line = {([shift = {(sOld)}]top) -- ([shift = {(sOld)}]right)}
											, second line = {([shift = {(s)}]right)  -- ([shift = {(s)}]left)});


				\matrix[draw = nt, edge from parent/.style = {draw = black}, rounded corners, cells = {rounded corners = 0}] (lhs) at (11.9em, -2em) {
					\node (t) {\(σ\)}
						child { node[tri] {\(t_1\)} edge from parent[child anchor = apex]}
						child {node {\(\dots\)} edge from parent[draw = none]}
						child { node[tri] {\(t_k\)} edge from parent[child anchor = apex]};
				\\
				};
				\node[right = 0 of lhs, inner sep = 0] (arr) {\({} → σ\Big(\)};
				\matrix[base right = 0 of arr, draw = nt, edge from parent/.style = {draw = black}, rounded corners, cells = {rounded corners = 0}]
					(t1) {\node[tri] {\(t_1\)};\\};
				\node[base right = 0 of t1, inner sep = 0] (dots) {\({}, \dots,{}\)};
				\matrix[base right = 0 of dots, draw = nt, edge from parent/.style = {draw = black}, rounded corners, cells = {rounded corners = 0}]
					(tk) {\node[tri] {\(t_k\)};\\};
				\node[base right = 0 of tk, inner sep = 0] {\(\Big)\)};

				\node[left = of lhs] {\(⟹\)};
			\end{tikzpicture}
		\end{block}
	}%
	\only<\tlThCanonicalMerges,\tlAssA,\tlAssB,\tlAssC>{%
		\begin{overprint}
		\onslide<\tlThCanonicalMerges>
			\begin{block}{Theorem \hfill [TD]}
				\it
				Let \(C\) be a finite, non-empty set of trees, and \(\mathcal{A}_C\) the canonical ta of \(C\).

				For every \(C\)-restricted, bottom-up deterministic ta \(\mathcal{A}\) such that \(C ⊆ ⟦\mathcal{A}⟧\) there is an \(\mathcal{A}_C\)-merger \(π\) such that
				\setlength{\abovedisplayskip}     {0.5em}%
				\setlength{\belowdisplayskip}     {0pt}%
				\[
					\mathcal{A} = π(\mathcal{A}_C)
					\text{.}
				\]
			\end{block}
		\onslide<\tlAssA,\tlAssB,\tlAssC>
			\vspace{1em}
			\begin{description}[Assumption 3]
			\item<\tlAssA>[Assumption 1]
				The pta\s are bottom-up deterministic.
			\item<\tlAssB>[Assumption 2]
				We can merge greedily.
			\item<\tlAssC>[Assumption 3]
				Merging a pta that has optimal weights cannot lead to a better likelihood if the result is bottom-up deterministic.
			\end{description}
		\end{overprint}
		\vspace{5.25em}
		\hfill%
		\smash{\alt<\tlAlgorithm>{%
			\footnotesize% SPDX-License-Identifier: CC-BY-4.0
% Copyright 2018 Toni Dietze
\documentclass[tikz]{standalone}
\usepackage{luatex85}% workaround problems with standalone
\usepackage{amsmath}
\usepackage[colon=literal, math-style=ISO]
           {unicode-math}
\usetikzlibrary{positioning}
\DeclareMathOperator{\mleOp}{mle}
\begin{document}
\begin{tikzpicture}
[ data/.style={align=center, draw, rounded corners}
, func/.style={align=center, draw}
, node distance = 1.2em and 3em
]

\node (merge) [func] {\strut find and apply best merger};

\node (c)     [data, above = of merge.north east, anchor = south east] {corpus};
\node (mle)   [func, above = of c] {\strut \(\mleOp\)};
\node (Mi)    [data, right = of mle] {\strut \(ℳ_i\)};

\node (Ai)    [data, anchor = west] at (merge.west |- mle) {\strut \(\mathcal{A}_i\)};

\draw[<-] (Ai) -- ++(-5em, 0) node[midway, above] {\(i ≔ 0\)};
\begin{scope}[->, line cap = rect, rounded corners]
	\draw (Ai)  -- (mle);
	\draw (Ai)  -| (merge);
	\draw (mle) -- (Mi);

	\draw (Ai |- merge.north) -- (Ai) node[midway, left] {\(i ≔ i + 1\)};

	\draw (c) to[bend left=20] (mle);
	\draw (c) to[bend left=20] (c |- merge.north);
\end{scope}
\end{tikzpicture}
\end{document}

		}{%
			\begin{tikzpicture}[every node/.style = tight]
				\matrix[ampersand replacement = \&, column sep = 3em] {
					\node (n)    {\strut\(ℳ_n\)};
				\&	\node (n-1)  {\strut\(ℳ_{n-1}\)};
				\&	\node (dots) {\strut\quad\(…\)\quad\null};
				\&	\node (2)    {\strut\(ℳ_2\)};
				\&	\node (1)    {\strut\(ℳ_1\)};
				\\};
				\visible<\tlCanonicalMerges>{
					\draw[->] (n.north east) to[bend left = 15] (n-1.north west);
					\draw[->] (n.north east) to[bend left = 15] (2.north west);
					\draw[->] (n.north east) to[bend left = 15] node[above] {merge} (1.north west);
				}
				\visible<\tlAssB>{
					\draw[->] (n.north east)    to[bend left = 15] node[above] {merge} (n-1.north west);
					\draw[->] (n-1.north east)  to[bend left = 15] node[above] {merge} (dots.north west);
					\draw[->] (dots.north east) to[bend left = 15] node[above] {merge} (2.north west);
					\draw[->] (2.north east)    to[bend left = 15] node[above] {merge} (1.north west);
				}
			\end{tikzpicture}%
		}}%
		\hfill\null%
	}%
\end{standaloneframe}
\end{document}

\end{frame}


\subsection{Experiments}

\begin{frame}[t]{\secname{} – \subsecname}{}
	\centering
	\documentclass[beamer]{standalone}
% SPDX-License-Identifier: CC-BY-4.0 OR MIT-0
% Copyright 2018 Toni Dietze
%
\usefonttheme{professionalfonts}

% LuaLaTeX specific packages
\usepackage{fontspec}
	\defaultfontfeatures{Ligatures=TeX}
\usepackage{polyglossia}
	\setdefaultlanguage{english}
\usepackage{amsmath}  % has to be loaded before unicode-math
\usepackage[math-style=ISO]{unicode-math}
	\setmathfont{Latin Modern Math}
% 	\setmathfont[range={\mathcal,\mathbfcal},StylisticSet=1]{xits-math.otf}
% 	\setmathfont[range={"029F5}]{XITS Math}  % ⧵
% 	\setmathfont[range={\mathscr,\mathbfscr},StylisticSet=1]{Latin Modern Math}  % make \mathscr use the correct font

\usepackage[noend]{algpseudocode}
	\algrenewcommand\algorithmicrequire{\textbf{Input:}}
	\algrenewcommand\algorithmicensure{\textbf{Output:}}
\usepackage[backend=biber, maxbibnames=42, maxcitenames=42, sorting=ynt, style=authoryear]{biblatex}
\usepackage{csquotes}
\usepackage{mathtools}
\usepackage{media9}
\usepackage{scalerel}
\usepackage{standalone}
\usepackage{tikz}
	\usetikzlibrary{arrows.meta}
	\usetikzlibrary{backgrounds}
	\usetikzlibrary{calc}
	\usetikzlibrary{decorations}
	\usetikzlibrary{decorations.pathmorphing}
	\usetikzlibrary{decorations.pathreplacing}
	\usetikzlibrary{fadings}
	\usetikzlibrary{fit}
	\usetikzlibrary{graphs}
	\usetikzlibrary{graphdrawing}
	\usetikzlibrary{intersections}
	\usetikzlibrary{positioning}
	\usetikzlibrary{quotes}
	\usetikzlibrary{shadows.blur}
	\usetikzlibrary{shapes.arrows}
	\usetikzlibrary{shapes.geometric}
	\usegdlibrary{trees}
\usepackage{xifthen}
\usepackage{xspace}

\usepackage{pgfplots}
	\pgfplotsset
		{ compat = 1.15
		, /pgf/number format/1000 sep = {\,}
		, /pgf/number format/assume math mode = true
		, every axis plot/.append style =
			{ mark options = {fill opacity = 0.25}
			}
		}
	\usepgfplotslibrary{groupplots}
\usepackage{pgfplotstable}

\hypersetup
	{ bookmarksopen
	, pdflang = en
	, unicode
	}


%%%%%%%%%%%%%%%%%%%%%%%%%%%%%%%%%%%%%%%%%%%%%%%%%%%%%%%%%%%%%%%%%%%%%%%%%%%%%%


% always show bad boxes
%\overfullrule=1em


%%%%%%%%%%%%%%%%%%%%%%%%%%%%%%%%%%%%%%%%%%%%%%%%%%%%%%%%%%%%%%%%%%%%%%%%%%%%%%
% biblatex
%%%%%%%%%%%%%%%%%%%%%%%%%%%%%%%%%%%%%%%%%%%%%%%%%%%%%%%%%%%%%%%%%%%%%%%%%%%%%%

\addbibresource{slides-dissertation-defense.bib}
% \renewcommand*{\finalnamedelim}{\addcomma\space}
% \setlength{\bibitemsep}{1em}
% 
\AtEveryBibitem{% Clean up the bibtex rather than editing it
 \clearlist{address}
 \clearfield{date}
 \clearfield{eprint}
 \clearfield{isbn}
 \clearfield{issn}
 \clearlist{language}
 \clearlist{location}
 \clearfield{month}
 \clearfield{series}
%  \clearfield{url}
%  \clearfield{doi}
 \clearfield{organization}

%  \ifentrytype{book}{}{% Remove stuff except for books
%   \clearfield{booktitle}
%   \clearfield{pages}
  \clearlist{publisher}
  \clearname{editor}
%  }
}
% do not print url if doi is present
% http://tex.stackexchange.com/questions/154864/biblatex-use-doi-only-if-there-is-no-url
\DeclareSourcemap{
	\maps[datatype=bibtex]{
		\map{
			\step[fieldsource=doi,final]
			\step[fieldset=url,null]
}	}	}
%
% remove qoutes around titles
\DeclareFieldFormat
	[article,inbook,incollection,inproceedings,patent,thesis,unpublished]
	{title}{#1\isdot}
% 
% \DeclareFieldFormat{url}{\mkbibacro{URL}\addcolon\addnbspace\url{#1}}
% 
% \DeclareNameAlias{sortname}{first-last}
% 
\renewbibmacro{in:}{\ifentrytype{article}{}{}}


%%%%%%%%%%%%%%%%%%%%%%%%%%%%%%%%%%%%%%%%%%%%%%%%%%%%%%%%%%%%%%%%%%%%%%%%%%%%%%
% beamer
%%%%%%%%%%%%%%%%%%%%%%%%%%%%%%%%%%%%%%%%%%%%%%%%%%%%%%%%%%%%%%%%%%%%%%%%%%%%%%

\useoutertheme{infolines}
\makeatletter
% based on
% /usr/share/texmf-dist/tex/latex/beamer/beamerouterthemeinfolines.sty
\setbeamertemplate{footline}
{%
	\leavevmode%
	\hbox{%
	\begin{beamercolorbox}[wd=.333333\paperwidth,ht=2.25ex,dp=1ex,center]{author in head/foot}%
		\usebeamerfont{author in head/foot}\insertshortauthor\expandafter\beamer@ifempty\expandafter{\beamer@shortinstitute}{}{~~(\insertshortinstitute)}
	\end{beamercolorbox}%
	\begin{beamercolorbox}[wd=.333333\paperwidth,ht=2.25ex,dp=1ex,center]{title in head/foot}%
		\usebeamerfont{title in head/foot}\insertshorttitle
	\end{beamercolorbox}%
	\begin{beamercolorbox}[wd=.333333\paperwidth,ht=2.25ex,dp=1ex,right]{date in head/foot}%
		\usebeamerfont{date in head/foot}%
		\hfill\insertshortdate\hfill\hfill%
		%\hspace*{2ex}%
		%\insertshortdate%
		%\hspace{0pt plus 1 filll}%
		%(\insertframenumber.\insertoverlaynumber{} / \insertmainframenumber)%
		%\hspace{0pt plus 1 filll}%
		\phantom{000}\llap{\insertpagenumber} / \insertpresentationendpage%
		\hspace*{2ex}%
	\end{beamercolorbox}}%
	\vskip0pt%
}
\makeatother
\useinnertheme{circles}
\beamertemplatenavigationsymbolsempty
\setbeamertemplate{bibliography item}{}
\setbeamertemplate{headline}[default]

\input{tudcolors.tex}
\setbeamercolor*{alerted text}{fg=HKS07K100}
\usecolortheme[named=HKS41K100]{structure}

\setbeamercolor*{palette primary}{use=structure,fg=white,bg=structure.fg}
\setbeamercolor*{palette secondary}{use=structure,fg=white,bg=structure.fg!80}
\setbeamercolor*{palette tertiary}{use=structure,fg=white,bg=structure.fg!60}
\setbeamercolor*{palette quaternary}{fg=white,bg=black}

\setbeamercolor*{sidebar}{use=structure,bg=structure.fg}

\setbeamercolor*{palette sidebar primary}{use=structure,fg=structure.fg!20}
\setbeamercolor*{palette sidebar secondary}{fg=white}
\setbeamercolor*{palette sidebar tertiary}{use=structure,fg=structure.fg!40}
\setbeamercolor*{palette sidebar quaternary}{fg=white}

\setbeamercolor*{titlelike}{parent=palette primary}

\setbeamercolor*{separation line}{}
\setbeamercolor*{fine separation line}{}

\setbeamercolor{block title}{use=structure,fg=white,bg=structure.fg}
\setbeamercolor{block title alerted}{use=alerted text,fg=white,bg=alerted text.fg!75!black}
\setbeamercolor{block title example}{use=example text,fg=white,bg=example text.fg!75!black}

\setbeamercolor{block body}{parent=normal text,use=block title,bg=block title.bg!10!bg}
\setbeamercolor{block body alerted}{parent=normal text,use=block title alerted,bg=block title alerted.bg!10!bg}
\setbeamercolor{block body example}{parent=normal text,use=block title example,bg=block title example.bg!10!bg}

% \setbeamertemplate{itemize items}[default]


%%%%%%%%%%%%%%%%%%%%%%%%%%%%%%%%%%%%%%%%%%%%%%%%%%%%%%%%%%%%%%%%%%%%%%%%%%%%%%
% TikZ
%%%%%%%%%%%%%%%%%%%%%%%%%%%%%%%%%%%%%%%%%%%%%%%%%%%%%%%%%%%%%%%%%%%%%%%%%%%%%%

\tikzset
	{ > = Stealth
	}


%%%%%%%%%%%%%%%%%%%%%%%%%%%%%%%%%%%%%%%%%%%%%%%%%%%%%%%%%%%%%%%%%%%%%%%%%%%%%%
% general commands and styles
%%%%%%%%%%%%%%%%%%%%%%%%%%%%%%%%%%%%%%%%%%%%%%%%%%%%%%%%%%%%%%%%%%%%%%%%%%%%%%

% \delegateStyle and \inheritStyle command
% usage: \delegateStyle{… \inheritStyle{…} …}
% example: \(X_{\delegateStyle{\fbox{\inheritStyle{X}}}}\)
% Save the current style and regain it in the argument.
% This works both for math and text mode, and can be nested.
% Acknowledgments: Based on \ThisStyle and \SavedStyle from scalerel package.
\makeatletter
\newcommand*{\@inheritStyle@D}[1]{\(\displaystyle      #1\)}
\newcommand*{\@inheritStyle@T}[1]{\(\textstyle         #1\)}
\newcommand*{\@inheritStyle@S}[1]{\(\scriptstyle       #1\)}
\newcommand*{\@inheritStyle@s}[1]{\(\scriptscriptstyle #1\)}
\newcommand*{\@inheritStyle@t}[1]{#1}
\newcommand*{\inheritStyle}{\csname @inheritStyle@\@inheritStyleSwitch\endcsname}
\newcommand*{\delegateStyle}[1]{%
	\ifmmode%
		\mathchoice%
		{\edef\@inheritStyleSwitch{D}#1}%
		{\edef\@inheritStyleSwitch{T}#1}%
		{\edef\@inheritStyleSwitch{S}#1}%
		{\edef\@inheritStyleSwitch{s}#1}%
	\else%
		\edef\@inheritStyleSwitch{t}#1%
	\fi%
}
\makeatother


% \oalt command
% requires: \delegateStyle and \inheritStyle command
% usage: \oalt<…>[…]{…}{…} (cf. \alt)
% Behaves like \alt, but reserves space according to largest overlays.
% The optional argument defines the alignment inside the reserved space;
% it is one of c, l, r, s (cf. \makebox); the default is c.
\makeatletter
\newlength{\oalt@dp}
\newlength{\oalt@ht}
\newlength{\oalt@wd}
\newbox{\oalt@a}
\newbox{\oalt@b}
\newbox{\oalt@empty}
\newcommand<>*{\oalt}[3][c]{%
	\delegateStyle{%
		% based on \setto… in /usr/share/texmf-dist/tex/latex/base/latex.ltx
		\setbox\oalt@a\hbox{\inheritStyle{#2}}%
		\setbox\oalt@b\hbox{\inheritStyle{#3}}%
		\pgfmathsetlength{\oalt@dp}{max(\dp\oalt@a,\dp\oalt@b)}%
		\pgfmathsetlength{\oalt@ht}{max(\ht\oalt@a,\ht\oalt@b)}%
		\pgfmathsetlength{\oalt@wd}{max(\wd\oalt@a,\wd\oalt@b)}%
		\raisebox{0pt}[\oalt@ht][\oalt@dp]{%
			\makebox[\oalt@wd][#1]{%
				\alt#4{\unhbox\oalt@a}{\unhbox\oalt@b}%
			}%
		}%
		\setbox\oalt@a\box\oalt@empty%
		\setbox\oalt@b\box\oalt@empty%
	}%
}
\makeatother


% \otemporal command
% requires: \delegateStyle and \inheritStyle command
% usage: \otemporal<…>[…]{…}{…}{…} (cf. \temporal)
% Behaves like \temporal, but reserves space according to largest overlays.
% The optional argument defines the alignment inside the reserved space;
% it is one of c, l, r, s (cf. \makebox); the default is c.
\makeatletter
\newlength{\ot@dp}
\newlength{\ot@ht}
\newlength{\ot@wd}
\newbox{\ot@a}
\newbox{\ot@b}
\newbox{\ot@c}
\newbox{\ot@empty}
\newcommand<>*{\otemporal}[4][c]{%
	\delegateStyle{%
		% based on \setto… in /usr/share/texmf-dist/tex/latex/base/latex.ltx
		\setbox\ot@a\hbox{\inheritStyle{#2}}%
		\setbox\ot@b\hbox{\inheritStyle{#3}}%
		\setbox\ot@c\hbox{\inheritStyle{#4}}%
		\pgfmathsetlength{\ot@dp}{max(\dp\ot@a,\dp\ot@b,\dp\ot@c)}%
		\pgfmathsetlength{\ot@ht}{max(\ht\ot@a,\ht\ot@b,\ht\ot@c)}%
		\pgfmathsetlength{\ot@wd}{max(\wd\ot@a,\wd\ot@b,\wd\ot@c)}%
		\raisebox{0pt}[\ot@ht][\ot@dp]{%
			\makebox[\ot@wd][#1]{%
				\temporal#5{\unhbox\ot@a}{\unhbox\ot@b}{\unhbox\ot@c}%
			}%
		}%
		\setbox\ot@a\box\ot@empty%
		\setbox\ot@b\box\ot@empty%
		\setbox\ot@c\box\ot@empty%
	}%
}
\makeatother


% Resize delimiters like braces, brackets, etc.
% Parameters: size, left delimiter, formula, right delimiter
% Example: \delim2({\frac{1}{2}})
\newcommand*{\delim}[4]{%
	\ifcase#1%
		#2#3#4%
	\or%
		\bigl#2#3\bigr#4%
	\or%
		\Bigl#2#3\Bigr#4%
	\or%
		\biggl#2#3\biggr#4%
	\or%
		\Biggl#2#3\Biggr#4%
	\else%
		\left#2#3\right#4%
	\fi%
}


% similar to \fullcite, but using the formatting of \printbibliography
\newcommand*{\printfullcite}[1]{%
	\begin{refsection}%
		\nocite{#1}%
		\DeclareNameAlias{author}{first-last}%
		\printbibliography[heading = none]%
	\end{refsection}%
}


\colorlet{light alert}{HKS07K60}
\tikzset{alert.bg/.style={rounded corners, fill=light alert}}
\tikzset{every picture/.style={line cap=round, semithick}}
% http://tex.stackexchange.com/questions/6135/how-to-make-beamer-overlays-with-tikz-node
\tikzset{onslide/.code args={<#1>#2}{\only<#1>{\pgfkeysalso{#2}}}}
\tikzset{invisible/.code args={<#1>}{\alt<#1>{\pgfkeysalso{transparent}}{\pgfkeysalso{opaque}}}}
\tikzset{uncover/.code args={<#1>}{\alt<#1>{\pgfkeysalso{opaque}}{\pgfkeysalso{opacity=0.25}}}}
\tikzset{visible/.code args={<#1>}{\alt<#1>{\pgfkeysalso{opaque}}{\pgfkeysalso{transparent}}}}
\tikzset{vuncover/.code args=%
	{<#1><#2>}%
	{\alt<#1>%
		{\alt<#2>%
			{\pgfkeysalso{opaque}}%
			{\pgfkeysalso{opacity=0.25}}%
		}{\pgfkeysalso{transparent}}%
	}%
}

\newcommand<%
	>{\tikzhighlight}[2][]{%
	\delegateStyle{\alt#3%
		{\tikz[baseline=0, anchor=base, inner sep=0.2em, text height=, text depth=]{\node[alert.bg, #1]{\inheritStyle{#2}};}}%
		{\tikz[baseline=0, anchor=base, inner sep=0.2em, text height=, text depth=]{\node[#1, fill=none]{\inheritStyle{#2}};}}%
	}%
}

\newcommand{\mathhighlight}{\tikzhighlight}

\newcommand<>{\mhl}[2][]{\mathhighlight#3[inner sep=0.2em, #1]{#2}}


\newcommand<>{\inlineblock}[2][]{{%
	\usebeamercolor*[fg]{block body}%
	\tikzhighlight#3[fill=block body.bg, #1]{#2}%
}}


% a small letter s for plurals of abbreviations
\newcommand*{\s}{{\scriptsize s}\xspace}


\newcommand<>*{\sout}[2][opacity=0.75, ultra thick]{%
	\delegateStyle{%
		\tikz[baseline=0, anchor=base, inner sep=0, outer sep=0]{
			\useasboundingbox node (n) {\inheritStyle{#2}};
			\only#3{
				\node (h) {\inheritStyle{\ifmmode\mathstrut\else\strut\fi}};
				\draw[#1] (n.west |- {$(h.south)!0.5!(h.north)$}) -- (n.east |- {$(h.south)!0.5!(h.north)$});
			}
		}%
	}%
}


% tight style
% Sets outer sep to default inner sep and inner sep to 0.
% Use this style for nodes that are neither drawn nor filled to prevent
% unwanted growth of the bounding box.
\tikzset{tight/.style={inner sep=0, outer sep=0.3333em}}


% rounded tree edges style
% usage: rounded tree edges={⟨direction⟩}{⟨looseness⟩}{⟨strength⟩}
\tikzset{
	rounded tree edges/.style n args={3}{
	edge from parent path={
	let
		\n{direction}={#1},
		\n{looseness}={#2},
		\n{strength}={#3},
		\p1=(\tikzparentnode),
		\p2=(\tikzchildnode),
		\p3=(\n{direction}:1pt),
		\p4=(\x2 - \x1, \y2 - \y1),
		\n{dist}={veclen(\p4)},
		\p4=(\x4 / \n{dist}, \y4 / \n{dist}),
		\n{angle}={atan2(\y4, \x4)},
		\n{delta}={Mod(\n{angle} - \n{direction}, 360)},
		\n{delta}={\n{delta} > 180 ? \n{delta} - 360  : \n{delta}},
		\n{delta}={\n{delta} >  90 ?  180 - \n{delta} : \n{delta}},
		\n{delta}={\n{delta} < -90 ? -180 - \n{delta} : \n{delta}}
	in (\tikzparentnode) .. controls
		+(    \n{angle}+\n{strength}*\n{delta}:\n{looseness}*0.3915*\n{dist}) and
		+(180+\n{angle}-\n{strength}*\n{delta}:\n{looseness}*0.3915*\n{dist}) ..
		(\tikzchildnode)
	}
	}
}


% Tear out snippets from PDFs.
% Usage: \tear[…]{file.pdf}
% The optional parameter is the same as for \includegraphics.
% Useful Arguments:
%   * page=‹pagenumber›
%   * trim=‹left› ‹bottom› ‹right› ‹top›
%   * width=0.98\linewidth
\newcommand*{\tear}[2][]{%
	\begin{tikzpicture}
		\node
			[ blur shadow
			, clip
			, decorate
			, decoration=random steps
			, draw
			, inner sep=0
			, preaction={fill=white}% hide the shadow if paper is transparent
			] {\includegraphics[#1]{#2}};
	\end{tikzpicture}%
}


\makeatletter
\newcommand*{\timeline}[3][0]{%
	\ifcsname timeline@cmd@#3\endcsname%
		\@timeline[#1]{#2}{#3}%
		\PackageWarning{timeline}{redefining timeline \@backslashchar\string#3}%
	\else%
		\ifcsname#3\endcsname%
			\errmessage{Command \@backslashchar\string#3 already defined}%
		\else%
			\@timeline[#1]{#2}{#3}%
		\fi%
	\fi%
}%
\newcommand*{\@timeline}[3][0]{%
	% mark command as timeline command – they can be overwritten
	\expandafter\def\csname timeline@cmd@#3\endcsname{}%
	\setcounter{@timeline}{#1}%
	\def\timeline@cmd{#3}%
	\timeline@reset%
	\timeline@append{0}%
	\@tfor\timeline@next:=#2\do{%
		\if\timeline@next+%
			\stepcounter{@timeline}%
			\timeline@append{,\the@timeline}%
		\else\if\timeline@next-%
			\stepcounter{@timeline}%
		\else%
			%\timeline@append{\timeline@next}%
			\GenericError{}{\protect\timeline: ignoring unknown character: \timeline@next}%
		\fi\fi%
	}%
}%
% \newcommand*{\tl}[1]{%
% 	\ifcsname timeline@cmd@#1\endcsname%
% 		\csname timeline@cmd@#1\endcsname%
% 	\else%
% 		0%
% 		%\GenericError{}{\protect\tl: timeline not defined: #1}%
% 	\fi%
% }%
\newcounter{@timeline}%
\def\timeline@reset{%
	\expandafter\def\csname\timeline@cmd\endcsname{}%
}%
\def\timeline@append#1{%
	\expandafter\edef\csname\timeline@cmd\endcsname{%
		\csname\timeline@cmd\endcsname#1%
	}%
}%
\makeatother


\newcommand*{\xminus}[1]{%
	\mathrel{\tikz[baseline={([yshift=-0.25em]n.south)}, inner sep=0, outer sep=0.2em]{%
		\node (n) {\(\scriptstyle #1\)};
		\draw (n.south west) -- (n.south east);
	}}%
}
\newcommand*{\tikzrightarrow}[1]{%
	\mathrel{\tikz[baseline={([yshift=-0.25em]n.south)}, inner sep=0, outer sep=0.2em]{%
		\node (n) {\(\scriptstyle #1\)};
		\draw[->, > = Computer Modern Rightarrow, line width = 0.4pt] (n.south west) -- (n.south east);
	}}%
}


%%%%%%%%%%%%%%%%%%%%%%%%%%%%%%%%%%%%%%%%%%%%%%%%%%%%%%%%%%%%%%%%%%%%%%%%%%%%%%
% document specific commands
%%%%%%%%%%%%%%%%%%%%%%%%%%%%%%%%%%%%%%%%%%%%%%%%%%%%%%%%%%%%%%%%%%%%%%%%%%%%%%

\newcommand<>*{\mycite}[1]{\uncover#2{{\color{HKS57K100}[\cite{#1}]}}}


\newcommand{\statetree}[1]{
	\tikz
	[ anchor=base
	, baseline=(current bounding box.center)
	, level distance=2em
	, sibling distance=2em
	]{
		\matrix
		[ draw=nt
		, edge from parent/.style={draw=black}
		, inner sep=0
		, nodes={inner sep=0.2em, rounded corners=0}
		, rounded corners
		] {#1\\}
	}
}


\newcommand*{\mylargeleaf}[1]{{\LARGE\color{HKS41K70}#1}}

\definecolor{state s}{named}{HKS57K80}
\definecolor{state t}{named}{HKS41K70}
\newcommand*{\stateS}[1]{{\color{state s}#1}}
\newcommand*{\stateT}[1]{{\color{state t}#1}}

\tikzset{
	subtree/.style =
		{ fill=lightgray
		, inner sep=0.02em
		, isosceles triangle apex angle=60
		, shape=isosceles triangle
		, shape border rotate=90
		}
	, state/.style = {circle, draw, inner sep=0.1em}
	, trans/.style = {rectangle, draw}
}

\newcommand*{\srBool}{\mathbb{B}}
\newcommand*{\srProb}{ℙ}


%%%%%%%%%%%%%%%%%%%%%%%%%%%%%%%%%%%%%%%%%%%%%%%%%%%%%%%%%%%%%%%%%%%%%%%%%%%%%%
% commands for specific notations
%%%%%%%%%%%%%%%%%%%%%%%%%%%%%%%%%%%%%%%%%%%%%%%%%%%%%%%%%%%%%%%%%%%%%%%%%%%%%%

\DeclareMathOperator*{\argmax}{argmax}

\newcommand*{\cardinality}[1]{\lvert#1\rvert}
\newcommand*{\corpussize}[1]{\lvert#1\rvert}

\DeclareMathOperator{\crispOp}{crisp}
\newcommand*        {\crisp}[2][0]{\crispOp\delim{#1}({#2})}

\DeclareMathOperator{\lhsOp}{lhs}
\newcommand*{\lhs}[1]{\lhsOp(#1)}

\DeclareMathOperator{\lklhdOp}{L}
\newcommand*{\lklhd}[2]{\lklhdOp(#1 ∣ #2)}

\DeclareMathOperator{\mleOp}{mle}
\newcommand*{\mle}[2][]{%
	\ifthenelse{\isempty{#1}}{%
		\mleOp(#2)%
	}{%
		\mleOp_{#1}(#2)%
	}%
}

\DeclareMathOperator{\mrg}{merge}

% CVD: color vision deficiencies
\definecolor{CVD light red}   {HTML}{FF8080}
\definecolor{CVD light yellow}{HTML}{FFFF80}
\definecolor{CVD light green} {HTML}{40FFC0}

\definecolor{nt}{named}{HKS41K70}
\newcommand*{\nt}[1]{{\color{nt}#1}}

% set of all probability distributions over #1
\DeclareMathOperator{\pdsOp}{Pd}
\newcommand*{\pds}[1]{\pdsOp(#1)}

\DeclareMathOperator{\positionsOp}{pos}
\newcommand*{\positions}[1]{\positionsOp(#1)}

\DeclareMathOperator{\rankOp}{rk}
\newcommand*{\rank}[1]{\rankOp(#1)}

\DeclareMathOperator{\runsOp}{run}
\newcommand*{\runs}[2][]{%
	\ifthenelse%
		{\isempty{#1}}%
		{\runsOp(#2)}%
		{\runsOp_{#1}(#2)}%
}

\newcommand*{\semantics}[1]{⟦#1⟧}

\DeclareMathOperator{\splt}{split}

\newcommand*{\subtree}[2]{#1|_{#2}}

\DeclareMathOperator{\supportOp}{supp}
\newcommand*{\support}[1]{\supportOp(#1)}

\newcommand*{\symId}{\textsc{\color{gray}Id}}
\newcommand*{\symCons}{\textsc{\color{gray}Cons}}
\newcommand*{\symFlip}{\textsc{\color{gray}Flip}}
\newcommand*{\symNull}{\textsc{\color{gray}Null}}
\newcommand*{\symNullR}{\textsc{\color{gray}N\(\overline{\textsc{ull}}\)}}
\newcommand*{\symSnoc}{\textsc{\color{gray}Snoc}}

\newcommand*{\transWTA}[4][]{#3 \xrightarrow{#1} #2(#4)}

\DeclareMathOperator{\uniqueRunOp}{r}
\newcommand*{\uniqueRun}[2][]{%
	\ifthenelse%
		{\isempty{#1}}%
		{\uniqueRunOp^{#2}}%
		{\uniqueRunOp_{\!#1}^{#2}}%
}

\DeclareMathOperator{\treesOp}{T}
\newcommand*{\trees}[2][]{%
	\ifthenelse%
		{\isempty{#1}}%
		{\treesOp_{\!#2}}%
		{\treesOp_{\!#2}(#1)}%
}
\DeclareMathOperator{\treesUOp}{U}
\newcommand*{\treesU}[2][]{%
	\ifthenelse%
		{\isempty{#1}}%
		{\treesUOp_{#2}}%
		{\treesUOp_{#2}(#1)}%
}


%%%%%%%%%%%%%%%%%%%%%%%%%%%%%%%%%%%%%%%%%%%%%%%%%%%%%%%%%%%%%%%%%%%%%%%%%%%%%%
% metadata
%%%%%%%%%%%%%%%%%%%%%%%%%%%%%%%%%%%%%%%%%%%%%%%%%%%%%%%%%%%%%%%%%%%%%%%%%%%%%%

\ifstandalonebeamer\else
	\title[Defense of Dissertation]{A Formal View on Training of Weighted Tree Automata by Likelihood-Driven State Splitting and Merging}
	\subtitle{Defense of Dissertation}
\fi
\author{Toni Dietze}
\institute[TU Dresden]{%
	\href{https://www.orchid.inf.tu-dresden.de/index.en/}{Chair for Foundations of Programming}
\\	\href{https://tu-dresden.de/ing/informatik/thi}{Institute of Theoretical Computer Science}
\\	\href{https://tu-dresden.de/ing/informatik}{Faculty of Computer Science}
\\	\href{https://tu-dresden.de/}{Technische Universität Dresden}
\\	01062 Dresden, Germany
}
\date[2018-09-27]{September 27, 2018}

\begin{document}
\begin{standaloneframe}[t]{\jobname}
	\centering\small
	\begin{tikzpicture}
	[ data/.style = {align = center, draw, rounded corners}
	, func/.style = {align = center, draw}
	, node distance = 1.5em
	]
		\newcommand{\stacklist}
		{ 1.25em/0.390625%
		, 1.1580882352941178em/0.78125%
		, 1.0546875em/1.5625%
		, 0.9375em/3.125%
		, 0.8035714285714286em/6.25%
		, 0.6490384615384616em/12.5%
		, 0.46875em/25.0%
		, 0.2556818181818182em/50.0%
		, 0.0em/100.0%
		}

		\node[data, text = HKS41K70] (ptb0)  {Penn Treebank  section 0 \\ (1\,921 trees)};

		\node[data, right = 0.3333em of ptb0, text = HKS57K100] (ptb23) {Penn Treebank section 23 \\ (2\,416 trees)};
		\node[data, below = 0.6666em of ptb23.south west, anchor = north west, xshift = 0.3333em + \pgflinewidth] (sents) {sequence of \\ sentences};
		\node[func, right = of sents] (yield) {\strut yield};
		\node[func, below = of sents] (parse) {\strut parse};
		\node[data, right = of parse] (trees) {sequence \\ of trees};
		\node[func, right = of trees] (evalb) {\strut evalb};

		\foreach \t/\c in \stacklist {
			\node[xshift = \t, yshift = \t, color = black!\c, fill = white, data] (ptas) at (ptb0 |- parse) {\strut sequence of pta\s};
			\node[xshift = \t, yshift = \t, color = black!\c, fill = white, data, right = of evalb, text = HKS57K100] (Fmeas) {bracketing \\ F-measure};
		}

		\node[func, above = of ptas] (cbsm)  {\strut cbsm};

		\node[func, fit = (sents) (yield) (parse) (trees) (evalb)] {};

		\begin{scope}[->, rounded corners]
			\draw (ptb0)  -- (cbsm) ;
			\draw (cbsm)  -- (ptas) ;
			\draw (ptas)  -- (parse);
			\draw (sents) -- (parse);
			\draw (parse) -- (trees);
			\draw (yield) -- (sents);
			\draw (trees) -- (evalb);
			\draw (evalb) -- (Fmeas);
			%\draw (yield |- ptb23.south) -- (yield);
			%\draw (yield |- ptb23.south) -- ++(0, -0.8em) -| (evalb);
			\draw (ptb23) -- (evalb |- ptb23.east) |- (yield);
			\draw (ptb23) -| (evalb);
		\end{scope}
		%\node[below = 0.8em of ptas, draw, shape = ellipse, inner sep = 0.1em, fill = white] {\emph{for each pta}};
	\end{tikzpicture}
	\\[0.5em]
	\only<2->{%
	\begin{tikzpicture}
		\pgfplotstableread[col sep = comma]{csv-07-evalb-wsj-2300-2320-len-restricted.csv}\dataEvalb
		\begin{axis}
			[ extra tick style = {grid = major, ticklabel pos = upper, tick style = {draw = none}}
			, group/group size = 1 by 3
			, group/vertical sep = 1.5cm
			, group/xlabels at = edge bottom
			, height = 11.5em
			, scaled ticks = false
			, set layers
			, tick pos = lower
			, width = 0.85\linewidth
			, xlabel = {iteration (from 11\,729 to 11\,808)}
			, xmin = 11726
			, xmax = 11811
			, xtick distance = 20
			, extra x ticks = {11803}
			, extra y ticks = {69.25, 71.06}
			, only marks
			, point meta min = 0
			, point meta max = 1
			, scatter
			, scatter src = explicit
			, ylabel = {Bracketing \\ F-measure [\%]}
			, ylabel style = {align = left}
			]
			\addplot table
				[ meta expr = \thisrow{Number of Skip  sentence}/\thisrow{Number of sentence}
				, x = iteration
				, y = Bracketing FMeasure
				] {\dataEvalb};
		\end{axis}
	\end{tikzpicture}%
	}
\end{standaloneframe}
\end{document}

\end{frame}


\iffalse
\section{Bottlenecks and Open Problems}

\begin{frame}{\secname}
	%\begin{block}{what we have seen}
	%	\begin{itemize}
	%	\item
	%		merging for (p)rtg
	%	\item
	%		count-based state merging algorithm:
	%		\\ abstracting away from a corpus in small steps
	%	\item
	%		bottom-up determinism simplifies the calculations
	%	\end{itemize}
	%\end{block}
	%\pause
	\begin{block}{}%{bottlenecks/problems}
		\begin{itemize}
		\item
			finding the best merger
		\item
			saturating mergers until they preserve determinism
		\item
			heuristics cannot distinguish states with same count
		\item
			Is bottom-up determinism too restrictive?
		\end{itemize}
	\end{block}
\end{frame}
\fi


\againframe<7>{toc}


\section{Binarization – Motivation}

\begin{frame}{\secname}
	% SPDX-License-Identifier: CC-BY-4.0
% Copyright 2018 Toni Dietze
\documentclass[beamer]{standalone}
% SPDX-License-Identifier: CC-BY-4.0 OR MIT-0
% Copyright 2018 Toni Dietze
%
\usefonttheme{professionalfonts}

% LuaLaTeX specific packages
\usepackage{fontspec}
	\defaultfontfeatures{Ligatures=TeX}
\usepackage{polyglossia}
	\setdefaultlanguage{english}
\usepackage{amsmath}  % has to be loaded before unicode-math
\usepackage[math-style=ISO]{unicode-math}
	\setmathfont{Latin Modern Math}
% 	\setmathfont[range={\mathcal,\mathbfcal},StylisticSet=1]{xits-math.otf}
% 	\setmathfont[range={"029F5}]{XITS Math}  % ⧵
% 	\setmathfont[range={\mathscr,\mathbfscr},StylisticSet=1]{Latin Modern Math}  % make \mathscr use the correct font

\usepackage[noend]{algpseudocode}
	\algrenewcommand\algorithmicrequire{\textbf{Input:}}
	\algrenewcommand\algorithmicensure{\textbf{Output:}}
\usepackage[backend=biber, maxbibnames=42, maxcitenames=42, sorting=ynt, style=authoryear]{biblatex}
\usepackage{csquotes}
\usepackage{mathtools}
\usepackage{media9}
\usepackage{scalerel}
\usepackage{standalone}
\usepackage{tikz}
	\usetikzlibrary{arrows.meta}
	\usetikzlibrary{backgrounds}
	\usetikzlibrary{calc}
	\usetikzlibrary{decorations}
	\usetikzlibrary{decorations.pathmorphing}
	\usetikzlibrary{decorations.pathreplacing}
	\usetikzlibrary{fadings}
	\usetikzlibrary{fit}
	\usetikzlibrary{graphs}
	\usetikzlibrary{graphdrawing}
	\usetikzlibrary{intersections}
	\usetikzlibrary{positioning}
	\usetikzlibrary{quotes}
	\usetikzlibrary{shadows.blur}
	\usetikzlibrary{shapes.arrows}
	\usetikzlibrary{shapes.geometric}
	\usegdlibrary{trees}
\usepackage{xifthen}
\usepackage{xspace}

\usepackage{pgfplots}
	\pgfplotsset
		{ compat = 1.15
		, /pgf/number format/1000 sep = {\,}
		, /pgf/number format/assume math mode = true
		, every axis plot/.append style =
			{ mark options = {fill opacity = 0.25}
			}
		}
	\usepgfplotslibrary{groupplots}
\usepackage{pgfplotstable}

\hypersetup
	{ bookmarksopen
	, pdflang = en
	, unicode
	}


%%%%%%%%%%%%%%%%%%%%%%%%%%%%%%%%%%%%%%%%%%%%%%%%%%%%%%%%%%%%%%%%%%%%%%%%%%%%%%


% always show bad boxes
%\overfullrule=1em


%%%%%%%%%%%%%%%%%%%%%%%%%%%%%%%%%%%%%%%%%%%%%%%%%%%%%%%%%%%%%%%%%%%%%%%%%%%%%%
% biblatex
%%%%%%%%%%%%%%%%%%%%%%%%%%%%%%%%%%%%%%%%%%%%%%%%%%%%%%%%%%%%%%%%%%%%%%%%%%%%%%

\addbibresource{slides-dissertation-defense.bib}
% \renewcommand*{\finalnamedelim}{\addcomma\space}
% \setlength{\bibitemsep}{1em}
% 
\AtEveryBibitem{% Clean up the bibtex rather than editing it
 \clearlist{address}
 \clearfield{date}
 \clearfield{eprint}
 \clearfield{isbn}
 \clearfield{issn}
 \clearlist{language}
 \clearlist{location}
 \clearfield{month}
 \clearfield{series}
%  \clearfield{url}
%  \clearfield{doi}
 \clearfield{organization}

%  \ifentrytype{book}{}{% Remove stuff except for books
%   \clearfield{booktitle}
%   \clearfield{pages}
  \clearlist{publisher}
  \clearname{editor}
%  }
}
% do not print url if doi is present
% http://tex.stackexchange.com/questions/154864/biblatex-use-doi-only-if-there-is-no-url
\DeclareSourcemap{
	\maps[datatype=bibtex]{
		\map{
			\step[fieldsource=doi,final]
			\step[fieldset=url,null]
}	}	}
%
% remove qoutes around titles
\DeclareFieldFormat
	[article,inbook,incollection,inproceedings,patent,thesis,unpublished]
	{title}{#1\isdot}
% 
% \DeclareFieldFormat{url}{\mkbibacro{URL}\addcolon\addnbspace\url{#1}}
% 
% \DeclareNameAlias{sortname}{first-last}
% 
\renewbibmacro{in:}{\ifentrytype{article}{}{}}


%%%%%%%%%%%%%%%%%%%%%%%%%%%%%%%%%%%%%%%%%%%%%%%%%%%%%%%%%%%%%%%%%%%%%%%%%%%%%%
% beamer
%%%%%%%%%%%%%%%%%%%%%%%%%%%%%%%%%%%%%%%%%%%%%%%%%%%%%%%%%%%%%%%%%%%%%%%%%%%%%%

\useoutertheme{infolines}
\makeatletter
% based on
% /usr/share/texmf-dist/tex/latex/beamer/beamerouterthemeinfolines.sty
\setbeamertemplate{footline}
{%
	\leavevmode%
	\hbox{%
	\begin{beamercolorbox}[wd=.333333\paperwidth,ht=2.25ex,dp=1ex,center]{author in head/foot}%
		\usebeamerfont{author in head/foot}\insertshortauthor\expandafter\beamer@ifempty\expandafter{\beamer@shortinstitute}{}{~~(\insertshortinstitute)}
	\end{beamercolorbox}%
	\begin{beamercolorbox}[wd=.333333\paperwidth,ht=2.25ex,dp=1ex,center]{title in head/foot}%
		\usebeamerfont{title in head/foot}\insertshorttitle
	\end{beamercolorbox}%
	\begin{beamercolorbox}[wd=.333333\paperwidth,ht=2.25ex,dp=1ex,right]{date in head/foot}%
		\usebeamerfont{date in head/foot}%
		\hfill\insertshortdate\hfill\hfill%
		%\hspace*{2ex}%
		%\insertshortdate%
		%\hspace{0pt plus 1 filll}%
		%(\insertframenumber.\insertoverlaynumber{} / \insertmainframenumber)%
		%\hspace{0pt plus 1 filll}%
		\phantom{000}\llap{\insertpagenumber} / \insertpresentationendpage%
		\hspace*{2ex}%
	\end{beamercolorbox}}%
	\vskip0pt%
}
\makeatother
\useinnertheme{circles}
\beamertemplatenavigationsymbolsempty
\setbeamertemplate{bibliography item}{}
\setbeamertemplate{headline}[default]

\input{tudcolors.tex}
\setbeamercolor*{alerted text}{fg=HKS07K100}
\usecolortheme[named=HKS41K100]{structure}

\setbeamercolor*{palette primary}{use=structure,fg=white,bg=structure.fg}
\setbeamercolor*{palette secondary}{use=structure,fg=white,bg=structure.fg!80}
\setbeamercolor*{palette tertiary}{use=structure,fg=white,bg=structure.fg!60}
\setbeamercolor*{palette quaternary}{fg=white,bg=black}

\setbeamercolor*{sidebar}{use=structure,bg=structure.fg}

\setbeamercolor*{palette sidebar primary}{use=structure,fg=structure.fg!20}
\setbeamercolor*{palette sidebar secondary}{fg=white}
\setbeamercolor*{palette sidebar tertiary}{use=structure,fg=structure.fg!40}
\setbeamercolor*{palette sidebar quaternary}{fg=white}

\setbeamercolor*{titlelike}{parent=palette primary}

\setbeamercolor*{separation line}{}
\setbeamercolor*{fine separation line}{}

\setbeamercolor{block title}{use=structure,fg=white,bg=structure.fg}
\setbeamercolor{block title alerted}{use=alerted text,fg=white,bg=alerted text.fg!75!black}
\setbeamercolor{block title example}{use=example text,fg=white,bg=example text.fg!75!black}

\setbeamercolor{block body}{parent=normal text,use=block title,bg=block title.bg!10!bg}
\setbeamercolor{block body alerted}{parent=normal text,use=block title alerted,bg=block title alerted.bg!10!bg}
\setbeamercolor{block body example}{parent=normal text,use=block title example,bg=block title example.bg!10!bg}

% \setbeamertemplate{itemize items}[default]


%%%%%%%%%%%%%%%%%%%%%%%%%%%%%%%%%%%%%%%%%%%%%%%%%%%%%%%%%%%%%%%%%%%%%%%%%%%%%%
% TikZ
%%%%%%%%%%%%%%%%%%%%%%%%%%%%%%%%%%%%%%%%%%%%%%%%%%%%%%%%%%%%%%%%%%%%%%%%%%%%%%

\tikzset
	{ > = Stealth
	}


%%%%%%%%%%%%%%%%%%%%%%%%%%%%%%%%%%%%%%%%%%%%%%%%%%%%%%%%%%%%%%%%%%%%%%%%%%%%%%
% general commands and styles
%%%%%%%%%%%%%%%%%%%%%%%%%%%%%%%%%%%%%%%%%%%%%%%%%%%%%%%%%%%%%%%%%%%%%%%%%%%%%%

% \delegateStyle and \inheritStyle command
% usage: \delegateStyle{… \inheritStyle{…} …}
% example: \(X_{\delegateStyle{\fbox{\inheritStyle{X}}}}\)
% Save the current style and regain it in the argument.
% This works both for math and text mode, and can be nested.
% Acknowledgments: Based on \ThisStyle and \SavedStyle from scalerel package.
\makeatletter
\newcommand*{\@inheritStyle@D}[1]{\(\displaystyle      #1\)}
\newcommand*{\@inheritStyle@T}[1]{\(\textstyle         #1\)}
\newcommand*{\@inheritStyle@S}[1]{\(\scriptstyle       #1\)}
\newcommand*{\@inheritStyle@s}[1]{\(\scriptscriptstyle #1\)}
\newcommand*{\@inheritStyle@t}[1]{#1}
\newcommand*{\inheritStyle}{\csname @inheritStyle@\@inheritStyleSwitch\endcsname}
\newcommand*{\delegateStyle}[1]{%
	\ifmmode%
		\mathchoice%
		{\edef\@inheritStyleSwitch{D}#1}%
		{\edef\@inheritStyleSwitch{T}#1}%
		{\edef\@inheritStyleSwitch{S}#1}%
		{\edef\@inheritStyleSwitch{s}#1}%
	\else%
		\edef\@inheritStyleSwitch{t}#1%
	\fi%
}
\makeatother


% \oalt command
% requires: \delegateStyle and \inheritStyle command
% usage: \oalt<…>[…]{…}{…} (cf. \alt)
% Behaves like \alt, but reserves space according to largest overlays.
% The optional argument defines the alignment inside the reserved space;
% it is one of c, l, r, s (cf. \makebox); the default is c.
\makeatletter
\newlength{\oalt@dp}
\newlength{\oalt@ht}
\newlength{\oalt@wd}
\newbox{\oalt@a}
\newbox{\oalt@b}
\newbox{\oalt@empty}
\newcommand<>*{\oalt}[3][c]{%
	\delegateStyle{%
		% based on \setto… in /usr/share/texmf-dist/tex/latex/base/latex.ltx
		\setbox\oalt@a\hbox{\inheritStyle{#2}}%
		\setbox\oalt@b\hbox{\inheritStyle{#3}}%
		\pgfmathsetlength{\oalt@dp}{max(\dp\oalt@a,\dp\oalt@b)}%
		\pgfmathsetlength{\oalt@ht}{max(\ht\oalt@a,\ht\oalt@b)}%
		\pgfmathsetlength{\oalt@wd}{max(\wd\oalt@a,\wd\oalt@b)}%
		\raisebox{0pt}[\oalt@ht][\oalt@dp]{%
			\makebox[\oalt@wd][#1]{%
				\alt#4{\unhbox\oalt@a}{\unhbox\oalt@b}%
			}%
		}%
		\setbox\oalt@a\box\oalt@empty%
		\setbox\oalt@b\box\oalt@empty%
	}%
}
\makeatother


% \otemporal command
% requires: \delegateStyle and \inheritStyle command
% usage: \otemporal<…>[…]{…}{…}{…} (cf. \temporal)
% Behaves like \temporal, but reserves space according to largest overlays.
% The optional argument defines the alignment inside the reserved space;
% it is one of c, l, r, s (cf. \makebox); the default is c.
\makeatletter
\newlength{\ot@dp}
\newlength{\ot@ht}
\newlength{\ot@wd}
\newbox{\ot@a}
\newbox{\ot@b}
\newbox{\ot@c}
\newbox{\ot@empty}
\newcommand<>*{\otemporal}[4][c]{%
	\delegateStyle{%
		% based on \setto… in /usr/share/texmf-dist/tex/latex/base/latex.ltx
		\setbox\ot@a\hbox{\inheritStyle{#2}}%
		\setbox\ot@b\hbox{\inheritStyle{#3}}%
		\setbox\ot@c\hbox{\inheritStyle{#4}}%
		\pgfmathsetlength{\ot@dp}{max(\dp\ot@a,\dp\ot@b,\dp\ot@c)}%
		\pgfmathsetlength{\ot@ht}{max(\ht\ot@a,\ht\ot@b,\ht\ot@c)}%
		\pgfmathsetlength{\ot@wd}{max(\wd\ot@a,\wd\ot@b,\wd\ot@c)}%
		\raisebox{0pt}[\ot@ht][\ot@dp]{%
			\makebox[\ot@wd][#1]{%
				\temporal#5{\unhbox\ot@a}{\unhbox\ot@b}{\unhbox\ot@c}%
			}%
		}%
		\setbox\ot@a\box\ot@empty%
		\setbox\ot@b\box\ot@empty%
		\setbox\ot@c\box\ot@empty%
	}%
}
\makeatother


% Resize delimiters like braces, brackets, etc.
% Parameters: size, left delimiter, formula, right delimiter
% Example: \delim2({\frac{1}{2}})
\newcommand*{\delim}[4]{%
	\ifcase#1%
		#2#3#4%
	\or%
		\bigl#2#3\bigr#4%
	\or%
		\Bigl#2#3\Bigr#4%
	\or%
		\biggl#2#3\biggr#4%
	\or%
		\Biggl#2#3\Biggr#4%
	\else%
		\left#2#3\right#4%
	\fi%
}


% similar to \fullcite, but using the formatting of \printbibliography
\newcommand*{\printfullcite}[1]{%
	\begin{refsection}%
		\nocite{#1}%
		\DeclareNameAlias{author}{first-last}%
		\printbibliography[heading = none]%
	\end{refsection}%
}


\colorlet{light alert}{HKS07K60}
\tikzset{alert.bg/.style={rounded corners, fill=light alert}}
\tikzset{every picture/.style={line cap=round, semithick}}
% http://tex.stackexchange.com/questions/6135/how-to-make-beamer-overlays-with-tikz-node
\tikzset{onslide/.code args={<#1>#2}{\only<#1>{\pgfkeysalso{#2}}}}
\tikzset{invisible/.code args={<#1>}{\alt<#1>{\pgfkeysalso{transparent}}{\pgfkeysalso{opaque}}}}
\tikzset{uncover/.code args={<#1>}{\alt<#1>{\pgfkeysalso{opaque}}{\pgfkeysalso{opacity=0.25}}}}
\tikzset{visible/.code args={<#1>}{\alt<#1>{\pgfkeysalso{opaque}}{\pgfkeysalso{transparent}}}}
\tikzset{vuncover/.code args=%
	{<#1><#2>}%
	{\alt<#1>%
		{\alt<#2>%
			{\pgfkeysalso{opaque}}%
			{\pgfkeysalso{opacity=0.25}}%
		}{\pgfkeysalso{transparent}}%
	}%
}

\newcommand<%
	>{\tikzhighlight}[2][]{%
	\delegateStyle{\alt#3%
		{\tikz[baseline=0, anchor=base, inner sep=0.2em, text height=, text depth=]{\node[alert.bg, #1]{\inheritStyle{#2}};}}%
		{\tikz[baseline=0, anchor=base, inner sep=0.2em, text height=, text depth=]{\node[#1, fill=none]{\inheritStyle{#2}};}}%
	}%
}

\newcommand{\mathhighlight}{\tikzhighlight}

\newcommand<>{\mhl}[2][]{\mathhighlight#3[inner sep=0.2em, #1]{#2}}


\newcommand<>{\inlineblock}[2][]{{%
	\usebeamercolor*[fg]{block body}%
	\tikzhighlight#3[fill=block body.bg, #1]{#2}%
}}


% a small letter s for plurals of abbreviations
\newcommand*{\s}{{\scriptsize s}\xspace}


\newcommand<>*{\sout}[2][opacity=0.75, ultra thick]{%
	\delegateStyle{%
		\tikz[baseline=0, anchor=base, inner sep=0, outer sep=0]{
			\useasboundingbox node (n) {\inheritStyle{#2}};
			\only#3{
				\node (h) {\inheritStyle{\ifmmode\mathstrut\else\strut\fi}};
				\draw[#1] (n.west |- {$(h.south)!0.5!(h.north)$}) -- (n.east |- {$(h.south)!0.5!(h.north)$});
			}
		}%
	}%
}


% tight style
% Sets outer sep to default inner sep and inner sep to 0.
% Use this style for nodes that are neither drawn nor filled to prevent
% unwanted growth of the bounding box.
\tikzset{tight/.style={inner sep=0, outer sep=0.3333em}}


% rounded tree edges style
% usage: rounded tree edges={⟨direction⟩}{⟨looseness⟩}{⟨strength⟩}
\tikzset{
	rounded tree edges/.style n args={3}{
	edge from parent path={
	let
		\n{direction}={#1},
		\n{looseness}={#2},
		\n{strength}={#3},
		\p1=(\tikzparentnode),
		\p2=(\tikzchildnode),
		\p3=(\n{direction}:1pt),
		\p4=(\x2 - \x1, \y2 - \y1),
		\n{dist}={veclen(\p4)},
		\p4=(\x4 / \n{dist}, \y4 / \n{dist}),
		\n{angle}={atan2(\y4, \x4)},
		\n{delta}={Mod(\n{angle} - \n{direction}, 360)},
		\n{delta}={\n{delta} > 180 ? \n{delta} - 360  : \n{delta}},
		\n{delta}={\n{delta} >  90 ?  180 - \n{delta} : \n{delta}},
		\n{delta}={\n{delta} < -90 ? -180 - \n{delta} : \n{delta}}
	in (\tikzparentnode) .. controls
		+(    \n{angle}+\n{strength}*\n{delta}:\n{looseness}*0.3915*\n{dist}) and
		+(180+\n{angle}-\n{strength}*\n{delta}:\n{looseness}*0.3915*\n{dist}) ..
		(\tikzchildnode)
	}
	}
}


% Tear out snippets from PDFs.
% Usage: \tear[…]{file.pdf}
% The optional parameter is the same as for \includegraphics.
% Useful Arguments:
%   * page=‹pagenumber›
%   * trim=‹left› ‹bottom› ‹right› ‹top›
%   * width=0.98\linewidth
\newcommand*{\tear}[2][]{%
	\begin{tikzpicture}
		\node
			[ blur shadow
			, clip
			, decorate
			, decoration=random steps
			, draw
			, inner sep=0
			, preaction={fill=white}% hide the shadow if paper is transparent
			] {\includegraphics[#1]{#2}};
	\end{tikzpicture}%
}


\makeatletter
\newcommand*{\timeline}[3][0]{%
	\ifcsname timeline@cmd@#3\endcsname%
		\@timeline[#1]{#2}{#3}%
		\PackageWarning{timeline}{redefining timeline \@backslashchar\string#3}%
	\else%
		\ifcsname#3\endcsname%
			\errmessage{Command \@backslashchar\string#3 already defined}%
		\else%
			\@timeline[#1]{#2}{#3}%
		\fi%
	\fi%
}%
\newcommand*{\@timeline}[3][0]{%
	% mark command as timeline command – they can be overwritten
	\expandafter\def\csname timeline@cmd@#3\endcsname{}%
	\setcounter{@timeline}{#1}%
	\def\timeline@cmd{#3}%
	\timeline@reset%
	\timeline@append{0}%
	\@tfor\timeline@next:=#2\do{%
		\if\timeline@next+%
			\stepcounter{@timeline}%
			\timeline@append{,\the@timeline}%
		\else\if\timeline@next-%
			\stepcounter{@timeline}%
		\else%
			%\timeline@append{\timeline@next}%
			\GenericError{}{\protect\timeline: ignoring unknown character: \timeline@next}%
		\fi\fi%
	}%
}%
% \newcommand*{\tl}[1]{%
% 	\ifcsname timeline@cmd@#1\endcsname%
% 		\csname timeline@cmd@#1\endcsname%
% 	\else%
% 		0%
% 		%\GenericError{}{\protect\tl: timeline not defined: #1}%
% 	\fi%
% }%
\newcounter{@timeline}%
\def\timeline@reset{%
	\expandafter\def\csname\timeline@cmd\endcsname{}%
}%
\def\timeline@append#1{%
	\expandafter\edef\csname\timeline@cmd\endcsname{%
		\csname\timeline@cmd\endcsname#1%
	}%
}%
\makeatother


\newcommand*{\xminus}[1]{%
	\mathrel{\tikz[baseline={([yshift=-0.25em]n.south)}, inner sep=0, outer sep=0.2em]{%
		\node (n) {\(\scriptstyle #1\)};
		\draw (n.south west) -- (n.south east);
	}}%
}
\newcommand*{\tikzrightarrow}[1]{%
	\mathrel{\tikz[baseline={([yshift=-0.25em]n.south)}, inner sep=0, outer sep=0.2em]{%
		\node (n) {\(\scriptstyle #1\)};
		\draw[->, > = Computer Modern Rightarrow, line width = 0.4pt] (n.south west) -- (n.south east);
	}}%
}


%%%%%%%%%%%%%%%%%%%%%%%%%%%%%%%%%%%%%%%%%%%%%%%%%%%%%%%%%%%%%%%%%%%%%%%%%%%%%%
% document specific commands
%%%%%%%%%%%%%%%%%%%%%%%%%%%%%%%%%%%%%%%%%%%%%%%%%%%%%%%%%%%%%%%%%%%%%%%%%%%%%%

\newcommand<>*{\mycite}[1]{\uncover#2{{\color{HKS57K100}[\cite{#1}]}}}


\newcommand{\statetree}[1]{
	\tikz
	[ anchor=base
	, baseline=(current bounding box.center)
	, level distance=2em
	, sibling distance=2em
	]{
		\matrix
		[ draw=nt
		, edge from parent/.style={draw=black}
		, inner sep=0
		, nodes={inner sep=0.2em, rounded corners=0}
		, rounded corners
		] {#1\\}
	}
}


\newcommand*{\mylargeleaf}[1]{{\LARGE\color{HKS41K70}#1}}

\definecolor{state s}{named}{HKS57K80}
\definecolor{state t}{named}{HKS41K70}
\newcommand*{\stateS}[1]{{\color{state s}#1}}
\newcommand*{\stateT}[1]{{\color{state t}#1}}

\tikzset{
	subtree/.style =
		{ fill=lightgray
		, inner sep=0.02em
		, isosceles triangle apex angle=60
		, shape=isosceles triangle
		, shape border rotate=90
		}
	, state/.style = {circle, draw, inner sep=0.1em}
	, trans/.style = {rectangle, draw}
}

\newcommand*{\srBool}{\mathbb{B}}
\newcommand*{\srProb}{ℙ}


%%%%%%%%%%%%%%%%%%%%%%%%%%%%%%%%%%%%%%%%%%%%%%%%%%%%%%%%%%%%%%%%%%%%%%%%%%%%%%
% commands for specific notations
%%%%%%%%%%%%%%%%%%%%%%%%%%%%%%%%%%%%%%%%%%%%%%%%%%%%%%%%%%%%%%%%%%%%%%%%%%%%%%

\DeclareMathOperator*{\argmax}{argmax}

\newcommand*{\cardinality}[1]{\lvert#1\rvert}
\newcommand*{\corpussize}[1]{\lvert#1\rvert}

\DeclareMathOperator{\crispOp}{crisp}
\newcommand*        {\crisp}[2][0]{\crispOp\delim{#1}({#2})}

\DeclareMathOperator{\lhsOp}{lhs}
\newcommand*{\lhs}[1]{\lhsOp(#1)}

\DeclareMathOperator{\lklhdOp}{L}
\newcommand*{\lklhd}[2]{\lklhdOp(#1 ∣ #2)}

\DeclareMathOperator{\mleOp}{mle}
\newcommand*{\mle}[2][]{%
	\ifthenelse{\isempty{#1}}{%
		\mleOp(#2)%
	}{%
		\mleOp_{#1}(#2)%
	}%
}

\DeclareMathOperator{\mrg}{merge}

% CVD: color vision deficiencies
\definecolor{CVD light red}   {HTML}{FF8080}
\definecolor{CVD light yellow}{HTML}{FFFF80}
\definecolor{CVD light green} {HTML}{40FFC0}

\definecolor{nt}{named}{HKS41K70}
\newcommand*{\nt}[1]{{\color{nt}#1}}

% set of all probability distributions over #1
\DeclareMathOperator{\pdsOp}{Pd}
\newcommand*{\pds}[1]{\pdsOp(#1)}

\DeclareMathOperator{\positionsOp}{pos}
\newcommand*{\positions}[1]{\positionsOp(#1)}

\DeclareMathOperator{\rankOp}{rk}
\newcommand*{\rank}[1]{\rankOp(#1)}

\DeclareMathOperator{\runsOp}{run}
\newcommand*{\runs}[2][]{%
	\ifthenelse%
		{\isempty{#1}}%
		{\runsOp(#2)}%
		{\runsOp_{#1}(#2)}%
}

\newcommand*{\semantics}[1]{⟦#1⟧}

\DeclareMathOperator{\splt}{split}

\newcommand*{\subtree}[2]{#1|_{#2}}

\DeclareMathOperator{\supportOp}{supp}
\newcommand*{\support}[1]{\supportOp(#1)}

\newcommand*{\symId}{\textsc{\color{gray}Id}}
\newcommand*{\symCons}{\textsc{\color{gray}Cons}}
\newcommand*{\symFlip}{\textsc{\color{gray}Flip}}
\newcommand*{\symNull}{\textsc{\color{gray}Null}}
\newcommand*{\symNullR}{\textsc{\color{gray}N\(\overline{\textsc{ull}}\)}}
\newcommand*{\symSnoc}{\textsc{\color{gray}Snoc}}

\newcommand*{\transWTA}[4][]{#3 \xrightarrow{#1} #2(#4)}

\DeclareMathOperator{\uniqueRunOp}{r}
\newcommand*{\uniqueRun}[2][]{%
	\ifthenelse%
		{\isempty{#1}}%
		{\uniqueRunOp^{#2}}%
		{\uniqueRunOp_{\!#1}^{#2}}%
}

\DeclareMathOperator{\treesOp}{T}
\newcommand*{\trees}[2][]{%
	\ifthenelse%
		{\isempty{#1}}%
		{\treesOp_{\!#2}}%
		{\treesOp_{\!#2}(#1)}%
}
\DeclareMathOperator{\treesUOp}{U}
\newcommand*{\treesU}[2][]{%
	\ifthenelse%
		{\isempty{#1}}%
		{\treesUOp_{#2}}%
		{\treesUOp_{#2}(#1)}%
}


%%%%%%%%%%%%%%%%%%%%%%%%%%%%%%%%%%%%%%%%%%%%%%%%%%%%%%%%%%%%%%%%%%%%%%%%%%%%%%
% metadata
%%%%%%%%%%%%%%%%%%%%%%%%%%%%%%%%%%%%%%%%%%%%%%%%%%%%%%%%%%%%%%%%%%%%%%%%%%%%%%

\ifstandalonebeamer\else
	\title[Defense of Dissertation]{A Formal View on Training of Weighted Tree Automata by Likelihood-Driven State Splitting and Merging}
	\subtitle{Defense of Dissertation}
\fi
\author{Toni Dietze}
\institute[TU Dresden]{%
	\href{https://www.orchid.inf.tu-dresden.de/index.en/}{Chair for Foundations of Programming}
\\	\href{https://tu-dresden.de/ing/informatik/thi}{Institute of Theoretical Computer Science}
\\	\href{https://tu-dresden.de/ing/informatik}{Faculty of Computer Science}
\\	\href{https://tu-dresden.de/}{Technische Universität Dresden}
\\	01062 Dresden, Germany
}
\date[2018-09-27]{September 27, 2018}

\begin{document}
\begin{standaloneframe}{\jobname}
	\centering
	\begin{tikzpicture}
	[ anchor = base
	, level sep = 1em
	, node distance = 1.5em
	, sibling sep = 0.25em
	]
		\graph[fresh nodes, tree layout]
		{
			NP --
			{ DT -- "the"
			, JJ -- "little"
			, JJ -- "red"
			, NN [x = 0, y = 0] -- "flower"
			}
		};

		\visible<2->{
			\node[base right = of NN, draw, shape = single arrow, minimum height = 5.5em] (training) {\strut training};
			\node[base right = of training] {\(\nt{A} → \text{NP}(\nt{B}, \nt{C}, \nt{D}, \nt{E})\)};
		}
		\visible<3->{
			\graph[fresh nodes, tree layout]
			{
				NP --
				{ DT -- "the"
				, JJ -- "nice"
				, JJ -- "little"
				, JJ -- "red"
				, NN [x = 0, y = -9em] -- "flower"
				}
			};
		}
		\visible<4->{
			\node[base right = of NN, draw, shape = single arrow, minimum height = 5.5em] (parse) {\strut accepted?};
			\node[base right = of parse] {no};
		}
	\end{tikzpicture}
\end{standaloneframe}
\end{document}

\end{frame}


\section{Binarization}

\begin{frame}{\secname}
	\documentclass[beamer]{standalone}
% SPDX-License-Identifier: CC-BY-4.0 OR MIT-0
% Copyright 2018 Toni Dietze
%
\usefonttheme{professionalfonts}

% LuaLaTeX specific packages
\usepackage{fontspec}
	\defaultfontfeatures{Ligatures=TeX}
\usepackage{polyglossia}
	\setdefaultlanguage{english}
\usepackage{amsmath}  % has to be loaded before unicode-math
\usepackage[math-style=ISO]{unicode-math}
	\setmathfont{Latin Modern Math}
% 	\setmathfont[range={\mathcal,\mathbfcal},StylisticSet=1]{xits-math.otf}
% 	\setmathfont[range={"029F5}]{XITS Math}  % ⧵
% 	\setmathfont[range={\mathscr,\mathbfscr},StylisticSet=1]{Latin Modern Math}  % make \mathscr use the correct font

\usepackage[noend]{algpseudocode}
	\algrenewcommand\algorithmicrequire{\textbf{Input:}}
	\algrenewcommand\algorithmicensure{\textbf{Output:}}
\usepackage[backend=biber, maxbibnames=42, maxcitenames=42, sorting=ynt, style=authoryear]{biblatex}
\usepackage{csquotes}
\usepackage{mathtools}
\usepackage{media9}
\usepackage{scalerel}
\usepackage{standalone}
\usepackage{tikz}
	\usetikzlibrary{arrows.meta}
	\usetikzlibrary{backgrounds}
	\usetikzlibrary{calc}
	\usetikzlibrary{decorations}
	\usetikzlibrary{decorations.pathmorphing}
	\usetikzlibrary{decorations.pathreplacing}
	\usetikzlibrary{fadings}
	\usetikzlibrary{fit}
	\usetikzlibrary{graphs}
	\usetikzlibrary{graphdrawing}
	\usetikzlibrary{intersections}
	\usetikzlibrary{positioning}
	\usetikzlibrary{quotes}
	\usetikzlibrary{shadows.blur}
	\usetikzlibrary{shapes.arrows}
	\usetikzlibrary{shapes.geometric}
	\usegdlibrary{trees}
\usepackage{xifthen}
\usepackage{xspace}

\usepackage{pgfplots}
	\pgfplotsset
		{ compat = 1.15
		, /pgf/number format/1000 sep = {\,}
		, /pgf/number format/assume math mode = true
		, every axis plot/.append style =
			{ mark options = {fill opacity = 0.25}
			}
		}
	\usepgfplotslibrary{groupplots}
\usepackage{pgfplotstable}

\hypersetup
	{ bookmarksopen
	, pdflang = en
	, unicode
	}


%%%%%%%%%%%%%%%%%%%%%%%%%%%%%%%%%%%%%%%%%%%%%%%%%%%%%%%%%%%%%%%%%%%%%%%%%%%%%%


% always show bad boxes
%\overfullrule=1em


%%%%%%%%%%%%%%%%%%%%%%%%%%%%%%%%%%%%%%%%%%%%%%%%%%%%%%%%%%%%%%%%%%%%%%%%%%%%%%
% biblatex
%%%%%%%%%%%%%%%%%%%%%%%%%%%%%%%%%%%%%%%%%%%%%%%%%%%%%%%%%%%%%%%%%%%%%%%%%%%%%%

\addbibresource{slides-dissertation-defense.bib}
% \renewcommand*{\finalnamedelim}{\addcomma\space}
% \setlength{\bibitemsep}{1em}
% 
\AtEveryBibitem{% Clean up the bibtex rather than editing it
 \clearlist{address}
 \clearfield{date}
 \clearfield{eprint}
 \clearfield{isbn}
 \clearfield{issn}
 \clearlist{language}
 \clearlist{location}
 \clearfield{month}
 \clearfield{series}
%  \clearfield{url}
%  \clearfield{doi}
 \clearfield{organization}

%  \ifentrytype{book}{}{% Remove stuff except for books
%   \clearfield{booktitle}
%   \clearfield{pages}
  \clearlist{publisher}
  \clearname{editor}
%  }
}
% do not print url if doi is present
% http://tex.stackexchange.com/questions/154864/biblatex-use-doi-only-if-there-is-no-url
\DeclareSourcemap{
	\maps[datatype=bibtex]{
		\map{
			\step[fieldsource=doi,final]
			\step[fieldset=url,null]
}	}	}
%
% remove qoutes around titles
\DeclareFieldFormat
	[article,inbook,incollection,inproceedings,patent,thesis,unpublished]
	{title}{#1\isdot}
% 
% \DeclareFieldFormat{url}{\mkbibacro{URL}\addcolon\addnbspace\url{#1}}
% 
% \DeclareNameAlias{sortname}{first-last}
% 
\renewbibmacro{in:}{\ifentrytype{article}{}{}}


%%%%%%%%%%%%%%%%%%%%%%%%%%%%%%%%%%%%%%%%%%%%%%%%%%%%%%%%%%%%%%%%%%%%%%%%%%%%%%
% beamer
%%%%%%%%%%%%%%%%%%%%%%%%%%%%%%%%%%%%%%%%%%%%%%%%%%%%%%%%%%%%%%%%%%%%%%%%%%%%%%

\useoutertheme{infolines}
\makeatletter
% based on
% /usr/share/texmf-dist/tex/latex/beamer/beamerouterthemeinfolines.sty
\setbeamertemplate{footline}
{%
	\leavevmode%
	\hbox{%
	\begin{beamercolorbox}[wd=.333333\paperwidth,ht=2.25ex,dp=1ex,center]{author in head/foot}%
		\usebeamerfont{author in head/foot}\insertshortauthor\expandafter\beamer@ifempty\expandafter{\beamer@shortinstitute}{}{~~(\insertshortinstitute)}
	\end{beamercolorbox}%
	\begin{beamercolorbox}[wd=.333333\paperwidth,ht=2.25ex,dp=1ex,center]{title in head/foot}%
		\usebeamerfont{title in head/foot}\insertshorttitle
	\end{beamercolorbox}%
	\begin{beamercolorbox}[wd=.333333\paperwidth,ht=2.25ex,dp=1ex,right]{date in head/foot}%
		\usebeamerfont{date in head/foot}%
		\hfill\insertshortdate\hfill\hfill%
		%\hspace*{2ex}%
		%\insertshortdate%
		%\hspace{0pt plus 1 filll}%
		%(\insertframenumber.\insertoverlaynumber{} / \insertmainframenumber)%
		%\hspace{0pt plus 1 filll}%
		\phantom{000}\llap{\insertpagenumber} / \insertpresentationendpage%
		\hspace*{2ex}%
	\end{beamercolorbox}}%
	\vskip0pt%
}
\makeatother
\useinnertheme{circles}
\beamertemplatenavigationsymbolsempty
\setbeamertemplate{bibliography item}{}
\setbeamertemplate{headline}[default]

\input{tudcolors.tex}
\setbeamercolor*{alerted text}{fg=HKS07K100}
\usecolortheme[named=HKS41K100]{structure}

\setbeamercolor*{palette primary}{use=structure,fg=white,bg=structure.fg}
\setbeamercolor*{palette secondary}{use=structure,fg=white,bg=structure.fg!80}
\setbeamercolor*{palette tertiary}{use=structure,fg=white,bg=structure.fg!60}
\setbeamercolor*{palette quaternary}{fg=white,bg=black}

\setbeamercolor*{sidebar}{use=structure,bg=structure.fg}

\setbeamercolor*{palette sidebar primary}{use=structure,fg=structure.fg!20}
\setbeamercolor*{palette sidebar secondary}{fg=white}
\setbeamercolor*{palette sidebar tertiary}{use=structure,fg=structure.fg!40}
\setbeamercolor*{palette sidebar quaternary}{fg=white}

\setbeamercolor*{titlelike}{parent=palette primary}

\setbeamercolor*{separation line}{}
\setbeamercolor*{fine separation line}{}

\setbeamercolor{block title}{use=structure,fg=white,bg=structure.fg}
\setbeamercolor{block title alerted}{use=alerted text,fg=white,bg=alerted text.fg!75!black}
\setbeamercolor{block title example}{use=example text,fg=white,bg=example text.fg!75!black}

\setbeamercolor{block body}{parent=normal text,use=block title,bg=block title.bg!10!bg}
\setbeamercolor{block body alerted}{parent=normal text,use=block title alerted,bg=block title alerted.bg!10!bg}
\setbeamercolor{block body example}{parent=normal text,use=block title example,bg=block title example.bg!10!bg}

% \setbeamertemplate{itemize items}[default]


%%%%%%%%%%%%%%%%%%%%%%%%%%%%%%%%%%%%%%%%%%%%%%%%%%%%%%%%%%%%%%%%%%%%%%%%%%%%%%
% TikZ
%%%%%%%%%%%%%%%%%%%%%%%%%%%%%%%%%%%%%%%%%%%%%%%%%%%%%%%%%%%%%%%%%%%%%%%%%%%%%%

\tikzset
	{ > = Stealth
	}


%%%%%%%%%%%%%%%%%%%%%%%%%%%%%%%%%%%%%%%%%%%%%%%%%%%%%%%%%%%%%%%%%%%%%%%%%%%%%%
% general commands and styles
%%%%%%%%%%%%%%%%%%%%%%%%%%%%%%%%%%%%%%%%%%%%%%%%%%%%%%%%%%%%%%%%%%%%%%%%%%%%%%

% \delegateStyle and \inheritStyle command
% usage: \delegateStyle{… \inheritStyle{…} …}
% example: \(X_{\delegateStyle{\fbox{\inheritStyle{X}}}}\)
% Save the current style and regain it in the argument.
% This works both for math and text mode, and can be nested.
% Acknowledgments: Based on \ThisStyle and \SavedStyle from scalerel package.
\makeatletter
\newcommand*{\@inheritStyle@D}[1]{\(\displaystyle      #1\)}
\newcommand*{\@inheritStyle@T}[1]{\(\textstyle         #1\)}
\newcommand*{\@inheritStyle@S}[1]{\(\scriptstyle       #1\)}
\newcommand*{\@inheritStyle@s}[1]{\(\scriptscriptstyle #1\)}
\newcommand*{\@inheritStyle@t}[1]{#1}
\newcommand*{\inheritStyle}{\csname @inheritStyle@\@inheritStyleSwitch\endcsname}
\newcommand*{\delegateStyle}[1]{%
	\ifmmode%
		\mathchoice%
		{\edef\@inheritStyleSwitch{D}#1}%
		{\edef\@inheritStyleSwitch{T}#1}%
		{\edef\@inheritStyleSwitch{S}#1}%
		{\edef\@inheritStyleSwitch{s}#1}%
	\else%
		\edef\@inheritStyleSwitch{t}#1%
	\fi%
}
\makeatother


% \oalt command
% requires: \delegateStyle and \inheritStyle command
% usage: \oalt<…>[…]{…}{…} (cf. \alt)
% Behaves like \alt, but reserves space according to largest overlays.
% The optional argument defines the alignment inside the reserved space;
% it is one of c, l, r, s (cf. \makebox); the default is c.
\makeatletter
\newlength{\oalt@dp}
\newlength{\oalt@ht}
\newlength{\oalt@wd}
\newbox{\oalt@a}
\newbox{\oalt@b}
\newbox{\oalt@empty}
\newcommand<>*{\oalt}[3][c]{%
	\delegateStyle{%
		% based on \setto… in /usr/share/texmf-dist/tex/latex/base/latex.ltx
		\setbox\oalt@a\hbox{\inheritStyle{#2}}%
		\setbox\oalt@b\hbox{\inheritStyle{#3}}%
		\pgfmathsetlength{\oalt@dp}{max(\dp\oalt@a,\dp\oalt@b)}%
		\pgfmathsetlength{\oalt@ht}{max(\ht\oalt@a,\ht\oalt@b)}%
		\pgfmathsetlength{\oalt@wd}{max(\wd\oalt@a,\wd\oalt@b)}%
		\raisebox{0pt}[\oalt@ht][\oalt@dp]{%
			\makebox[\oalt@wd][#1]{%
				\alt#4{\unhbox\oalt@a}{\unhbox\oalt@b}%
			}%
		}%
		\setbox\oalt@a\box\oalt@empty%
		\setbox\oalt@b\box\oalt@empty%
	}%
}
\makeatother


% \otemporal command
% requires: \delegateStyle and \inheritStyle command
% usage: \otemporal<…>[…]{…}{…}{…} (cf. \temporal)
% Behaves like \temporal, but reserves space according to largest overlays.
% The optional argument defines the alignment inside the reserved space;
% it is one of c, l, r, s (cf. \makebox); the default is c.
\makeatletter
\newlength{\ot@dp}
\newlength{\ot@ht}
\newlength{\ot@wd}
\newbox{\ot@a}
\newbox{\ot@b}
\newbox{\ot@c}
\newbox{\ot@empty}
\newcommand<>*{\otemporal}[4][c]{%
	\delegateStyle{%
		% based on \setto… in /usr/share/texmf-dist/tex/latex/base/latex.ltx
		\setbox\ot@a\hbox{\inheritStyle{#2}}%
		\setbox\ot@b\hbox{\inheritStyle{#3}}%
		\setbox\ot@c\hbox{\inheritStyle{#4}}%
		\pgfmathsetlength{\ot@dp}{max(\dp\ot@a,\dp\ot@b,\dp\ot@c)}%
		\pgfmathsetlength{\ot@ht}{max(\ht\ot@a,\ht\ot@b,\ht\ot@c)}%
		\pgfmathsetlength{\ot@wd}{max(\wd\ot@a,\wd\ot@b,\wd\ot@c)}%
		\raisebox{0pt}[\ot@ht][\ot@dp]{%
			\makebox[\ot@wd][#1]{%
				\temporal#5{\unhbox\ot@a}{\unhbox\ot@b}{\unhbox\ot@c}%
			}%
		}%
		\setbox\ot@a\box\ot@empty%
		\setbox\ot@b\box\ot@empty%
		\setbox\ot@c\box\ot@empty%
	}%
}
\makeatother


% Resize delimiters like braces, brackets, etc.
% Parameters: size, left delimiter, formula, right delimiter
% Example: \delim2({\frac{1}{2}})
\newcommand*{\delim}[4]{%
	\ifcase#1%
		#2#3#4%
	\or%
		\bigl#2#3\bigr#4%
	\or%
		\Bigl#2#3\Bigr#4%
	\or%
		\biggl#2#3\biggr#4%
	\or%
		\Biggl#2#3\Biggr#4%
	\else%
		\left#2#3\right#4%
	\fi%
}


% similar to \fullcite, but using the formatting of \printbibliography
\newcommand*{\printfullcite}[1]{%
	\begin{refsection}%
		\nocite{#1}%
		\DeclareNameAlias{author}{first-last}%
		\printbibliography[heading = none]%
	\end{refsection}%
}


\colorlet{light alert}{HKS07K60}
\tikzset{alert.bg/.style={rounded corners, fill=light alert}}
\tikzset{every picture/.style={line cap=round, semithick}}
% http://tex.stackexchange.com/questions/6135/how-to-make-beamer-overlays-with-tikz-node
\tikzset{onslide/.code args={<#1>#2}{\only<#1>{\pgfkeysalso{#2}}}}
\tikzset{invisible/.code args={<#1>}{\alt<#1>{\pgfkeysalso{transparent}}{\pgfkeysalso{opaque}}}}
\tikzset{uncover/.code args={<#1>}{\alt<#1>{\pgfkeysalso{opaque}}{\pgfkeysalso{opacity=0.25}}}}
\tikzset{visible/.code args={<#1>}{\alt<#1>{\pgfkeysalso{opaque}}{\pgfkeysalso{transparent}}}}
\tikzset{vuncover/.code args=%
	{<#1><#2>}%
	{\alt<#1>%
		{\alt<#2>%
			{\pgfkeysalso{opaque}}%
			{\pgfkeysalso{opacity=0.25}}%
		}{\pgfkeysalso{transparent}}%
	}%
}

\newcommand<%
	>{\tikzhighlight}[2][]{%
	\delegateStyle{\alt#3%
		{\tikz[baseline=0, anchor=base, inner sep=0.2em, text height=, text depth=]{\node[alert.bg, #1]{\inheritStyle{#2}};}}%
		{\tikz[baseline=0, anchor=base, inner sep=0.2em, text height=, text depth=]{\node[#1, fill=none]{\inheritStyle{#2}};}}%
	}%
}

\newcommand{\mathhighlight}{\tikzhighlight}

\newcommand<>{\mhl}[2][]{\mathhighlight#3[inner sep=0.2em, #1]{#2}}


\newcommand<>{\inlineblock}[2][]{{%
	\usebeamercolor*[fg]{block body}%
	\tikzhighlight#3[fill=block body.bg, #1]{#2}%
}}


% a small letter s for plurals of abbreviations
\newcommand*{\s}{{\scriptsize s}\xspace}


\newcommand<>*{\sout}[2][opacity=0.75, ultra thick]{%
	\delegateStyle{%
		\tikz[baseline=0, anchor=base, inner sep=0, outer sep=0]{
			\useasboundingbox node (n) {\inheritStyle{#2}};
			\only#3{
				\node (h) {\inheritStyle{\ifmmode\mathstrut\else\strut\fi}};
				\draw[#1] (n.west |- {$(h.south)!0.5!(h.north)$}) -- (n.east |- {$(h.south)!0.5!(h.north)$});
			}
		}%
	}%
}


% tight style
% Sets outer sep to default inner sep and inner sep to 0.
% Use this style for nodes that are neither drawn nor filled to prevent
% unwanted growth of the bounding box.
\tikzset{tight/.style={inner sep=0, outer sep=0.3333em}}


% rounded tree edges style
% usage: rounded tree edges={⟨direction⟩}{⟨looseness⟩}{⟨strength⟩}
\tikzset{
	rounded tree edges/.style n args={3}{
	edge from parent path={
	let
		\n{direction}={#1},
		\n{looseness}={#2},
		\n{strength}={#3},
		\p1=(\tikzparentnode),
		\p2=(\tikzchildnode),
		\p3=(\n{direction}:1pt),
		\p4=(\x2 - \x1, \y2 - \y1),
		\n{dist}={veclen(\p4)},
		\p4=(\x4 / \n{dist}, \y4 / \n{dist}),
		\n{angle}={atan2(\y4, \x4)},
		\n{delta}={Mod(\n{angle} - \n{direction}, 360)},
		\n{delta}={\n{delta} > 180 ? \n{delta} - 360  : \n{delta}},
		\n{delta}={\n{delta} >  90 ?  180 - \n{delta} : \n{delta}},
		\n{delta}={\n{delta} < -90 ? -180 - \n{delta} : \n{delta}}
	in (\tikzparentnode) .. controls
		+(    \n{angle}+\n{strength}*\n{delta}:\n{looseness}*0.3915*\n{dist}) and
		+(180+\n{angle}-\n{strength}*\n{delta}:\n{looseness}*0.3915*\n{dist}) ..
		(\tikzchildnode)
	}
	}
}


% Tear out snippets from PDFs.
% Usage: \tear[…]{file.pdf}
% The optional parameter is the same as for \includegraphics.
% Useful Arguments:
%   * page=‹pagenumber›
%   * trim=‹left› ‹bottom› ‹right› ‹top›
%   * width=0.98\linewidth
\newcommand*{\tear}[2][]{%
	\begin{tikzpicture}
		\node
			[ blur shadow
			, clip
			, decorate
			, decoration=random steps
			, draw
			, inner sep=0
			, preaction={fill=white}% hide the shadow if paper is transparent
			] {\includegraphics[#1]{#2}};
	\end{tikzpicture}%
}


\makeatletter
\newcommand*{\timeline}[3][0]{%
	\ifcsname timeline@cmd@#3\endcsname%
		\@timeline[#1]{#2}{#3}%
		\PackageWarning{timeline}{redefining timeline \@backslashchar\string#3}%
	\else%
		\ifcsname#3\endcsname%
			\errmessage{Command \@backslashchar\string#3 already defined}%
		\else%
			\@timeline[#1]{#2}{#3}%
		\fi%
	\fi%
}%
\newcommand*{\@timeline}[3][0]{%
	% mark command as timeline command – they can be overwritten
	\expandafter\def\csname timeline@cmd@#3\endcsname{}%
	\setcounter{@timeline}{#1}%
	\def\timeline@cmd{#3}%
	\timeline@reset%
	\timeline@append{0}%
	\@tfor\timeline@next:=#2\do{%
		\if\timeline@next+%
			\stepcounter{@timeline}%
			\timeline@append{,\the@timeline}%
		\else\if\timeline@next-%
			\stepcounter{@timeline}%
		\else%
			%\timeline@append{\timeline@next}%
			\GenericError{}{\protect\timeline: ignoring unknown character: \timeline@next}%
		\fi\fi%
	}%
}%
% \newcommand*{\tl}[1]{%
% 	\ifcsname timeline@cmd@#1\endcsname%
% 		\csname timeline@cmd@#1\endcsname%
% 	\else%
% 		0%
% 		%\GenericError{}{\protect\tl: timeline not defined: #1}%
% 	\fi%
% }%
\newcounter{@timeline}%
\def\timeline@reset{%
	\expandafter\def\csname\timeline@cmd\endcsname{}%
}%
\def\timeline@append#1{%
	\expandafter\edef\csname\timeline@cmd\endcsname{%
		\csname\timeline@cmd\endcsname#1%
	}%
}%
\makeatother


\newcommand*{\xminus}[1]{%
	\mathrel{\tikz[baseline={([yshift=-0.25em]n.south)}, inner sep=0, outer sep=0.2em]{%
		\node (n) {\(\scriptstyle #1\)};
		\draw (n.south west) -- (n.south east);
	}}%
}
\newcommand*{\tikzrightarrow}[1]{%
	\mathrel{\tikz[baseline={([yshift=-0.25em]n.south)}, inner sep=0, outer sep=0.2em]{%
		\node (n) {\(\scriptstyle #1\)};
		\draw[->, > = Computer Modern Rightarrow, line width = 0.4pt] (n.south west) -- (n.south east);
	}}%
}


%%%%%%%%%%%%%%%%%%%%%%%%%%%%%%%%%%%%%%%%%%%%%%%%%%%%%%%%%%%%%%%%%%%%%%%%%%%%%%
% document specific commands
%%%%%%%%%%%%%%%%%%%%%%%%%%%%%%%%%%%%%%%%%%%%%%%%%%%%%%%%%%%%%%%%%%%%%%%%%%%%%%

\newcommand<>*{\mycite}[1]{\uncover#2{{\color{HKS57K100}[\cite{#1}]}}}


\newcommand{\statetree}[1]{
	\tikz
	[ anchor=base
	, baseline=(current bounding box.center)
	, level distance=2em
	, sibling distance=2em
	]{
		\matrix
		[ draw=nt
		, edge from parent/.style={draw=black}
		, inner sep=0
		, nodes={inner sep=0.2em, rounded corners=0}
		, rounded corners
		] {#1\\}
	}
}


\newcommand*{\mylargeleaf}[1]{{\LARGE\color{HKS41K70}#1}}

\definecolor{state s}{named}{HKS57K80}
\definecolor{state t}{named}{HKS41K70}
\newcommand*{\stateS}[1]{{\color{state s}#1}}
\newcommand*{\stateT}[1]{{\color{state t}#1}}

\tikzset{
	subtree/.style =
		{ fill=lightgray
		, inner sep=0.02em
		, isosceles triangle apex angle=60
		, shape=isosceles triangle
		, shape border rotate=90
		}
	, state/.style = {circle, draw, inner sep=0.1em}
	, trans/.style = {rectangle, draw}
}

\newcommand*{\srBool}{\mathbb{B}}
\newcommand*{\srProb}{ℙ}


%%%%%%%%%%%%%%%%%%%%%%%%%%%%%%%%%%%%%%%%%%%%%%%%%%%%%%%%%%%%%%%%%%%%%%%%%%%%%%
% commands for specific notations
%%%%%%%%%%%%%%%%%%%%%%%%%%%%%%%%%%%%%%%%%%%%%%%%%%%%%%%%%%%%%%%%%%%%%%%%%%%%%%

\DeclareMathOperator*{\argmax}{argmax}

\newcommand*{\cardinality}[1]{\lvert#1\rvert}
\newcommand*{\corpussize}[1]{\lvert#1\rvert}

\DeclareMathOperator{\crispOp}{crisp}
\newcommand*        {\crisp}[2][0]{\crispOp\delim{#1}({#2})}

\DeclareMathOperator{\lhsOp}{lhs}
\newcommand*{\lhs}[1]{\lhsOp(#1)}

\DeclareMathOperator{\lklhdOp}{L}
\newcommand*{\lklhd}[2]{\lklhdOp(#1 ∣ #2)}

\DeclareMathOperator{\mleOp}{mle}
\newcommand*{\mle}[2][]{%
	\ifthenelse{\isempty{#1}}{%
		\mleOp(#2)%
	}{%
		\mleOp_{#1}(#2)%
	}%
}

\DeclareMathOperator{\mrg}{merge}

% CVD: color vision deficiencies
\definecolor{CVD light red}   {HTML}{FF8080}
\definecolor{CVD light yellow}{HTML}{FFFF80}
\definecolor{CVD light green} {HTML}{40FFC0}

\definecolor{nt}{named}{HKS41K70}
\newcommand*{\nt}[1]{{\color{nt}#1}}

% set of all probability distributions over #1
\DeclareMathOperator{\pdsOp}{Pd}
\newcommand*{\pds}[1]{\pdsOp(#1)}

\DeclareMathOperator{\positionsOp}{pos}
\newcommand*{\positions}[1]{\positionsOp(#1)}

\DeclareMathOperator{\rankOp}{rk}
\newcommand*{\rank}[1]{\rankOp(#1)}

\DeclareMathOperator{\runsOp}{run}
\newcommand*{\runs}[2][]{%
	\ifthenelse%
		{\isempty{#1}}%
		{\runsOp(#2)}%
		{\runsOp_{#1}(#2)}%
}

\newcommand*{\semantics}[1]{⟦#1⟧}

\DeclareMathOperator{\splt}{split}

\newcommand*{\subtree}[2]{#1|_{#2}}

\DeclareMathOperator{\supportOp}{supp}
\newcommand*{\support}[1]{\supportOp(#1)}

\newcommand*{\symId}{\textsc{\color{gray}Id}}
\newcommand*{\symCons}{\textsc{\color{gray}Cons}}
\newcommand*{\symFlip}{\textsc{\color{gray}Flip}}
\newcommand*{\symNull}{\textsc{\color{gray}Null}}
\newcommand*{\symNullR}{\textsc{\color{gray}N\(\overline{\textsc{ull}}\)}}
\newcommand*{\symSnoc}{\textsc{\color{gray}Snoc}}

\newcommand*{\transWTA}[4][]{#3 \xrightarrow{#1} #2(#4)}

\DeclareMathOperator{\uniqueRunOp}{r}
\newcommand*{\uniqueRun}[2][]{%
	\ifthenelse%
		{\isempty{#1}}%
		{\uniqueRunOp^{#2}}%
		{\uniqueRunOp_{\!#1}^{#2}}%
}

\DeclareMathOperator{\treesOp}{T}
\newcommand*{\trees}[2][]{%
	\ifthenelse%
		{\isempty{#1}}%
		{\treesOp_{\!#2}}%
		{\treesOp_{\!#2}(#1)}%
}
\DeclareMathOperator{\treesUOp}{U}
\newcommand*{\treesU}[2][]{%
	\ifthenelse%
		{\isempty{#1}}%
		{\treesUOp_{#2}}%
		{\treesUOp_{#2}(#1)}%
}


%%%%%%%%%%%%%%%%%%%%%%%%%%%%%%%%%%%%%%%%%%%%%%%%%%%%%%%%%%%%%%%%%%%%%%%%%%%%%%
% metadata
%%%%%%%%%%%%%%%%%%%%%%%%%%%%%%%%%%%%%%%%%%%%%%%%%%%%%%%%%%%%%%%%%%%%%%%%%%%%%%

\ifstandalonebeamer\else
	\title[Defense of Dissertation]{A Formal View on Training of Weighted Tree Automata by Likelihood-Driven State Splitting and Merging}
	\subtitle{Defense of Dissertation}
\fi
\author{Toni Dietze}
\institute[TU Dresden]{%
	\href{https://www.orchid.inf.tu-dresden.de/index.en/}{Chair for Foundations of Programming}
\\	\href{https://tu-dresden.de/ing/informatik/thi}{Institute of Theoretical Computer Science}
\\	\href{https://tu-dresden.de/ing/informatik}{Faculty of Computer Science}
\\	\href{https://tu-dresden.de/}{Technische Universität Dresden}
\\	01062 Dresden, Germany
}
\date[2018-09-27]{September 27, 2018}

\begin{document}
\begin{standaloneframe}{\jobname}
	\begin{tikzpicture}
		[ baseline = (base.base)
		, level distance=2em
		, level 1/.style={sibling distance=3em}
		, inner sep = 0
		, outer sep = 0.25em
		]
		% δ(α, γ(α), β)
		\node {\(σ\)}
			child {node {\(α\)}
			}
			child {node {\(γ\)}
				child {node {\(α\)}
				}
			}
			child {node (base) {\(β\)}
			};
	\end{tikzpicture}
	\hfill
	\begin{tikzpicture}[baseline = (base.base)]
		\node[draw, shape = single arrow] (base) {\strut binarize};
	\end{tikzpicture}
	\hfill
	\begin{tikzpicture}
		[ baseline = (base.base)
		, level distance=2em
		, level 2/.style={sibling distance=7em}
		, level 4/.style={sibling distance=3em}
		, inner sep = 0
		, outer sep = 0.25em
		]
		% σ(\symId(\symCons(α(\symId(\symNull)), \symCons(γ(\symId(\symCons(α(\symId(\symNull)),\symNull))),\symCons(β(\symId(\symNull)), \symNull)))))
		\node {\(σ\)}
			child {node[gray] {\(\symCons\)}
				child {node {\(α\)}
					child {node[gray] (base) {\(\symNull\)}
					}
				}
				child {node[gray] {\(\symCons\)}
					child {node {\(γ\)}
						child {node[gray] {\(\symCons\)}
							child {node {\(α\)}
								child {node[gray] {\(\symNull\)}
								}
							}
							child {node[gray] {\(\symNull\)}
							}
						}
					}
					child {node[gray] {\(\symCons\)}
						child {node {\(β\)}
							child {node[gray] {\(\symNull\)}
							}
						}
						child {node[gray] {\(\symNull\)}
						}
					}
				}
			};
	\end{tikzpicture}
	\begin{block}<2->{I analyzed three binarization strategies motivated by}
		\printfullcite{2005MatsuzakiMiyaoTsujii}
	\end{block}
\end{standaloneframe}
\end{document}

\end{frame}


\section{Binarization – Results}

\begin{frame}[t]{\secname}
	\centering
	% SPDX-License-Identifier: CC-BY-4.0
% Copyright 2018 Toni Dietze
\documentclass[beamer]{standalone}
% SPDX-License-Identifier: CC-BY-4.0 OR MIT-0
% Copyright 2018 Toni Dietze
%
\usefonttheme{professionalfonts}

% LuaLaTeX specific packages
\usepackage{fontspec}
	\defaultfontfeatures{Ligatures=TeX}
\usepackage{polyglossia}
	\setdefaultlanguage{english}
\usepackage{amsmath}  % has to be loaded before unicode-math
\usepackage[math-style=ISO]{unicode-math}
	\setmathfont{Latin Modern Math}
% 	\setmathfont[range={\mathcal,\mathbfcal},StylisticSet=1]{xits-math.otf}
% 	\setmathfont[range={"029F5}]{XITS Math}  % ⧵
% 	\setmathfont[range={\mathscr,\mathbfscr},StylisticSet=1]{Latin Modern Math}  % make \mathscr use the correct font

\usepackage[noend]{algpseudocode}
	\algrenewcommand\algorithmicrequire{\textbf{Input:}}
	\algrenewcommand\algorithmicensure{\textbf{Output:}}
\usepackage[backend=biber, maxbibnames=42, maxcitenames=42, sorting=ynt, style=authoryear]{biblatex}
\usepackage{csquotes}
\usepackage{mathtools}
\usepackage{media9}
\usepackage{scalerel}
\usepackage{standalone}
\usepackage{tikz}
	\usetikzlibrary{arrows.meta}
	\usetikzlibrary{backgrounds}
	\usetikzlibrary{calc}
	\usetikzlibrary{decorations}
	\usetikzlibrary{decorations.pathmorphing}
	\usetikzlibrary{decorations.pathreplacing}
	\usetikzlibrary{fadings}
	\usetikzlibrary{fit}
	\usetikzlibrary{graphs}
	\usetikzlibrary{graphdrawing}
	\usetikzlibrary{intersections}
	\usetikzlibrary{positioning}
	\usetikzlibrary{quotes}
	\usetikzlibrary{shadows.blur}
	\usetikzlibrary{shapes.arrows}
	\usetikzlibrary{shapes.geometric}
	\usegdlibrary{trees}
\usepackage{xifthen}
\usepackage{xspace}

\usepackage{pgfplots}
	\pgfplotsset
		{ compat = 1.15
		, /pgf/number format/1000 sep = {\,}
		, /pgf/number format/assume math mode = true
		, every axis plot/.append style =
			{ mark options = {fill opacity = 0.25}
			}
		}
	\usepgfplotslibrary{groupplots}
\usepackage{pgfplotstable}

\hypersetup
	{ bookmarksopen
	, pdflang = en
	, unicode
	}


%%%%%%%%%%%%%%%%%%%%%%%%%%%%%%%%%%%%%%%%%%%%%%%%%%%%%%%%%%%%%%%%%%%%%%%%%%%%%%


% always show bad boxes
%\overfullrule=1em


%%%%%%%%%%%%%%%%%%%%%%%%%%%%%%%%%%%%%%%%%%%%%%%%%%%%%%%%%%%%%%%%%%%%%%%%%%%%%%
% biblatex
%%%%%%%%%%%%%%%%%%%%%%%%%%%%%%%%%%%%%%%%%%%%%%%%%%%%%%%%%%%%%%%%%%%%%%%%%%%%%%

\addbibresource{slides-dissertation-defense.bib}
% \renewcommand*{\finalnamedelim}{\addcomma\space}
% \setlength{\bibitemsep}{1em}
% 
\AtEveryBibitem{% Clean up the bibtex rather than editing it
 \clearlist{address}
 \clearfield{date}
 \clearfield{eprint}
 \clearfield{isbn}
 \clearfield{issn}
 \clearlist{language}
 \clearlist{location}
 \clearfield{month}
 \clearfield{series}
%  \clearfield{url}
%  \clearfield{doi}
 \clearfield{organization}

%  \ifentrytype{book}{}{% Remove stuff except for books
%   \clearfield{booktitle}
%   \clearfield{pages}
  \clearlist{publisher}
  \clearname{editor}
%  }
}
% do not print url if doi is present
% http://tex.stackexchange.com/questions/154864/biblatex-use-doi-only-if-there-is-no-url
\DeclareSourcemap{
	\maps[datatype=bibtex]{
		\map{
			\step[fieldsource=doi,final]
			\step[fieldset=url,null]
}	}	}
%
% remove qoutes around titles
\DeclareFieldFormat
	[article,inbook,incollection,inproceedings,patent,thesis,unpublished]
	{title}{#1\isdot}
% 
% \DeclareFieldFormat{url}{\mkbibacro{URL}\addcolon\addnbspace\url{#1}}
% 
% \DeclareNameAlias{sortname}{first-last}
% 
\renewbibmacro{in:}{\ifentrytype{article}{}{}}


%%%%%%%%%%%%%%%%%%%%%%%%%%%%%%%%%%%%%%%%%%%%%%%%%%%%%%%%%%%%%%%%%%%%%%%%%%%%%%
% beamer
%%%%%%%%%%%%%%%%%%%%%%%%%%%%%%%%%%%%%%%%%%%%%%%%%%%%%%%%%%%%%%%%%%%%%%%%%%%%%%

\useoutertheme{infolines}
\makeatletter
% based on
% /usr/share/texmf-dist/tex/latex/beamer/beamerouterthemeinfolines.sty
\setbeamertemplate{footline}
{%
	\leavevmode%
	\hbox{%
	\begin{beamercolorbox}[wd=.333333\paperwidth,ht=2.25ex,dp=1ex,center]{author in head/foot}%
		\usebeamerfont{author in head/foot}\insertshortauthor\expandafter\beamer@ifempty\expandafter{\beamer@shortinstitute}{}{~~(\insertshortinstitute)}
	\end{beamercolorbox}%
	\begin{beamercolorbox}[wd=.333333\paperwidth,ht=2.25ex,dp=1ex,center]{title in head/foot}%
		\usebeamerfont{title in head/foot}\insertshorttitle
	\end{beamercolorbox}%
	\begin{beamercolorbox}[wd=.333333\paperwidth,ht=2.25ex,dp=1ex,right]{date in head/foot}%
		\usebeamerfont{date in head/foot}%
		\hfill\insertshortdate\hfill\hfill%
		%\hspace*{2ex}%
		%\insertshortdate%
		%\hspace{0pt plus 1 filll}%
		%(\insertframenumber.\insertoverlaynumber{} / \insertmainframenumber)%
		%\hspace{0pt plus 1 filll}%
		\phantom{000}\llap{\insertpagenumber} / \insertpresentationendpage%
		\hspace*{2ex}%
	\end{beamercolorbox}}%
	\vskip0pt%
}
\makeatother
\useinnertheme{circles}
\beamertemplatenavigationsymbolsempty
\setbeamertemplate{bibliography item}{}
\setbeamertemplate{headline}[default]

\input{tudcolors.tex}
\setbeamercolor*{alerted text}{fg=HKS07K100}
\usecolortheme[named=HKS41K100]{structure}

\setbeamercolor*{palette primary}{use=structure,fg=white,bg=structure.fg}
\setbeamercolor*{palette secondary}{use=structure,fg=white,bg=structure.fg!80}
\setbeamercolor*{palette tertiary}{use=structure,fg=white,bg=structure.fg!60}
\setbeamercolor*{palette quaternary}{fg=white,bg=black}

\setbeamercolor*{sidebar}{use=structure,bg=structure.fg}

\setbeamercolor*{palette sidebar primary}{use=structure,fg=structure.fg!20}
\setbeamercolor*{palette sidebar secondary}{fg=white}
\setbeamercolor*{palette sidebar tertiary}{use=structure,fg=structure.fg!40}
\setbeamercolor*{palette sidebar quaternary}{fg=white}

\setbeamercolor*{titlelike}{parent=palette primary}

\setbeamercolor*{separation line}{}
\setbeamercolor*{fine separation line}{}

\setbeamercolor{block title}{use=structure,fg=white,bg=structure.fg}
\setbeamercolor{block title alerted}{use=alerted text,fg=white,bg=alerted text.fg!75!black}
\setbeamercolor{block title example}{use=example text,fg=white,bg=example text.fg!75!black}

\setbeamercolor{block body}{parent=normal text,use=block title,bg=block title.bg!10!bg}
\setbeamercolor{block body alerted}{parent=normal text,use=block title alerted,bg=block title alerted.bg!10!bg}
\setbeamercolor{block body example}{parent=normal text,use=block title example,bg=block title example.bg!10!bg}

% \setbeamertemplate{itemize items}[default]


%%%%%%%%%%%%%%%%%%%%%%%%%%%%%%%%%%%%%%%%%%%%%%%%%%%%%%%%%%%%%%%%%%%%%%%%%%%%%%
% TikZ
%%%%%%%%%%%%%%%%%%%%%%%%%%%%%%%%%%%%%%%%%%%%%%%%%%%%%%%%%%%%%%%%%%%%%%%%%%%%%%

\tikzset
	{ > = Stealth
	}


%%%%%%%%%%%%%%%%%%%%%%%%%%%%%%%%%%%%%%%%%%%%%%%%%%%%%%%%%%%%%%%%%%%%%%%%%%%%%%
% general commands and styles
%%%%%%%%%%%%%%%%%%%%%%%%%%%%%%%%%%%%%%%%%%%%%%%%%%%%%%%%%%%%%%%%%%%%%%%%%%%%%%

% \delegateStyle and \inheritStyle command
% usage: \delegateStyle{… \inheritStyle{…} …}
% example: \(X_{\delegateStyle{\fbox{\inheritStyle{X}}}}\)
% Save the current style and regain it in the argument.
% This works both for math and text mode, and can be nested.
% Acknowledgments: Based on \ThisStyle and \SavedStyle from scalerel package.
\makeatletter
\newcommand*{\@inheritStyle@D}[1]{\(\displaystyle      #1\)}
\newcommand*{\@inheritStyle@T}[1]{\(\textstyle         #1\)}
\newcommand*{\@inheritStyle@S}[1]{\(\scriptstyle       #1\)}
\newcommand*{\@inheritStyle@s}[1]{\(\scriptscriptstyle #1\)}
\newcommand*{\@inheritStyle@t}[1]{#1}
\newcommand*{\inheritStyle}{\csname @inheritStyle@\@inheritStyleSwitch\endcsname}
\newcommand*{\delegateStyle}[1]{%
	\ifmmode%
		\mathchoice%
		{\edef\@inheritStyleSwitch{D}#1}%
		{\edef\@inheritStyleSwitch{T}#1}%
		{\edef\@inheritStyleSwitch{S}#1}%
		{\edef\@inheritStyleSwitch{s}#1}%
	\else%
		\edef\@inheritStyleSwitch{t}#1%
	\fi%
}
\makeatother


% \oalt command
% requires: \delegateStyle and \inheritStyle command
% usage: \oalt<…>[…]{…}{…} (cf. \alt)
% Behaves like \alt, but reserves space according to largest overlays.
% The optional argument defines the alignment inside the reserved space;
% it is one of c, l, r, s (cf. \makebox); the default is c.
\makeatletter
\newlength{\oalt@dp}
\newlength{\oalt@ht}
\newlength{\oalt@wd}
\newbox{\oalt@a}
\newbox{\oalt@b}
\newbox{\oalt@empty}
\newcommand<>*{\oalt}[3][c]{%
	\delegateStyle{%
		% based on \setto… in /usr/share/texmf-dist/tex/latex/base/latex.ltx
		\setbox\oalt@a\hbox{\inheritStyle{#2}}%
		\setbox\oalt@b\hbox{\inheritStyle{#3}}%
		\pgfmathsetlength{\oalt@dp}{max(\dp\oalt@a,\dp\oalt@b)}%
		\pgfmathsetlength{\oalt@ht}{max(\ht\oalt@a,\ht\oalt@b)}%
		\pgfmathsetlength{\oalt@wd}{max(\wd\oalt@a,\wd\oalt@b)}%
		\raisebox{0pt}[\oalt@ht][\oalt@dp]{%
			\makebox[\oalt@wd][#1]{%
				\alt#4{\unhbox\oalt@a}{\unhbox\oalt@b}%
			}%
		}%
		\setbox\oalt@a\box\oalt@empty%
		\setbox\oalt@b\box\oalt@empty%
	}%
}
\makeatother


% \otemporal command
% requires: \delegateStyle and \inheritStyle command
% usage: \otemporal<…>[…]{…}{…}{…} (cf. \temporal)
% Behaves like \temporal, but reserves space according to largest overlays.
% The optional argument defines the alignment inside the reserved space;
% it is one of c, l, r, s (cf. \makebox); the default is c.
\makeatletter
\newlength{\ot@dp}
\newlength{\ot@ht}
\newlength{\ot@wd}
\newbox{\ot@a}
\newbox{\ot@b}
\newbox{\ot@c}
\newbox{\ot@empty}
\newcommand<>*{\otemporal}[4][c]{%
	\delegateStyle{%
		% based on \setto… in /usr/share/texmf-dist/tex/latex/base/latex.ltx
		\setbox\ot@a\hbox{\inheritStyle{#2}}%
		\setbox\ot@b\hbox{\inheritStyle{#3}}%
		\setbox\ot@c\hbox{\inheritStyle{#4}}%
		\pgfmathsetlength{\ot@dp}{max(\dp\ot@a,\dp\ot@b,\dp\ot@c)}%
		\pgfmathsetlength{\ot@ht}{max(\ht\ot@a,\ht\ot@b,\ht\ot@c)}%
		\pgfmathsetlength{\ot@wd}{max(\wd\ot@a,\wd\ot@b,\wd\ot@c)}%
		\raisebox{0pt}[\ot@ht][\ot@dp]{%
			\makebox[\ot@wd][#1]{%
				\temporal#5{\unhbox\ot@a}{\unhbox\ot@b}{\unhbox\ot@c}%
			}%
		}%
		\setbox\ot@a\box\ot@empty%
		\setbox\ot@b\box\ot@empty%
		\setbox\ot@c\box\ot@empty%
	}%
}
\makeatother


% Resize delimiters like braces, brackets, etc.
% Parameters: size, left delimiter, formula, right delimiter
% Example: \delim2({\frac{1}{2}})
\newcommand*{\delim}[4]{%
	\ifcase#1%
		#2#3#4%
	\or%
		\bigl#2#3\bigr#4%
	\or%
		\Bigl#2#3\Bigr#4%
	\or%
		\biggl#2#3\biggr#4%
	\or%
		\Biggl#2#3\Biggr#4%
	\else%
		\left#2#3\right#4%
	\fi%
}


% similar to \fullcite, but using the formatting of \printbibliography
\newcommand*{\printfullcite}[1]{%
	\begin{refsection}%
		\nocite{#1}%
		\DeclareNameAlias{author}{first-last}%
		\printbibliography[heading = none]%
	\end{refsection}%
}


\colorlet{light alert}{HKS07K60}
\tikzset{alert.bg/.style={rounded corners, fill=light alert}}
\tikzset{every picture/.style={line cap=round, semithick}}
% http://tex.stackexchange.com/questions/6135/how-to-make-beamer-overlays-with-tikz-node
\tikzset{onslide/.code args={<#1>#2}{\only<#1>{\pgfkeysalso{#2}}}}
\tikzset{invisible/.code args={<#1>}{\alt<#1>{\pgfkeysalso{transparent}}{\pgfkeysalso{opaque}}}}
\tikzset{uncover/.code args={<#1>}{\alt<#1>{\pgfkeysalso{opaque}}{\pgfkeysalso{opacity=0.25}}}}
\tikzset{visible/.code args={<#1>}{\alt<#1>{\pgfkeysalso{opaque}}{\pgfkeysalso{transparent}}}}
\tikzset{vuncover/.code args=%
	{<#1><#2>}%
	{\alt<#1>%
		{\alt<#2>%
			{\pgfkeysalso{opaque}}%
			{\pgfkeysalso{opacity=0.25}}%
		}{\pgfkeysalso{transparent}}%
	}%
}

\newcommand<%
	>{\tikzhighlight}[2][]{%
	\delegateStyle{\alt#3%
		{\tikz[baseline=0, anchor=base, inner sep=0.2em, text height=, text depth=]{\node[alert.bg, #1]{\inheritStyle{#2}};}}%
		{\tikz[baseline=0, anchor=base, inner sep=0.2em, text height=, text depth=]{\node[#1, fill=none]{\inheritStyle{#2}};}}%
	}%
}

\newcommand{\mathhighlight}{\tikzhighlight}

\newcommand<>{\mhl}[2][]{\mathhighlight#3[inner sep=0.2em, #1]{#2}}


\newcommand<>{\inlineblock}[2][]{{%
	\usebeamercolor*[fg]{block body}%
	\tikzhighlight#3[fill=block body.bg, #1]{#2}%
}}


% a small letter s for plurals of abbreviations
\newcommand*{\s}{{\scriptsize s}\xspace}


\newcommand<>*{\sout}[2][opacity=0.75, ultra thick]{%
	\delegateStyle{%
		\tikz[baseline=0, anchor=base, inner sep=0, outer sep=0]{
			\useasboundingbox node (n) {\inheritStyle{#2}};
			\only#3{
				\node (h) {\inheritStyle{\ifmmode\mathstrut\else\strut\fi}};
				\draw[#1] (n.west |- {$(h.south)!0.5!(h.north)$}) -- (n.east |- {$(h.south)!0.5!(h.north)$});
			}
		}%
	}%
}


% tight style
% Sets outer sep to default inner sep and inner sep to 0.
% Use this style for nodes that are neither drawn nor filled to prevent
% unwanted growth of the bounding box.
\tikzset{tight/.style={inner sep=0, outer sep=0.3333em}}


% rounded tree edges style
% usage: rounded tree edges={⟨direction⟩}{⟨looseness⟩}{⟨strength⟩}
\tikzset{
	rounded tree edges/.style n args={3}{
	edge from parent path={
	let
		\n{direction}={#1},
		\n{looseness}={#2},
		\n{strength}={#3},
		\p1=(\tikzparentnode),
		\p2=(\tikzchildnode),
		\p3=(\n{direction}:1pt),
		\p4=(\x2 - \x1, \y2 - \y1),
		\n{dist}={veclen(\p4)},
		\p4=(\x4 / \n{dist}, \y4 / \n{dist}),
		\n{angle}={atan2(\y4, \x4)},
		\n{delta}={Mod(\n{angle} - \n{direction}, 360)},
		\n{delta}={\n{delta} > 180 ? \n{delta} - 360  : \n{delta}},
		\n{delta}={\n{delta} >  90 ?  180 - \n{delta} : \n{delta}},
		\n{delta}={\n{delta} < -90 ? -180 - \n{delta} : \n{delta}}
	in (\tikzparentnode) .. controls
		+(    \n{angle}+\n{strength}*\n{delta}:\n{looseness}*0.3915*\n{dist}) and
		+(180+\n{angle}-\n{strength}*\n{delta}:\n{looseness}*0.3915*\n{dist}) ..
		(\tikzchildnode)
	}
	}
}


% Tear out snippets from PDFs.
% Usage: \tear[…]{file.pdf}
% The optional parameter is the same as for \includegraphics.
% Useful Arguments:
%   * page=‹pagenumber›
%   * trim=‹left› ‹bottom› ‹right› ‹top›
%   * width=0.98\linewidth
\newcommand*{\tear}[2][]{%
	\begin{tikzpicture}
		\node
			[ blur shadow
			, clip
			, decorate
			, decoration=random steps
			, draw
			, inner sep=0
			, preaction={fill=white}% hide the shadow if paper is transparent
			] {\includegraphics[#1]{#2}};
	\end{tikzpicture}%
}


\makeatletter
\newcommand*{\timeline}[3][0]{%
	\ifcsname timeline@cmd@#3\endcsname%
		\@timeline[#1]{#2}{#3}%
		\PackageWarning{timeline}{redefining timeline \@backslashchar\string#3}%
	\else%
		\ifcsname#3\endcsname%
			\errmessage{Command \@backslashchar\string#3 already defined}%
		\else%
			\@timeline[#1]{#2}{#3}%
		\fi%
	\fi%
}%
\newcommand*{\@timeline}[3][0]{%
	% mark command as timeline command – they can be overwritten
	\expandafter\def\csname timeline@cmd@#3\endcsname{}%
	\setcounter{@timeline}{#1}%
	\def\timeline@cmd{#3}%
	\timeline@reset%
	\timeline@append{0}%
	\@tfor\timeline@next:=#2\do{%
		\if\timeline@next+%
			\stepcounter{@timeline}%
			\timeline@append{,\the@timeline}%
		\else\if\timeline@next-%
			\stepcounter{@timeline}%
		\else%
			%\timeline@append{\timeline@next}%
			\GenericError{}{\protect\timeline: ignoring unknown character: \timeline@next}%
		\fi\fi%
	}%
}%
% \newcommand*{\tl}[1]{%
% 	\ifcsname timeline@cmd@#1\endcsname%
% 		\csname timeline@cmd@#1\endcsname%
% 	\else%
% 		0%
% 		%\GenericError{}{\protect\tl: timeline not defined: #1}%
% 	\fi%
% }%
\newcounter{@timeline}%
\def\timeline@reset{%
	\expandafter\def\csname\timeline@cmd\endcsname{}%
}%
\def\timeline@append#1{%
	\expandafter\edef\csname\timeline@cmd\endcsname{%
		\csname\timeline@cmd\endcsname#1%
	}%
}%
\makeatother


\newcommand*{\xminus}[1]{%
	\mathrel{\tikz[baseline={([yshift=-0.25em]n.south)}, inner sep=0, outer sep=0.2em]{%
		\node (n) {\(\scriptstyle #1\)};
		\draw (n.south west) -- (n.south east);
	}}%
}
\newcommand*{\tikzrightarrow}[1]{%
	\mathrel{\tikz[baseline={([yshift=-0.25em]n.south)}, inner sep=0, outer sep=0.2em]{%
		\node (n) {\(\scriptstyle #1\)};
		\draw[->, > = Computer Modern Rightarrow, line width = 0.4pt] (n.south west) -- (n.south east);
	}}%
}


%%%%%%%%%%%%%%%%%%%%%%%%%%%%%%%%%%%%%%%%%%%%%%%%%%%%%%%%%%%%%%%%%%%%%%%%%%%%%%
% document specific commands
%%%%%%%%%%%%%%%%%%%%%%%%%%%%%%%%%%%%%%%%%%%%%%%%%%%%%%%%%%%%%%%%%%%%%%%%%%%%%%

\newcommand<>*{\mycite}[1]{\uncover#2{{\color{HKS57K100}[\cite{#1}]}}}


\newcommand{\statetree}[1]{
	\tikz
	[ anchor=base
	, baseline=(current bounding box.center)
	, level distance=2em
	, sibling distance=2em
	]{
		\matrix
		[ draw=nt
		, edge from parent/.style={draw=black}
		, inner sep=0
		, nodes={inner sep=0.2em, rounded corners=0}
		, rounded corners
		] {#1\\}
	}
}


\newcommand*{\mylargeleaf}[1]{{\LARGE\color{HKS41K70}#1}}

\definecolor{state s}{named}{HKS57K80}
\definecolor{state t}{named}{HKS41K70}
\newcommand*{\stateS}[1]{{\color{state s}#1}}
\newcommand*{\stateT}[1]{{\color{state t}#1}}

\tikzset{
	subtree/.style =
		{ fill=lightgray
		, inner sep=0.02em
		, isosceles triangle apex angle=60
		, shape=isosceles triangle
		, shape border rotate=90
		}
	, state/.style = {circle, draw, inner sep=0.1em}
	, trans/.style = {rectangle, draw}
}

\newcommand*{\srBool}{\mathbb{B}}
\newcommand*{\srProb}{ℙ}


%%%%%%%%%%%%%%%%%%%%%%%%%%%%%%%%%%%%%%%%%%%%%%%%%%%%%%%%%%%%%%%%%%%%%%%%%%%%%%
% commands for specific notations
%%%%%%%%%%%%%%%%%%%%%%%%%%%%%%%%%%%%%%%%%%%%%%%%%%%%%%%%%%%%%%%%%%%%%%%%%%%%%%

\DeclareMathOperator*{\argmax}{argmax}

\newcommand*{\cardinality}[1]{\lvert#1\rvert}
\newcommand*{\corpussize}[1]{\lvert#1\rvert}

\DeclareMathOperator{\crispOp}{crisp}
\newcommand*        {\crisp}[2][0]{\crispOp\delim{#1}({#2})}

\DeclareMathOperator{\lhsOp}{lhs}
\newcommand*{\lhs}[1]{\lhsOp(#1)}

\DeclareMathOperator{\lklhdOp}{L}
\newcommand*{\lklhd}[2]{\lklhdOp(#1 ∣ #2)}

\DeclareMathOperator{\mleOp}{mle}
\newcommand*{\mle}[2][]{%
	\ifthenelse{\isempty{#1}}{%
		\mleOp(#2)%
	}{%
		\mleOp_{#1}(#2)%
	}%
}

\DeclareMathOperator{\mrg}{merge}

% CVD: color vision deficiencies
\definecolor{CVD light red}   {HTML}{FF8080}
\definecolor{CVD light yellow}{HTML}{FFFF80}
\definecolor{CVD light green} {HTML}{40FFC0}

\definecolor{nt}{named}{HKS41K70}
\newcommand*{\nt}[1]{{\color{nt}#1}}

% set of all probability distributions over #1
\DeclareMathOperator{\pdsOp}{Pd}
\newcommand*{\pds}[1]{\pdsOp(#1)}

\DeclareMathOperator{\positionsOp}{pos}
\newcommand*{\positions}[1]{\positionsOp(#1)}

\DeclareMathOperator{\rankOp}{rk}
\newcommand*{\rank}[1]{\rankOp(#1)}

\DeclareMathOperator{\runsOp}{run}
\newcommand*{\runs}[2][]{%
	\ifthenelse%
		{\isempty{#1}}%
		{\runsOp(#2)}%
		{\runsOp_{#1}(#2)}%
}

\newcommand*{\semantics}[1]{⟦#1⟧}

\DeclareMathOperator{\splt}{split}

\newcommand*{\subtree}[2]{#1|_{#2}}

\DeclareMathOperator{\supportOp}{supp}
\newcommand*{\support}[1]{\supportOp(#1)}

\newcommand*{\symId}{\textsc{\color{gray}Id}}
\newcommand*{\symCons}{\textsc{\color{gray}Cons}}
\newcommand*{\symFlip}{\textsc{\color{gray}Flip}}
\newcommand*{\symNull}{\textsc{\color{gray}Null}}
\newcommand*{\symNullR}{\textsc{\color{gray}N\(\overline{\textsc{ull}}\)}}
\newcommand*{\symSnoc}{\textsc{\color{gray}Snoc}}

\newcommand*{\transWTA}[4][]{#3 \xrightarrow{#1} #2(#4)}

\DeclareMathOperator{\uniqueRunOp}{r}
\newcommand*{\uniqueRun}[2][]{%
	\ifthenelse%
		{\isempty{#1}}%
		{\uniqueRunOp^{#2}}%
		{\uniqueRunOp_{\!#1}^{#2}}%
}

\DeclareMathOperator{\treesOp}{T}
\newcommand*{\trees}[2][]{%
	\ifthenelse%
		{\isempty{#1}}%
		{\treesOp_{\!#2}}%
		{\treesOp_{\!#2}(#1)}%
}
\DeclareMathOperator{\treesUOp}{U}
\newcommand*{\treesU}[2][]{%
	\ifthenelse%
		{\isempty{#1}}%
		{\treesUOp_{#2}}%
		{\treesUOp_{#2}(#1)}%
}


%%%%%%%%%%%%%%%%%%%%%%%%%%%%%%%%%%%%%%%%%%%%%%%%%%%%%%%%%%%%%%%%%%%%%%%%%%%%%%
% metadata
%%%%%%%%%%%%%%%%%%%%%%%%%%%%%%%%%%%%%%%%%%%%%%%%%%%%%%%%%%%%%%%%%%%%%%%%%%%%%%

\ifstandalonebeamer\else
	\title[Defense of Dissertation]{A Formal View on Training of Weighted Tree Automata by Likelihood-Driven State Splitting and Merging}
	\subtitle{Defense of Dissertation}
\fi
\author{Toni Dietze}
\institute[TU Dresden]{%
	\href{https://www.orchid.inf.tu-dresden.de/index.en/}{Chair for Foundations of Programming}
\\	\href{https://tu-dresden.de/ing/informatik/thi}{Institute of Theoretical Computer Science}
\\	\href{https://tu-dresden.de/ing/informatik}{Faculty of Computer Science}
\\	\href{https://tu-dresden.de/}{Technische Universität Dresden}
\\	01062 Dresden, Germany
}
\date[2018-09-27]{September 27, 2018}

\begin{document}
\begin{standaloneframe}[t]{\jobname}
\centering
\begin{tikzpicture}[ampersand replacement=\&]
	\matrix[column sep=0em, row sep=1em]{
		\node (b) {binarization \(h\)};
	\&
		\node[subtree, inner sep=0.3em] (t) {\(t\)};
	\\
		\node[align=center] (wta) {weighted (\alert{ranked}) \\ tree automaton \(ℳ'\)};
	\&\&
		\node[align=center, visible=<2->] (wuta) {weighted \alert{unranked} \\ tree automaton \(ℳ\)};
	\\\&
		\node (w) {\(\displaystyle ∑_{t' ∈ h^{-1}(t)} ⟦ℳ'⟧(t') \uncover<2->{= ⟦ℳ⟧(t)}\)};
	\\};

	\begin{scope}[rounded corners]
		\draw[->] (t) -- (b);
		\draw[->] (b) -- (wta);
		\draw[->] (wta) |- (w.mid west);

		\draw[->, visible=<2->] (t) -| (wuta);
		\draw[->, visible=<2->] (wuta) |- (w.mid east);
	\end{scope}
\end{tikzpicture}%
\only<3->{%
	\\[1em]
	For each of three binarizations \(h\):
	\begin{block}{Theorem \hfill [TD]}
		%We show that wsta together with any of the presented binarizations and wuta are equally powerful
		For every \only<4>{\alert{prob.}\ }wuta \(ℳ\) there is a \only<4>{\alert{prob.}\ }wta \(ℳ'\) and vice versa such that
		\setlength{\belowdisplayskip}{0pt}%
		\[
			⟦ℳ⟧(t) = ∑_{t' ∈ h^{-1}(t)} ⟦ℳ'⟧(t')
			\qquad
			\text{for every \(t ∈ \trees{Σ}\).}
		\]
	\end{block}
}%
\end{standaloneframe}
\end{document}

\end{frame}


\iffalse
\section{Conclusion}

\begin{frame}{\secname}
	\documentclass[beamer]{standalone}
% SPDX-License-Identifier: CC-BY-4.0 OR MIT-0
% Copyright 2018 Toni Dietze
%
\usefonttheme{professionalfonts}

% LuaLaTeX specific packages
\usepackage{fontspec}
	\defaultfontfeatures{Ligatures=TeX}
\usepackage{polyglossia}
	\setdefaultlanguage{english}
\usepackage{amsmath}  % has to be loaded before unicode-math
\usepackage[math-style=ISO]{unicode-math}
	\setmathfont{Latin Modern Math}
% 	\setmathfont[range={\mathcal,\mathbfcal},StylisticSet=1]{xits-math.otf}
% 	\setmathfont[range={"029F5}]{XITS Math}  % ⧵
% 	\setmathfont[range={\mathscr,\mathbfscr},StylisticSet=1]{Latin Modern Math}  % make \mathscr use the correct font

\usepackage[noend]{algpseudocode}
	\algrenewcommand\algorithmicrequire{\textbf{Input:}}
	\algrenewcommand\algorithmicensure{\textbf{Output:}}
\usepackage[backend=biber, maxbibnames=42, maxcitenames=42, sorting=ynt, style=authoryear]{biblatex}
\usepackage{csquotes}
\usepackage{mathtools}
\usepackage{media9}
\usepackage{scalerel}
\usepackage{standalone}
\usepackage{tikz}
	\usetikzlibrary{arrows.meta}
	\usetikzlibrary{backgrounds}
	\usetikzlibrary{calc}
	\usetikzlibrary{decorations}
	\usetikzlibrary{decorations.pathmorphing}
	\usetikzlibrary{decorations.pathreplacing}
	\usetikzlibrary{fadings}
	\usetikzlibrary{fit}
	\usetikzlibrary{graphs}
	\usetikzlibrary{graphdrawing}
	\usetikzlibrary{intersections}
	\usetikzlibrary{positioning}
	\usetikzlibrary{quotes}
	\usetikzlibrary{shadows.blur}
	\usetikzlibrary{shapes.arrows}
	\usetikzlibrary{shapes.geometric}
	\usegdlibrary{trees}
\usepackage{xifthen}
\usepackage{xspace}

\usepackage{pgfplots}
	\pgfplotsset
		{ compat = 1.15
		, /pgf/number format/1000 sep = {\,}
		, /pgf/number format/assume math mode = true
		, every axis plot/.append style =
			{ mark options = {fill opacity = 0.25}
			}
		}
	\usepgfplotslibrary{groupplots}
\usepackage{pgfplotstable}

\hypersetup
	{ bookmarksopen
	, pdflang = en
	, unicode
	}


%%%%%%%%%%%%%%%%%%%%%%%%%%%%%%%%%%%%%%%%%%%%%%%%%%%%%%%%%%%%%%%%%%%%%%%%%%%%%%


% always show bad boxes
%\overfullrule=1em


%%%%%%%%%%%%%%%%%%%%%%%%%%%%%%%%%%%%%%%%%%%%%%%%%%%%%%%%%%%%%%%%%%%%%%%%%%%%%%
% biblatex
%%%%%%%%%%%%%%%%%%%%%%%%%%%%%%%%%%%%%%%%%%%%%%%%%%%%%%%%%%%%%%%%%%%%%%%%%%%%%%

\addbibresource{slides-dissertation-defense.bib}
% \renewcommand*{\finalnamedelim}{\addcomma\space}
% \setlength{\bibitemsep}{1em}
% 
\AtEveryBibitem{% Clean up the bibtex rather than editing it
 \clearlist{address}
 \clearfield{date}
 \clearfield{eprint}
 \clearfield{isbn}
 \clearfield{issn}
 \clearlist{language}
 \clearlist{location}
 \clearfield{month}
 \clearfield{series}
%  \clearfield{url}
%  \clearfield{doi}
 \clearfield{organization}

%  \ifentrytype{book}{}{% Remove stuff except for books
%   \clearfield{booktitle}
%   \clearfield{pages}
  \clearlist{publisher}
  \clearname{editor}
%  }
}
% do not print url if doi is present
% http://tex.stackexchange.com/questions/154864/biblatex-use-doi-only-if-there-is-no-url
\DeclareSourcemap{
	\maps[datatype=bibtex]{
		\map{
			\step[fieldsource=doi,final]
			\step[fieldset=url,null]
}	}	}
%
% remove qoutes around titles
\DeclareFieldFormat
	[article,inbook,incollection,inproceedings,patent,thesis,unpublished]
	{title}{#1\isdot}
% 
% \DeclareFieldFormat{url}{\mkbibacro{URL}\addcolon\addnbspace\url{#1}}
% 
% \DeclareNameAlias{sortname}{first-last}
% 
\renewbibmacro{in:}{\ifentrytype{article}{}{}}


%%%%%%%%%%%%%%%%%%%%%%%%%%%%%%%%%%%%%%%%%%%%%%%%%%%%%%%%%%%%%%%%%%%%%%%%%%%%%%
% beamer
%%%%%%%%%%%%%%%%%%%%%%%%%%%%%%%%%%%%%%%%%%%%%%%%%%%%%%%%%%%%%%%%%%%%%%%%%%%%%%

\useoutertheme{infolines}
\makeatletter
% based on
% /usr/share/texmf-dist/tex/latex/beamer/beamerouterthemeinfolines.sty
\setbeamertemplate{footline}
{%
	\leavevmode%
	\hbox{%
	\begin{beamercolorbox}[wd=.333333\paperwidth,ht=2.25ex,dp=1ex,center]{author in head/foot}%
		\usebeamerfont{author in head/foot}\insertshortauthor\expandafter\beamer@ifempty\expandafter{\beamer@shortinstitute}{}{~~(\insertshortinstitute)}
	\end{beamercolorbox}%
	\begin{beamercolorbox}[wd=.333333\paperwidth,ht=2.25ex,dp=1ex,center]{title in head/foot}%
		\usebeamerfont{title in head/foot}\insertshorttitle
	\end{beamercolorbox}%
	\begin{beamercolorbox}[wd=.333333\paperwidth,ht=2.25ex,dp=1ex,right]{date in head/foot}%
		\usebeamerfont{date in head/foot}%
		\hfill\insertshortdate\hfill\hfill%
		%\hspace*{2ex}%
		%\insertshortdate%
		%\hspace{0pt plus 1 filll}%
		%(\insertframenumber.\insertoverlaynumber{} / \insertmainframenumber)%
		%\hspace{0pt plus 1 filll}%
		\phantom{000}\llap{\insertpagenumber} / \insertpresentationendpage%
		\hspace*{2ex}%
	\end{beamercolorbox}}%
	\vskip0pt%
}
\makeatother
\useinnertheme{circles}
\beamertemplatenavigationsymbolsempty
\setbeamertemplate{bibliography item}{}
\setbeamertemplate{headline}[default]

\input{tudcolors.tex}
\setbeamercolor*{alerted text}{fg=HKS07K100}
\usecolortheme[named=HKS41K100]{structure}

\setbeamercolor*{palette primary}{use=structure,fg=white,bg=structure.fg}
\setbeamercolor*{palette secondary}{use=structure,fg=white,bg=structure.fg!80}
\setbeamercolor*{palette tertiary}{use=structure,fg=white,bg=structure.fg!60}
\setbeamercolor*{palette quaternary}{fg=white,bg=black}

\setbeamercolor*{sidebar}{use=structure,bg=structure.fg}

\setbeamercolor*{palette sidebar primary}{use=structure,fg=structure.fg!20}
\setbeamercolor*{palette sidebar secondary}{fg=white}
\setbeamercolor*{palette sidebar tertiary}{use=structure,fg=structure.fg!40}
\setbeamercolor*{palette sidebar quaternary}{fg=white}

\setbeamercolor*{titlelike}{parent=palette primary}

\setbeamercolor*{separation line}{}
\setbeamercolor*{fine separation line}{}

\setbeamercolor{block title}{use=structure,fg=white,bg=structure.fg}
\setbeamercolor{block title alerted}{use=alerted text,fg=white,bg=alerted text.fg!75!black}
\setbeamercolor{block title example}{use=example text,fg=white,bg=example text.fg!75!black}

\setbeamercolor{block body}{parent=normal text,use=block title,bg=block title.bg!10!bg}
\setbeamercolor{block body alerted}{parent=normal text,use=block title alerted,bg=block title alerted.bg!10!bg}
\setbeamercolor{block body example}{parent=normal text,use=block title example,bg=block title example.bg!10!bg}

% \setbeamertemplate{itemize items}[default]


%%%%%%%%%%%%%%%%%%%%%%%%%%%%%%%%%%%%%%%%%%%%%%%%%%%%%%%%%%%%%%%%%%%%%%%%%%%%%%
% TikZ
%%%%%%%%%%%%%%%%%%%%%%%%%%%%%%%%%%%%%%%%%%%%%%%%%%%%%%%%%%%%%%%%%%%%%%%%%%%%%%

\tikzset
	{ > = Stealth
	}


%%%%%%%%%%%%%%%%%%%%%%%%%%%%%%%%%%%%%%%%%%%%%%%%%%%%%%%%%%%%%%%%%%%%%%%%%%%%%%
% general commands and styles
%%%%%%%%%%%%%%%%%%%%%%%%%%%%%%%%%%%%%%%%%%%%%%%%%%%%%%%%%%%%%%%%%%%%%%%%%%%%%%

% \delegateStyle and \inheritStyle command
% usage: \delegateStyle{… \inheritStyle{…} …}
% example: \(X_{\delegateStyle{\fbox{\inheritStyle{X}}}}\)
% Save the current style and regain it in the argument.
% This works both for math and text mode, and can be nested.
% Acknowledgments: Based on \ThisStyle and \SavedStyle from scalerel package.
\makeatletter
\newcommand*{\@inheritStyle@D}[1]{\(\displaystyle      #1\)}
\newcommand*{\@inheritStyle@T}[1]{\(\textstyle         #1\)}
\newcommand*{\@inheritStyle@S}[1]{\(\scriptstyle       #1\)}
\newcommand*{\@inheritStyle@s}[1]{\(\scriptscriptstyle #1\)}
\newcommand*{\@inheritStyle@t}[1]{#1}
\newcommand*{\inheritStyle}{\csname @inheritStyle@\@inheritStyleSwitch\endcsname}
\newcommand*{\delegateStyle}[1]{%
	\ifmmode%
		\mathchoice%
		{\edef\@inheritStyleSwitch{D}#1}%
		{\edef\@inheritStyleSwitch{T}#1}%
		{\edef\@inheritStyleSwitch{S}#1}%
		{\edef\@inheritStyleSwitch{s}#1}%
	\else%
		\edef\@inheritStyleSwitch{t}#1%
	\fi%
}
\makeatother


% \oalt command
% requires: \delegateStyle and \inheritStyle command
% usage: \oalt<…>[…]{…}{…} (cf. \alt)
% Behaves like \alt, but reserves space according to largest overlays.
% The optional argument defines the alignment inside the reserved space;
% it is one of c, l, r, s (cf. \makebox); the default is c.
\makeatletter
\newlength{\oalt@dp}
\newlength{\oalt@ht}
\newlength{\oalt@wd}
\newbox{\oalt@a}
\newbox{\oalt@b}
\newbox{\oalt@empty}
\newcommand<>*{\oalt}[3][c]{%
	\delegateStyle{%
		% based on \setto… in /usr/share/texmf-dist/tex/latex/base/latex.ltx
		\setbox\oalt@a\hbox{\inheritStyle{#2}}%
		\setbox\oalt@b\hbox{\inheritStyle{#3}}%
		\pgfmathsetlength{\oalt@dp}{max(\dp\oalt@a,\dp\oalt@b)}%
		\pgfmathsetlength{\oalt@ht}{max(\ht\oalt@a,\ht\oalt@b)}%
		\pgfmathsetlength{\oalt@wd}{max(\wd\oalt@a,\wd\oalt@b)}%
		\raisebox{0pt}[\oalt@ht][\oalt@dp]{%
			\makebox[\oalt@wd][#1]{%
				\alt#4{\unhbox\oalt@a}{\unhbox\oalt@b}%
			}%
		}%
		\setbox\oalt@a\box\oalt@empty%
		\setbox\oalt@b\box\oalt@empty%
	}%
}
\makeatother


% \otemporal command
% requires: \delegateStyle and \inheritStyle command
% usage: \otemporal<…>[…]{…}{…}{…} (cf. \temporal)
% Behaves like \temporal, but reserves space according to largest overlays.
% The optional argument defines the alignment inside the reserved space;
% it is one of c, l, r, s (cf. \makebox); the default is c.
\makeatletter
\newlength{\ot@dp}
\newlength{\ot@ht}
\newlength{\ot@wd}
\newbox{\ot@a}
\newbox{\ot@b}
\newbox{\ot@c}
\newbox{\ot@empty}
\newcommand<>*{\otemporal}[4][c]{%
	\delegateStyle{%
		% based on \setto… in /usr/share/texmf-dist/tex/latex/base/latex.ltx
		\setbox\ot@a\hbox{\inheritStyle{#2}}%
		\setbox\ot@b\hbox{\inheritStyle{#3}}%
		\setbox\ot@c\hbox{\inheritStyle{#4}}%
		\pgfmathsetlength{\ot@dp}{max(\dp\ot@a,\dp\ot@b,\dp\ot@c)}%
		\pgfmathsetlength{\ot@ht}{max(\ht\ot@a,\ht\ot@b,\ht\ot@c)}%
		\pgfmathsetlength{\ot@wd}{max(\wd\ot@a,\wd\ot@b,\wd\ot@c)}%
		\raisebox{0pt}[\ot@ht][\ot@dp]{%
			\makebox[\ot@wd][#1]{%
				\temporal#5{\unhbox\ot@a}{\unhbox\ot@b}{\unhbox\ot@c}%
			}%
		}%
		\setbox\ot@a\box\ot@empty%
		\setbox\ot@b\box\ot@empty%
		\setbox\ot@c\box\ot@empty%
	}%
}
\makeatother


% Resize delimiters like braces, brackets, etc.
% Parameters: size, left delimiter, formula, right delimiter
% Example: \delim2({\frac{1}{2}})
\newcommand*{\delim}[4]{%
	\ifcase#1%
		#2#3#4%
	\or%
		\bigl#2#3\bigr#4%
	\or%
		\Bigl#2#3\Bigr#4%
	\or%
		\biggl#2#3\biggr#4%
	\or%
		\Biggl#2#3\Biggr#4%
	\else%
		\left#2#3\right#4%
	\fi%
}


% similar to \fullcite, but using the formatting of \printbibliography
\newcommand*{\printfullcite}[1]{%
	\begin{refsection}%
		\nocite{#1}%
		\DeclareNameAlias{author}{first-last}%
		\printbibliography[heading = none]%
	\end{refsection}%
}


\colorlet{light alert}{HKS07K60}
\tikzset{alert.bg/.style={rounded corners, fill=light alert}}
\tikzset{every picture/.style={line cap=round, semithick}}
% http://tex.stackexchange.com/questions/6135/how-to-make-beamer-overlays-with-tikz-node
\tikzset{onslide/.code args={<#1>#2}{\only<#1>{\pgfkeysalso{#2}}}}
\tikzset{invisible/.code args={<#1>}{\alt<#1>{\pgfkeysalso{transparent}}{\pgfkeysalso{opaque}}}}
\tikzset{uncover/.code args={<#1>}{\alt<#1>{\pgfkeysalso{opaque}}{\pgfkeysalso{opacity=0.25}}}}
\tikzset{visible/.code args={<#1>}{\alt<#1>{\pgfkeysalso{opaque}}{\pgfkeysalso{transparent}}}}
\tikzset{vuncover/.code args=%
	{<#1><#2>}%
	{\alt<#1>%
		{\alt<#2>%
			{\pgfkeysalso{opaque}}%
			{\pgfkeysalso{opacity=0.25}}%
		}{\pgfkeysalso{transparent}}%
	}%
}

\newcommand<%
	>{\tikzhighlight}[2][]{%
	\delegateStyle{\alt#3%
		{\tikz[baseline=0, anchor=base, inner sep=0.2em, text height=, text depth=]{\node[alert.bg, #1]{\inheritStyle{#2}};}}%
		{\tikz[baseline=0, anchor=base, inner sep=0.2em, text height=, text depth=]{\node[#1, fill=none]{\inheritStyle{#2}};}}%
	}%
}

\newcommand{\mathhighlight}{\tikzhighlight}

\newcommand<>{\mhl}[2][]{\mathhighlight#3[inner sep=0.2em, #1]{#2}}


\newcommand<>{\inlineblock}[2][]{{%
	\usebeamercolor*[fg]{block body}%
	\tikzhighlight#3[fill=block body.bg, #1]{#2}%
}}


% a small letter s for plurals of abbreviations
\newcommand*{\s}{{\scriptsize s}\xspace}


\newcommand<>*{\sout}[2][opacity=0.75, ultra thick]{%
	\delegateStyle{%
		\tikz[baseline=0, anchor=base, inner sep=0, outer sep=0]{
			\useasboundingbox node (n) {\inheritStyle{#2}};
			\only#3{
				\node (h) {\inheritStyle{\ifmmode\mathstrut\else\strut\fi}};
				\draw[#1] (n.west |- {$(h.south)!0.5!(h.north)$}) -- (n.east |- {$(h.south)!0.5!(h.north)$});
			}
		}%
	}%
}


% tight style
% Sets outer sep to default inner sep and inner sep to 0.
% Use this style for nodes that are neither drawn nor filled to prevent
% unwanted growth of the bounding box.
\tikzset{tight/.style={inner sep=0, outer sep=0.3333em}}


% rounded tree edges style
% usage: rounded tree edges={⟨direction⟩}{⟨looseness⟩}{⟨strength⟩}
\tikzset{
	rounded tree edges/.style n args={3}{
	edge from parent path={
	let
		\n{direction}={#1},
		\n{looseness}={#2},
		\n{strength}={#3},
		\p1=(\tikzparentnode),
		\p2=(\tikzchildnode),
		\p3=(\n{direction}:1pt),
		\p4=(\x2 - \x1, \y2 - \y1),
		\n{dist}={veclen(\p4)},
		\p4=(\x4 / \n{dist}, \y4 / \n{dist}),
		\n{angle}={atan2(\y4, \x4)},
		\n{delta}={Mod(\n{angle} - \n{direction}, 360)},
		\n{delta}={\n{delta} > 180 ? \n{delta} - 360  : \n{delta}},
		\n{delta}={\n{delta} >  90 ?  180 - \n{delta} : \n{delta}},
		\n{delta}={\n{delta} < -90 ? -180 - \n{delta} : \n{delta}}
	in (\tikzparentnode) .. controls
		+(    \n{angle}+\n{strength}*\n{delta}:\n{looseness}*0.3915*\n{dist}) and
		+(180+\n{angle}-\n{strength}*\n{delta}:\n{looseness}*0.3915*\n{dist}) ..
		(\tikzchildnode)
	}
	}
}


% Tear out snippets from PDFs.
% Usage: \tear[…]{file.pdf}
% The optional parameter is the same as for \includegraphics.
% Useful Arguments:
%   * page=‹pagenumber›
%   * trim=‹left› ‹bottom› ‹right› ‹top›
%   * width=0.98\linewidth
\newcommand*{\tear}[2][]{%
	\begin{tikzpicture}
		\node
			[ blur shadow
			, clip
			, decorate
			, decoration=random steps
			, draw
			, inner sep=0
			, preaction={fill=white}% hide the shadow if paper is transparent
			] {\includegraphics[#1]{#2}};
	\end{tikzpicture}%
}


\makeatletter
\newcommand*{\timeline}[3][0]{%
	\ifcsname timeline@cmd@#3\endcsname%
		\@timeline[#1]{#2}{#3}%
		\PackageWarning{timeline}{redefining timeline \@backslashchar\string#3}%
	\else%
		\ifcsname#3\endcsname%
			\errmessage{Command \@backslashchar\string#3 already defined}%
		\else%
			\@timeline[#1]{#2}{#3}%
		\fi%
	\fi%
}%
\newcommand*{\@timeline}[3][0]{%
	% mark command as timeline command – they can be overwritten
	\expandafter\def\csname timeline@cmd@#3\endcsname{}%
	\setcounter{@timeline}{#1}%
	\def\timeline@cmd{#3}%
	\timeline@reset%
	\timeline@append{0}%
	\@tfor\timeline@next:=#2\do{%
		\if\timeline@next+%
			\stepcounter{@timeline}%
			\timeline@append{,\the@timeline}%
		\else\if\timeline@next-%
			\stepcounter{@timeline}%
		\else%
			%\timeline@append{\timeline@next}%
			\GenericError{}{\protect\timeline: ignoring unknown character: \timeline@next}%
		\fi\fi%
	}%
}%
% \newcommand*{\tl}[1]{%
% 	\ifcsname timeline@cmd@#1\endcsname%
% 		\csname timeline@cmd@#1\endcsname%
% 	\else%
% 		0%
% 		%\GenericError{}{\protect\tl: timeline not defined: #1}%
% 	\fi%
% }%
\newcounter{@timeline}%
\def\timeline@reset{%
	\expandafter\def\csname\timeline@cmd\endcsname{}%
}%
\def\timeline@append#1{%
	\expandafter\edef\csname\timeline@cmd\endcsname{%
		\csname\timeline@cmd\endcsname#1%
	}%
}%
\makeatother


\newcommand*{\xminus}[1]{%
	\mathrel{\tikz[baseline={([yshift=-0.25em]n.south)}, inner sep=0, outer sep=0.2em]{%
		\node (n) {\(\scriptstyle #1\)};
		\draw (n.south west) -- (n.south east);
	}}%
}
\newcommand*{\tikzrightarrow}[1]{%
	\mathrel{\tikz[baseline={([yshift=-0.25em]n.south)}, inner sep=0, outer sep=0.2em]{%
		\node (n) {\(\scriptstyle #1\)};
		\draw[->, > = Computer Modern Rightarrow, line width = 0.4pt] (n.south west) -- (n.south east);
	}}%
}


%%%%%%%%%%%%%%%%%%%%%%%%%%%%%%%%%%%%%%%%%%%%%%%%%%%%%%%%%%%%%%%%%%%%%%%%%%%%%%
% document specific commands
%%%%%%%%%%%%%%%%%%%%%%%%%%%%%%%%%%%%%%%%%%%%%%%%%%%%%%%%%%%%%%%%%%%%%%%%%%%%%%

\newcommand<>*{\mycite}[1]{\uncover#2{{\color{HKS57K100}[\cite{#1}]}}}


\newcommand{\statetree}[1]{
	\tikz
	[ anchor=base
	, baseline=(current bounding box.center)
	, level distance=2em
	, sibling distance=2em
	]{
		\matrix
		[ draw=nt
		, edge from parent/.style={draw=black}
		, inner sep=0
		, nodes={inner sep=0.2em, rounded corners=0}
		, rounded corners
		] {#1\\}
	}
}


\newcommand*{\mylargeleaf}[1]{{\LARGE\color{HKS41K70}#1}}

\definecolor{state s}{named}{HKS57K80}
\definecolor{state t}{named}{HKS41K70}
\newcommand*{\stateS}[1]{{\color{state s}#1}}
\newcommand*{\stateT}[1]{{\color{state t}#1}}

\tikzset{
	subtree/.style =
		{ fill=lightgray
		, inner sep=0.02em
		, isosceles triangle apex angle=60
		, shape=isosceles triangle
		, shape border rotate=90
		}
	, state/.style = {circle, draw, inner sep=0.1em}
	, trans/.style = {rectangle, draw}
}

\newcommand*{\srBool}{\mathbb{B}}
\newcommand*{\srProb}{ℙ}


%%%%%%%%%%%%%%%%%%%%%%%%%%%%%%%%%%%%%%%%%%%%%%%%%%%%%%%%%%%%%%%%%%%%%%%%%%%%%%
% commands for specific notations
%%%%%%%%%%%%%%%%%%%%%%%%%%%%%%%%%%%%%%%%%%%%%%%%%%%%%%%%%%%%%%%%%%%%%%%%%%%%%%

\DeclareMathOperator*{\argmax}{argmax}

\newcommand*{\cardinality}[1]{\lvert#1\rvert}
\newcommand*{\corpussize}[1]{\lvert#1\rvert}

\DeclareMathOperator{\crispOp}{crisp}
\newcommand*        {\crisp}[2][0]{\crispOp\delim{#1}({#2})}

\DeclareMathOperator{\lhsOp}{lhs}
\newcommand*{\lhs}[1]{\lhsOp(#1)}

\DeclareMathOperator{\lklhdOp}{L}
\newcommand*{\lklhd}[2]{\lklhdOp(#1 ∣ #2)}

\DeclareMathOperator{\mleOp}{mle}
\newcommand*{\mle}[2][]{%
	\ifthenelse{\isempty{#1}}{%
		\mleOp(#2)%
	}{%
		\mleOp_{#1}(#2)%
	}%
}

\DeclareMathOperator{\mrg}{merge}

% CVD: color vision deficiencies
\definecolor{CVD light red}   {HTML}{FF8080}
\definecolor{CVD light yellow}{HTML}{FFFF80}
\definecolor{CVD light green} {HTML}{40FFC0}

\definecolor{nt}{named}{HKS41K70}
\newcommand*{\nt}[1]{{\color{nt}#1}}

% set of all probability distributions over #1
\DeclareMathOperator{\pdsOp}{Pd}
\newcommand*{\pds}[1]{\pdsOp(#1)}

\DeclareMathOperator{\positionsOp}{pos}
\newcommand*{\positions}[1]{\positionsOp(#1)}

\DeclareMathOperator{\rankOp}{rk}
\newcommand*{\rank}[1]{\rankOp(#1)}

\DeclareMathOperator{\runsOp}{run}
\newcommand*{\runs}[2][]{%
	\ifthenelse%
		{\isempty{#1}}%
		{\runsOp(#2)}%
		{\runsOp_{#1}(#2)}%
}

\newcommand*{\semantics}[1]{⟦#1⟧}

\DeclareMathOperator{\splt}{split}

\newcommand*{\subtree}[2]{#1|_{#2}}

\DeclareMathOperator{\supportOp}{supp}
\newcommand*{\support}[1]{\supportOp(#1)}

\newcommand*{\symId}{\textsc{\color{gray}Id}}
\newcommand*{\symCons}{\textsc{\color{gray}Cons}}
\newcommand*{\symFlip}{\textsc{\color{gray}Flip}}
\newcommand*{\symNull}{\textsc{\color{gray}Null}}
\newcommand*{\symNullR}{\textsc{\color{gray}N\(\overline{\textsc{ull}}\)}}
\newcommand*{\symSnoc}{\textsc{\color{gray}Snoc}}

\newcommand*{\transWTA}[4][]{#3 \xrightarrow{#1} #2(#4)}

\DeclareMathOperator{\uniqueRunOp}{r}
\newcommand*{\uniqueRun}[2][]{%
	\ifthenelse%
		{\isempty{#1}}%
		{\uniqueRunOp^{#2}}%
		{\uniqueRunOp_{\!#1}^{#2}}%
}

\DeclareMathOperator{\treesOp}{T}
\newcommand*{\trees}[2][]{%
	\ifthenelse%
		{\isempty{#1}}%
		{\treesOp_{\!#2}}%
		{\treesOp_{\!#2}(#1)}%
}
\DeclareMathOperator{\treesUOp}{U}
\newcommand*{\treesU}[2][]{%
	\ifthenelse%
		{\isempty{#1}}%
		{\treesUOp_{#2}}%
		{\treesUOp_{#2}(#1)}%
}


%%%%%%%%%%%%%%%%%%%%%%%%%%%%%%%%%%%%%%%%%%%%%%%%%%%%%%%%%%%%%%%%%%%%%%%%%%%%%%
% metadata
%%%%%%%%%%%%%%%%%%%%%%%%%%%%%%%%%%%%%%%%%%%%%%%%%%%%%%%%%%%%%%%%%%%%%%%%%%%%%%

\ifstandalonebeamer\else
	\title[Defense of Dissertation]{A Formal View on Training of Weighted Tree Automata by Likelihood-Driven State Splitting and Merging}
	\subtitle{Defense of Dissertation}
\fi
\author{Toni Dietze}
\institute[TU Dresden]{%
	\href{https://www.orchid.inf.tu-dresden.de/index.en/}{Chair for Foundations of Programming}
\\	\href{https://tu-dresden.de/ing/informatik/thi}{Institute of Theoretical Computer Science}
\\	\href{https://tu-dresden.de/ing/informatik}{Faculty of Computer Science}
\\	\href{https://tu-dresden.de/}{Technische Universität Dresden}
\\	01062 Dresden, Germany
}
\date[2018-09-27]{September 27, 2018}

\title{\jobname}
\begin{document}
\begin{standaloneframe}{\jobname}
	\only<+->{}
	\begin{itemize}[<1-|alert@+>]
	\item (weighted) tree automata
	\item read-off automaton
	\item adapting weights:
		\begin{itemize}[<1-|alert@.>]
		\item EM algorithm
		\end{itemize}
	\item adapting state behavior (and weights):
		\begin{itemize}[<1-|alert@.>]
		\item state splitting and merging algorithm
		\item count-based state merging
		\end{itemize}
	\item adapting branching behavior:
		\begin{itemize}[<1-|alert@.>]
		\item binarization
		\end{itemize}
	\end{itemize}
\end{standaloneframe}
\end{document}

\end{frame}
\fi


\section{Publications}

\begin{frame}{\secname}
	% SPDX-License-Identifier: CC-BY-4.0
% Copyright 2018 Toni Dietze
\documentclass[beamer]{standalone}
% SPDX-License-Identifier: CC-BY-4.0 OR MIT-0
% Copyright 2018 Toni Dietze
%
\usefonttheme{professionalfonts}

% LuaLaTeX specific packages
\usepackage{fontspec}
	\defaultfontfeatures{Ligatures=TeX}
\usepackage{polyglossia}
	\setdefaultlanguage{english}
\usepackage{amsmath}  % has to be loaded before unicode-math
\usepackage[math-style=ISO]{unicode-math}
	\setmathfont{Latin Modern Math}
% 	\setmathfont[range={\mathcal,\mathbfcal},StylisticSet=1]{xits-math.otf}
% 	\setmathfont[range={"029F5}]{XITS Math}  % ⧵
% 	\setmathfont[range={\mathscr,\mathbfscr},StylisticSet=1]{Latin Modern Math}  % make \mathscr use the correct font

\usepackage[noend]{algpseudocode}
	\algrenewcommand\algorithmicrequire{\textbf{Input:}}
	\algrenewcommand\algorithmicensure{\textbf{Output:}}
\usepackage[backend=biber, maxbibnames=42, maxcitenames=42, sorting=ynt, style=authoryear]{biblatex}
\usepackage{csquotes}
\usepackage{mathtools}
\usepackage{media9}
\usepackage{scalerel}
\usepackage{standalone}
\usepackage{tikz}
	\usetikzlibrary{arrows.meta}
	\usetikzlibrary{backgrounds}
	\usetikzlibrary{calc}
	\usetikzlibrary{decorations}
	\usetikzlibrary{decorations.pathmorphing}
	\usetikzlibrary{decorations.pathreplacing}
	\usetikzlibrary{fadings}
	\usetikzlibrary{fit}
	\usetikzlibrary{graphs}
	\usetikzlibrary{graphdrawing}
	\usetikzlibrary{intersections}
	\usetikzlibrary{positioning}
	\usetikzlibrary{quotes}
	\usetikzlibrary{shadows.blur}
	\usetikzlibrary{shapes.arrows}
	\usetikzlibrary{shapes.geometric}
	\usegdlibrary{trees}
\usepackage{xifthen}
\usepackage{xspace}

\usepackage{pgfplots}
	\pgfplotsset
		{ compat = 1.15
		, /pgf/number format/1000 sep = {\,}
		, /pgf/number format/assume math mode = true
		, every axis plot/.append style =
			{ mark options = {fill opacity = 0.25}
			}
		}
	\usepgfplotslibrary{groupplots}
\usepackage{pgfplotstable}

\hypersetup
	{ bookmarksopen
	, pdflang = en
	, unicode
	}


%%%%%%%%%%%%%%%%%%%%%%%%%%%%%%%%%%%%%%%%%%%%%%%%%%%%%%%%%%%%%%%%%%%%%%%%%%%%%%


% always show bad boxes
%\overfullrule=1em


%%%%%%%%%%%%%%%%%%%%%%%%%%%%%%%%%%%%%%%%%%%%%%%%%%%%%%%%%%%%%%%%%%%%%%%%%%%%%%
% biblatex
%%%%%%%%%%%%%%%%%%%%%%%%%%%%%%%%%%%%%%%%%%%%%%%%%%%%%%%%%%%%%%%%%%%%%%%%%%%%%%

\addbibresource{slides-dissertation-defense.bib}
% \renewcommand*{\finalnamedelim}{\addcomma\space}
% \setlength{\bibitemsep}{1em}
% 
\AtEveryBibitem{% Clean up the bibtex rather than editing it
 \clearlist{address}
 \clearfield{date}
 \clearfield{eprint}
 \clearfield{isbn}
 \clearfield{issn}
 \clearlist{language}
 \clearlist{location}
 \clearfield{month}
 \clearfield{series}
%  \clearfield{url}
%  \clearfield{doi}
 \clearfield{organization}

%  \ifentrytype{book}{}{% Remove stuff except for books
%   \clearfield{booktitle}
%   \clearfield{pages}
  \clearlist{publisher}
  \clearname{editor}
%  }
}
% do not print url if doi is present
% http://tex.stackexchange.com/questions/154864/biblatex-use-doi-only-if-there-is-no-url
\DeclareSourcemap{
	\maps[datatype=bibtex]{
		\map{
			\step[fieldsource=doi,final]
			\step[fieldset=url,null]
}	}	}
%
% remove qoutes around titles
\DeclareFieldFormat
	[article,inbook,incollection,inproceedings,patent,thesis,unpublished]
	{title}{#1\isdot}
% 
% \DeclareFieldFormat{url}{\mkbibacro{URL}\addcolon\addnbspace\url{#1}}
% 
% \DeclareNameAlias{sortname}{first-last}
% 
\renewbibmacro{in:}{\ifentrytype{article}{}{}}


%%%%%%%%%%%%%%%%%%%%%%%%%%%%%%%%%%%%%%%%%%%%%%%%%%%%%%%%%%%%%%%%%%%%%%%%%%%%%%
% beamer
%%%%%%%%%%%%%%%%%%%%%%%%%%%%%%%%%%%%%%%%%%%%%%%%%%%%%%%%%%%%%%%%%%%%%%%%%%%%%%

\useoutertheme{infolines}
\makeatletter
% based on
% /usr/share/texmf-dist/tex/latex/beamer/beamerouterthemeinfolines.sty
\setbeamertemplate{footline}
{%
	\leavevmode%
	\hbox{%
	\begin{beamercolorbox}[wd=.333333\paperwidth,ht=2.25ex,dp=1ex,center]{author in head/foot}%
		\usebeamerfont{author in head/foot}\insertshortauthor\expandafter\beamer@ifempty\expandafter{\beamer@shortinstitute}{}{~~(\insertshortinstitute)}
	\end{beamercolorbox}%
	\begin{beamercolorbox}[wd=.333333\paperwidth,ht=2.25ex,dp=1ex,center]{title in head/foot}%
		\usebeamerfont{title in head/foot}\insertshorttitle
	\end{beamercolorbox}%
	\begin{beamercolorbox}[wd=.333333\paperwidth,ht=2.25ex,dp=1ex,right]{date in head/foot}%
		\usebeamerfont{date in head/foot}%
		\hfill\insertshortdate\hfill\hfill%
		%\hspace*{2ex}%
		%\insertshortdate%
		%\hspace{0pt plus 1 filll}%
		%(\insertframenumber.\insertoverlaynumber{} / \insertmainframenumber)%
		%\hspace{0pt plus 1 filll}%
		\phantom{000}\llap{\insertpagenumber} / \insertpresentationendpage%
		\hspace*{2ex}%
	\end{beamercolorbox}}%
	\vskip0pt%
}
\makeatother
\useinnertheme{circles}
\beamertemplatenavigationsymbolsempty
\setbeamertemplate{bibliography item}{}
\setbeamertemplate{headline}[default]

\input{tudcolors.tex}
\setbeamercolor*{alerted text}{fg=HKS07K100}
\usecolortheme[named=HKS41K100]{structure}

\setbeamercolor*{palette primary}{use=structure,fg=white,bg=structure.fg}
\setbeamercolor*{palette secondary}{use=structure,fg=white,bg=structure.fg!80}
\setbeamercolor*{palette tertiary}{use=structure,fg=white,bg=structure.fg!60}
\setbeamercolor*{palette quaternary}{fg=white,bg=black}

\setbeamercolor*{sidebar}{use=structure,bg=structure.fg}

\setbeamercolor*{palette sidebar primary}{use=structure,fg=structure.fg!20}
\setbeamercolor*{palette sidebar secondary}{fg=white}
\setbeamercolor*{palette sidebar tertiary}{use=structure,fg=structure.fg!40}
\setbeamercolor*{palette sidebar quaternary}{fg=white}

\setbeamercolor*{titlelike}{parent=palette primary}

\setbeamercolor*{separation line}{}
\setbeamercolor*{fine separation line}{}

\setbeamercolor{block title}{use=structure,fg=white,bg=structure.fg}
\setbeamercolor{block title alerted}{use=alerted text,fg=white,bg=alerted text.fg!75!black}
\setbeamercolor{block title example}{use=example text,fg=white,bg=example text.fg!75!black}

\setbeamercolor{block body}{parent=normal text,use=block title,bg=block title.bg!10!bg}
\setbeamercolor{block body alerted}{parent=normal text,use=block title alerted,bg=block title alerted.bg!10!bg}
\setbeamercolor{block body example}{parent=normal text,use=block title example,bg=block title example.bg!10!bg}

% \setbeamertemplate{itemize items}[default]


%%%%%%%%%%%%%%%%%%%%%%%%%%%%%%%%%%%%%%%%%%%%%%%%%%%%%%%%%%%%%%%%%%%%%%%%%%%%%%
% TikZ
%%%%%%%%%%%%%%%%%%%%%%%%%%%%%%%%%%%%%%%%%%%%%%%%%%%%%%%%%%%%%%%%%%%%%%%%%%%%%%

\tikzset
	{ > = Stealth
	}


%%%%%%%%%%%%%%%%%%%%%%%%%%%%%%%%%%%%%%%%%%%%%%%%%%%%%%%%%%%%%%%%%%%%%%%%%%%%%%
% general commands and styles
%%%%%%%%%%%%%%%%%%%%%%%%%%%%%%%%%%%%%%%%%%%%%%%%%%%%%%%%%%%%%%%%%%%%%%%%%%%%%%

% \delegateStyle and \inheritStyle command
% usage: \delegateStyle{… \inheritStyle{…} …}
% example: \(X_{\delegateStyle{\fbox{\inheritStyle{X}}}}\)
% Save the current style and regain it in the argument.
% This works both for math and text mode, and can be nested.
% Acknowledgments: Based on \ThisStyle and \SavedStyle from scalerel package.
\makeatletter
\newcommand*{\@inheritStyle@D}[1]{\(\displaystyle      #1\)}
\newcommand*{\@inheritStyle@T}[1]{\(\textstyle         #1\)}
\newcommand*{\@inheritStyle@S}[1]{\(\scriptstyle       #1\)}
\newcommand*{\@inheritStyle@s}[1]{\(\scriptscriptstyle #1\)}
\newcommand*{\@inheritStyle@t}[1]{#1}
\newcommand*{\inheritStyle}{\csname @inheritStyle@\@inheritStyleSwitch\endcsname}
\newcommand*{\delegateStyle}[1]{%
	\ifmmode%
		\mathchoice%
		{\edef\@inheritStyleSwitch{D}#1}%
		{\edef\@inheritStyleSwitch{T}#1}%
		{\edef\@inheritStyleSwitch{S}#1}%
		{\edef\@inheritStyleSwitch{s}#1}%
	\else%
		\edef\@inheritStyleSwitch{t}#1%
	\fi%
}
\makeatother


% \oalt command
% requires: \delegateStyle and \inheritStyle command
% usage: \oalt<…>[…]{…}{…} (cf. \alt)
% Behaves like \alt, but reserves space according to largest overlays.
% The optional argument defines the alignment inside the reserved space;
% it is one of c, l, r, s (cf. \makebox); the default is c.
\makeatletter
\newlength{\oalt@dp}
\newlength{\oalt@ht}
\newlength{\oalt@wd}
\newbox{\oalt@a}
\newbox{\oalt@b}
\newbox{\oalt@empty}
\newcommand<>*{\oalt}[3][c]{%
	\delegateStyle{%
		% based on \setto… in /usr/share/texmf-dist/tex/latex/base/latex.ltx
		\setbox\oalt@a\hbox{\inheritStyle{#2}}%
		\setbox\oalt@b\hbox{\inheritStyle{#3}}%
		\pgfmathsetlength{\oalt@dp}{max(\dp\oalt@a,\dp\oalt@b)}%
		\pgfmathsetlength{\oalt@ht}{max(\ht\oalt@a,\ht\oalt@b)}%
		\pgfmathsetlength{\oalt@wd}{max(\wd\oalt@a,\wd\oalt@b)}%
		\raisebox{0pt}[\oalt@ht][\oalt@dp]{%
			\makebox[\oalt@wd][#1]{%
				\alt#4{\unhbox\oalt@a}{\unhbox\oalt@b}%
			}%
		}%
		\setbox\oalt@a\box\oalt@empty%
		\setbox\oalt@b\box\oalt@empty%
	}%
}
\makeatother


% \otemporal command
% requires: \delegateStyle and \inheritStyle command
% usage: \otemporal<…>[…]{…}{…}{…} (cf. \temporal)
% Behaves like \temporal, but reserves space according to largest overlays.
% The optional argument defines the alignment inside the reserved space;
% it is one of c, l, r, s (cf. \makebox); the default is c.
\makeatletter
\newlength{\ot@dp}
\newlength{\ot@ht}
\newlength{\ot@wd}
\newbox{\ot@a}
\newbox{\ot@b}
\newbox{\ot@c}
\newbox{\ot@empty}
\newcommand<>*{\otemporal}[4][c]{%
	\delegateStyle{%
		% based on \setto… in /usr/share/texmf-dist/tex/latex/base/latex.ltx
		\setbox\ot@a\hbox{\inheritStyle{#2}}%
		\setbox\ot@b\hbox{\inheritStyle{#3}}%
		\setbox\ot@c\hbox{\inheritStyle{#4}}%
		\pgfmathsetlength{\ot@dp}{max(\dp\ot@a,\dp\ot@b,\dp\ot@c)}%
		\pgfmathsetlength{\ot@ht}{max(\ht\ot@a,\ht\ot@b,\ht\ot@c)}%
		\pgfmathsetlength{\ot@wd}{max(\wd\ot@a,\wd\ot@b,\wd\ot@c)}%
		\raisebox{0pt}[\ot@ht][\ot@dp]{%
			\makebox[\ot@wd][#1]{%
				\temporal#5{\unhbox\ot@a}{\unhbox\ot@b}{\unhbox\ot@c}%
			}%
		}%
		\setbox\ot@a\box\ot@empty%
		\setbox\ot@b\box\ot@empty%
		\setbox\ot@c\box\ot@empty%
	}%
}
\makeatother


% Resize delimiters like braces, brackets, etc.
% Parameters: size, left delimiter, formula, right delimiter
% Example: \delim2({\frac{1}{2}})
\newcommand*{\delim}[4]{%
	\ifcase#1%
		#2#3#4%
	\or%
		\bigl#2#3\bigr#4%
	\or%
		\Bigl#2#3\Bigr#4%
	\or%
		\biggl#2#3\biggr#4%
	\or%
		\Biggl#2#3\Biggr#4%
	\else%
		\left#2#3\right#4%
	\fi%
}


% similar to \fullcite, but using the formatting of \printbibliography
\newcommand*{\printfullcite}[1]{%
	\begin{refsection}%
		\nocite{#1}%
		\DeclareNameAlias{author}{first-last}%
		\printbibliography[heading = none]%
	\end{refsection}%
}


\colorlet{light alert}{HKS07K60}
\tikzset{alert.bg/.style={rounded corners, fill=light alert}}
\tikzset{every picture/.style={line cap=round, semithick}}
% http://tex.stackexchange.com/questions/6135/how-to-make-beamer-overlays-with-tikz-node
\tikzset{onslide/.code args={<#1>#2}{\only<#1>{\pgfkeysalso{#2}}}}
\tikzset{invisible/.code args={<#1>}{\alt<#1>{\pgfkeysalso{transparent}}{\pgfkeysalso{opaque}}}}
\tikzset{uncover/.code args={<#1>}{\alt<#1>{\pgfkeysalso{opaque}}{\pgfkeysalso{opacity=0.25}}}}
\tikzset{visible/.code args={<#1>}{\alt<#1>{\pgfkeysalso{opaque}}{\pgfkeysalso{transparent}}}}
\tikzset{vuncover/.code args=%
	{<#1><#2>}%
	{\alt<#1>%
		{\alt<#2>%
			{\pgfkeysalso{opaque}}%
			{\pgfkeysalso{opacity=0.25}}%
		}{\pgfkeysalso{transparent}}%
	}%
}

\newcommand<%
	>{\tikzhighlight}[2][]{%
	\delegateStyle{\alt#3%
		{\tikz[baseline=0, anchor=base, inner sep=0.2em, text height=, text depth=]{\node[alert.bg, #1]{\inheritStyle{#2}};}}%
		{\tikz[baseline=0, anchor=base, inner sep=0.2em, text height=, text depth=]{\node[#1, fill=none]{\inheritStyle{#2}};}}%
	}%
}

\newcommand{\mathhighlight}{\tikzhighlight}

\newcommand<>{\mhl}[2][]{\mathhighlight#3[inner sep=0.2em, #1]{#2}}


\newcommand<>{\inlineblock}[2][]{{%
	\usebeamercolor*[fg]{block body}%
	\tikzhighlight#3[fill=block body.bg, #1]{#2}%
}}


% a small letter s for plurals of abbreviations
\newcommand*{\s}{{\scriptsize s}\xspace}


\newcommand<>*{\sout}[2][opacity=0.75, ultra thick]{%
	\delegateStyle{%
		\tikz[baseline=0, anchor=base, inner sep=0, outer sep=0]{
			\useasboundingbox node (n) {\inheritStyle{#2}};
			\only#3{
				\node (h) {\inheritStyle{\ifmmode\mathstrut\else\strut\fi}};
				\draw[#1] (n.west |- {$(h.south)!0.5!(h.north)$}) -- (n.east |- {$(h.south)!0.5!(h.north)$});
			}
		}%
	}%
}


% tight style
% Sets outer sep to default inner sep and inner sep to 0.
% Use this style for nodes that are neither drawn nor filled to prevent
% unwanted growth of the bounding box.
\tikzset{tight/.style={inner sep=0, outer sep=0.3333em}}


% rounded tree edges style
% usage: rounded tree edges={⟨direction⟩}{⟨looseness⟩}{⟨strength⟩}
\tikzset{
	rounded tree edges/.style n args={3}{
	edge from parent path={
	let
		\n{direction}={#1},
		\n{looseness}={#2},
		\n{strength}={#3},
		\p1=(\tikzparentnode),
		\p2=(\tikzchildnode),
		\p3=(\n{direction}:1pt),
		\p4=(\x2 - \x1, \y2 - \y1),
		\n{dist}={veclen(\p4)},
		\p4=(\x4 / \n{dist}, \y4 / \n{dist}),
		\n{angle}={atan2(\y4, \x4)},
		\n{delta}={Mod(\n{angle} - \n{direction}, 360)},
		\n{delta}={\n{delta} > 180 ? \n{delta} - 360  : \n{delta}},
		\n{delta}={\n{delta} >  90 ?  180 - \n{delta} : \n{delta}},
		\n{delta}={\n{delta} < -90 ? -180 - \n{delta} : \n{delta}}
	in (\tikzparentnode) .. controls
		+(    \n{angle}+\n{strength}*\n{delta}:\n{looseness}*0.3915*\n{dist}) and
		+(180+\n{angle}-\n{strength}*\n{delta}:\n{looseness}*0.3915*\n{dist}) ..
		(\tikzchildnode)
	}
	}
}


% Tear out snippets from PDFs.
% Usage: \tear[…]{file.pdf}
% The optional parameter is the same as for \includegraphics.
% Useful Arguments:
%   * page=‹pagenumber›
%   * trim=‹left› ‹bottom› ‹right› ‹top›
%   * width=0.98\linewidth
\newcommand*{\tear}[2][]{%
	\begin{tikzpicture}
		\node
			[ blur shadow
			, clip
			, decorate
			, decoration=random steps
			, draw
			, inner sep=0
			, preaction={fill=white}% hide the shadow if paper is transparent
			] {\includegraphics[#1]{#2}};
	\end{tikzpicture}%
}


\makeatletter
\newcommand*{\timeline}[3][0]{%
	\ifcsname timeline@cmd@#3\endcsname%
		\@timeline[#1]{#2}{#3}%
		\PackageWarning{timeline}{redefining timeline \@backslashchar\string#3}%
	\else%
		\ifcsname#3\endcsname%
			\errmessage{Command \@backslashchar\string#3 already defined}%
		\else%
			\@timeline[#1]{#2}{#3}%
		\fi%
	\fi%
}%
\newcommand*{\@timeline}[3][0]{%
	% mark command as timeline command – they can be overwritten
	\expandafter\def\csname timeline@cmd@#3\endcsname{}%
	\setcounter{@timeline}{#1}%
	\def\timeline@cmd{#3}%
	\timeline@reset%
	\timeline@append{0}%
	\@tfor\timeline@next:=#2\do{%
		\if\timeline@next+%
			\stepcounter{@timeline}%
			\timeline@append{,\the@timeline}%
		\else\if\timeline@next-%
			\stepcounter{@timeline}%
		\else%
			%\timeline@append{\timeline@next}%
			\GenericError{}{\protect\timeline: ignoring unknown character: \timeline@next}%
		\fi\fi%
	}%
}%
% \newcommand*{\tl}[1]{%
% 	\ifcsname timeline@cmd@#1\endcsname%
% 		\csname timeline@cmd@#1\endcsname%
% 	\else%
% 		0%
% 		%\GenericError{}{\protect\tl: timeline not defined: #1}%
% 	\fi%
% }%
\newcounter{@timeline}%
\def\timeline@reset{%
	\expandafter\def\csname\timeline@cmd\endcsname{}%
}%
\def\timeline@append#1{%
	\expandafter\edef\csname\timeline@cmd\endcsname{%
		\csname\timeline@cmd\endcsname#1%
	}%
}%
\makeatother


\newcommand*{\xminus}[1]{%
	\mathrel{\tikz[baseline={([yshift=-0.25em]n.south)}, inner sep=0, outer sep=0.2em]{%
		\node (n) {\(\scriptstyle #1\)};
		\draw (n.south west) -- (n.south east);
	}}%
}
\newcommand*{\tikzrightarrow}[1]{%
	\mathrel{\tikz[baseline={([yshift=-0.25em]n.south)}, inner sep=0, outer sep=0.2em]{%
		\node (n) {\(\scriptstyle #1\)};
		\draw[->, > = Computer Modern Rightarrow, line width = 0.4pt] (n.south west) -- (n.south east);
	}}%
}


%%%%%%%%%%%%%%%%%%%%%%%%%%%%%%%%%%%%%%%%%%%%%%%%%%%%%%%%%%%%%%%%%%%%%%%%%%%%%%
% document specific commands
%%%%%%%%%%%%%%%%%%%%%%%%%%%%%%%%%%%%%%%%%%%%%%%%%%%%%%%%%%%%%%%%%%%%%%%%%%%%%%

\newcommand<>*{\mycite}[1]{\uncover#2{{\color{HKS57K100}[\cite{#1}]}}}


\newcommand{\statetree}[1]{
	\tikz
	[ anchor=base
	, baseline=(current bounding box.center)
	, level distance=2em
	, sibling distance=2em
	]{
		\matrix
		[ draw=nt
		, edge from parent/.style={draw=black}
		, inner sep=0
		, nodes={inner sep=0.2em, rounded corners=0}
		, rounded corners
		] {#1\\}
	}
}


\newcommand*{\mylargeleaf}[1]{{\LARGE\color{HKS41K70}#1}}

\definecolor{state s}{named}{HKS57K80}
\definecolor{state t}{named}{HKS41K70}
\newcommand*{\stateS}[1]{{\color{state s}#1}}
\newcommand*{\stateT}[1]{{\color{state t}#1}}

\tikzset{
	subtree/.style =
		{ fill=lightgray
		, inner sep=0.02em
		, isosceles triangle apex angle=60
		, shape=isosceles triangle
		, shape border rotate=90
		}
	, state/.style = {circle, draw, inner sep=0.1em}
	, trans/.style = {rectangle, draw}
}

\newcommand*{\srBool}{\mathbb{B}}
\newcommand*{\srProb}{ℙ}


%%%%%%%%%%%%%%%%%%%%%%%%%%%%%%%%%%%%%%%%%%%%%%%%%%%%%%%%%%%%%%%%%%%%%%%%%%%%%%
% commands for specific notations
%%%%%%%%%%%%%%%%%%%%%%%%%%%%%%%%%%%%%%%%%%%%%%%%%%%%%%%%%%%%%%%%%%%%%%%%%%%%%%

\DeclareMathOperator*{\argmax}{argmax}

\newcommand*{\cardinality}[1]{\lvert#1\rvert}
\newcommand*{\corpussize}[1]{\lvert#1\rvert}

\DeclareMathOperator{\crispOp}{crisp}
\newcommand*        {\crisp}[2][0]{\crispOp\delim{#1}({#2})}

\DeclareMathOperator{\lhsOp}{lhs}
\newcommand*{\lhs}[1]{\lhsOp(#1)}

\DeclareMathOperator{\lklhdOp}{L}
\newcommand*{\lklhd}[2]{\lklhdOp(#1 ∣ #2)}

\DeclareMathOperator{\mleOp}{mle}
\newcommand*{\mle}[2][]{%
	\ifthenelse{\isempty{#1}}{%
		\mleOp(#2)%
	}{%
		\mleOp_{#1}(#2)%
	}%
}

\DeclareMathOperator{\mrg}{merge}

% CVD: color vision deficiencies
\definecolor{CVD light red}   {HTML}{FF8080}
\definecolor{CVD light yellow}{HTML}{FFFF80}
\definecolor{CVD light green} {HTML}{40FFC0}

\definecolor{nt}{named}{HKS41K70}
\newcommand*{\nt}[1]{{\color{nt}#1}}

% set of all probability distributions over #1
\DeclareMathOperator{\pdsOp}{Pd}
\newcommand*{\pds}[1]{\pdsOp(#1)}

\DeclareMathOperator{\positionsOp}{pos}
\newcommand*{\positions}[1]{\positionsOp(#1)}

\DeclareMathOperator{\rankOp}{rk}
\newcommand*{\rank}[1]{\rankOp(#1)}

\DeclareMathOperator{\runsOp}{run}
\newcommand*{\runs}[2][]{%
	\ifthenelse%
		{\isempty{#1}}%
		{\runsOp(#2)}%
		{\runsOp_{#1}(#2)}%
}

\newcommand*{\semantics}[1]{⟦#1⟧}

\DeclareMathOperator{\splt}{split}

\newcommand*{\subtree}[2]{#1|_{#2}}

\DeclareMathOperator{\supportOp}{supp}
\newcommand*{\support}[1]{\supportOp(#1)}

\newcommand*{\symId}{\textsc{\color{gray}Id}}
\newcommand*{\symCons}{\textsc{\color{gray}Cons}}
\newcommand*{\symFlip}{\textsc{\color{gray}Flip}}
\newcommand*{\symNull}{\textsc{\color{gray}Null}}
\newcommand*{\symNullR}{\textsc{\color{gray}N\(\overline{\textsc{ull}}\)}}
\newcommand*{\symSnoc}{\textsc{\color{gray}Snoc}}

\newcommand*{\transWTA}[4][]{#3 \xrightarrow{#1} #2(#4)}

\DeclareMathOperator{\uniqueRunOp}{r}
\newcommand*{\uniqueRun}[2][]{%
	\ifthenelse%
		{\isempty{#1}}%
		{\uniqueRunOp^{#2}}%
		{\uniqueRunOp_{\!#1}^{#2}}%
}

\DeclareMathOperator{\treesOp}{T}
\newcommand*{\trees}[2][]{%
	\ifthenelse%
		{\isempty{#1}}%
		{\treesOp_{\!#2}}%
		{\treesOp_{\!#2}(#1)}%
}
\DeclareMathOperator{\treesUOp}{U}
\newcommand*{\treesU}[2][]{%
	\ifthenelse%
		{\isempty{#1}}%
		{\treesUOp_{#2}}%
		{\treesUOp_{#2}(#1)}%
}


%%%%%%%%%%%%%%%%%%%%%%%%%%%%%%%%%%%%%%%%%%%%%%%%%%%%%%%%%%%%%%%%%%%%%%%%%%%%%%
% metadata
%%%%%%%%%%%%%%%%%%%%%%%%%%%%%%%%%%%%%%%%%%%%%%%%%%%%%%%%%%%%%%%%%%%%%%%%%%%%%%

\ifstandalonebeamer\else
	\title[Defense of Dissertation]{A Formal View on Training of Weighted Tree Automata by Likelihood-Driven State Splitting and Merging}
	\subtitle{Defense of Dissertation}
\fi
\author{Toni Dietze}
\institute[TU Dresden]{%
	\href{https://www.orchid.inf.tu-dresden.de/index.en/}{Chair for Foundations of Programming}
\\	\href{https://tu-dresden.de/ing/informatik/thi}{Institute of Theoretical Computer Science}
\\	\href{https://tu-dresden.de/ing/informatik}{Faculty of Computer Science}
\\	\href{https://tu-dresden.de/}{Technische Universität Dresden}
\\	01062 Dresden, Germany
}
\date[2018-09-27]{September 27, 2018}

\title{\jobname}
\begin{document}
\begin{standaloneframe}{\jobname}
	\printfullcite{2015DietzeNederhof}
	\printfullcite{2016Dietze}
	\printfullcite{2016OsterholzerDietzeHerrmann}
\end{standaloneframe}
\end{document}

\end{frame}


\section{Thank You}

\begin{frame}
	\centering
	%Thank you for your attention.
	% SPDX-License-Identifier: CC-BY-4.0
% Copyright 2018 Toni Dietze
\documentclass[beamer]{standalone}
% SPDX-License-Identifier: CC-BY-4.0 OR MIT-0
% Copyright 2018 Toni Dietze
%
\usefonttheme{professionalfonts}

% LuaLaTeX specific packages
\usepackage{fontspec}
	\defaultfontfeatures{Ligatures=TeX}
\usepackage{polyglossia}
	\setdefaultlanguage{english}
\usepackage{amsmath}  % has to be loaded before unicode-math
\usepackage[math-style=ISO]{unicode-math}
	\setmathfont{Latin Modern Math}
% 	\setmathfont[range={\mathcal,\mathbfcal},StylisticSet=1]{xits-math.otf}
% 	\setmathfont[range={"029F5}]{XITS Math}  % ⧵
% 	\setmathfont[range={\mathscr,\mathbfscr},StylisticSet=1]{Latin Modern Math}  % make \mathscr use the correct font

\usepackage[noend]{algpseudocode}
	\algrenewcommand\algorithmicrequire{\textbf{Input:}}
	\algrenewcommand\algorithmicensure{\textbf{Output:}}
\usepackage[backend=biber, maxbibnames=42, maxcitenames=42, sorting=ynt, style=authoryear]{biblatex}
\usepackage{csquotes}
\usepackage{mathtools}
\usepackage{media9}
\usepackage{scalerel}
\usepackage{standalone}
\usepackage{tikz}
	\usetikzlibrary{arrows.meta}
	\usetikzlibrary{backgrounds}
	\usetikzlibrary{calc}
	\usetikzlibrary{decorations}
	\usetikzlibrary{decorations.pathmorphing}
	\usetikzlibrary{decorations.pathreplacing}
	\usetikzlibrary{fadings}
	\usetikzlibrary{fit}
	\usetikzlibrary{graphs}
	\usetikzlibrary{graphdrawing}
	\usetikzlibrary{intersections}
	\usetikzlibrary{positioning}
	\usetikzlibrary{quotes}
	\usetikzlibrary{shadows.blur}
	\usetikzlibrary{shapes.arrows}
	\usetikzlibrary{shapes.geometric}
	\usegdlibrary{trees}
\usepackage{xifthen}
\usepackage{xspace}

\usepackage{pgfplots}
	\pgfplotsset
		{ compat = 1.15
		, /pgf/number format/1000 sep = {\,}
		, /pgf/number format/assume math mode = true
		, every axis plot/.append style =
			{ mark options = {fill opacity = 0.25}
			}
		}
	\usepgfplotslibrary{groupplots}
\usepackage{pgfplotstable}

\hypersetup
	{ bookmarksopen
	, pdflang = en
	, unicode
	}


%%%%%%%%%%%%%%%%%%%%%%%%%%%%%%%%%%%%%%%%%%%%%%%%%%%%%%%%%%%%%%%%%%%%%%%%%%%%%%


% always show bad boxes
%\overfullrule=1em


%%%%%%%%%%%%%%%%%%%%%%%%%%%%%%%%%%%%%%%%%%%%%%%%%%%%%%%%%%%%%%%%%%%%%%%%%%%%%%
% biblatex
%%%%%%%%%%%%%%%%%%%%%%%%%%%%%%%%%%%%%%%%%%%%%%%%%%%%%%%%%%%%%%%%%%%%%%%%%%%%%%

\addbibresource{slides-dissertation-defense.bib}
% \renewcommand*{\finalnamedelim}{\addcomma\space}
% \setlength{\bibitemsep}{1em}
% 
\AtEveryBibitem{% Clean up the bibtex rather than editing it
 \clearlist{address}
 \clearfield{date}
 \clearfield{eprint}
 \clearfield{isbn}
 \clearfield{issn}
 \clearlist{language}
 \clearlist{location}
 \clearfield{month}
 \clearfield{series}
%  \clearfield{url}
%  \clearfield{doi}
 \clearfield{organization}

%  \ifentrytype{book}{}{% Remove stuff except for books
%   \clearfield{booktitle}
%   \clearfield{pages}
  \clearlist{publisher}
  \clearname{editor}
%  }
}
% do not print url if doi is present
% http://tex.stackexchange.com/questions/154864/biblatex-use-doi-only-if-there-is-no-url
\DeclareSourcemap{
	\maps[datatype=bibtex]{
		\map{
			\step[fieldsource=doi,final]
			\step[fieldset=url,null]
}	}	}
%
% remove qoutes around titles
\DeclareFieldFormat
	[article,inbook,incollection,inproceedings,patent,thesis,unpublished]
	{title}{#1\isdot}
% 
% \DeclareFieldFormat{url}{\mkbibacro{URL}\addcolon\addnbspace\url{#1}}
% 
% \DeclareNameAlias{sortname}{first-last}
% 
\renewbibmacro{in:}{\ifentrytype{article}{}{}}


%%%%%%%%%%%%%%%%%%%%%%%%%%%%%%%%%%%%%%%%%%%%%%%%%%%%%%%%%%%%%%%%%%%%%%%%%%%%%%
% beamer
%%%%%%%%%%%%%%%%%%%%%%%%%%%%%%%%%%%%%%%%%%%%%%%%%%%%%%%%%%%%%%%%%%%%%%%%%%%%%%

\useoutertheme{infolines}
\makeatletter
% based on
% /usr/share/texmf-dist/tex/latex/beamer/beamerouterthemeinfolines.sty
\setbeamertemplate{footline}
{%
	\leavevmode%
	\hbox{%
	\begin{beamercolorbox}[wd=.333333\paperwidth,ht=2.25ex,dp=1ex,center]{author in head/foot}%
		\usebeamerfont{author in head/foot}\insertshortauthor\expandafter\beamer@ifempty\expandafter{\beamer@shortinstitute}{}{~~(\insertshortinstitute)}
	\end{beamercolorbox}%
	\begin{beamercolorbox}[wd=.333333\paperwidth,ht=2.25ex,dp=1ex,center]{title in head/foot}%
		\usebeamerfont{title in head/foot}\insertshorttitle
	\end{beamercolorbox}%
	\begin{beamercolorbox}[wd=.333333\paperwidth,ht=2.25ex,dp=1ex,right]{date in head/foot}%
		\usebeamerfont{date in head/foot}%
		\hfill\insertshortdate\hfill\hfill%
		%\hspace*{2ex}%
		%\insertshortdate%
		%\hspace{0pt plus 1 filll}%
		%(\insertframenumber.\insertoverlaynumber{} / \insertmainframenumber)%
		%\hspace{0pt plus 1 filll}%
		\phantom{000}\llap{\insertpagenumber} / \insertpresentationendpage%
		\hspace*{2ex}%
	\end{beamercolorbox}}%
	\vskip0pt%
}
\makeatother
\useinnertheme{circles}
\beamertemplatenavigationsymbolsempty
\setbeamertemplate{bibliography item}{}
\setbeamertemplate{headline}[default]

\input{tudcolors.tex}
\setbeamercolor*{alerted text}{fg=HKS07K100}
\usecolortheme[named=HKS41K100]{structure}

\setbeamercolor*{palette primary}{use=structure,fg=white,bg=structure.fg}
\setbeamercolor*{palette secondary}{use=structure,fg=white,bg=structure.fg!80}
\setbeamercolor*{palette tertiary}{use=structure,fg=white,bg=structure.fg!60}
\setbeamercolor*{palette quaternary}{fg=white,bg=black}

\setbeamercolor*{sidebar}{use=structure,bg=structure.fg}

\setbeamercolor*{palette sidebar primary}{use=structure,fg=structure.fg!20}
\setbeamercolor*{palette sidebar secondary}{fg=white}
\setbeamercolor*{palette sidebar tertiary}{use=structure,fg=structure.fg!40}
\setbeamercolor*{palette sidebar quaternary}{fg=white}

\setbeamercolor*{titlelike}{parent=palette primary}

\setbeamercolor*{separation line}{}
\setbeamercolor*{fine separation line}{}

\setbeamercolor{block title}{use=structure,fg=white,bg=structure.fg}
\setbeamercolor{block title alerted}{use=alerted text,fg=white,bg=alerted text.fg!75!black}
\setbeamercolor{block title example}{use=example text,fg=white,bg=example text.fg!75!black}

\setbeamercolor{block body}{parent=normal text,use=block title,bg=block title.bg!10!bg}
\setbeamercolor{block body alerted}{parent=normal text,use=block title alerted,bg=block title alerted.bg!10!bg}
\setbeamercolor{block body example}{parent=normal text,use=block title example,bg=block title example.bg!10!bg}

% \setbeamertemplate{itemize items}[default]


%%%%%%%%%%%%%%%%%%%%%%%%%%%%%%%%%%%%%%%%%%%%%%%%%%%%%%%%%%%%%%%%%%%%%%%%%%%%%%
% TikZ
%%%%%%%%%%%%%%%%%%%%%%%%%%%%%%%%%%%%%%%%%%%%%%%%%%%%%%%%%%%%%%%%%%%%%%%%%%%%%%

\tikzset
	{ > = Stealth
	}


%%%%%%%%%%%%%%%%%%%%%%%%%%%%%%%%%%%%%%%%%%%%%%%%%%%%%%%%%%%%%%%%%%%%%%%%%%%%%%
% general commands and styles
%%%%%%%%%%%%%%%%%%%%%%%%%%%%%%%%%%%%%%%%%%%%%%%%%%%%%%%%%%%%%%%%%%%%%%%%%%%%%%

% \delegateStyle and \inheritStyle command
% usage: \delegateStyle{… \inheritStyle{…} …}
% example: \(X_{\delegateStyle{\fbox{\inheritStyle{X}}}}\)
% Save the current style and regain it in the argument.
% This works both for math and text mode, and can be nested.
% Acknowledgments: Based on \ThisStyle and \SavedStyle from scalerel package.
\makeatletter
\newcommand*{\@inheritStyle@D}[1]{\(\displaystyle      #1\)}
\newcommand*{\@inheritStyle@T}[1]{\(\textstyle         #1\)}
\newcommand*{\@inheritStyle@S}[1]{\(\scriptstyle       #1\)}
\newcommand*{\@inheritStyle@s}[1]{\(\scriptscriptstyle #1\)}
\newcommand*{\@inheritStyle@t}[1]{#1}
\newcommand*{\inheritStyle}{\csname @inheritStyle@\@inheritStyleSwitch\endcsname}
\newcommand*{\delegateStyle}[1]{%
	\ifmmode%
		\mathchoice%
		{\edef\@inheritStyleSwitch{D}#1}%
		{\edef\@inheritStyleSwitch{T}#1}%
		{\edef\@inheritStyleSwitch{S}#1}%
		{\edef\@inheritStyleSwitch{s}#1}%
	\else%
		\edef\@inheritStyleSwitch{t}#1%
	\fi%
}
\makeatother


% \oalt command
% requires: \delegateStyle and \inheritStyle command
% usage: \oalt<…>[…]{…}{…} (cf. \alt)
% Behaves like \alt, but reserves space according to largest overlays.
% The optional argument defines the alignment inside the reserved space;
% it is one of c, l, r, s (cf. \makebox); the default is c.
\makeatletter
\newlength{\oalt@dp}
\newlength{\oalt@ht}
\newlength{\oalt@wd}
\newbox{\oalt@a}
\newbox{\oalt@b}
\newbox{\oalt@empty}
\newcommand<>*{\oalt}[3][c]{%
	\delegateStyle{%
		% based on \setto… in /usr/share/texmf-dist/tex/latex/base/latex.ltx
		\setbox\oalt@a\hbox{\inheritStyle{#2}}%
		\setbox\oalt@b\hbox{\inheritStyle{#3}}%
		\pgfmathsetlength{\oalt@dp}{max(\dp\oalt@a,\dp\oalt@b)}%
		\pgfmathsetlength{\oalt@ht}{max(\ht\oalt@a,\ht\oalt@b)}%
		\pgfmathsetlength{\oalt@wd}{max(\wd\oalt@a,\wd\oalt@b)}%
		\raisebox{0pt}[\oalt@ht][\oalt@dp]{%
			\makebox[\oalt@wd][#1]{%
				\alt#4{\unhbox\oalt@a}{\unhbox\oalt@b}%
			}%
		}%
		\setbox\oalt@a\box\oalt@empty%
		\setbox\oalt@b\box\oalt@empty%
	}%
}
\makeatother


% \otemporal command
% requires: \delegateStyle and \inheritStyle command
% usage: \otemporal<…>[…]{…}{…}{…} (cf. \temporal)
% Behaves like \temporal, but reserves space according to largest overlays.
% The optional argument defines the alignment inside the reserved space;
% it is one of c, l, r, s (cf. \makebox); the default is c.
\makeatletter
\newlength{\ot@dp}
\newlength{\ot@ht}
\newlength{\ot@wd}
\newbox{\ot@a}
\newbox{\ot@b}
\newbox{\ot@c}
\newbox{\ot@empty}
\newcommand<>*{\otemporal}[4][c]{%
	\delegateStyle{%
		% based on \setto… in /usr/share/texmf-dist/tex/latex/base/latex.ltx
		\setbox\ot@a\hbox{\inheritStyle{#2}}%
		\setbox\ot@b\hbox{\inheritStyle{#3}}%
		\setbox\ot@c\hbox{\inheritStyle{#4}}%
		\pgfmathsetlength{\ot@dp}{max(\dp\ot@a,\dp\ot@b,\dp\ot@c)}%
		\pgfmathsetlength{\ot@ht}{max(\ht\ot@a,\ht\ot@b,\ht\ot@c)}%
		\pgfmathsetlength{\ot@wd}{max(\wd\ot@a,\wd\ot@b,\wd\ot@c)}%
		\raisebox{0pt}[\ot@ht][\ot@dp]{%
			\makebox[\ot@wd][#1]{%
				\temporal#5{\unhbox\ot@a}{\unhbox\ot@b}{\unhbox\ot@c}%
			}%
		}%
		\setbox\ot@a\box\ot@empty%
		\setbox\ot@b\box\ot@empty%
		\setbox\ot@c\box\ot@empty%
	}%
}
\makeatother


% Resize delimiters like braces, brackets, etc.
% Parameters: size, left delimiter, formula, right delimiter
% Example: \delim2({\frac{1}{2}})
\newcommand*{\delim}[4]{%
	\ifcase#1%
		#2#3#4%
	\or%
		\bigl#2#3\bigr#4%
	\or%
		\Bigl#2#3\Bigr#4%
	\or%
		\biggl#2#3\biggr#4%
	\or%
		\Biggl#2#3\Biggr#4%
	\else%
		\left#2#3\right#4%
	\fi%
}


% similar to \fullcite, but using the formatting of \printbibliography
\newcommand*{\printfullcite}[1]{%
	\begin{refsection}%
		\nocite{#1}%
		\DeclareNameAlias{author}{first-last}%
		\printbibliography[heading = none]%
	\end{refsection}%
}


\colorlet{light alert}{HKS07K60}
\tikzset{alert.bg/.style={rounded corners, fill=light alert}}
\tikzset{every picture/.style={line cap=round, semithick}}
% http://tex.stackexchange.com/questions/6135/how-to-make-beamer-overlays-with-tikz-node
\tikzset{onslide/.code args={<#1>#2}{\only<#1>{\pgfkeysalso{#2}}}}
\tikzset{invisible/.code args={<#1>}{\alt<#1>{\pgfkeysalso{transparent}}{\pgfkeysalso{opaque}}}}
\tikzset{uncover/.code args={<#1>}{\alt<#1>{\pgfkeysalso{opaque}}{\pgfkeysalso{opacity=0.25}}}}
\tikzset{visible/.code args={<#1>}{\alt<#1>{\pgfkeysalso{opaque}}{\pgfkeysalso{transparent}}}}
\tikzset{vuncover/.code args=%
	{<#1><#2>}%
	{\alt<#1>%
		{\alt<#2>%
			{\pgfkeysalso{opaque}}%
			{\pgfkeysalso{opacity=0.25}}%
		}{\pgfkeysalso{transparent}}%
	}%
}

\newcommand<%
	>{\tikzhighlight}[2][]{%
	\delegateStyle{\alt#3%
		{\tikz[baseline=0, anchor=base, inner sep=0.2em, text height=, text depth=]{\node[alert.bg, #1]{\inheritStyle{#2}};}}%
		{\tikz[baseline=0, anchor=base, inner sep=0.2em, text height=, text depth=]{\node[#1, fill=none]{\inheritStyle{#2}};}}%
	}%
}

\newcommand{\mathhighlight}{\tikzhighlight}

\newcommand<>{\mhl}[2][]{\mathhighlight#3[inner sep=0.2em, #1]{#2}}


\newcommand<>{\inlineblock}[2][]{{%
	\usebeamercolor*[fg]{block body}%
	\tikzhighlight#3[fill=block body.bg, #1]{#2}%
}}


% a small letter s for plurals of abbreviations
\newcommand*{\s}{{\scriptsize s}\xspace}


\newcommand<>*{\sout}[2][opacity=0.75, ultra thick]{%
	\delegateStyle{%
		\tikz[baseline=0, anchor=base, inner sep=0, outer sep=0]{
			\useasboundingbox node (n) {\inheritStyle{#2}};
			\only#3{
				\node (h) {\inheritStyle{\ifmmode\mathstrut\else\strut\fi}};
				\draw[#1] (n.west |- {$(h.south)!0.5!(h.north)$}) -- (n.east |- {$(h.south)!0.5!(h.north)$});
			}
		}%
	}%
}


% tight style
% Sets outer sep to default inner sep and inner sep to 0.
% Use this style for nodes that are neither drawn nor filled to prevent
% unwanted growth of the bounding box.
\tikzset{tight/.style={inner sep=0, outer sep=0.3333em}}


% rounded tree edges style
% usage: rounded tree edges={⟨direction⟩}{⟨looseness⟩}{⟨strength⟩}
\tikzset{
	rounded tree edges/.style n args={3}{
	edge from parent path={
	let
		\n{direction}={#1},
		\n{looseness}={#2},
		\n{strength}={#3},
		\p1=(\tikzparentnode),
		\p2=(\tikzchildnode),
		\p3=(\n{direction}:1pt),
		\p4=(\x2 - \x1, \y2 - \y1),
		\n{dist}={veclen(\p4)},
		\p4=(\x4 / \n{dist}, \y4 / \n{dist}),
		\n{angle}={atan2(\y4, \x4)},
		\n{delta}={Mod(\n{angle} - \n{direction}, 360)},
		\n{delta}={\n{delta} > 180 ? \n{delta} - 360  : \n{delta}},
		\n{delta}={\n{delta} >  90 ?  180 - \n{delta} : \n{delta}},
		\n{delta}={\n{delta} < -90 ? -180 - \n{delta} : \n{delta}}
	in (\tikzparentnode) .. controls
		+(    \n{angle}+\n{strength}*\n{delta}:\n{looseness}*0.3915*\n{dist}) and
		+(180+\n{angle}-\n{strength}*\n{delta}:\n{looseness}*0.3915*\n{dist}) ..
		(\tikzchildnode)
	}
	}
}


% Tear out snippets from PDFs.
% Usage: \tear[…]{file.pdf}
% The optional parameter is the same as for \includegraphics.
% Useful Arguments:
%   * page=‹pagenumber›
%   * trim=‹left› ‹bottom› ‹right› ‹top›
%   * width=0.98\linewidth
\newcommand*{\tear}[2][]{%
	\begin{tikzpicture}
		\node
			[ blur shadow
			, clip
			, decorate
			, decoration=random steps
			, draw
			, inner sep=0
			, preaction={fill=white}% hide the shadow if paper is transparent
			] {\includegraphics[#1]{#2}};
	\end{tikzpicture}%
}


\makeatletter
\newcommand*{\timeline}[3][0]{%
	\ifcsname timeline@cmd@#3\endcsname%
		\@timeline[#1]{#2}{#3}%
		\PackageWarning{timeline}{redefining timeline \@backslashchar\string#3}%
	\else%
		\ifcsname#3\endcsname%
			\errmessage{Command \@backslashchar\string#3 already defined}%
		\else%
			\@timeline[#1]{#2}{#3}%
		\fi%
	\fi%
}%
\newcommand*{\@timeline}[3][0]{%
	% mark command as timeline command – they can be overwritten
	\expandafter\def\csname timeline@cmd@#3\endcsname{}%
	\setcounter{@timeline}{#1}%
	\def\timeline@cmd{#3}%
	\timeline@reset%
	\timeline@append{0}%
	\@tfor\timeline@next:=#2\do{%
		\if\timeline@next+%
			\stepcounter{@timeline}%
			\timeline@append{,\the@timeline}%
		\else\if\timeline@next-%
			\stepcounter{@timeline}%
		\else%
			%\timeline@append{\timeline@next}%
			\GenericError{}{\protect\timeline: ignoring unknown character: \timeline@next}%
		\fi\fi%
	}%
}%
% \newcommand*{\tl}[1]{%
% 	\ifcsname timeline@cmd@#1\endcsname%
% 		\csname timeline@cmd@#1\endcsname%
% 	\else%
% 		0%
% 		%\GenericError{}{\protect\tl: timeline not defined: #1}%
% 	\fi%
% }%
\newcounter{@timeline}%
\def\timeline@reset{%
	\expandafter\def\csname\timeline@cmd\endcsname{}%
}%
\def\timeline@append#1{%
	\expandafter\edef\csname\timeline@cmd\endcsname{%
		\csname\timeline@cmd\endcsname#1%
	}%
}%
\makeatother


\newcommand*{\xminus}[1]{%
	\mathrel{\tikz[baseline={([yshift=-0.25em]n.south)}, inner sep=0, outer sep=0.2em]{%
		\node (n) {\(\scriptstyle #1\)};
		\draw (n.south west) -- (n.south east);
	}}%
}
\newcommand*{\tikzrightarrow}[1]{%
	\mathrel{\tikz[baseline={([yshift=-0.25em]n.south)}, inner sep=0, outer sep=0.2em]{%
		\node (n) {\(\scriptstyle #1\)};
		\draw[->, > = Computer Modern Rightarrow, line width = 0.4pt] (n.south west) -- (n.south east);
	}}%
}


%%%%%%%%%%%%%%%%%%%%%%%%%%%%%%%%%%%%%%%%%%%%%%%%%%%%%%%%%%%%%%%%%%%%%%%%%%%%%%
% document specific commands
%%%%%%%%%%%%%%%%%%%%%%%%%%%%%%%%%%%%%%%%%%%%%%%%%%%%%%%%%%%%%%%%%%%%%%%%%%%%%%

\newcommand<>*{\mycite}[1]{\uncover#2{{\color{HKS57K100}[\cite{#1}]}}}


\newcommand{\statetree}[1]{
	\tikz
	[ anchor=base
	, baseline=(current bounding box.center)
	, level distance=2em
	, sibling distance=2em
	]{
		\matrix
		[ draw=nt
		, edge from parent/.style={draw=black}
		, inner sep=0
		, nodes={inner sep=0.2em, rounded corners=0}
		, rounded corners
		] {#1\\}
	}
}


\newcommand*{\mylargeleaf}[1]{{\LARGE\color{HKS41K70}#1}}

\definecolor{state s}{named}{HKS57K80}
\definecolor{state t}{named}{HKS41K70}
\newcommand*{\stateS}[1]{{\color{state s}#1}}
\newcommand*{\stateT}[1]{{\color{state t}#1}}

\tikzset{
	subtree/.style =
		{ fill=lightgray
		, inner sep=0.02em
		, isosceles triangle apex angle=60
		, shape=isosceles triangle
		, shape border rotate=90
		}
	, state/.style = {circle, draw, inner sep=0.1em}
	, trans/.style = {rectangle, draw}
}

\newcommand*{\srBool}{\mathbb{B}}
\newcommand*{\srProb}{ℙ}


%%%%%%%%%%%%%%%%%%%%%%%%%%%%%%%%%%%%%%%%%%%%%%%%%%%%%%%%%%%%%%%%%%%%%%%%%%%%%%
% commands for specific notations
%%%%%%%%%%%%%%%%%%%%%%%%%%%%%%%%%%%%%%%%%%%%%%%%%%%%%%%%%%%%%%%%%%%%%%%%%%%%%%

\DeclareMathOperator*{\argmax}{argmax}

\newcommand*{\cardinality}[1]{\lvert#1\rvert}
\newcommand*{\corpussize}[1]{\lvert#1\rvert}

\DeclareMathOperator{\crispOp}{crisp}
\newcommand*        {\crisp}[2][0]{\crispOp\delim{#1}({#2})}

\DeclareMathOperator{\lhsOp}{lhs}
\newcommand*{\lhs}[1]{\lhsOp(#1)}

\DeclareMathOperator{\lklhdOp}{L}
\newcommand*{\lklhd}[2]{\lklhdOp(#1 ∣ #2)}

\DeclareMathOperator{\mleOp}{mle}
\newcommand*{\mle}[2][]{%
	\ifthenelse{\isempty{#1}}{%
		\mleOp(#2)%
	}{%
		\mleOp_{#1}(#2)%
	}%
}

\DeclareMathOperator{\mrg}{merge}

% CVD: color vision deficiencies
\definecolor{CVD light red}   {HTML}{FF8080}
\definecolor{CVD light yellow}{HTML}{FFFF80}
\definecolor{CVD light green} {HTML}{40FFC0}

\definecolor{nt}{named}{HKS41K70}
\newcommand*{\nt}[1]{{\color{nt}#1}}

% set of all probability distributions over #1
\DeclareMathOperator{\pdsOp}{Pd}
\newcommand*{\pds}[1]{\pdsOp(#1)}

\DeclareMathOperator{\positionsOp}{pos}
\newcommand*{\positions}[1]{\positionsOp(#1)}

\DeclareMathOperator{\rankOp}{rk}
\newcommand*{\rank}[1]{\rankOp(#1)}

\DeclareMathOperator{\runsOp}{run}
\newcommand*{\runs}[2][]{%
	\ifthenelse%
		{\isempty{#1}}%
		{\runsOp(#2)}%
		{\runsOp_{#1}(#2)}%
}

\newcommand*{\semantics}[1]{⟦#1⟧}

\DeclareMathOperator{\splt}{split}

\newcommand*{\subtree}[2]{#1|_{#2}}

\DeclareMathOperator{\supportOp}{supp}
\newcommand*{\support}[1]{\supportOp(#1)}

\newcommand*{\symId}{\textsc{\color{gray}Id}}
\newcommand*{\symCons}{\textsc{\color{gray}Cons}}
\newcommand*{\symFlip}{\textsc{\color{gray}Flip}}
\newcommand*{\symNull}{\textsc{\color{gray}Null}}
\newcommand*{\symNullR}{\textsc{\color{gray}N\(\overline{\textsc{ull}}\)}}
\newcommand*{\symSnoc}{\textsc{\color{gray}Snoc}}

\newcommand*{\transWTA}[4][]{#3 \xrightarrow{#1} #2(#4)}

\DeclareMathOperator{\uniqueRunOp}{r}
\newcommand*{\uniqueRun}[2][]{%
	\ifthenelse%
		{\isempty{#1}}%
		{\uniqueRunOp^{#2}}%
		{\uniqueRunOp_{\!#1}^{#2}}%
}

\DeclareMathOperator{\treesOp}{T}
\newcommand*{\trees}[2][]{%
	\ifthenelse%
		{\isempty{#1}}%
		{\treesOp_{\!#2}}%
		{\treesOp_{\!#2}(#1)}%
}
\DeclareMathOperator{\treesUOp}{U}
\newcommand*{\treesU}[2][]{%
	\ifthenelse%
		{\isempty{#1}}%
		{\treesUOp_{#2}}%
		{\treesUOp_{#2}(#1)}%
}


%%%%%%%%%%%%%%%%%%%%%%%%%%%%%%%%%%%%%%%%%%%%%%%%%%%%%%%%%%%%%%%%%%%%%%%%%%%%%%
% metadata
%%%%%%%%%%%%%%%%%%%%%%%%%%%%%%%%%%%%%%%%%%%%%%%%%%%%%%%%%%%%%%%%%%%%%%%%%%%%%%

\ifstandalonebeamer\else
	\title[Defense of Dissertation]{A Formal View on Training of Weighted Tree Automata by Likelihood-Driven State Splitting and Merging}
	\subtitle{Defense of Dissertation}
\fi
\author{Toni Dietze}
\institute[TU Dresden]{%
	\href{https://www.orchid.inf.tu-dresden.de/index.en/}{Chair for Foundations of Programming}
\\	\href{https://tu-dresden.de/ing/informatik/thi}{Institute of Theoretical Computer Science}
\\	\href{https://tu-dresden.de/ing/informatik}{Faculty of Computer Science}
\\	\href{https://tu-dresden.de/}{Technische Universität Dresden}
\\	01062 Dresden, Germany
}
\date[2018-09-27]{September 27, 2018}

\title{\jobname}
\begin{document}
\begin{standaloneframe}
	\centering
	\begin{tikzpicture}[anchor=base, level sep=1em, tree layout]
		\node {S}
			child { node {VP}
				child { node {VB} child { node {\mylargeleaf{Thank}} } }
				child { node {NP}
					child { node {PRP} child { node {\mylargeleaf{you}} } }
				}
				child { node {PP}
					child { node {IN} child { node {\mylargeleaf{for}} } }
					child { node {NP}
						child { node {PRP\$} child { node {\mylargeleaf{your}} } }
						child { node {NN} child { node {\mylargeleaf{attention}} } }
					}
				}
			};
	\end{tikzpicture}
\end{standaloneframe}
\end{document}
% (ROOT
%   (S
%     (VP (VB Thank)
%       (NP (PRP you))
%       (PP (IN for)
%         (NP (PRP$ your) (NN attention))))
%     (. .)))

\end{frame}


%%%%%%%%%%%%%%%%%%%%%%%%%%%%%%%%%%%%%%%%%%%%%%%%%%%%%%%%%%%%%%%%%%%%%%%%%%%%%%
\appendix
%%%%%%%%%%%%%%%%%%%%%%%%%%%%%%%%%%%%%%%%%%%%%%%%%%%%%%%%%%%%%%%%%%%%%%%%%%%%%%


\section{References}

\begin{frame}{\secname}
	\printbibliography
\end{frame}


\section{I see the man with the telescope.}

\begin{frame}{\secname}
	\centering
	% SPDX-License-Identifier: CC-BY-4.0
% Copyright 2018 Toni Dietze
\documentclass[beamer]{standalone}
% SPDX-License-Identifier: CC-BY-4.0 OR MIT-0
% Copyright 2018 Toni Dietze
%
\usefonttheme{professionalfonts}

% LuaLaTeX specific packages
\usepackage{fontspec}
	\defaultfontfeatures{Ligatures=TeX}
\usepackage{polyglossia}
	\setdefaultlanguage{english}
\usepackage{amsmath}  % has to be loaded before unicode-math
\usepackage[math-style=ISO]{unicode-math}
	\setmathfont{Latin Modern Math}
% 	\setmathfont[range={\mathcal,\mathbfcal},StylisticSet=1]{xits-math.otf}
% 	\setmathfont[range={"029F5}]{XITS Math}  % ⧵
% 	\setmathfont[range={\mathscr,\mathbfscr},StylisticSet=1]{Latin Modern Math}  % make \mathscr use the correct font

\usepackage[noend]{algpseudocode}
	\algrenewcommand\algorithmicrequire{\textbf{Input:}}
	\algrenewcommand\algorithmicensure{\textbf{Output:}}
\usepackage[backend=biber, maxbibnames=42, maxcitenames=42, sorting=ynt, style=authoryear]{biblatex}
\usepackage{csquotes}
\usepackage{mathtools}
\usepackage{media9}
\usepackage{scalerel}
\usepackage{standalone}
\usepackage{tikz}
	\usetikzlibrary{arrows.meta}
	\usetikzlibrary{backgrounds}
	\usetikzlibrary{calc}
	\usetikzlibrary{decorations}
	\usetikzlibrary{decorations.pathmorphing}
	\usetikzlibrary{decorations.pathreplacing}
	\usetikzlibrary{fadings}
	\usetikzlibrary{fit}
	\usetikzlibrary{graphs}
	\usetikzlibrary{graphdrawing}
	\usetikzlibrary{intersections}
	\usetikzlibrary{positioning}
	\usetikzlibrary{quotes}
	\usetikzlibrary{shadows.blur}
	\usetikzlibrary{shapes.arrows}
	\usetikzlibrary{shapes.geometric}
	\usegdlibrary{trees}
\usepackage{xifthen}
\usepackage{xspace}

\usepackage{pgfplots}
	\pgfplotsset
		{ compat = 1.15
		, /pgf/number format/1000 sep = {\,}
		, /pgf/number format/assume math mode = true
		, every axis plot/.append style =
			{ mark options = {fill opacity = 0.25}
			}
		}
	\usepgfplotslibrary{groupplots}
\usepackage{pgfplotstable}

\hypersetup
	{ bookmarksopen
	, pdflang = en
	, unicode
	}


%%%%%%%%%%%%%%%%%%%%%%%%%%%%%%%%%%%%%%%%%%%%%%%%%%%%%%%%%%%%%%%%%%%%%%%%%%%%%%


% always show bad boxes
%\overfullrule=1em


%%%%%%%%%%%%%%%%%%%%%%%%%%%%%%%%%%%%%%%%%%%%%%%%%%%%%%%%%%%%%%%%%%%%%%%%%%%%%%
% biblatex
%%%%%%%%%%%%%%%%%%%%%%%%%%%%%%%%%%%%%%%%%%%%%%%%%%%%%%%%%%%%%%%%%%%%%%%%%%%%%%

\addbibresource{slides-dissertation-defense.bib}
% \renewcommand*{\finalnamedelim}{\addcomma\space}
% \setlength{\bibitemsep}{1em}
% 
\AtEveryBibitem{% Clean up the bibtex rather than editing it
 \clearlist{address}
 \clearfield{date}
 \clearfield{eprint}
 \clearfield{isbn}
 \clearfield{issn}
 \clearlist{language}
 \clearlist{location}
 \clearfield{month}
 \clearfield{series}
%  \clearfield{url}
%  \clearfield{doi}
 \clearfield{organization}

%  \ifentrytype{book}{}{% Remove stuff except for books
%   \clearfield{booktitle}
%   \clearfield{pages}
  \clearlist{publisher}
  \clearname{editor}
%  }
}
% do not print url if doi is present
% http://tex.stackexchange.com/questions/154864/biblatex-use-doi-only-if-there-is-no-url
\DeclareSourcemap{
	\maps[datatype=bibtex]{
		\map{
			\step[fieldsource=doi,final]
			\step[fieldset=url,null]
}	}	}
%
% remove qoutes around titles
\DeclareFieldFormat
	[article,inbook,incollection,inproceedings,patent,thesis,unpublished]
	{title}{#1\isdot}
% 
% \DeclareFieldFormat{url}{\mkbibacro{URL}\addcolon\addnbspace\url{#1}}
% 
% \DeclareNameAlias{sortname}{first-last}
% 
\renewbibmacro{in:}{\ifentrytype{article}{}{}}


%%%%%%%%%%%%%%%%%%%%%%%%%%%%%%%%%%%%%%%%%%%%%%%%%%%%%%%%%%%%%%%%%%%%%%%%%%%%%%
% beamer
%%%%%%%%%%%%%%%%%%%%%%%%%%%%%%%%%%%%%%%%%%%%%%%%%%%%%%%%%%%%%%%%%%%%%%%%%%%%%%

\useoutertheme{infolines}
\makeatletter
% based on
% /usr/share/texmf-dist/tex/latex/beamer/beamerouterthemeinfolines.sty
\setbeamertemplate{footline}
{%
	\leavevmode%
	\hbox{%
	\begin{beamercolorbox}[wd=.333333\paperwidth,ht=2.25ex,dp=1ex,center]{author in head/foot}%
		\usebeamerfont{author in head/foot}\insertshortauthor\expandafter\beamer@ifempty\expandafter{\beamer@shortinstitute}{}{~~(\insertshortinstitute)}
	\end{beamercolorbox}%
	\begin{beamercolorbox}[wd=.333333\paperwidth,ht=2.25ex,dp=1ex,center]{title in head/foot}%
		\usebeamerfont{title in head/foot}\insertshorttitle
	\end{beamercolorbox}%
	\begin{beamercolorbox}[wd=.333333\paperwidth,ht=2.25ex,dp=1ex,right]{date in head/foot}%
		\usebeamerfont{date in head/foot}%
		\hfill\insertshortdate\hfill\hfill%
		%\hspace*{2ex}%
		%\insertshortdate%
		%\hspace{0pt plus 1 filll}%
		%(\insertframenumber.\insertoverlaynumber{} / \insertmainframenumber)%
		%\hspace{0pt plus 1 filll}%
		\phantom{000}\llap{\insertpagenumber} / \insertpresentationendpage%
		\hspace*{2ex}%
	\end{beamercolorbox}}%
	\vskip0pt%
}
\makeatother
\useinnertheme{circles}
\beamertemplatenavigationsymbolsempty
\setbeamertemplate{bibliography item}{}
\setbeamertemplate{headline}[default]

\input{tudcolors.tex}
\setbeamercolor*{alerted text}{fg=HKS07K100}
\usecolortheme[named=HKS41K100]{structure}

\setbeamercolor*{palette primary}{use=structure,fg=white,bg=structure.fg}
\setbeamercolor*{palette secondary}{use=structure,fg=white,bg=structure.fg!80}
\setbeamercolor*{palette tertiary}{use=structure,fg=white,bg=structure.fg!60}
\setbeamercolor*{palette quaternary}{fg=white,bg=black}

\setbeamercolor*{sidebar}{use=structure,bg=structure.fg}

\setbeamercolor*{palette sidebar primary}{use=structure,fg=structure.fg!20}
\setbeamercolor*{palette sidebar secondary}{fg=white}
\setbeamercolor*{palette sidebar tertiary}{use=structure,fg=structure.fg!40}
\setbeamercolor*{palette sidebar quaternary}{fg=white}

\setbeamercolor*{titlelike}{parent=palette primary}

\setbeamercolor*{separation line}{}
\setbeamercolor*{fine separation line}{}

\setbeamercolor{block title}{use=structure,fg=white,bg=structure.fg}
\setbeamercolor{block title alerted}{use=alerted text,fg=white,bg=alerted text.fg!75!black}
\setbeamercolor{block title example}{use=example text,fg=white,bg=example text.fg!75!black}

\setbeamercolor{block body}{parent=normal text,use=block title,bg=block title.bg!10!bg}
\setbeamercolor{block body alerted}{parent=normal text,use=block title alerted,bg=block title alerted.bg!10!bg}
\setbeamercolor{block body example}{parent=normal text,use=block title example,bg=block title example.bg!10!bg}

% \setbeamertemplate{itemize items}[default]


%%%%%%%%%%%%%%%%%%%%%%%%%%%%%%%%%%%%%%%%%%%%%%%%%%%%%%%%%%%%%%%%%%%%%%%%%%%%%%
% TikZ
%%%%%%%%%%%%%%%%%%%%%%%%%%%%%%%%%%%%%%%%%%%%%%%%%%%%%%%%%%%%%%%%%%%%%%%%%%%%%%

\tikzset
	{ > = Stealth
	}


%%%%%%%%%%%%%%%%%%%%%%%%%%%%%%%%%%%%%%%%%%%%%%%%%%%%%%%%%%%%%%%%%%%%%%%%%%%%%%
% general commands and styles
%%%%%%%%%%%%%%%%%%%%%%%%%%%%%%%%%%%%%%%%%%%%%%%%%%%%%%%%%%%%%%%%%%%%%%%%%%%%%%

% \delegateStyle and \inheritStyle command
% usage: \delegateStyle{… \inheritStyle{…} …}
% example: \(X_{\delegateStyle{\fbox{\inheritStyle{X}}}}\)
% Save the current style and regain it in the argument.
% This works both for math and text mode, and can be nested.
% Acknowledgments: Based on \ThisStyle and \SavedStyle from scalerel package.
\makeatletter
\newcommand*{\@inheritStyle@D}[1]{\(\displaystyle      #1\)}
\newcommand*{\@inheritStyle@T}[1]{\(\textstyle         #1\)}
\newcommand*{\@inheritStyle@S}[1]{\(\scriptstyle       #1\)}
\newcommand*{\@inheritStyle@s}[1]{\(\scriptscriptstyle #1\)}
\newcommand*{\@inheritStyle@t}[1]{#1}
\newcommand*{\inheritStyle}{\csname @inheritStyle@\@inheritStyleSwitch\endcsname}
\newcommand*{\delegateStyle}[1]{%
	\ifmmode%
		\mathchoice%
		{\edef\@inheritStyleSwitch{D}#1}%
		{\edef\@inheritStyleSwitch{T}#1}%
		{\edef\@inheritStyleSwitch{S}#1}%
		{\edef\@inheritStyleSwitch{s}#1}%
	\else%
		\edef\@inheritStyleSwitch{t}#1%
	\fi%
}
\makeatother


% \oalt command
% requires: \delegateStyle and \inheritStyle command
% usage: \oalt<…>[…]{…}{…} (cf. \alt)
% Behaves like \alt, but reserves space according to largest overlays.
% The optional argument defines the alignment inside the reserved space;
% it is one of c, l, r, s (cf. \makebox); the default is c.
\makeatletter
\newlength{\oalt@dp}
\newlength{\oalt@ht}
\newlength{\oalt@wd}
\newbox{\oalt@a}
\newbox{\oalt@b}
\newbox{\oalt@empty}
\newcommand<>*{\oalt}[3][c]{%
	\delegateStyle{%
		% based on \setto… in /usr/share/texmf-dist/tex/latex/base/latex.ltx
		\setbox\oalt@a\hbox{\inheritStyle{#2}}%
		\setbox\oalt@b\hbox{\inheritStyle{#3}}%
		\pgfmathsetlength{\oalt@dp}{max(\dp\oalt@a,\dp\oalt@b)}%
		\pgfmathsetlength{\oalt@ht}{max(\ht\oalt@a,\ht\oalt@b)}%
		\pgfmathsetlength{\oalt@wd}{max(\wd\oalt@a,\wd\oalt@b)}%
		\raisebox{0pt}[\oalt@ht][\oalt@dp]{%
			\makebox[\oalt@wd][#1]{%
				\alt#4{\unhbox\oalt@a}{\unhbox\oalt@b}%
			}%
		}%
		\setbox\oalt@a\box\oalt@empty%
		\setbox\oalt@b\box\oalt@empty%
	}%
}
\makeatother


% \otemporal command
% requires: \delegateStyle and \inheritStyle command
% usage: \otemporal<…>[…]{…}{…}{…} (cf. \temporal)
% Behaves like \temporal, but reserves space according to largest overlays.
% The optional argument defines the alignment inside the reserved space;
% it is one of c, l, r, s (cf. \makebox); the default is c.
\makeatletter
\newlength{\ot@dp}
\newlength{\ot@ht}
\newlength{\ot@wd}
\newbox{\ot@a}
\newbox{\ot@b}
\newbox{\ot@c}
\newbox{\ot@empty}
\newcommand<>*{\otemporal}[4][c]{%
	\delegateStyle{%
		% based on \setto… in /usr/share/texmf-dist/tex/latex/base/latex.ltx
		\setbox\ot@a\hbox{\inheritStyle{#2}}%
		\setbox\ot@b\hbox{\inheritStyle{#3}}%
		\setbox\ot@c\hbox{\inheritStyle{#4}}%
		\pgfmathsetlength{\ot@dp}{max(\dp\ot@a,\dp\ot@b,\dp\ot@c)}%
		\pgfmathsetlength{\ot@ht}{max(\ht\ot@a,\ht\ot@b,\ht\ot@c)}%
		\pgfmathsetlength{\ot@wd}{max(\wd\ot@a,\wd\ot@b,\wd\ot@c)}%
		\raisebox{0pt}[\ot@ht][\ot@dp]{%
			\makebox[\ot@wd][#1]{%
				\temporal#5{\unhbox\ot@a}{\unhbox\ot@b}{\unhbox\ot@c}%
			}%
		}%
		\setbox\ot@a\box\ot@empty%
		\setbox\ot@b\box\ot@empty%
		\setbox\ot@c\box\ot@empty%
	}%
}
\makeatother


% Resize delimiters like braces, brackets, etc.
% Parameters: size, left delimiter, formula, right delimiter
% Example: \delim2({\frac{1}{2}})
\newcommand*{\delim}[4]{%
	\ifcase#1%
		#2#3#4%
	\or%
		\bigl#2#3\bigr#4%
	\or%
		\Bigl#2#3\Bigr#4%
	\or%
		\biggl#2#3\biggr#4%
	\or%
		\Biggl#2#3\Biggr#4%
	\else%
		\left#2#3\right#4%
	\fi%
}


% similar to \fullcite, but using the formatting of \printbibliography
\newcommand*{\printfullcite}[1]{%
	\begin{refsection}%
		\nocite{#1}%
		\DeclareNameAlias{author}{first-last}%
		\printbibliography[heading = none]%
	\end{refsection}%
}


\colorlet{light alert}{HKS07K60}
\tikzset{alert.bg/.style={rounded corners, fill=light alert}}
\tikzset{every picture/.style={line cap=round, semithick}}
% http://tex.stackexchange.com/questions/6135/how-to-make-beamer-overlays-with-tikz-node
\tikzset{onslide/.code args={<#1>#2}{\only<#1>{\pgfkeysalso{#2}}}}
\tikzset{invisible/.code args={<#1>}{\alt<#1>{\pgfkeysalso{transparent}}{\pgfkeysalso{opaque}}}}
\tikzset{uncover/.code args={<#1>}{\alt<#1>{\pgfkeysalso{opaque}}{\pgfkeysalso{opacity=0.25}}}}
\tikzset{visible/.code args={<#1>}{\alt<#1>{\pgfkeysalso{opaque}}{\pgfkeysalso{transparent}}}}
\tikzset{vuncover/.code args=%
	{<#1><#2>}%
	{\alt<#1>%
		{\alt<#2>%
			{\pgfkeysalso{opaque}}%
			{\pgfkeysalso{opacity=0.25}}%
		}{\pgfkeysalso{transparent}}%
	}%
}

\newcommand<%
	>{\tikzhighlight}[2][]{%
	\delegateStyle{\alt#3%
		{\tikz[baseline=0, anchor=base, inner sep=0.2em, text height=, text depth=]{\node[alert.bg, #1]{\inheritStyle{#2}};}}%
		{\tikz[baseline=0, anchor=base, inner sep=0.2em, text height=, text depth=]{\node[#1, fill=none]{\inheritStyle{#2}};}}%
	}%
}

\newcommand{\mathhighlight}{\tikzhighlight}

\newcommand<>{\mhl}[2][]{\mathhighlight#3[inner sep=0.2em, #1]{#2}}


\newcommand<>{\inlineblock}[2][]{{%
	\usebeamercolor*[fg]{block body}%
	\tikzhighlight#3[fill=block body.bg, #1]{#2}%
}}


% a small letter s for plurals of abbreviations
\newcommand*{\s}{{\scriptsize s}\xspace}


\newcommand<>*{\sout}[2][opacity=0.75, ultra thick]{%
	\delegateStyle{%
		\tikz[baseline=0, anchor=base, inner sep=0, outer sep=0]{
			\useasboundingbox node (n) {\inheritStyle{#2}};
			\only#3{
				\node (h) {\inheritStyle{\ifmmode\mathstrut\else\strut\fi}};
				\draw[#1] (n.west |- {$(h.south)!0.5!(h.north)$}) -- (n.east |- {$(h.south)!0.5!(h.north)$});
			}
		}%
	}%
}


% tight style
% Sets outer sep to default inner sep and inner sep to 0.
% Use this style for nodes that are neither drawn nor filled to prevent
% unwanted growth of the bounding box.
\tikzset{tight/.style={inner sep=0, outer sep=0.3333em}}


% rounded tree edges style
% usage: rounded tree edges={⟨direction⟩}{⟨looseness⟩}{⟨strength⟩}
\tikzset{
	rounded tree edges/.style n args={3}{
	edge from parent path={
	let
		\n{direction}={#1},
		\n{looseness}={#2},
		\n{strength}={#3},
		\p1=(\tikzparentnode),
		\p2=(\tikzchildnode),
		\p3=(\n{direction}:1pt),
		\p4=(\x2 - \x1, \y2 - \y1),
		\n{dist}={veclen(\p4)},
		\p4=(\x4 / \n{dist}, \y4 / \n{dist}),
		\n{angle}={atan2(\y4, \x4)},
		\n{delta}={Mod(\n{angle} - \n{direction}, 360)},
		\n{delta}={\n{delta} > 180 ? \n{delta} - 360  : \n{delta}},
		\n{delta}={\n{delta} >  90 ?  180 - \n{delta} : \n{delta}},
		\n{delta}={\n{delta} < -90 ? -180 - \n{delta} : \n{delta}}
	in (\tikzparentnode) .. controls
		+(    \n{angle}+\n{strength}*\n{delta}:\n{looseness}*0.3915*\n{dist}) and
		+(180+\n{angle}-\n{strength}*\n{delta}:\n{looseness}*0.3915*\n{dist}) ..
		(\tikzchildnode)
	}
	}
}


% Tear out snippets from PDFs.
% Usage: \tear[…]{file.pdf}
% The optional parameter is the same as for \includegraphics.
% Useful Arguments:
%   * page=‹pagenumber›
%   * trim=‹left› ‹bottom› ‹right› ‹top›
%   * width=0.98\linewidth
\newcommand*{\tear}[2][]{%
	\begin{tikzpicture}
		\node
			[ blur shadow
			, clip
			, decorate
			, decoration=random steps
			, draw
			, inner sep=0
			, preaction={fill=white}% hide the shadow if paper is transparent
			] {\includegraphics[#1]{#2}};
	\end{tikzpicture}%
}


\makeatletter
\newcommand*{\timeline}[3][0]{%
	\ifcsname timeline@cmd@#3\endcsname%
		\@timeline[#1]{#2}{#3}%
		\PackageWarning{timeline}{redefining timeline \@backslashchar\string#3}%
	\else%
		\ifcsname#3\endcsname%
			\errmessage{Command \@backslashchar\string#3 already defined}%
		\else%
			\@timeline[#1]{#2}{#3}%
		\fi%
	\fi%
}%
\newcommand*{\@timeline}[3][0]{%
	% mark command as timeline command – they can be overwritten
	\expandafter\def\csname timeline@cmd@#3\endcsname{}%
	\setcounter{@timeline}{#1}%
	\def\timeline@cmd{#3}%
	\timeline@reset%
	\timeline@append{0}%
	\@tfor\timeline@next:=#2\do{%
		\if\timeline@next+%
			\stepcounter{@timeline}%
			\timeline@append{,\the@timeline}%
		\else\if\timeline@next-%
			\stepcounter{@timeline}%
		\else%
			%\timeline@append{\timeline@next}%
			\GenericError{}{\protect\timeline: ignoring unknown character: \timeline@next}%
		\fi\fi%
	}%
}%
% \newcommand*{\tl}[1]{%
% 	\ifcsname timeline@cmd@#1\endcsname%
% 		\csname timeline@cmd@#1\endcsname%
% 	\else%
% 		0%
% 		%\GenericError{}{\protect\tl: timeline not defined: #1}%
% 	\fi%
% }%
\newcounter{@timeline}%
\def\timeline@reset{%
	\expandafter\def\csname\timeline@cmd\endcsname{}%
}%
\def\timeline@append#1{%
	\expandafter\edef\csname\timeline@cmd\endcsname{%
		\csname\timeline@cmd\endcsname#1%
	}%
}%
\makeatother


\newcommand*{\xminus}[1]{%
	\mathrel{\tikz[baseline={([yshift=-0.25em]n.south)}, inner sep=0, outer sep=0.2em]{%
		\node (n) {\(\scriptstyle #1\)};
		\draw (n.south west) -- (n.south east);
	}}%
}
\newcommand*{\tikzrightarrow}[1]{%
	\mathrel{\tikz[baseline={([yshift=-0.25em]n.south)}, inner sep=0, outer sep=0.2em]{%
		\node (n) {\(\scriptstyle #1\)};
		\draw[->, > = Computer Modern Rightarrow, line width = 0.4pt] (n.south west) -- (n.south east);
	}}%
}


%%%%%%%%%%%%%%%%%%%%%%%%%%%%%%%%%%%%%%%%%%%%%%%%%%%%%%%%%%%%%%%%%%%%%%%%%%%%%%
% document specific commands
%%%%%%%%%%%%%%%%%%%%%%%%%%%%%%%%%%%%%%%%%%%%%%%%%%%%%%%%%%%%%%%%%%%%%%%%%%%%%%

\newcommand<>*{\mycite}[1]{\uncover#2{{\color{HKS57K100}[\cite{#1}]}}}


\newcommand{\statetree}[1]{
	\tikz
	[ anchor=base
	, baseline=(current bounding box.center)
	, level distance=2em
	, sibling distance=2em
	]{
		\matrix
		[ draw=nt
		, edge from parent/.style={draw=black}
		, inner sep=0
		, nodes={inner sep=0.2em, rounded corners=0}
		, rounded corners
		] {#1\\}
	}
}


\newcommand*{\mylargeleaf}[1]{{\LARGE\color{HKS41K70}#1}}

\definecolor{state s}{named}{HKS57K80}
\definecolor{state t}{named}{HKS41K70}
\newcommand*{\stateS}[1]{{\color{state s}#1}}
\newcommand*{\stateT}[1]{{\color{state t}#1}}

\tikzset{
	subtree/.style =
		{ fill=lightgray
		, inner sep=0.02em
		, isosceles triangle apex angle=60
		, shape=isosceles triangle
		, shape border rotate=90
		}
	, state/.style = {circle, draw, inner sep=0.1em}
	, trans/.style = {rectangle, draw}
}

\newcommand*{\srBool}{\mathbb{B}}
\newcommand*{\srProb}{ℙ}


%%%%%%%%%%%%%%%%%%%%%%%%%%%%%%%%%%%%%%%%%%%%%%%%%%%%%%%%%%%%%%%%%%%%%%%%%%%%%%
% commands for specific notations
%%%%%%%%%%%%%%%%%%%%%%%%%%%%%%%%%%%%%%%%%%%%%%%%%%%%%%%%%%%%%%%%%%%%%%%%%%%%%%

\DeclareMathOperator*{\argmax}{argmax}

\newcommand*{\cardinality}[1]{\lvert#1\rvert}
\newcommand*{\corpussize}[1]{\lvert#1\rvert}

\DeclareMathOperator{\crispOp}{crisp}
\newcommand*        {\crisp}[2][0]{\crispOp\delim{#1}({#2})}

\DeclareMathOperator{\lhsOp}{lhs}
\newcommand*{\lhs}[1]{\lhsOp(#1)}

\DeclareMathOperator{\lklhdOp}{L}
\newcommand*{\lklhd}[2]{\lklhdOp(#1 ∣ #2)}

\DeclareMathOperator{\mleOp}{mle}
\newcommand*{\mle}[2][]{%
	\ifthenelse{\isempty{#1}}{%
		\mleOp(#2)%
	}{%
		\mleOp_{#1}(#2)%
	}%
}

\DeclareMathOperator{\mrg}{merge}

% CVD: color vision deficiencies
\definecolor{CVD light red}   {HTML}{FF8080}
\definecolor{CVD light yellow}{HTML}{FFFF80}
\definecolor{CVD light green} {HTML}{40FFC0}

\definecolor{nt}{named}{HKS41K70}
\newcommand*{\nt}[1]{{\color{nt}#1}}

% set of all probability distributions over #1
\DeclareMathOperator{\pdsOp}{Pd}
\newcommand*{\pds}[1]{\pdsOp(#1)}

\DeclareMathOperator{\positionsOp}{pos}
\newcommand*{\positions}[1]{\positionsOp(#1)}

\DeclareMathOperator{\rankOp}{rk}
\newcommand*{\rank}[1]{\rankOp(#1)}

\DeclareMathOperator{\runsOp}{run}
\newcommand*{\runs}[2][]{%
	\ifthenelse%
		{\isempty{#1}}%
		{\runsOp(#2)}%
		{\runsOp_{#1}(#2)}%
}

\newcommand*{\semantics}[1]{⟦#1⟧}

\DeclareMathOperator{\splt}{split}

\newcommand*{\subtree}[2]{#1|_{#2}}

\DeclareMathOperator{\supportOp}{supp}
\newcommand*{\support}[1]{\supportOp(#1)}

\newcommand*{\symId}{\textsc{\color{gray}Id}}
\newcommand*{\symCons}{\textsc{\color{gray}Cons}}
\newcommand*{\symFlip}{\textsc{\color{gray}Flip}}
\newcommand*{\symNull}{\textsc{\color{gray}Null}}
\newcommand*{\symNullR}{\textsc{\color{gray}N\(\overline{\textsc{ull}}\)}}
\newcommand*{\symSnoc}{\textsc{\color{gray}Snoc}}

\newcommand*{\transWTA}[4][]{#3 \xrightarrow{#1} #2(#4)}

\DeclareMathOperator{\uniqueRunOp}{r}
\newcommand*{\uniqueRun}[2][]{%
	\ifthenelse%
		{\isempty{#1}}%
		{\uniqueRunOp^{#2}}%
		{\uniqueRunOp_{\!#1}^{#2}}%
}

\DeclareMathOperator{\treesOp}{T}
\newcommand*{\trees}[2][]{%
	\ifthenelse%
		{\isempty{#1}}%
		{\treesOp_{\!#2}}%
		{\treesOp_{\!#2}(#1)}%
}
\DeclareMathOperator{\treesUOp}{U}
\newcommand*{\treesU}[2][]{%
	\ifthenelse%
		{\isempty{#1}}%
		{\treesUOp_{#2}}%
		{\treesUOp_{#2}(#1)}%
}


%%%%%%%%%%%%%%%%%%%%%%%%%%%%%%%%%%%%%%%%%%%%%%%%%%%%%%%%%%%%%%%%%%%%%%%%%%%%%%
% metadata
%%%%%%%%%%%%%%%%%%%%%%%%%%%%%%%%%%%%%%%%%%%%%%%%%%%%%%%%%%%%%%%%%%%%%%%%%%%%%%

\ifstandalonebeamer\else
	\title[Defense of Dissertation]{A Formal View on Training of Weighted Tree Automata by Likelihood-Driven State Splitting and Merging}
	\subtitle{Defense of Dissertation}
\fi
\author{Toni Dietze}
\institute[TU Dresden]{%
	\href{https://www.orchid.inf.tu-dresden.de/index.en/}{Chair for Foundations of Programming}
\\	\href{https://tu-dresden.de/ing/informatik/thi}{Institute of Theoretical Computer Science}
\\	\href{https://tu-dresden.de/ing/informatik}{Faculty of Computer Science}
\\	\href{https://tu-dresden.de/}{Technische Universität Dresden}
\\	01062 Dresden, Germany
}
\date[2018-09-27]{September 27, 2018}

\begin{document}
\begin{standaloneframe}{\jobname}
	\centering
	\begin{overprint}
	\onslide<2-3>\centering
		\begin{tikzpicture}
			\begin{scope}[anchor=base, level distance=1em, tree layout]
			\node {S}
				child { node {NP}
					child { node {PRP} child { node (I) {I} }
					}
				}
				child { node {VP}
					child { node {VBP} child { node {see} }
					}
					child { node {NP}
						child { node {DT} child { node {the} }
						}
						child { node {NN} child { node {man} }
						}
					}
					child { node {PP}
						child { node {IN} child { node {with} }
						}
						child { node {NP}
							child { node {DT} child { node {the} }
							}
							child { node {NN} child { node {telescope} }
							}
						}
					}
				};
			\end{scope}
			\path (I) -- ++(21.5em, 0);
		\end{tikzpicture}
	\onslide<4>\centering
		\begin{tikzpicture}
			\begin{scope}[anchor=base, level distance=1em, tree layout]
			\node {S}
				child { node {NP}
					child { node {PRP} child { node (I) {I} }
					}
				}
				child { node {VP}
					child { node {VBP} child { node {see} }
					}
					child[sibling pre sep=4.5em] { node {NP}
						child[sibling pre sep=0] { node {NP}
							child { node {DT} child { node {the} }
							}
							child { node {NN} child { node {man} }
							}
						}
						child[sibling pre sep=0] { node {PP}
							child { node {IN} child { node {with} }
							}
							child { node {NP}
								child { node {DT} child { node {the} }
								}
								child { node {NN} child { node {telescope} }
								}
							}
						}
					}
				};
			\end{scope}
			\path (I) -- ++(21.5em, 0);
		\end{tikzpicture}
	\end{overprint}

	\setcounter{beamerpauses}{3}
	\uncover<.(1)->{%
		I \(→\)
		\raisebox{-0.4\height}{%
			\alt<.(1)>{%
				\includegraphics[width=5em]{figure-telescope-reflected.png}%
				\qquad%
				\includegraphics[width=5em]{figure-telescope-without.png}%
			}{%
				\includegraphics[width=5em]{figure-telescope-without-reflected.png}%
				\qquad%
				\includegraphics[width=5em]{figure-telescope.png}%
			}%
		}
		\(←\) man

		{\hfill\tiny\color{gray}\url{http://www.clipartpanda.com/clipart_images/cartoon-boy-with-telescope-11578232}}
	}%
\end{standaloneframe}
\end{document}

\end{frame}


\section{State Splitting and Merging Algorithm – Details}

\againframe<7>{frame:algorithm-split-merge}

\begin{frame}{\secname}
	\centering
	% SPDX-License-Identifier: CC-BY-4.0
% Copyright 2018 Toni Dietze
\documentclass[beamer]{standalone}
% SPDX-License-Identifier: CC-BY-4.0 OR MIT-0
% Copyright 2018 Toni Dietze
%
\usefonttheme{professionalfonts}

% LuaLaTeX specific packages
\usepackage{fontspec}
	\defaultfontfeatures{Ligatures=TeX}
\usepackage{polyglossia}
	\setdefaultlanguage{english}
\usepackage{amsmath}  % has to be loaded before unicode-math
\usepackage[math-style=ISO]{unicode-math}
	\setmathfont{Latin Modern Math}
% 	\setmathfont[range={\mathcal,\mathbfcal},StylisticSet=1]{xits-math.otf}
% 	\setmathfont[range={"029F5}]{XITS Math}  % ⧵
% 	\setmathfont[range={\mathscr,\mathbfscr},StylisticSet=1]{Latin Modern Math}  % make \mathscr use the correct font

\usepackage[noend]{algpseudocode}
	\algrenewcommand\algorithmicrequire{\textbf{Input:}}
	\algrenewcommand\algorithmicensure{\textbf{Output:}}
\usepackage[backend=biber, maxbibnames=42, maxcitenames=42, sorting=ynt, style=authoryear]{biblatex}
\usepackage{csquotes}
\usepackage{mathtools}
\usepackage{media9}
\usepackage{scalerel}
\usepackage{standalone}
\usepackage{tikz}
	\usetikzlibrary{arrows.meta}
	\usetikzlibrary{backgrounds}
	\usetikzlibrary{calc}
	\usetikzlibrary{decorations}
	\usetikzlibrary{decorations.pathmorphing}
	\usetikzlibrary{decorations.pathreplacing}
	\usetikzlibrary{fadings}
	\usetikzlibrary{fit}
	\usetikzlibrary{graphs}
	\usetikzlibrary{graphdrawing}
	\usetikzlibrary{intersections}
	\usetikzlibrary{positioning}
	\usetikzlibrary{quotes}
	\usetikzlibrary{shadows.blur}
	\usetikzlibrary{shapes.arrows}
	\usetikzlibrary{shapes.geometric}
	\usegdlibrary{trees}
\usepackage{xifthen}
\usepackage{xspace}

\usepackage{pgfplots}
	\pgfplotsset
		{ compat = 1.15
		, /pgf/number format/1000 sep = {\,}
		, /pgf/number format/assume math mode = true
		, every axis plot/.append style =
			{ mark options = {fill opacity = 0.25}
			}
		}
	\usepgfplotslibrary{groupplots}
\usepackage{pgfplotstable}

\hypersetup
	{ bookmarksopen
	, pdflang = en
	, unicode
	}


%%%%%%%%%%%%%%%%%%%%%%%%%%%%%%%%%%%%%%%%%%%%%%%%%%%%%%%%%%%%%%%%%%%%%%%%%%%%%%


% always show bad boxes
%\overfullrule=1em


%%%%%%%%%%%%%%%%%%%%%%%%%%%%%%%%%%%%%%%%%%%%%%%%%%%%%%%%%%%%%%%%%%%%%%%%%%%%%%
% biblatex
%%%%%%%%%%%%%%%%%%%%%%%%%%%%%%%%%%%%%%%%%%%%%%%%%%%%%%%%%%%%%%%%%%%%%%%%%%%%%%

\addbibresource{slides-dissertation-defense.bib}
% \renewcommand*{\finalnamedelim}{\addcomma\space}
% \setlength{\bibitemsep}{1em}
% 
\AtEveryBibitem{% Clean up the bibtex rather than editing it
 \clearlist{address}
 \clearfield{date}
 \clearfield{eprint}
 \clearfield{isbn}
 \clearfield{issn}
 \clearlist{language}
 \clearlist{location}
 \clearfield{month}
 \clearfield{series}
%  \clearfield{url}
%  \clearfield{doi}
 \clearfield{organization}

%  \ifentrytype{book}{}{% Remove stuff except for books
%   \clearfield{booktitle}
%   \clearfield{pages}
  \clearlist{publisher}
  \clearname{editor}
%  }
}
% do not print url if doi is present
% http://tex.stackexchange.com/questions/154864/biblatex-use-doi-only-if-there-is-no-url
\DeclareSourcemap{
	\maps[datatype=bibtex]{
		\map{
			\step[fieldsource=doi,final]
			\step[fieldset=url,null]
}	}	}
%
% remove qoutes around titles
\DeclareFieldFormat
	[article,inbook,incollection,inproceedings,patent,thesis,unpublished]
	{title}{#1\isdot}
% 
% \DeclareFieldFormat{url}{\mkbibacro{URL}\addcolon\addnbspace\url{#1}}
% 
% \DeclareNameAlias{sortname}{first-last}
% 
\renewbibmacro{in:}{\ifentrytype{article}{}{}}


%%%%%%%%%%%%%%%%%%%%%%%%%%%%%%%%%%%%%%%%%%%%%%%%%%%%%%%%%%%%%%%%%%%%%%%%%%%%%%
% beamer
%%%%%%%%%%%%%%%%%%%%%%%%%%%%%%%%%%%%%%%%%%%%%%%%%%%%%%%%%%%%%%%%%%%%%%%%%%%%%%

\useoutertheme{infolines}
\makeatletter
% based on
% /usr/share/texmf-dist/tex/latex/beamer/beamerouterthemeinfolines.sty
\setbeamertemplate{footline}
{%
	\leavevmode%
	\hbox{%
	\begin{beamercolorbox}[wd=.333333\paperwidth,ht=2.25ex,dp=1ex,center]{author in head/foot}%
		\usebeamerfont{author in head/foot}\insertshortauthor\expandafter\beamer@ifempty\expandafter{\beamer@shortinstitute}{}{~~(\insertshortinstitute)}
	\end{beamercolorbox}%
	\begin{beamercolorbox}[wd=.333333\paperwidth,ht=2.25ex,dp=1ex,center]{title in head/foot}%
		\usebeamerfont{title in head/foot}\insertshorttitle
	\end{beamercolorbox}%
	\begin{beamercolorbox}[wd=.333333\paperwidth,ht=2.25ex,dp=1ex,right]{date in head/foot}%
		\usebeamerfont{date in head/foot}%
		\hfill\insertshortdate\hfill\hfill%
		%\hspace*{2ex}%
		%\insertshortdate%
		%\hspace{0pt plus 1 filll}%
		%(\insertframenumber.\insertoverlaynumber{} / \insertmainframenumber)%
		%\hspace{0pt plus 1 filll}%
		\phantom{000}\llap{\insertpagenumber} / \insertpresentationendpage%
		\hspace*{2ex}%
	\end{beamercolorbox}}%
	\vskip0pt%
}
\makeatother
\useinnertheme{circles}
\beamertemplatenavigationsymbolsempty
\setbeamertemplate{bibliography item}{}
\setbeamertemplate{headline}[default]

\input{tudcolors.tex}
\setbeamercolor*{alerted text}{fg=HKS07K100}
\usecolortheme[named=HKS41K100]{structure}

\setbeamercolor*{palette primary}{use=structure,fg=white,bg=structure.fg}
\setbeamercolor*{palette secondary}{use=structure,fg=white,bg=structure.fg!80}
\setbeamercolor*{palette tertiary}{use=structure,fg=white,bg=structure.fg!60}
\setbeamercolor*{palette quaternary}{fg=white,bg=black}

\setbeamercolor*{sidebar}{use=structure,bg=structure.fg}

\setbeamercolor*{palette sidebar primary}{use=structure,fg=structure.fg!20}
\setbeamercolor*{palette sidebar secondary}{fg=white}
\setbeamercolor*{palette sidebar tertiary}{use=structure,fg=structure.fg!40}
\setbeamercolor*{palette sidebar quaternary}{fg=white}

\setbeamercolor*{titlelike}{parent=palette primary}

\setbeamercolor*{separation line}{}
\setbeamercolor*{fine separation line}{}

\setbeamercolor{block title}{use=structure,fg=white,bg=structure.fg}
\setbeamercolor{block title alerted}{use=alerted text,fg=white,bg=alerted text.fg!75!black}
\setbeamercolor{block title example}{use=example text,fg=white,bg=example text.fg!75!black}

\setbeamercolor{block body}{parent=normal text,use=block title,bg=block title.bg!10!bg}
\setbeamercolor{block body alerted}{parent=normal text,use=block title alerted,bg=block title alerted.bg!10!bg}
\setbeamercolor{block body example}{parent=normal text,use=block title example,bg=block title example.bg!10!bg}

% \setbeamertemplate{itemize items}[default]


%%%%%%%%%%%%%%%%%%%%%%%%%%%%%%%%%%%%%%%%%%%%%%%%%%%%%%%%%%%%%%%%%%%%%%%%%%%%%%
% TikZ
%%%%%%%%%%%%%%%%%%%%%%%%%%%%%%%%%%%%%%%%%%%%%%%%%%%%%%%%%%%%%%%%%%%%%%%%%%%%%%

\tikzset
	{ > = Stealth
	}


%%%%%%%%%%%%%%%%%%%%%%%%%%%%%%%%%%%%%%%%%%%%%%%%%%%%%%%%%%%%%%%%%%%%%%%%%%%%%%
% general commands and styles
%%%%%%%%%%%%%%%%%%%%%%%%%%%%%%%%%%%%%%%%%%%%%%%%%%%%%%%%%%%%%%%%%%%%%%%%%%%%%%

% \delegateStyle and \inheritStyle command
% usage: \delegateStyle{… \inheritStyle{…} …}
% example: \(X_{\delegateStyle{\fbox{\inheritStyle{X}}}}\)
% Save the current style and regain it in the argument.
% This works both for math and text mode, and can be nested.
% Acknowledgments: Based on \ThisStyle and \SavedStyle from scalerel package.
\makeatletter
\newcommand*{\@inheritStyle@D}[1]{\(\displaystyle      #1\)}
\newcommand*{\@inheritStyle@T}[1]{\(\textstyle         #1\)}
\newcommand*{\@inheritStyle@S}[1]{\(\scriptstyle       #1\)}
\newcommand*{\@inheritStyle@s}[1]{\(\scriptscriptstyle #1\)}
\newcommand*{\@inheritStyle@t}[1]{#1}
\newcommand*{\inheritStyle}{\csname @inheritStyle@\@inheritStyleSwitch\endcsname}
\newcommand*{\delegateStyle}[1]{%
	\ifmmode%
		\mathchoice%
		{\edef\@inheritStyleSwitch{D}#1}%
		{\edef\@inheritStyleSwitch{T}#1}%
		{\edef\@inheritStyleSwitch{S}#1}%
		{\edef\@inheritStyleSwitch{s}#1}%
	\else%
		\edef\@inheritStyleSwitch{t}#1%
	\fi%
}
\makeatother


% \oalt command
% requires: \delegateStyle and \inheritStyle command
% usage: \oalt<…>[…]{…}{…} (cf. \alt)
% Behaves like \alt, but reserves space according to largest overlays.
% The optional argument defines the alignment inside the reserved space;
% it is one of c, l, r, s (cf. \makebox); the default is c.
\makeatletter
\newlength{\oalt@dp}
\newlength{\oalt@ht}
\newlength{\oalt@wd}
\newbox{\oalt@a}
\newbox{\oalt@b}
\newbox{\oalt@empty}
\newcommand<>*{\oalt}[3][c]{%
	\delegateStyle{%
		% based on \setto… in /usr/share/texmf-dist/tex/latex/base/latex.ltx
		\setbox\oalt@a\hbox{\inheritStyle{#2}}%
		\setbox\oalt@b\hbox{\inheritStyle{#3}}%
		\pgfmathsetlength{\oalt@dp}{max(\dp\oalt@a,\dp\oalt@b)}%
		\pgfmathsetlength{\oalt@ht}{max(\ht\oalt@a,\ht\oalt@b)}%
		\pgfmathsetlength{\oalt@wd}{max(\wd\oalt@a,\wd\oalt@b)}%
		\raisebox{0pt}[\oalt@ht][\oalt@dp]{%
			\makebox[\oalt@wd][#1]{%
				\alt#4{\unhbox\oalt@a}{\unhbox\oalt@b}%
			}%
		}%
		\setbox\oalt@a\box\oalt@empty%
		\setbox\oalt@b\box\oalt@empty%
	}%
}
\makeatother


% \otemporal command
% requires: \delegateStyle and \inheritStyle command
% usage: \otemporal<…>[…]{…}{…}{…} (cf. \temporal)
% Behaves like \temporal, but reserves space according to largest overlays.
% The optional argument defines the alignment inside the reserved space;
% it is one of c, l, r, s (cf. \makebox); the default is c.
\makeatletter
\newlength{\ot@dp}
\newlength{\ot@ht}
\newlength{\ot@wd}
\newbox{\ot@a}
\newbox{\ot@b}
\newbox{\ot@c}
\newbox{\ot@empty}
\newcommand<>*{\otemporal}[4][c]{%
	\delegateStyle{%
		% based on \setto… in /usr/share/texmf-dist/tex/latex/base/latex.ltx
		\setbox\ot@a\hbox{\inheritStyle{#2}}%
		\setbox\ot@b\hbox{\inheritStyle{#3}}%
		\setbox\ot@c\hbox{\inheritStyle{#4}}%
		\pgfmathsetlength{\ot@dp}{max(\dp\ot@a,\dp\ot@b,\dp\ot@c)}%
		\pgfmathsetlength{\ot@ht}{max(\ht\ot@a,\ht\ot@b,\ht\ot@c)}%
		\pgfmathsetlength{\ot@wd}{max(\wd\ot@a,\wd\ot@b,\wd\ot@c)}%
		\raisebox{0pt}[\ot@ht][\ot@dp]{%
			\makebox[\ot@wd][#1]{%
				\temporal#5{\unhbox\ot@a}{\unhbox\ot@b}{\unhbox\ot@c}%
			}%
		}%
		\setbox\ot@a\box\ot@empty%
		\setbox\ot@b\box\ot@empty%
		\setbox\ot@c\box\ot@empty%
	}%
}
\makeatother


% Resize delimiters like braces, brackets, etc.
% Parameters: size, left delimiter, formula, right delimiter
% Example: \delim2({\frac{1}{2}})
\newcommand*{\delim}[4]{%
	\ifcase#1%
		#2#3#4%
	\or%
		\bigl#2#3\bigr#4%
	\or%
		\Bigl#2#3\Bigr#4%
	\or%
		\biggl#2#3\biggr#4%
	\or%
		\Biggl#2#3\Biggr#4%
	\else%
		\left#2#3\right#4%
	\fi%
}


% similar to \fullcite, but using the formatting of \printbibliography
\newcommand*{\printfullcite}[1]{%
	\begin{refsection}%
		\nocite{#1}%
		\DeclareNameAlias{author}{first-last}%
		\printbibliography[heading = none]%
	\end{refsection}%
}


\colorlet{light alert}{HKS07K60}
\tikzset{alert.bg/.style={rounded corners, fill=light alert}}
\tikzset{every picture/.style={line cap=round, semithick}}
% http://tex.stackexchange.com/questions/6135/how-to-make-beamer-overlays-with-tikz-node
\tikzset{onslide/.code args={<#1>#2}{\only<#1>{\pgfkeysalso{#2}}}}
\tikzset{invisible/.code args={<#1>}{\alt<#1>{\pgfkeysalso{transparent}}{\pgfkeysalso{opaque}}}}
\tikzset{uncover/.code args={<#1>}{\alt<#1>{\pgfkeysalso{opaque}}{\pgfkeysalso{opacity=0.25}}}}
\tikzset{visible/.code args={<#1>}{\alt<#1>{\pgfkeysalso{opaque}}{\pgfkeysalso{transparent}}}}
\tikzset{vuncover/.code args=%
	{<#1><#2>}%
	{\alt<#1>%
		{\alt<#2>%
			{\pgfkeysalso{opaque}}%
			{\pgfkeysalso{opacity=0.25}}%
		}{\pgfkeysalso{transparent}}%
	}%
}

\newcommand<%
	>{\tikzhighlight}[2][]{%
	\delegateStyle{\alt#3%
		{\tikz[baseline=0, anchor=base, inner sep=0.2em, text height=, text depth=]{\node[alert.bg, #1]{\inheritStyle{#2}};}}%
		{\tikz[baseline=0, anchor=base, inner sep=0.2em, text height=, text depth=]{\node[#1, fill=none]{\inheritStyle{#2}};}}%
	}%
}

\newcommand{\mathhighlight}{\tikzhighlight}

\newcommand<>{\mhl}[2][]{\mathhighlight#3[inner sep=0.2em, #1]{#2}}


\newcommand<>{\inlineblock}[2][]{{%
	\usebeamercolor*[fg]{block body}%
	\tikzhighlight#3[fill=block body.bg, #1]{#2}%
}}


% a small letter s for plurals of abbreviations
\newcommand*{\s}{{\scriptsize s}\xspace}


\newcommand<>*{\sout}[2][opacity=0.75, ultra thick]{%
	\delegateStyle{%
		\tikz[baseline=0, anchor=base, inner sep=0, outer sep=0]{
			\useasboundingbox node (n) {\inheritStyle{#2}};
			\only#3{
				\node (h) {\inheritStyle{\ifmmode\mathstrut\else\strut\fi}};
				\draw[#1] (n.west |- {$(h.south)!0.5!(h.north)$}) -- (n.east |- {$(h.south)!0.5!(h.north)$});
			}
		}%
	}%
}


% tight style
% Sets outer sep to default inner sep and inner sep to 0.
% Use this style for nodes that are neither drawn nor filled to prevent
% unwanted growth of the bounding box.
\tikzset{tight/.style={inner sep=0, outer sep=0.3333em}}


% rounded tree edges style
% usage: rounded tree edges={⟨direction⟩}{⟨looseness⟩}{⟨strength⟩}
\tikzset{
	rounded tree edges/.style n args={3}{
	edge from parent path={
	let
		\n{direction}={#1},
		\n{looseness}={#2},
		\n{strength}={#3},
		\p1=(\tikzparentnode),
		\p2=(\tikzchildnode),
		\p3=(\n{direction}:1pt),
		\p4=(\x2 - \x1, \y2 - \y1),
		\n{dist}={veclen(\p4)},
		\p4=(\x4 / \n{dist}, \y4 / \n{dist}),
		\n{angle}={atan2(\y4, \x4)},
		\n{delta}={Mod(\n{angle} - \n{direction}, 360)},
		\n{delta}={\n{delta} > 180 ? \n{delta} - 360  : \n{delta}},
		\n{delta}={\n{delta} >  90 ?  180 - \n{delta} : \n{delta}},
		\n{delta}={\n{delta} < -90 ? -180 - \n{delta} : \n{delta}}
	in (\tikzparentnode) .. controls
		+(    \n{angle}+\n{strength}*\n{delta}:\n{looseness}*0.3915*\n{dist}) and
		+(180+\n{angle}-\n{strength}*\n{delta}:\n{looseness}*0.3915*\n{dist}) ..
		(\tikzchildnode)
	}
	}
}


% Tear out snippets from PDFs.
% Usage: \tear[…]{file.pdf}
% The optional parameter is the same as for \includegraphics.
% Useful Arguments:
%   * page=‹pagenumber›
%   * trim=‹left› ‹bottom› ‹right› ‹top›
%   * width=0.98\linewidth
\newcommand*{\tear}[2][]{%
	\begin{tikzpicture}
		\node
			[ blur shadow
			, clip
			, decorate
			, decoration=random steps
			, draw
			, inner sep=0
			, preaction={fill=white}% hide the shadow if paper is transparent
			] {\includegraphics[#1]{#2}};
	\end{tikzpicture}%
}


\makeatletter
\newcommand*{\timeline}[3][0]{%
	\ifcsname timeline@cmd@#3\endcsname%
		\@timeline[#1]{#2}{#3}%
		\PackageWarning{timeline}{redefining timeline \@backslashchar\string#3}%
	\else%
		\ifcsname#3\endcsname%
			\errmessage{Command \@backslashchar\string#3 already defined}%
		\else%
			\@timeline[#1]{#2}{#3}%
		\fi%
	\fi%
}%
\newcommand*{\@timeline}[3][0]{%
	% mark command as timeline command – they can be overwritten
	\expandafter\def\csname timeline@cmd@#3\endcsname{}%
	\setcounter{@timeline}{#1}%
	\def\timeline@cmd{#3}%
	\timeline@reset%
	\timeline@append{0}%
	\@tfor\timeline@next:=#2\do{%
		\if\timeline@next+%
			\stepcounter{@timeline}%
			\timeline@append{,\the@timeline}%
		\else\if\timeline@next-%
			\stepcounter{@timeline}%
		\else%
			%\timeline@append{\timeline@next}%
			\GenericError{}{\protect\timeline: ignoring unknown character: \timeline@next}%
		\fi\fi%
	}%
}%
% \newcommand*{\tl}[1]{%
% 	\ifcsname timeline@cmd@#1\endcsname%
% 		\csname timeline@cmd@#1\endcsname%
% 	\else%
% 		0%
% 		%\GenericError{}{\protect\tl: timeline not defined: #1}%
% 	\fi%
% }%
\newcounter{@timeline}%
\def\timeline@reset{%
	\expandafter\def\csname\timeline@cmd\endcsname{}%
}%
\def\timeline@append#1{%
	\expandafter\edef\csname\timeline@cmd\endcsname{%
		\csname\timeline@cmd\endcsname#1%
	}%
}%
\makeatother


\newcommand*{\xminus}[1]{%
	\mathrel{\tikz[baseline={([yshift=-0.25em]n.south)}, inner sep=0, outer sep=0.2em]{%
		\node (n) {\(\scriptstyle #1\)};
		\draw (n.south west) -- (n.south east);
	}}%
}
\newcommand*{\tikzrightarrow}[1]{%
	\mathrel{\tikz[baseline={([yshift=-0.25em]n.south)}, inner sep=0, outer sep=0.2em]{%
		\node (n) {\(\scriptstyle #1\)};
		\draw[->, > = Computer Modern Rightarrow, line width = 0.4pt] (n.south west) -- (n.south east);
	}}%
}


%%%%%%%%%%%%%%%%%%%%%%%%%%%%%%%%%%%%%%%%%%%%%%%%%%%%%%%%%%%%%%%%%%%%%%%%%%%%%%
% document specific commands
%%%%%%%%%%%%%%%%%%%%%%%%%%%%%%%%%%%%%%%%%%%%%%%%%%%%%%%%%%%%%%%%%%%%%%%%%%%%%%

\newcommand<>*{\mycite}[1]{\uncover#2{{\color{HKS57K100}[\cite{#1}]}}}


\newcommand{\statetree}[1]{
	\tikz
	[ anchor=base
	, baseline=(current bounding box.center)
	, level distance=2em
	, sibling distance=2em
	]{
		\matrix
		[ draw=nt
		, edge from parent/.style={draw=black}
		, inner sep=0
		, nodes={inner sep=0.2em, rounded corners=0}
		, rounded corners
		] {#1\\}
	}
}


\newcommand*{\mylargeleaf}[1]{{\LARGE\color{HKS41K70}#1}}

\definecolor{state s}{named}{HKS57K80}
\definecolor{state t}{named}{HKS41K70}
\newcommand*{\stateS}[1]{{\color{state s}#1}}
\newcommand*{\stateT}[1]{{\color{state t}#1}}

\tikzset{
	subtree/.style =
		{ fill=lightgray
		, inner sep=0.02em
		, isosceles triangle apex angle=60
		, shape=isosceles triangle
		, shape border rotate=90
		}
	, state/.style = {circle, draw, inner sep=0.1em}
	, trans/.style = {rectangle, draw}
}

\newcommand*{\srBool}{\mathbb{B}}
\newcommand*{\srProb}{ℙ}


%%%%%%%%%%%%%%%%%%%%%%%%%%%%%%%%%%%%%%%%%%%%%%%%%%%%%%%%%%%%%%%%%%%%%%%%%%%%%%
% commands for specific notations
%%%%%%%%%%%%%%%%%%%%%%%%%%%%%%%%%%%%%%%%%%%%%%%%%%%%%%%%%%%%%%%%%%%%%%%%%%%%%%

\DeclareMathOperator*{\argmax}{argmax}

\newcommand*{\cardinality}[1]{\lvert#1\rvert}
\newcommand*{\corpussize}[1]{\lvert#1\rvert}

\DeclareMathOperator{\crispOp}{crisp}
\newcommand*        {\crisp}[2][0]{\crispOp\delim{#1}({#2})}

\DeclareMathOperator{\lhsOp}{lhs}
\newcommand*{\lhs}[1]{\lhsOp(#1)}

\DeclareMathOperator{\lklhdOp}{L}
\newcommand*{\lklhd}[2]{\lklhdOp(#1 ∣ #2)}

\DeclareMathOperator{\mleOp}{mle}
\newcommand*{\mle}[2][]{%
	\ifthenelse{\isempty{#1}}{%
		\mleOp(#2)%
	}{%
		\mleOp_{#1}(#2)%
	}%
}

\DeclareMathOperator{\mrg}{merge}

% CVD: color vision deficiencies
\definecolor{CVD light red}   {HTML}{FF8080}
\definecolor{CVD light yellow}{HTML}{FFFF80}
\definecolor{CVD light green} {HTML}{40FFC0}

\definecolor{nt}{named}{HKS41K70}
\newcommand*{\nt}[1]{{\color{nt}#1}}

% set of all probability distributions over #1
\DeclareMathOperator{\pdsOp}{Pd}
\newcommand*{\pds}[1]{\pdsOp(#1)}

\DeclareMathOperator{\positionsOp}{pos}
\newcommand*{\positions}[1]{\positionsOp(#1)}

\DeclareMathOperator{\rankOp}{rk}
\newcommand*{\rank}[1]{\rankOp(#1)}

\DeclareMathOperator{\runsOp}{run}
\newcommand*{\runs}[2][]{%
	\ifthenelse%
		{\isempty{#1}}%
		{\runsOp(#2)}%
		{\runsOp_{#1}(#2)}%
}

\newcommand*{\semantics}[1]{⟦#1⟧}

\DeclareMathOperator{\splt}{split}

\newcommand*{\subtree}[2]{#1|_{#2}}

\DeclareMathOperator{\supportOp}{supp}
\newcommand*{\support}[1]{\supportOp(#1)}

\newcommand*{\symId}{\textsc{\color{gray}Id}}
\newcommand*{\symCons}{\textsc{\color{gray}Cons}}
\newcommand*{\symFlip}{\textsc{\color{gray}Flip}}
\newcommand*{\symNull}{\textsc{\color{gray}Null}}
\newcommand*{\symNullR}{\textsc{\color{gray}N\(\overline{\textsc{ull}}\)}}
\newcommand*{\symSnoc}{\textsc{\color{gray}Snoc}}

\newcommand*{\transWTA}[4][]{#3 \xrightarrow{#1} #2(#4)}

\DeclareMathOperator{\uniqueRunOp}{r}
\newcommand*{\uniqueRun}[2][]{%
	\ifthenelse%
		{\isempty{#1}}%
		{\uniqueRunOp^{#2}}%
		{\uniqueRunOp_{\!#1}^{#2}}%
}

\DeclareMathOperator{\treesOp}{T}
\newcommand*{\trees}[2][]{%
	\ifthenelse%
		{\isempty{#1}}%
		{\treesOp_{\!#2}}%
		{\treesOp_{\!#2}(#1)}%
}
\DeclareMathOperator{\treesUOp}{U}
\newcommand*{\treesU}[2][]{%
	\ifthenelse%
		{\isempty{#1}}%
		{\treesUOp_{#2}}%
		{\treesUOp_{#2}(#1)}%
}


%%%%%%%%%%%%%%%%%%%%%%%%%%%%%%%%%%%%%%%%%%%%%%%%%%%%%%%%%%%%%%%%%%%%%%%%%%%%%%
% metadata
%%%%%%%%%%%%%%%%%%%%%%%%%%%%%%%%%%%%%%%%%%%%%%%%%%%%%%%%%%%%%%%%%%%%%%%%%%%%%%

\ifstandalonebeamer\else
	\title[Defense of Dissertation]{A Formal View on Training of Weighted Tree Automata by Likelihood-Driven State Splitting and Merging}
	\subtitle{Defense of Dissertation}
\fi
\author{Toni Dietze}
\institute[TU Dresden]{%
	\href{https://www.orchid.inf.tu-dresden.de/index.en/}{Chair for Foundations of Programming}
\\	\href{https://tu-dresden.de/ing/informatik/thi}{Institute of Theoretical Computer Science}
\\	\href{https://tu-dresden.de/ing/informatik}{Faculty of Computer Science}
\\	\href{https://tu-dresden.de/}{Technische Universität Dresden}
\\	01062 Dresden, Germany
}
\date[2018-09-27]{September 27, 2018}

\begin{document}
\begin{standaloneframe}{\jobname}
\begin{algorithmic}[1]
	\Function{split}{$ℳ$}
		\State $\pi \gets$ $ℳ$-splitter splitting every state $A$ in $ℳ$ into $A^1$ and $A^2$
		\State \Return a proper $\pi$-split of $ℳ$
	\EndFunction
	\vspace{1em}
	\Function{merge}{$ℳ'$}
		\State $\pi \gets \text{identity mapping}$
		\ForAll{states $A$ s.t.\ $A^1$, $A^2$ in $ℳ'$}
			\State $\widehat{\pi} \gets \text{identity mapping}$
			\State $\widehat{\pi}(A^1) \gets A$ and $\widehat{\pi}(A^2) \gets A$
			\State $\lambda \gets \text{a good $\widehat{\pi}$-distributor}$
			\If{$\displaystyle\frac{\lklhd{c}{\mrg_{\widehat{\pi}}^\lambda(ℳ')}}{\lklhd{c}{ℳ'}} \geq \mu$}
				\label{algorithmic:lklhd-ratio}
				\State $\pi(A^1) \gets A$ and $\pi(A^2) \gets A$
			\EndIf
		\EndFor
		\State $\lambda \gets \text{a good $\pi$-distributor}$
		\State \Return $\mrg_\pi^\lambda(ℳ')$
	\EndFunction
\end{algorithmic}
\end{standaloneframe}
\end{document}

\end{frame}


\section{Some Numbers}

\begin{frame}{\secname}
	\documentclass[beamer]{standalone}
% SPDX-License-Identifier: CC-BY-4.0 OR MIT-0
% Copyright 2018 Toni Dietze
%
\usefonttheme{professionalfonts}

% LuaLaTeX specific packages
\usepackage{fontspec}
	\defaultfontfeatures{Ligatures=TeX}
\usepackage{polyglossia}
	\setdefaultlanguage{english}
\usepackage{amsmath}  % has to be loaded before unicode-math
\usepackage[math-style=ISO]{unicode-math}
	\setmathfont{Latin Modern Math}
% 	\setmathfont[range={\mathcal,\mathbfcal},StylisticSet=1]{xits-math.otf}
% 	\setmathfont[range={"029F5}]{XITS Math}  % ⧵
% 	\setmathfont[range={\mathscr,\mathbfscr},StylisticSet=1]{Latin Modern Math}  % make \mathscr use the correct font

\usepackage[noend]{algpseudocode}
	\algrenewcommand\algorithmicrequire{\textbf{Input:}}
	\algrenewcommand\algorithmicensure{\textbf{Output:}}
\usepackage[backend=biber, maxbibnames=42, maxcitenames=42, sorting=ynt, style=authoryear]{biblatex}
\usepackage{csquotes}
\usepackage{mathtools}
\usepackage{media9}
\usepackage{scalerel}
\usepackage{standalone}
\usepackage{tikz}
	\usetikzlibrary{arrows.meta}
	\usetikzlibrary{backgrounds}
	\usetikzlibrary{calc}
	\usetikzlibrary{decorations}
	\usetikzlibrary{decorations.pathmorphing}
	\usetikzlibrary{decorations.pathreplacing}
	\usetikzlibrary{fadings}
	\usetikzlibrary{fit}
	\usetikzlibrary{graphs}
	\usetikzlibrary{graphdrawing}
	\usetikzlibrary{intersections}
	\usetikzlibrary{positioning}
	\usetikzlibrary{quotes}
	\usetikzlibrary{shadows.blur}
	\usetikzlibrary{shapes.arrows}
	\usetikzlibrary{shapes.geometric}
	\usegdlibrary{trees}
\usepackage{xifthen}
\usepackage{xspace}

\usepackage{pgfplots}
	\pgfplotsset
		{ compat = 1.15
		, /pgf/number format/1000 sep = {\,}
		, /pgf/number format/assume math mode = true
		, every axis plot/.append style =
			{ mark options = {fill opacity = 0.25}
			}
		}
	\usepgfplotslibrary{groupplots}
\usepackage{pgfplotstable}

\hypersetup
	{ bookmarksopen
	, pdflang = en
	, unicode
	}


%%%%%%%%%%%%%%%%%%%%%%%%%%%%%%%%%%%%%%%%%%%%%%%%%%%%%%%%%%%%%%%%%%%%%%%%%%%%%%


% always show bad boxes
%\overfullrule=1em


%%%%%%%%%%%%%%%%%%%%%%%%%%%%%%%%%%%%%%%%%%%%%%%%%%%%%%%%%%%%%%%%%%%%%%%%%%%%%%
% biblatex
%%%%%%%%%%%%%%%%%%%%%%%%%%%%%%%%%%%%%%%%%%%%%%%%%%%%%%%%%%%%%%%%%%%%%%%%%%%%%%

\addbibresource{slides-dissertation-defense.bib}
% \renewcommand*{\finalnamedelim}{\addcomma\space}
% \setlength{\bibitemsep}{1em}
% 
\AtEveryBibitem{% Clean up the bibtex rather than editing it
 \clearlist{address}
 \clearfield{date}
 \clearfield{eprint}
 \clearfield{isbn}
 \clearfield{issn}
 \clearlist{language}
 \clearlist{location}
 \clearfield{month}
 \clearfield{series}
%  \clearfield{url}
%  \clearfield{doi}
 \clearfield{organization}

%  \ifentrytype{book}{}{% Remove stuff except for books
%   \clearfield{booktitle}
%   \clearfield{pages}
  \clearlist{publisher}
  \clearname{editor}
%  }
}
% do not print url if doi is present
% http://tex.stackexchange.com/questions/154864/biblatex-use-doi-only-if-there-is-no-url
\DeclareSourcemap{
	\maps[datatype=bibtex]{
		\map{
			\step[fieldsource=doi,final]
			\step[fieldset=url,null]
}	}	}
%
% remove qoutes around titles
\DeclareFieldFormat
	[article,inbook,incollection,inproceedings,patent,thesis,unpublished]
	{title}{#1\isdot}
% 
% \DeclareFieldFormat{url}{\mkbibacro{URL}\addcolon\addnbspace\url{#1}}
% 
% \DeclareNameAlias{sortname}{first-last}
% 
\renewbibmacro{in:}{\ifentrytype{article}{}{}}


%%%%%%%%%%%%%%%%%%%%%%%%%%%%%%%%%%%%%%%%%%%%%%%%%%%%%%%%%%%%%%%%%%%%%%%%%%%%%%
% beamer
%%%%%%%%%%%%%%%%%%%%%%%%%%%%%%%%%%%%%%%%%%%%%%%%%%%%%%%%%%%%%%%%%%%%%%%%%%%%%%

\useoutertheme{infolines}
\makeatletter
% based on
% /usr/share/texmf-dist/tex/latex/beamer/beamerouterthemeinfolines.sty
\setbeamertemplate{footline}
{%
	\leavevmode%
	\hbox{%
	\begin{beamercolorbox}[wd=.333333\paperwidth,ht=2.25ex,dp=1ex,center]{author in head/foot}%
		\usebeamerfont{author in head/foot}\insertshortauthor\expandafter\beamer@ifempty\expandafter{\beamer@shortinstitute}{}{~~(\insertshortinstitute)}
	\end{beamercolorbox}%
	\begin{beamercolorbox}[wd=.333333\paperwidth,ht=2.25ex,dp=1ex,center]{title in head/foot}%
		\usebeamerfont{title in head/foot}\insertshorttitle
	\end{beamercolorbox}%
	\begin{beamercolorbox}[wd=.333333\paperwidth,ht=2.25ex,dp=1ex,right]{date in head/foot}%
		\usebeamerfont{date in head/foot}%
		\hfill\insertshortdate\hfill\hfill%
		%\hspace*{2ex}%
		%\insertshortdate%
		%\hspace{0pt plus 1 filll}%
		%(\insertframenumber.\insertoverlaynumber{} / \insertmainframenumber)%
		%\hspace{0pt plus 1 filll}%
		\phantom{000}\llap{\insertpagenumber} / \insertpresentationendpage%
		\hspace*{2ex}%
	\end{beamercolorbox}}%
	\vskip0pt%
}
\makeatother
\useinnertheme{circles}
\beamertemplatenavigationsymbolsempty
\setbeamertemplate{bibliography item}{}
\setbeamertemplate{headline}[default]

\input{tudcolors.tex}
\setbeamercolor*{alerted text}{fg=HKS07K100}
\usecolortheme[named=HKS41K100]{structure}

\setbeamercolor*{palette primary}{use=structure,fg=white,bg=structure.fg}
\setbeamercolor*{palette secondary}{use=structure,fg=white,bg=structure.fg!80}
\setbeamercolor*{palette tertiary}{use=structure,fg=white,bg=structure.fg!60}
\setbeamercolor*{palette quaternary}{fg=white,bg=black}

\setbeamercolor*{sidebar}{use=structure,bg=structure.fg}

\setbeamercolor*{palette sidebar primary}{use=structure,fg=structure.fg!20}
\setbeamercolor*{palette sidebar secondary}{fg=white}
\setbeamercolor*{palette sidebar tertiary}{use=structure,fg=structure.fg!40}
\setbeamercolor*{palette sidebar quaternary}{fg=white}

\setbeamercolor*{titlelike}{parent=palette primary}

\setbeamercolor*{separation line}{}
\setbeamercolor*{fine separation line}{}

\setbeamercolor{block title}{use=structure,fg=white,bg=structure.fg}
\setbeamercolor{block title alerted}{use=alerted text,fg=white,bg=alerted text.fg!75!black}
\setbeamercolor{block title example}{use=example text,fg=white,bg=example text.fg!75!black}

\setbeamercolor{block body}{parent=normal text,use=block title,bg=block title.bg!10!bg}
\setbeamercolor{block body alerted}{parent=normal text,use=block title alerted,bg=block title alerted.bg!10!bg}
\setbeamercolor{block body example}{parent=normal text,use=block title example,bg=block title example.bg!10!bg}

% \setbeamertemplate{itemize items}[default]


%%%%%%%%%%%%%%%%%%%%%%%%%%%%%%%%%%%%%%%%%%%%%%%%%%%%%%%%%%%%%%%%%%%%%%%%%%%%%%
% TikZ
%%%%%%%%%%%%%%%%%%%%%%%%%%%%%%%%%%%%%%%%%%%%%%%%%%%%%%%%%%%%%%%%%%%%%%%%%%%%%%

\tikzset
	{ > = Stealth
	}


%%%%%%%%%%%%%%%%%%%%%%%%%%%%%%%%%%%%%%%%%%%%%%%%%%%%%%%%%%%%%%%%%%%%%%%%%%%%%%
% general commands and styles
%%%%%%%%%%%%%%%%%%%%%%%%%%%%%%%%%%%%%%%%%%%%%%%%%%%%%%%%%%%%%%%%%%%%%%%%%%%%%%

% \delegateStyle and \inheritStyle command
% usage: \delegateStyle{… \inheritStyle{…} …}
% example: \(X_{\delegateStyle{\fbox{\inheritStyle{X}}}}\)
% Save the current style and regain it in the argument.
% This works both for math and text mode, and can be nested.
% Acknowledgments: Based on \ThisStyle and \SavedStyle from scalerel package.
\makeatletter
\newcommand*{\@inheritStyle@D}[1]{\(\displaystyle      #1\)}
\newcommand*{\@inheritStyle@T}[1]{\(\textstyle         #1\)}
\newcommand*{\@inheritStyle@S}[1]{\(\scriptstyle       #1\)}
\newcommand*{\@inheritStyle@s}[1]{\(\scriptscriptstyle #1\)}
\newcommand*{\@inheritStyle@t}[1]{#1}
\newcommand*{\inheritStyle}{\csname @inheritStyle@\@inheritStyleSwitch\endcsname}
\newcommand*{\delegateStyle}[1]{%
	\ifmmode%
		\mathchoice%
		{\edef\@inheritStyleSwitch{D}#1}%
		{\edef\@inheritStyleSwitch{T}#1}%
		{\edef\@inheritStyleSwitch{S}#1}%
		{\edef\@inheritStyleSwitch{s}#1}%
	\else%
		\edef\@inheritStyleSwitch{t}#1%
	\fi%
}
\makeatother


% \oalt command
% requires: \delegateStyle and \inheritStyle command
% usage: \oalt<…>[…]{…}{…} (cf. \alt)
% Behaves like \alt, but reserves space according to largest overlays.
% The optional argument defines the alignment inside the reserved space;
% it is one of c, l, r, s (cf. \makebox); the default is c.
\makeatletter
\newlength{\oalt@dp}
\newlength{\oalt@ht}
\newlength{\oalt@wd}
\newbox{\oalt@a}
\newbox{\oalt@b}
\newbox{\oalt@empty}
\newcommand<>*{\oalt}[3][c]{%
	\delegateStyle{%
		% based on \setto… in /usr/share/texmf-dist/tex/latex/base/latex.ltx
		\setbox\oalt@a\hbox{\inheritStyle{#2}}%
		\setbox\oalt@b\hbox{\inheritStyle{#3}}%
		\pgfmathsetlength{\oalt@dp}{max(\dp\oalt@a,\dp\oalt@b)}%
		\pgfmathsetlength{\oalt@ht}{max(\ht\oalt@a,\ht\oalt@b)}%
		\pgfmathsetlength{\oalt@wd}{max(\wd\oalt@a,\wd\oalt@b)}%
		\raisebox{0pt}[\oalt@ht][\oalt@dp]{%
			\makebox[\oalt@wd][#1]{%
				\alt#4{\unhbox\oalt@a}{\unhbox\oalt@b}%
			}%
		}%
		\setbox\oalt@a\box\oalt@empty%
		\setbox\oalt@b\box\oalt@empty%
	}%
}
\makeatother


% \otemporal command
% requires: \delegateStyle and \inheritStyle command
% usage: \otemporal<…>[…]{…}{…}{…} (cf. \temporal)
% Behaves like \temporal, but reserves space according to largest overlays.
% The optional argument defines the alignment inside the reserved space;
% it is one of c, l, r, s (cf. \makebox); the default is c.
\makeatletter
\newlength{\ot@dp}
\newlength{\ot@ht}
\newlength{\ot@wd}
\newbox{\ot@a}
\newbox{\ot@b}
\newbox{\ot@c}
\newbox{\ot@empty}
\newcommand<>*{\otemporal}[4][c]{%
	\delegateStyle{%
		% based on \setto… in /usr/share/texmf-dist/tex/latex/base/latex.ltx
		\setbox\ot@a\hbox{\inheritStyle{#2}}%
		\setbox\ot@b\hbox{\inheritStyle{#3}}%
		\setbox\ot@c\hbox{\inheritStyle{#4}}%
		\pgfmathsetlength{\ot@dp}{max(\dp\ot@a,\dp\ot@b,\dp\ot@c)}%
		\pgfmathsetlength{\ot@ht}{max(\ht\ot@a,\ht\ot@b,\ht\ot@c)}%
		\pgfmathsetlength{\ot@wd}{max(\wd\ot@a,\wd\ot@b,\wd\ot@c)}%
		\raisebox{0pt}[\ot@ht][\ot@dp]{%
			\makebox[\ot@wd][#1]{%
				\temporal#5{\unhbox\ot@a}{\unhbox\ot@b}{\unhbox\ot@c}%
			}%
		}%
		\setbox\ot@a\box\ot@empty%
		\setbox\ot@b\box\ot@empty%
		\setbox\ot@c\box\ot@empty%
	}%
}
\makeatother


% Resize delimiters like braces, brackets, etc.
% Parameters: size, left delimiter, formula, right delimiter
% Example: \delim2({\frac{1}{2}})
\newcommand*{\delim}[4]{%
	\ifcase#1%
		#2#3#4%
	\or%
		\bigl#2#3\bigr#4%
	\or%
		\Bigl#2#3\Bigr#4%
	\or%
		\biggl#2#3\biggr#4%
	\or%
		\Biggl#2#3\Biggr#4%
	\else%
		\left#2#3\right#4%
	\fi%
}


% similar to \fullcite, but using the formatting of \printbibliography
\newcommand*{\printfullcite}[1]{%
	\begin{refsection}%
		\nocite{#1}%
		\DeclareNameAlias{author}{first-last}%
		\printbibliography[heading = none]%
	\end{refsection}%
}


\colorlet{light alert}{HKS07K60}
\tikzset{alert.bg/.style={rounded corners, fill=light alert}}
\tikzset{every picture/.style={line cap=round, semithick}}
% http://tex.stackexchange.com/questions/6135/how-to-make-beamer-overlays-with-tikz-node
\tikzset{onslide/.code args={<#1>#2}{\only<#1>{\pgfkeysalso{#2}}}}
\tikzset{invisible/.code args={<#1>}{\alt<#1>{\pgfkeysalso{transparent}}{\pgfkeysalso{opaque}}}}
\tikzset{uncover/.code args={<#1>}{\alt<#1>{\pgfkeysalso{opaque}}{\pgfkeysalso{opacity=0.25}}}}
\tikzset{visible/.code args={<#1>}{\alt<#1>{\pgfkeysalso{opaque}}{\pgfkeysalso{transparent}}}}
\tikzset{vuncover/.code args=%
	{<#1><#2>}%
	{\alt<#1>%
		{\alt<#2>%
			{\pgfkeysalso{opaque}}%
			{\pgfkeysalso{opacity=0.25}}%
		}{\pgfkeysalso{transparent}}%
	}%
}

\newcommand<%
	>{\tikzhighlight}[2][]{%
	\delegateStyle{\alt#3%
		{\tikz[baseline=0, anchor=base, inner sep=0.2em, text height=, text depth=]{\node[alert.bg, #1]{\inheritStyle{#2}};}}%
		{\tikz[baseline=0, anchor=base, inner sep=0.2em, text height=, text depth=]{\node[#1, fill=none]{\inheritStyle{#2}};}}%
	}%
}

\newcommand{\mathhighlight}{\tikzhighlight}

\newcommand<>{\mhl}[2][]{\mathhighlight#3[inner sep=0.2em, #1]{#2}}


\newcommand<>{\inlineblock}[2][]{{%
	\usebeamercolor*[fg]{block body}%
	\tikzhighlight#3[fill=block body.bg, #1]{#2}%
}}


% a small letter s for plurals of abbreviations
\newcommand*{\s}{{\scriptsize s}\xspace}


\newcommand<>*{\sout}[2][opacity=0.75, ultra thick]{%
	\delegateStyle{%
		\tikz[baseline=0, anchor=base, inner sep=0, outer sep=0]{
			\useasboundingbox node (n) {\inheritStyle{#2}};
			\only#3{
				\node (h) {\inheritStyle{\ifmmode\mathstrut\else\strut\fi}};
				\draw[#1] (n.west |- {$(h.south)!0.5!(h.north)$}) -- (n.east |- {$(h.south)!0.5!(h.north)$});
			}
		}%
	}%
}


% tight style
% Sets outer sep to default inner sep and inner sep to 0.
% Use this style for nodes that are neither drawn nor filled to prevent
% unwanted growth of the bounding box.
\tikzset{tight/.style={inner sep=0, outer sep=0.3333em}}


% rounded tree edges style
% usage: rounded tree edges={⟨direction⟩}{⟨looseness⟩}{⟨strength⟩}
\tikzset{
	rounded tree edges/.style n args={3}{
	edge from parent path={
	let
		\n{direction}={#1},
		\n{looseness}={#2},
		\n{strength}={#3},
		\p1=(\tikzparentnode),
		\p2=(\tikzchildnode),
		\p3=(\n{direction}:1pt),
		\p4=(\x2 - \x1, \y2 - \y1),
		\n{dist}={veclen(\p4)},
		\p4=(\x4 / \n{dist}, \y4 / \n{dist}),
		\n{angle}={atan2(\y4, \x4)},
		\n{delta}={Mod(\n{angle} - \n{direction}, 360)},
		\n{delta}={\n{delta} > 180 ? \n{delta} - 360  : \n{delta}},
		\n{delta}={\n{delta} >  90 ?  180 - \n{delta} : \n{delta}},
		\n{delta}={\n{delta} < -90 ? -180 - \n{delta} : \n{delta}}
	in (\tikzparentnode) .. controls
		+(    \n{angle}+\n{strength}*\n{delta}:\n{looseness}*0.3915*\n{dist}) and
		+(180+\n{angle}-\n{strength}*\n{delta}:\n{looseness}*0.3915*\n{dist}) ..
		(\tikzchildnode)
	}
	}
}


% Tear out snippets from PDFs.
% Usage: \tear[…]{file.pdf}
% The optional parameter is the same as for \includegraphics.
% Useful Arguments:
%   * page=‹pagenumber›
%   * trim=‹left› ‹bottom› ‹right› ‹top›
%   * width=0.98\linewidth
\newcommand*{\tear}[2][]{%
	\begin{tikzpicture}
		\node
			[ blur shadow
			, clip
			, decorate
			, decoration=random steps
			, draw
			, inner sep=0
			, preaction={fill=white}% hide the shadow if paper is transparent
			] {\includegraphics[#1]{#2}};
	\end{tikzpicture}%
}


\makeatletter
\newcommand*{\timeline}[3][0]{%
	\ifcsname timeline@cmd@#3\endcsname%
		\@timeline[#1]{#2}{#3}%
		\PackageWarning{timeline}{redefining timeline \@backslashchar\string#3}%
	\else%
		\ifcsname#3\endcsname%
			\errmessage{Command \@backslashchar\string#3 already defined}%
		\else%
			\@timeline[#1]{#2}{#3}%
		\fi%
	\fi%
}%
\newcommand*{\@timeline}[3][0]{%
	% mark command as timeline command – they can be overwritten
	\expandafter\def\csname timeline@cmd@#3\endcsname{}%
	\setcounter{@timeline}{#1}%
	\def\timeline@cmd{#3}%
	\timeline@reset%
	\timeline@append{0}%
	\@tfor\timeline@next:=#2\do{%
		\if\timeline@next+%
			\stepcounter{@timeline}%
			\timeline@append{,\the@timeline}%
		\else\if\timeline@next-%
			\stepcounter{@timeline}%
		\else%
			%\timeline@append{\timeline@next}%
			\GenericError{}{\protect\timeline: ignoring unknown character: \timeline@next}%
		\fi\fi%
	}%
}%
% \newcommand*{\tl}[1]{%
% 	\ifcsname timeline@cmd@#1\endcsname%
% 		\csname timeline@cmd@#1\endcsname%
% 	\else%
% 		0%
% 		%\GenericError{}{\protect\tl: timeline not defined: #1}%
% 	\fi%
% }%
\newcounter{@timeline}%
\def\timeline@reset{%
	\expandafter\def\csname\timeline@cmd\endcsname{}%
}%
\def\timeline@append#1{%
	\expandafter\edef\csname\timeline@cmd\endcsname{%
		\csname\timeline@cmd\endcsname#1%
	}%
}%
\makeatother


\newcommand*{\xminus}[1]{%
	\mathrel{\tikz[baseline={([yshift=-0.25em]n.south)}, inner sep=0, outer sep=0.2em]{%
		\node (n) {\(\scriptstyle #1\)};
		\draw (n.south west) -- (n.south east);
	}}%
}
\newcommand*{\tikzrightarrow}[1]{%
	\mathrel{\tikz[baseline={([yshift=-0.25em]n.south)}, inner sep=0, outer sep=0.2em]{%
		\node (n) {\(\scriptstyle #1\)};
		\draw[->, > = Computer Modern Rightarrow, line width = 0.4pt] (n.south west) -- (n.south east);
	}}%
}


%%%%%%%%%%%%%%%%%%%%%%%%%%%%%%%%%%%%%%%%%%%%%%%%%%%%%%%%%%%%%%%%%%%%%%%%%%%%%%
% document specific commands
%%%%%%%%%%%%%%%%%%%%%%%%%%%%%%%%%%%%%%%%%%%%%%%%%%%%%%%%%%%%%%%%%%%%%%%%%%%%%%

\newcommand<>*{\mycite}[1]{\uncover#2{{\color{HKS57K100}[\cite{#1}]}}}


\newcommand{\statetree}[1]{
	\tikz
	[ anchor=base
	, baseline=(current bounding box.center)
	, level distance=2em
	, sibling distance=2em
	]{
		\matrix
		[ draw=nt
		, edge from parent/.style={draw=black}
		, inner sep=0
		, nodes={inner sep=0.2em, rounded corners=0}
		, rounded corners
		] {#1\\}
	}
}


\newcommand*{\mylargeleaf}[1]{{\LARGE\color{HKS41K70}#1}}

\definecolor{state s}{named}{HKS57K80}
\definecolor{state t}{named}{HKS41K70}
\newcommand*{\stateS}[1]{{\color{state s}#1}}
\newcommand*{\stateT}[1]{{\color{state t}#1}}

\tikzset{
	subtree/.style =
		{ fill=lightgray
		, inner sep=0.02em
		, isosceles triangle apex angle=60
		, shape=isosceles triangle
		, shape border rotate=90
		}
	, state/.style = {circle, draw, inner sep=0.1em}
	, trans/.style = {rectangle, draw}
}

\newcommand*{\srBool}{\mathbb{B}}
\newcommand*{\srProb}{ℙ}


%%%%%%%%%%%%%%%%%%%%%%%%%%%%%%%%%%%%%%%%%%%%%%%%%%%%%%%%%%%%%%%%%%%%%%%%%%%%%%
% commands for specific notations
%%%%%%%%%%%%%%%%%%%%%%%%%%%%%%%%%%%%%%%%%%%%%%%%%%%%%%%%%%%%%%%%%%%%%%%%%%%%%%

\DeclareMathOperator*{\argmax}{argmax}

\newcommand*{\cardinality}[1]{\lvert#1\rvert}
\newcommand*{\corpussize}[1]{\lvert#1\rvert}

\DeclareMathOperator{\crispOp}{crisp}
\newcommand*        {\crisp}[2][0]{\crispOp\delim{#1}({#2})}

\DeclareMathOperator{\lhsOp}{lhs}
\newcommand*{\lhs}[1]{\lhsOp(#1)}

\DeclareMathOperator{\lklhdOp}{L}
\newcommand*{\lklhd}[2]{\lklhdOp(#1 ∣ #2)}

\DeclareMathOperator{\mleOp}{mle}
\newcommand*{\mle}[2][]{%
	\ifthenelse{\isempty{#1}}{%
		\mleOp(#2)%
	}{%
		\mleOp_{#1}(#2)%
	}%
}

\DeclareMathOperator{\mrg}{merge}

% CVD: color vision deficiencies
\definecolor{CVD light red}   {HTML}{FF8080}
\definecolor{CVD light yellow}{HTML}{FFFF80}
\definecolor{CVD light green} {HTML}{40FFC0}

\definecolor{nt}{named}{HKS41K70}
\newcommand*{\nt}[1]{{\color{nt}#1}}

% set of all probability distributions over #1
\DeclareMathOperator{\pdsOp}{Pd}
\newcommand*{\pds}[1]{\pdsOp(#1)}

\DeclareMathOperator{\positionsOp}{pos}
\newcommand*{\positions}[1]{\positionsOp(#1)}

\DeclareMathOperator{\rankOp}{rk}
\newcommand*{\rank}[1]{\rankOp(#1)}

\DeclareMathOperator{\runsOp}{run}
\newcommand*{\runs}[2][]{%
	\ifthenelse%
		{\isempty{#1}}%
		{\runsOp(#2)}%
		{\runsOp_{#1}(#2)}%
}

\newcommand*{\semantics}[1]{⟦#1⟧}

\DeclareMathOperator{\splt}{split}

\newcommand*{\subtree}[2]{#1|_{#2}}

\DeclareMathOperator{\supportOp}{supp}
\newcommand*{\support}[1]{\supportOp(#1)}

\newcommand*{\symId}{\textsc{\color{gray}Id}}
\newcommand*{\symCons}{\textsc{\color{gray}Cons}}
\newcommand*{\symFlip}{\textsc{\color{gray}Flip}}
\newcommand*{\symNull}{\textsc{\color{gray}Null}}
\newcommand*{\symNullR}{\textsc{\color{gray}N\(\overline{\textsc{ull}}\)}}
\newcommand*{\symSnoc}{\textsc{\color{gray}Snoc}}

\newcommand*{\transWTA}[4][]{#3 \xrightarrow{#1} #2(#4)}

\DeclareMathOperator{\uniqueRunOp}{r}
\newcommand*{\uniqueRun}[2][]{%
	\ifthenelse%
		{\isempty{#1}}%
		{\uniqueRunOp^{#2}}%
		{\uniqueRunOp_{\!#1}^{#2}}%
}

\DeclareMathOperator{\treesOp}{T}
\newcommand*{\trees}[2][]{%
	\ifthenelse%
		{\isempty{#1}}%
		{\treesOp_{\!#2}}%
		{\treesOp_{\!#2}(#1)}%
}
\DeclareMathOperator{\treesUOp}{U}
\newcommand*{\treesU}[2][]{%
	\ifthenelse%
		{\isempty{#1}}%
		{\treesUOp_{#2}}%
		{\treesUOp_{#2}(#1)}%
}


%%%%%%%%%%%%%%%%%%%%%%%%%%%%%%%%%%%%%%%%%%%%%%%%%%%%%%%%%%%%%%%%%%%%%%%%%%%%%%
% metadata
%%%%%%%%%%%%%%%%%%%%%%%%%%%%%%%%%%%%%%%%%%%%%%%%%%%%%%%%%%%%%%%%%%%%%%%%%%%%%%

\ifstandalonebeamer\else
	\title[Defense of Dissertation]{A Formal View on Training of Weighted Tree Automata by Likelihood-Driven State Splitting and Merging}
	\subtitle{Defense of Dissertation}
\fi
\author{Toni Dietze}
\institute[TU Dresden]{%
	\href{https://www.orchid.inf.tu-dresden.de/index.en/}{Chair for Foundations of Programming}
\\	\href{https://tu-dresden.de/ing/informatik/thi}{Institute of Theoretical Computer Science}
\\	\href{https://tu-dresden.de/ing/informatik}{Faculty of Computer Science}
\\	\href{https://tu-dresden.de/}{Technische Universität Dresden}
\\	01062 Dresden, Germany
}
\date[2018-09-27]{September 27, 2018}

\title{\jobname}
\begin{document}
\begin{standaloneframe}{\jobname}
	\begin{block}{some included grammars of the Berkeley Parser}
		\begin{center}
% 			\begin{tabular}{r|rrr}
% 				grammar & \texttt{arb\_sm5} & \texttt{eng\_sm6} & \texttt{ger\_sm5}
% 			\\\hline
% 				binary rules & 897,168 & 1,725,570 & 615,199
% 			\\
% 				unary rules & 48,315 & 116,648 & 1,577
% 			\\
% 				nullary rules & 2,300,883 & 2,425,059 & 1,495,873
% 			\\\hline
% 				sum & 3,246,366 & 4,267,277 & 2,112,649
% 			\end{tabular}
			\begin{tabular}{r|rrr|r}
				grammar & binary rules & unary rules & nullary rules & sum
			\\\hline
				\texttt{arb\_sm5} & 897,168 & 48,315 & 2,300,883 & 3,246,366
			\\
				\texttt{eng\_sm6} & 1,725,570 & 116,648 & 2,425,059 & 4,267,277
			\\
				\texttt{ger\_sm5} & 615,199 & 1,577 & 1,495,873 & 2,112,649
			\end{tabular}
		\end{center}
	\end{block}
\end{standaloneframe}
\end{document}

\end{frame}


\section{Canonical Tree Automaton}

\begin{frame}{\secname}
	\centering%
	\documentclass[beamer]{standalone}
% SPDX-License-Identifier: CC-BY-4.0 OR MIT-0
% Copyright 2018 Toni Dietze
%
\usefonttheme{professionalfonts}

% LuaLaTeX specific packages
\usepackage{fontspec}
	\defaultfontfeatures{Ligatures=TeX}
\usepackage{polyglossia}
	\setdefaultlanguage{english}
\usepackage{amsmath}  % has to be loaded before unicode-math
\usepackage[math-style=ISO]{unicode-math}
	\setmathfont{Latin Modern Math}
% 	\setmathfont[range={\mathcal,\mathbfcal},StylisticSet=1]{xits-math.otf}
% 	\setmathfont[range={"029F5}]{XITS Math}  % ⧵
% 	\setmathfont[range={\mathscr,\mathbfscr},StylisticSet=1]{Latin Modern Math}  % make \mathscr use the correct font

\usepackage[noend]{algpseudocode}
	\algrenewcommand\algorithmicrequire{\textbf{Input:}}
	\algrenewcommand\algorithmicensure{\textbf{Output:}}
\usepackage[backend=biber, maxbibnames=42, maxcitenames=42, sorting=ynt, style=authoryear]{biblatex}
\usepackage{csquotes}
\usepackage{mathtools}
\usepackage{media9}
\usepackage{scalerel}
\usepackage{standalone}
\usepackage{tikz}
	\usetikzlibrary{arrows.meta}
	\usetikzlibrary{backgrounds}
	\usetikzlibrary{calc}
	\usetikzlibrary{decorations}
	\usetikzlibrary{decorations.pathmorphing}
	\usetikzlibrary{decorations.pathreplacing}
	\usetikzlibrary{fadings}
	\usetikzlibrary{fit}
	\usetikzlibrary{graphs}
	\usetikzlibrary{graphdrawing}
	\usetikzlibrary{intersections}
	\usetikzlibrary{positioning}
	\usetikzlibrary{quotes}
	\usetikzlibrary{shadows.blur}
	\usetikzlibrary{shapes.arrows}
	\usetikzlibrary{shapes.geometric}
	\usegdlibrary{trees}
\usepackage{xifthen}
\usepackage{xspace}

\usepackage{pgfplots}
	\pgfplotsset
		{ compat = 1.15
		, /pgf/number format/1000 sep = {\,}
		, /pgf/number format/assume math mode = true
		, every axis plot/.append style =
			{ mark options = {fill opacity = 0.25}
			}
		}
	\usepgfplotslibrary{groupplots}
\usepackage{pgfplotstable}

\hypersetup
	{ bookmarksopen
	, pdflang = en
	, unicode
	}


%%%%%%%%%%%%%%%%%%%%%%%%%%%%%%%%%%%%%%%%%%%%%%%%%%%%%%%%%%%%%%%%%%%%%%%%%%%%%%


% always show bad boxes
%\overfullrule=1em


%%%%%%%%%%%%%%%%%%%%%%%%%%%%%%%%%%%%%%%%%%%%%%%%%%%%%%%%%%%%%%%%%%%%%%%%%%%%%%
% biblatex
%%%%%%%%%%%%%%%%%%%%%%%%%%%%%%%%%%%%%%%%%%%%%%%%%%%%%%%%%%%%%%%%%%%%%%%%%%%%%%

\addbibresource{slides-dissertation-defense.bib}
% \renewcommand*{\finalnamedelim}{\addcomma\space}
% \setlength{\bibitemsep}{1em}
% 
\AtEveryBibitem{% Clean up the bibtex rather than editing it
 \clearlist{address}
 \clearfield{date}
 \clearfield{eprint}
 \clearfield{isbn}
 \clearfield{issn}
 \clearlist{language}
 \clearlist{location}
 \clearfield{month}
 \clearfield{series}
%  \clearfield{url}
%  \clearfield{doi}
 \clearfield{organization}

%  \ifentrytype{book}{}{% Remove stuff except for books
%   \clearfield{booktitle}
%   \clearfield{pages}
  \clearlist{publisher}
  \clearname{editor}
%  }
}
% do not print url if doi is present
% http://tex.stackexchange.com/questions/154864/biblatex-use-doi-only-if-there-is-no-url
\DeclareSourcemap{
	\maps[datatype=bibtex]{
		\map{
			\step[fieldsource=doi,final]
			\step[fieldset=url,null]
}	}	}
%
% remove qoutes around titles
\DeclareFieldFormat
	[article,inbook,incollection,inproceedings,patent,thesis,unpublished]
	{title}{#1\isdot}
% 
% \DeclareFieldFormat{url}{\mkbibacro{URL}\addcolon\addnbspace\url{#1}}
% 
% \DeclareNameAlias{sortname}{first-last}
% 
\renewbibmacro{in:}{\ifentrytype{article}{}{}}


%%%%%%%%%%%%%%%%%%%%%%%%%%%%%%%%%%%%%%%%%%%%%%%%%%%%%%%%%%%%%%%%%%%%%%%%%%%%%%
% beamer
%%%%%%%%%%%%%%%%%%%%%%%%%%%%%%%%%%%%%%%%%%%%%%%%%%%%%%%%%%%%%%%%%%%%%%%%%%%%%%

\useoutertheme{infolines}
\makeatletter
% based on
% /usr/share/texmf-dist/tex/latex/beamer/beamerouterthemeinfolines.sty
\setbeamertemplate{footline}
{%
	\leavevmode%
	\hbox{%
	\begin{beamercolorbox}[wd=.333333\paperwidth,ht=2.25ex,dp=1ex,center]{author in head/foot}%
		\usebeamerfont{author in head/foot}\insertshortauthor\expandafter\beamer@ifempty\expandafter{\beamer@shortinstitute}{}{~~(\insertshortinstitute)}
	\end{beamercolorbox}%
	\begin{beamercolorbox}[wd=.333333\paperwidth,ht=2.25ex,dp=1ex,center]{title in head/foot}%
		\usebeamerfont{title in head/foot}\insertshorttitle
	\end{beamercolorbox}%
	\begin{beamercolorbox}[wd=.333333\paperwidth,ht=2.25ex,dp=1ex,right]{date in head/foot}%
		\usebeamerfont{date in head/foot}%
		\hfill\insertshortdate\hfill\hfill%
		%\hspace*{2ex}%
		%\insertshortdate%
		%\hspace{0pt plus 1 filll}%
		%(\insertframenumber.\insertoverlaynumber{} / \insertmainframenumber)%
		%\hspace{0pt plus 1 filll}%
		\phantom{000}\llap{\insertpagenumber} / \insertpresentationendpage%
		\hspace*{2ex}%
	\end{beamercolorbox}}%
	\vskip0pt%
}
\makeatother
\useinnertheme{circles}
\beamertemplatenavigationsymbolsempty
\setbeamertemplate{bibliography item}{}
\setbeamertemplate{headline}[default]

\input{tudcolors.tex}
\setbeamercolor*{alerted text}{fg=HKS07K100}
\usecolortheme[named=HKS41K100]{structure}

\setbeamercolor*{palette primary}{use=structure,fg=white,bg=structure.fg}
\setbeamercolor*{palette secondary}{use=structure,fg=white,bg=structure.fg!80}
\setbeamercolor*{palette tertiary}{use=structure,fg=white,bg=structure.fg!60}
\setbeamercolor*{palette quaternary}{fg=white,bg=black}

\setbeamercolor*{sidebar}{use=structure,bg=structure.fg}

\setbeamercolor*{palette sidebar primary}{use=structure,fg=structure.fg!20}
\setbeamercolor*{palette sidebar secondary}{fg=white}
\setbeamercolor*{palette sidebar tertiary}{use=structure,fg=structure.fg!40}
\setbeamercolor*{palette sidebar quaternary}{fg=white}

\setbeamercolor*{titlelike}{parent=palette primary}

\setbeamercolor*{separation line}{}
\setbeamercolor*{fine separation line}{}

\setbeamercolor{block title}{use=structure,fg=white,bg=structure.fg}
\setbeamercolor{block title alerted}{use=alerted text,fg=white,bg=alerted text.fg!75!black}
\setbeamercolor{block title example}{use=example text,fg=white,bg=example text.fg!75!black}

\setbeamercolor{block body}{parent=normal text,use=block title,bg=block title.bg!10!bg}
\setbeamercolor{block body alerted}{parent=normal text,use=block title alerted,bg=block title alerted.bg!10!bg}
\setbeamercolor{block body example}{parent=normal text,use=block title example,bg=block title example.bg!10!bg}

% \setbeamertemplate{itemize items}[default]


%%%%%%%%%%%%%%%%%%%%%%%%%%%%%%%%%%%%%%%%%%%%%%%%%%%%%%%%%%%%%%%%%%%%%%%%%%%%%%
% TikZ
%%%%%%%%%%%%%%%%%%%%%%%%%%%%%%%%%%%%%%%%%%%%%%%%%%%%%%%%%%%%%%%%%%%%%%%%%%%%%%

\tikzset
	{ > = Stealth
	}


%%%%%%%%%%%%%%%%%%%%%%%%%%%%%%%%%%%%%%%%%%%%%%%%%%%%%%%%%%%%%%%%%%%%%%%%%%%%%%
% general commands and styles
%%%%%%%%%%%%%%%%%%%%%%%%%%%%%%%%%%%%%%%%%%%%%%%%%%%%%%%%%%%%%%%%%%%%%%%%%%%%%%

% \delegateStyle and \inheritStyle command
% usage: \delegateStyle{… \inheritStyle{…} …}
% example: \(X_{\delegateStyle{\fbox{\inheritStyle{X}}}}\)
% Save the current style and regain it in the argument.
% This works both for math and text mode, and can be nested.
% Acknowledgments: Based on \ThisStyle and \SavedStyle from scalerel package.
\makeatletter
\newcommand*{\@inheritStyle@D}[1]{\(\displaystyle      #1\)}
\newcommand*{\@inheritStyle@T}[1]{\(\textstyle         #1\)}
\newcommand*{\@inheritStyle@S}[1]{\(\scriptstyle       #1\)}
\newcommand*{\@inheritStyle@s}[1]{\(\scriptscriptstyle #1\)}
\newcommand*{\@inheritStyle@t}[1]{#1}
\newcommand*{\inheritStyle}{\csname @inheritStyle@\@inheritStyleSwitch\endcsname}
\newcommand*{\delegateStyle}[1]{%
	\ifmmode%
		\mathchoice%
		{\edef\@inheritStyleSwitch{D}#1}%
		{\edef\@inheritStyleSwitch{T}#1}%
		{\edef\@inheritStyleSwitch{S}#1}%
		{\edef\@inheritStyleSwitch{s}#1}%
	\else%
		\edef\@inheritStyleSwitch{t}#1%
	\fi%
}
\makeatother


% \oalt command
% requires: \delegateStyle and \inheritStyle command
% usage: \oalt<…>[…]{…}{…} (cf. \alt)
% Behaves like \alt, but reserves space according to largest overlays.
% The optional argument defines the alignment inside the reserved space;
% it is one of c, l, r, s (cf. \makebox); the default is c.
\makeatletter
\newlength{\oalt@dp}
\newlength{\oalt@ht}
\newlength{\oalt@wd}
\newbox{\oalt@a}
\newbox{\oalt@b}
\newbox{\oalt@empty}
\newcommand<>*{\oalt}[3][c]{%
	\delegateStyle{%
		% based on \setto… in /usr/share/texmf-dist/tex/latex/base/latex.ltx
		\setbox\oalt@a\hbox{\inheritStyle{#2}}%
		\setbox\oalt@b\hbox{\inheritStyle{#3}}%
		\pgfmathsetlength{\oalt@dp}{max(\dp\oalt@a,\dp\oalt@b)}%
		\pgfmathsetlength{\oalt@ht}{max(\ht\oalt@a,\ht\oalt@b)}%
		\pgfmathsetlength{\oalt@wd}{max(\wd\oalt@a,\wd\oalt@b)}%
		\raisebox{0pt}[\oalt@ht][\oalt@dp]{%
			\makebox[\oalt@wd][#1]{%
				\alt#4{\unhbox\oalt@a}{\unhbox\oalt@b}%
			}%
		}%
		\setbox\oalt@a\box\oalt@empty%
		\setbox\oalt@b\box\oalt@empty%
	}%
}
\makeatother


% \otemporal command
% requires: \delegateStyle and \inheritStyle command
% usage: \otemporal<…>[…]{…}{…}{…} (cf. \temporal)
% Behaves like \temporal, but reserves space according to largest overlays.
% The optional argument defines the alignment inside the reserved space;
% it is one of c, l, r, s (cf. \makebox); the default is c.
\makeatletter
\newlength{\ot@dp}
\newlength{\ot@ht}
\newlength{\ot@wd}
\newbox{\ot@a}
\newbox{\ot@b}
\newbox{\ot@c}
\newbox{\ot@empty}
\newcommand<>*{\otemporal}[4][c]{%
	\delegateStyle{%
		% based on \setto… in /usr/share/texmf-dist/tex/latex/base/latex.ltx
		\setbox\ot@a\hbox{\inheritStyle{#2}}%
		\setbox\ot@b\hbox{\inheritStyle{#3}}%
		\setbox\ot@c\hbox{\inheritStyle{#4}}%
		\pgfmathsetlength{\ot@dp}{max(\dp\ot@a,\dp\ot@b,\dp\ot@c)}%
		\pgfmathsetlength{\ot@ht}{max(\ht\ot@a,\ht\ot@b,\ht\ot@c)}%
		\pgfmathsetlength{\ot@wd}{max(\wd\ot@a,\wd\ot@b,\wd\ot@c)}%
		\raisebox{0pt}[\ot@ht][\ot@dp]{%
			\makebox[\ot@wd][#1]{%
				\temporal#5{\unhbox\ot@a}{\unhbox\ot@b}{\unhbox\ot@c}%
			}%
		}%
		\setbox\ot@a\box\ot@empty%
		\setbox\ot@b\box\ot@empty%
		\setbox\ot@c\box\ot@empty%
	}%
}
\makeatother


% Resize delimiters like braces, brackets, etc.
% Parameters: size, left delimiter, formula, right delimiter
% Example: \delim2({\frac{1}{2}})
\newcommand*{\delim}[4]{%
	\ifcase#1%
		#2#3#4%
	\or%
		\bigl#2#3\bigr#4%
	\or%
		\Bigl#2#3\Bigr#4%
	\or%
		\biggl#2#3\biggr#4%
	\or%
		\Biggl#2#3\Biggr#4%
	\else%
		\left#2#3\right#4%
	\fi%
}


% similar to \fullcite, but using the formatting of \printbibliography
\newcommand*{\printfullcite}[1]{%
	\begin{refsection}%
		\nocite{#1}%
		\DeclareNameAlias{author}{first-last}%
		\printbibliography[heading = none]%
	\end{refsection}%
}


\colorlet{light alert}{HKS07K60}
\tikzset{alert.bg/.style={rounded corners, fill=light alert}}
\tikzset{every picture/.style={line cap=round, semithick}}
% http://tex.stackexchange.com/questions/6135/how-to-make-beamer-overlays-with-tikz-node
\tikzset{onslide/.code args={<#1>#2}{\only<#1>{\pgfkeysalso{#2}}}}
\tikzset{invisible/.code args={<#1>}{\alt<#1>{\pgfkeysalso{transparent}}{\pgfkeysalso{opaque}}}}
\tikzset{uncover/.code args={<#1>}{\alt<#1>{\pgfkeysalso{opaque}}{\pgfkeysalso{opacity=0.25}}}}
\tikzset{visible/.code args={<#1>}{\alt<#1>{\pgfkeysalso{opaque}}{\pgfkeysalso{transparent}}}}
\tikzset{vuncover/.code args=%
	{<#1><#2>}%
	{\alt<#1>%
		{\alt<#2>%
			{\pgfkeysalso{opaque}}%
			{\pgfkeysalso{opacity=0.25}}%
		}{\pgfkeysalso{transparent}}%
	}%
}

\newcommand<%
	>{\tikzhighlight}[2][]{%
	\delegateStyle{\alt#3%
		{\tikz[baseline=0, anchor=base, inner sep=0.2em, text height=, text depth=]{\node[alert.bg, #1]{\inheritStyle{#2}};}}%
		{\tikz[baseline=0, anchor=base, inner sep=0.2em, text height=, text depth=]{\node[#1, fill=none]{\inheritStyle{#2}};}}%
	}%
}

\newcommand{\mathhighlight}{\tikzhighlight}

\newcommand<>{\mhl}[2][]{\mathhighlight#3[inner sep=0.2em, #1]{#2}}


\newcommand<>{\inlineblock}[2][]{{%
	\usebeamercolor*[fg]{block body}%
	\tikzhighlight#3[fill=block body.bg, #1]{#2}%
}}


% a small letter s for plurals of abbreviations
\newcommand*{\s}{{\scriptsize s}\xspace}


\newcommand<>*{\sout}[2][opacity=0.75, ultra thick]{%
	\delegateStyle{%
		\tikz[baseline=0, anchor=base, inner sep=0, outer sep=0]{
			\useasboundingbox node (n) {\inheritStyle{#2}};
			\only#3{
				\node (h) {\inheritStyle{\ifmmode\mathstrut\else\strut\fi}};
				\draw[#1] (n.west |- {$(h.south)!0.5!(h.north)$}) -- (n.east |- {$(h.south)!0.5!(h.north)$});
			}
		}%
	}%
}


% tight style
% Sets outer sep to default inner sep and inner sep to 0.
% Use this style for nodes that are neither drawn nor filled to prevent
% unwanted growth of the bounding box.
\tikzset{tight/.style={inner sep=0, outer sep=0.3333em}}


% rounded tree edges style
% usage: rounded tree edges={⟨direction⟩}{⟨looseness⟩}{⟨strength⟩}
\tikzset{
	rounded tree edges/.style n args={3}{
	edge from parent path={
	let
		\n{direction}={#1},
		\n{looseness}={#2},
		\n{strength}={#3},
		\p1=(\tikzparentnode),
		\p2=(\tikzchildnode),
		\p3=(\n{direction}:1pt),
		\p4=(\x2 - \x1, \y2 - \y1),
		\n{dist}={veclen(\p4)},
		\p4=(\x4 / \n{dist}, \y4 / \n{dist}),
		\n{angle}={atan2(\y4, \x4)},
		\n{delta}={Mod(\n{angle} - \n{direction}, 360)},
		\n{delta}={\n{delta} > 180 ? \n{delta} - 360  : \n{delta}},
		\n{delta}={\n{delta} >  90 ?  180 - \n{delta} : \n{delta}},
		\n{delta}={\n{delta} < -90 ? -180 - \n{delta} : \n{delta}}
	in (\tikzparentnode) .. controls
		+(    \n{angle}+\n{strength}*\n{delta}:\n{looseness}*0.3915*\n{dist}) and
		+(180+\n{angle}-\n{strength}*\n{delta}:\n{looseness}*0.3915*\n{dist}) ..
		(\tikzchildnode)
	}
	}
}


% Tear out snippets from PDFs.
% Usage: \tear[…]{file.pdf}
% The optional parameter is the same as for \includegraphics.
% Useful Arguments:
%   * page=‹pagenumber›
%   * trim=‹left› ‹bottom› ‹right› ‹top›
%   * width=0.98\linewidth
\newcommand*{\tear}[2][]{%
	\begin{tikzpicture}
		\node
			[ blur shadow
			, clip
			, decorate
			, decoration=random steps
			, draw
			, inner sep=0
			, preaction={fill=white}% hide the shadow if paper is transparent
			] {\includegraphics[#1]{#2}};
	\end{tikzpicture}%
}


\makeatletter
\newcommand*{\timeline}[3][0]{%
	\ifcsname timeline@cmd@#3\endcsname%
		\@timeline[#1]{#2}{#3}%
		\PackageWarning{timeline}{redefining timeline \@backslashchar\string#3}%
	\else%
		\ifcsname#3\endcsname%
			\errmessage{Command \@backslashchar\string#3 already defined}%
		\else%
			\@timeline[#1]{#2}{#3}%
		\fi%
	\fi%
}%
\newcommand*{\@timeline}[3][0]{%
	% mark command as timeline command – they can be overwritten
	\expandafter\def\csname timeline@cmd@#3\endcsname{}%
	\setcounter{@timeline}{#1}%
	\def\timeline@cmd{#3}%
	\timeline@reset%
	\timeline@append{0}%
	\@tfor\timeline@next:=#2\do{%
		\if\timeline@next+%
			\stepcounter{@timeline}%
			\timeline@append{,\the@timeline}%
		\else\if\timeline@next-%
			\stepcounter{@timeline}%
		\else%
			%\timeline@append{\timeline@next}%
			\GenericError{}{\protect\timeline: ignoring unknown character: \timeline@next}%
		\fi\fi%
	}%
}%
% \newcommand*{\tl}[1]{%
% 	\ifcsname timeline@cmd@#1\endcsname%
% 		\csname timeline@cmd@#1\endcsname%
% 	\else%
% 		0%
% 		%\GenericError{}{\protect\tl: timeline not defined: #1}%
% 	\fi%
% }%
\newcounter{@timeline}%
\def\timeline@reset{%
	\expandafter\def\csname\timeline@cmd\endcsname{}%
}%
\def\timeline@append#1{%
	\expandafter\edef\csname\timeline@cmd\endcsname{%
		\csname\timeline@cmd\endcsname#1%
	}%
}%
\makeatother


\newcommand*{\xminus}[1]{%
	\mathrel{\tikz[baseline={([yshift=-0.25em]n.south)}, inner sep=0, outer sep=0.2em]{%
		\node (n) {\(\scriptstyle #1\)};
		\draw (n.south west) -- (n.south east);
	}}%
}
\newcommand*{\tikzrightarrow}[1]{%
	\mathrel{\tikz[baseline={([yshift=-0.25em]n.south)}, inner sep=0, outer sep=0.2em]{%
		\node (n) {\(\scriptstyle #1\)};
		\draw[->, > = Computer Modern Rightarrow, line width = 0.4pt] (n.south west) -- (n.south east);
	}}%
}


%%%%%%%%%%%%%%%%%%%%%%%%%%%%%%%%%%%%%%%%%%%%%%%%%%%%%%%%%%%%%%%%%%%%%%%%%%%%%%
% document specific commands
%%%%%%%%%%%%%%%%%%%%%%%%%%%%%%%%%%%%%%%%%%%%%%%%%%%%%%%%%%%%%%%%%%%%%%%%%%%%%%

\newcommand<>*{\mycite}[1]{\uncover#2{{\color{HKS57K100}[\cite{#1}]}}}


\newcommand{\statetree}[1]{
	\tikz
	[ anchor=base
	, baseline=(current bounding box.center)
	, level distance=2em
	, sibling distance=2em
	]{
		\matrix
		[ draw=nt
		, edge from parent/.style={draw=black}
		, inner sep=0
		, nodes={inner sep=0.2em, rounded corners=0}
		, rounded corners
		] {#1\\}
	}
}


\newcommand*{\mylargeleaf}[1]{{\LARGE\color{HKS41K70}#1}}

\definecolor{state s}{named}{HKS57K80}
\definecolor{state t}{named}{HKS41K70}
\newcommand*{\stateS}[1]{{\color{state s}#1}}
\newcommand*{\stateT}[1]{{\color{state t}#1}}

\tikzset{
	subtree/.style =
		{ fill=lightgray
		, inner sep=0.02em
		, isosceles triangle apex angle=60
		, shape=isosceles triangle
		, shape border rotate=90
		}
	, state/.style = {circle, draw, inner sep=0.1em}
	, trans/.style = {rectangle, draw}
}

\newcommand*{\srBool}{\mathbb{B}}
\newcommand*{\srProb}{ℙ}


%%%%%%%%%%%%%%%%%%%%%%%%%%%%%%%%%%%%%%%%%%%%%%%%%%%%%%%%%%%%%%%%%%%%%%%%%%%%%%
% commands for specific notations
%%%%%%%%%%%%%%%%%%%%%%%%%%%%%%%%%%%%%%%%%%%%%%%%%%%%%%%%%%%%%%%%%%%%%%%%%%%%%%

\DeclareMathOperator*{\argmax}{argmax}

\newcommand*{\cardinality}[1]{\lvert#1\rvert}
\newcommand*{\corpussize}[1]{\lvert#1\rvert}

\DeclareMathOperator{\crispOp}{crisp}
\newcommand*        {\crisp}[2][0]{\crispOp\delim{#1}({#2})}

\DeclareMathOperator{\lhsOp}{lhs}
\newcommand*{\lhs}[1]{\lhsOp(#1)}

\DeclareMathOperator{\lklhdOp}{L}
\newcommand*{\lklhd}[2]{\lklhdOp(#1 ∣ #2)}

\DeclareMathOperator{\mleOp}{mle}
\newcommand*{\mle}[2][]{%
	\ifthenelse{\isempty{#1}}{%
		\mleOp(#2)%
	}{%
		\mleOp_{#1}(#2)%
	}%
}

\DeclareMathOperator{\mrg}{merge}

% CVD: color vision deficiencies
\definecolor{CVD light red}   {HTML}{FF8080}
\definecolor{CVD light yellow}{HTML}{FFFF80}
\definecolor{CVD light green} {HTML}{40FFC0}

\definecolor{nt}{named}{HKS41K70}
\newcommand*{\nt}[1]{{\color{nt}#1}}

% set of all probability distributions over #1
\DeclareMathOperator{\pdsOp}{Pd}
\newcommand*{\pds}[1]{\pdsOp(#1)}

\DeclareMathOperator{\positionsOp}{pos}
\newcommand*{\positions}[1]{\positionsOp(#1)}

\DeclareMathOperator{\rankOp}{rk}
\newcommand*{\rank}[1]{\rankOp(#1)}

\DeclareMathOperator{\runsOp}{run}
\newcommand*{\runs}[2][]{%
	\ifthenelse%
		{\isempty{#1}}%
		{\runsOp(#2)}%
		{\runsOp_{#1}(#2)}%
}

\newcommand*{\semantics}[1]{⟦#1⟧}

\DeclareMathOperator{\splt}{split}

\newcommand*{\subtree}[2]{#1|_{#2}}

\DeclareMathOperator{\supportOp}{supp}
\newcommand*{\support}[1]{\supportOp(#1)}

\newcommand*{\symId}{\textsc{\color{gray}Id}}
\newcommand*{\symCons}{\textsc{\color{gray}Cons}}
\newcommand*{\symFlip}{\textsc{\color{gray}Flip}}
\newcommand*{\symNull}{\textsc{\color{gray}Null}}
\newcommand*{\symNullR}{\textsc{\color{gray}N\(\overline{\textsc{ull}}\)}}
\newcommand*{\symSnoc}{\textsc{\color{gray}Snoc}}

\newcommand*{\transWTA}[4][]{#3 \xrightarrow{#1} #2(#4)}

\DeclareMathOperator{\uniqueRunOp}{r}
\newcommand*{\uniqueRun}[2][]{%
	\ifthenelse%
		{\isempty{#1}}%
		{\uniqueRunOp^{#2}}%
		{\uniqueRunOp_{\!#1}^{#2}}%
}

\DeclareMathOperator{\treesOp}{T}
\newcommand*{\trees}[2][]{%
	\ifthenelse%
		{\isempty{#1}}%
		{\treesOp_{\!#2}}%
		{\treesOp_{\!#2}(#1)}%
}
\DeclareMathOperator{\treesUOp}{U}
\newcommand*{\treesU}[2][]{%
	\ifthenelse%
		{\isempty{#1}}%
		{\treesUOp_{#2}}%
		{\treesUOp_{#2}(#1)}%
}


%%%%%%%%%%%%%%%%%%%%%%%%%%%%%%%%%%%%%%%%%%%%%%%%%%%%%%%%%%%%%%%%%%%%%%%%%%%%%%
% metadata
%%%%%%%%%%%%%%%%%%%%%%%%%%%%%%%%%%%%%%%%%%%%%%%%%%%%%%%%%%%%%%%%%%%%%%%%%%%%%%

\ifstandalonebeamer\else
	\title[Defense of Dissertation]{A Formal View on Training of Weighted Tree Automata by Likelihood-Driven State Splitting and Merging}
	\subtitle{Defense of Dissertation}
\fi
\author{Toni Dietze}
\institute[TU Dresden]{%
	\href{https://www.orchid.inf.tu-dresden.de/index.en/}{Chair for Foundations of Programming}
\\	\href{https://tu-dresden.de/ing/informatik/thi}{Institute of Theoretical Computer Science}
\\	\href{https://tu-dresden.de/ing/informatik}{Faculty of Computer Science}
\\	\href{https://tu-dresden.de/}{Technische Universität Dresden}
\\	01062 Dresden, Germany
}
\date[2018-09-27]{September 27, 2018}

\begin{document}
\begin{standaloneframe}{\jobname}
\centering%
\begin{tikzpicture}
[ anchor=base
, level distance=3em
, node distance=0.5em
, sibling distance=2em
, tri/.style =
	{ inner sep=0.12em
	, outer sep=0
	, shape border rotate=90
	, isosceles triangle
	, isosceles triangle apex angle=60
	, draw=black
	}
]
\node (t) at (0, -4em) {\(σ\)}
	child { node[tri] {\(t_1\)} edge from parent[child anchor=apex]}
	child {node {\(\dots\)} edge from parent[draw=none]}
	child { node[tri] {\(t_k\)} edge from parent[child anchor=apex]};
\draw (t-1.right corner) -- (t-3.left corner);
\draw (t-1.left corner)
	-- ++(-1em, 0) coordinate (left)
	-- (0.5em, 0) coordinate (top)
	-- ($(t-3.right corner) + (2em, 0)$) coordinate (right)
	-- (t-3.right corner);
\draw[decorate, decoration={snake, amplitude=0.05em, segment length=1em}]
	(t) .. controls ++(0, 1em) and ++(0, -1em) .. (0.5em, 0);


% draw subsequent tree triangles
\coordinate (s) at (0, 0);
\foreach \s in {1, 2, ..., 4}
	\draw
		coordinate (sOld) at (s)
		coordinate (s) at (\s*0.5em, \s*0.125em)
		(intersection cs: first line = {([shift={(sOld)}]top) -- ([shift={(sOld)}]right)}
						, second line = {([shift={(s)}]top)    -- ([shift={(s)}]left)})
		-- ([shift={(s)}]top)
		-- ([shift={(s)}]right)
		-- (intersection cs: first line = {([shift={(sOld)}]top) -- ([shift={(sOld)}]right)}
							, second line = {([shift={(s)}]right)  -- ([shift={(s)}]left)});


\node at (0, -9em) {\(\Downarrow\)};


\matrix[draw=nt, edge from parent/.style={draw=black}, rounded corners, cells={rounded corners=0}] (lhs) at (-8em, -11em) {
	\node (t) {\(σ\)}
		child { node[tri] {\(t_1\)} edge from parent[child anchor=apex]}
		child {node {\(\dots\)} edge from parent[draw=none]}
		child { node[tri] {\(t_k\)} edge from parent[child anchor=apex]};
\\
};
\node[right=0 of lhs] (arr) {\({} → σ\Big(\)};
\matrix[base right=of arr, draw=nt, edge from parent/.style={draw=black}, rounded corners, cells={rounded corners=0}]
	(t1) {\node[tri] {\(t_1\)};\\};
\node[base right=of t1] (dots) {\(, \dots,\)};
\matrix[base right=of dots, draw=nt, edge from parent/.style={draw=black}, rounded corners, cells={rounded corners=0}]
	(tk) {\node[tri] {\(t_k\)};\\};
\node[base right=of tk] {\(\Big)\)};
\end{tikzpicture}
\pause
\begin{center}
	result is bottom-up deterministic
\end{center}
\end{standaloneframe}
\end{document}

\end{frame}


\begin{frame}<-3>[label=frame:ta]{\secname}
	% SPDX-License-Identifier: CC-BY-4.0
% Copyright 2018 Toni Dietze
\documentclass[beamer]{standalone}
% SPDX-License-Identifier: CC-BY-4.0 OR MIT-0
% Copyright 2018 Toni Dietze
%
\usefonttheme{professionalfonts}

% LuaLaTeX specific packages
\usepackage{fontspec}
	\defaultfontfeatures{Ligatures=TeX}
\usepackage{polyglossia}
	\setdefaultlanguage{english}
\usepackage{amsmath}  % has to be loaded before unicode-math
\usepackage[math-style=ISO]{unicode-math}
	\setmathfont{Latin Modern Math}
% 	\setmathfont[range={\mathcal,\mathbfcal},StylisticSet=1]{xits-math.otf}
% 	\setmathfont[range={"029F5}]{XITS Math}  % ⧵
% 	\setmathfont[range={\mathscr,\mathbfscr},StylisticSet=1]{Latin Modern Math}  % make \mathscr use the correct font

\usepackage[noend]{algpseudocode}
	\algrenewcommand\algorithmicrequire{\textbf{Input:}}
	\algrenewcommand\algorithmicensure{\textbf{Output:}}
\usepackage[backend=biber, maxbibnames=42, maxcitenames=42, sorting=ynt, style=authoryear]{biblatex}
\usepackage{csquotes}
\usepackage{mathtools}
\usepackage{media9}
\usepackage{scalerel}
\usepackage{standalone}
\usepackage{tikz}
	\usetikzlibrary{arrows.meta}
	\usetikzlibrary{backgrounds}
	\usetikzlibrary{calc}
	\usetikzlibrary{decorations}
	\usetikzlibrary{decorations.pathmorphing}
	\usetikzlibrary{decorations.pathreplacing}
	\usetikzlibrary{fadings}
	\usetikzlibrary{fit}
	\usetikzlibrary{graphs}
	\usetikzlibrary{graphdrawing}
	\usetikzlibrary{intersections}
	\usetikzlibrary{positioning}
	\usetikzlibrary{quotes}
	\usetikzlibrary{shadows.blur}
	\usetikzlibrary{shapes.arrows}
	\usetikzlibrary{shapes.geometric}
	\usegdlibrary{trees}
\usepackage{xifthen}
\usepackage{xspace}

\usepackage{pgfplots}
	\pgfplotsset
		{ compat = 1.15
		, /pgf/number format/1000 sep = {\,}
		, /pgf/number format/assume math mode = true
		, every axis plot/.append style =
			{ mark options = {fill opacity = 0.25}
			}
		}
	\usepgfplotslibrary{groupplots}
\usepackage{pgfplotstable}

\hypersetup
	{ bookmarksopen
	, pdflang = en
	, unicode
	}


%%%%%%%%%%%%%%%%%%%%%%%%%%%%%%%%%%%%%%%%%%%%%%%%%%%%%%%%%%%%%%%%%%%%%%%%%%%%%%


% always show bad boxes
%\overfullrule=1em


%%%%%%%%%%%%%%%%%%%%%%%%%%%%%%%%%%%%%%%%%%%%%%%%%%%%%%%%%%%%%%%%%%%%%%%%%%%%%%
% biblatex
%%%%%%%%%%%%%%%%%%%%%%%%%%%%%%%%%%%%%%%%%%%%%%%%%%%%%%%%%%%%%%%%%%%%%%%%%%%%%%

\addbibresource{slides-dissertation-defense.bib}
% \renewcommand*{\finalnamedelim}{\addcomma\space}
% \setlength{\bibitemsep}{1em}
% 
\AtEveryBibitem{% Clean up the bibtex rather than editing it
 \clearlist{address}
 \clearfield{date}
 \clearfield{eprint}
 \clearfield{isbn}
 \clearfield{issn}
 \clearlist{language}
 \clearlist{location}
 \clearfield{month}
 \clearfield{series}
%  \clearfield{url}
%  \clearfield{doi}
 \clearfield{organization}

%  \ifentrytype{book}{}{% Remove stuff except for books
%   \clearfield{booktitle}
%   \clearfield{pages}
  \clearlist{publisher}
  \clearname{editor}
%  }
}
% do not print url if doi is present
% http://tex.stackexchange.com/questions/154864/biblatex-use-doi-only-if-there-is-no-url
\DeclareSourcemap{
	\maps[datatype=bibtex]{
		\map{
			\step[fieldsource=doi,final]
			\step[fieldset=url,null]
}	}	}
%
% remove qoutes around titles
\DeclareFieldFormat
	[article,inbook,incollection,inproceedings,patent,thesis,unpublished]
	{title}{#1\isdot}
% 
% \DeclareFieldFormat{url}{\mkbibacro{URL}\addcolon\addnbspace\url{#1}}
% 
% \DeclareNameAlias{sortname}{first-last}
% 
\renewbibmacro{in:}{\ifentrytype{article}{}{}}


%%%%%%%%%%%%%%%%%%%%%%%%%%%%%%%%%%%%%%%%%%%%%%%%%%%%%%%%%%%%%%%%%%%%%%%%%%%%%%
% beamer
%%%%%%%%%%%%%%%%%%%%%%%%%%%%%%%%%%%%%%%%%%%%%%%%%%%%%%%%%%%%%%%%%%%%%%%%%%%%%%

\useoutertheme{infolines}
\makeatletter
% based on
% /usr/share/texmf-dist/tex/latex/beamer/beamerouterthemeinfolines.sty
\setbeamertemplate{footline}
{%
	\leavevmode%
	\hbox{%
	\begin{beamercolorbox}[wd=.333333\paperwidth,ht=2.25ex,dp=1ex,center]{author in head/foot}%
		\usebeamerfont{author in head/foot}\insertshortauthor\expandafter\beamer@ifempty\expandafter{\beamer@shortinstitute}{}{~~(\insertshortinstitute)}
	\end{beamercolorbox}%
	\begin{beamercolorbox}[wd=.333333\paperwidth,ht=2.25ex,dp=1ex,center]{title in head/foot}%
		\usebeamerfont{title in head/foot}\insertshorttitle
	\end{beamercolorbox}%
	\begin{beamercolorbox}[wd=.333333\paperwidth,ht=2.25ex,dp=1ex,right]{date in head/foot}%
		\usebeamerfont{date in head/foot}%
		\hfill\insertshortdate\hfill\hfill%
		%\hspace*{2ex}%
		%\insertshortdate%
		%\hspace{0pt plus 1 filll}%
		%(\insertframenumber.\insertoverlaynumber{} / \insertmainframenumber)%
		%\hspace{0pt plus 1 filll}%
		\phantom{000}\llap{\insertpagenumber} / \insertpresentationendpage%
		\hspace*{2ex}%
	\end{beamercolorbox}}%
	\vskip0pt%
}
\makeatother
\useinnertheme{circles}
\beamertemplatenavigationsymbolsempty
\setbeamertemplate{bibliography item}{}
\setbeamertemplate{headline}[default]

\input{tudcolors.tex}
\setbeamercolor*{alerted text}{fg=HKS07K100}
\usecolortheme[named=HKS41K100]{structure}

\setbeamercolor*{palette primary}{use=structure,fg=white,bg=structure.fg}
\setbeamercolor*{palette secondary}{use=structure,fg=white,bg=structure.fg!80}
\setbeamercolor*{palette tertiary}{use=structure,fg=white,bg=structure.fg!60}
\setbeamercolor*{palette quaternary}{fg=white,bg=black}

\setbeamercolor*{sidebar}{use=structure,bg=structure.fg}

\setbeamercolor*{palette sidebar primary}{use=structure,fg=structure.fg!20}
\setbeamercolor*{palette sidebar secondary}{fg=white}
\setbeamercolor*{palette sidebar tertiary}{use=structure,fg=structure.fg!40}
\setbeamercolor*{palette sidebar quaternary}{fg=white}

\setbeamercolor*{titlelike}{parent=palette primary}

\setbeamercolor*{separation line}{}
\setbeamercolor*{fine separation line}{}

\setbeamercolor{block title}{use=structure,fg=white,bg=structure.fg}
\setbeamercolor{block title alerted}{use=alerted text,fg=white,bg=alerted text.fg!75!black}
\setbeamercolor{block title example}{use=example text,fg=white,bg=example text.fg!75!black}

\setbeamercolor{block body}{parent=normal text,use=block title,bg=block title.bg!10!bg}
\setbeamercolor{block body alerted}{parent=normal text,use=block title alerted,bg=block title alerted.bg!10!bg}
\setbeamercolor{block body example}{parent=normal text,use=block title example,bg=block title example.bg!10!bg}

% \setbeamertemplate{itemize items}[default]


%%%%%%%%%%%%%%%%%%%%%%%%%%%%%%%%%%%%%%%%%%%%%%%%%%%%%%%%%%%%%%%%%%%%%%%%%%%%%%
% TikZ
%%%%%%%%%%%%%%%%%%%%%%%%%%%%%%%%%%%%%%%%%%%%%%%%%%%%%%%%%%%%%%%%%%%%%%%%%%%%%%

\tikzset
	{ > = Stealth
	}


%%%%%%%%%%%%%%%%%%%%%%%%%%%%%%%%%%%%%%%%%%%%%%%%%%%%%%%%%%%%%%%%%%%%%%%%%%%%%%
% general commands and styles
%%%%%%%%%%%%%%%%%%%%%%%%%%%%%%%%%%%%%%%%%%%%%%%%%%%%%%%%%%%%%%%%%%%%%%%%%%%%%%

% \delegateStyle and \inheritStyle command
% usage: \delegateStyle{… \inheritStyle{…} …}
% example: \(X_{\delegateStyle{\fbox{\inheritStyle{X}}}}\)
% Save the current style and regain it in the argument.
% This works both for math and text mode, and can be nested.
% Acknowledgments: Based on \ThisStyle and \SavedStyle from scalerel package.
\makeatletter
\newcommand*{\@inheritStyle@D}[1]{\(\displaystyle      #1\)}
\newcommand*{\@inheritStyle@T}[1]{\(\textstyle         #1\)}
\newcommand*{\@inheritStyle@S}[1]{\(\scriptstyle       #1\)}
\newcommand*{\@inheritStyle@s}[1]{\(\scriptscriptstyle #1\)}
\newcommand*{\@inheritStyle@t}[1]{#1}
\newcommand*{\inheritStyle}{\csname @inheritStyle@\@inheritStyleSwitch\endcsname}
\newcommand*{\delegateStyle}[1]{%
	\ifmmode%
		\mathchoice%
		{\edef\@inheritStyleSwitch{D}#1}%
		{\edef\@inheritStyleSwitch{T}#1}%
		{\edef\@inheritStyleSwitch{S}#1}%
		{\edef\@inheritStyleSwitch{s}#1}%
	\else%
		\edef\@inheritStyleSwitch{t}#1%
	\fi%
}
\makeatother


% \oalt command
% requires: \delegateStyle and \inheritStyle command
% usage: \oalt<…>[…]{…}{…} (cf. \alt)
% Behaves like \alt, but reserves space according to largest overlays.
% The optional argument defines the alignment inside the reserved space;
% it is one of c, l, r, s (cf. \makebox); the default is c.
\makeatletter
\newlength{\oalt@dp}
\newlength{\oalt@ht}
\newlength{\oalt@wd}
\newbox{\oalt@a}
\newbox{\oalt@b}
\newbox{\oalt@empty}
\newcommand<>*{\oalt}[3][c]{%
	\delegateStyle{%
		% based on \setto… in /usr/share/texmf-dist/tex/latex/base/latex.ltx
		\setbox\oalt@a\hbox{\inheritStyle{#2}}%
		\setbox\oalt@b\hbox{\inheritStyle{#3}}%
		\pgfmathsetlength{\oalt@dp}{max(\dp\oalt@a,\dp\oalt@b)}%
		\pgfmathsetlength{\oalt@ht}{max(\ht\oalt@a,\ht\oalt@b)}%
		\pgfmathsetlength{\oalt@wd}{max(\wd\oalt@a,\wd\oalt@b)}%
		\raisebox{0pt}[\oalt@ht][\oalt@dp]{%
			\makebox[\oalt@wd][#1]{%
				\alt#4{\unhbox\oalt@a}{\unhbox\oalt@b}%
			}%
		}%
		\setbox\oalt@a\box\oalt@empty%
		\setbox\oalt@b\box\oalt@empty%
	}%
}
\makeatother


% \otemporal command
% requires: \delegateStyle and \inheritStyle command
% usage: \otemporal<…>[…]{…}{…}{…} (cf. \temporal)
% Behaves like \temporal, but reserves space according to largest overlays.
% The optional argument defines the alignment inside the reserved space;
% it is one of c, l, r, s (cf. \makebox); the default is c.
\makeatletter
\newlength{\ot@dp}
\newlength{\ot@ht}
\newlength{\ot@wd}
\newbox{\ot@a}
\newbox{\ot@b}
\newbox{\ot@c}
\newbox{\ot@empty}
\newcommand<>*{\otemporal}[4][c]{%
	\delegateStyle{%
		% based on \setto… in /usr/share/texmf-dist/tex/latex/base/latex.ltx
		\setbox\ot@a\hbox{\inheritStyle{#2}}%
		\setbox\ot@b\hbox{\inheritStyle{#3}}%
		\setbox\ot@c\hbox{\inheritStyle{#4}}%
		\pgfmathsetlength{\ot@dp}{max(\dp\ot@a,\dp\ot@b,\dp\ot@c)}%
		\pgfmathsetlength{\ot@ht}{max(\ht\ot@a,\ht\ot@b,\ht\ot@c)}%
		\pgfmathsetlength{\ot@wd}{max(\wd\ot@a,\wd\ot@b,\wd\ot@c)}%
		\raisebox{0pt}[\ot@ht][\ot@dp]{%
			\makebox[\ot@wd][#1]{%
				\temporal#5{\unhbox\ot@a}{\unhbox\ot@b}{\unhbox\ot@c}%
			}%
		}%
		\setbox\ot@a\box\ot@empty%
		\setbox\ot@b\box\ot@empty%
		\setbox\ot@c\box\ot@empty%
	}%
}
\makeatother


% Resize delimiters like braces, brackets, etc.
% Parameters: size, left delimiter, formula, right delimiter
% Example: \delim2({\frac{1}{2}})
\newcommand*{\delim}[4]{%
	\ifcase#1%
		#2#3#4%
	\or%
		\bigl#2#3\bigr#4%
	\or%
		\Bigl#2#3\Bigr#4%
	\or%
		\biggl#2#3\biggr#4%
	\or%
		\Biggl#2#3\Biggr#4%
	\else%
		\left#2#3\right#4%
	\fi%
}


% similar to \fullcite, but using the formatting of \printbibliography
\newcommand*{\printfullcite}[1]{%
	\begin{refsection}%
		\nocite{#1}%
		\DeclareNameAlias{author}{first-last}%
		\printbibliography[heading = none]%
	\end{refsection}%
}


\colorlet{light alert}{HKS07K60}
\tikzset{alert.bg/.style={rounded corners, fill=light alert}}
\tikzset{every picture/.style={line cap=round, semithick}}
% http://tex.stackexchange.com/questions/6135/how-to-make-beamer-overlays-with-tikz-node
\tikzset{onslide/.code args={<#1>#2}{\only<#1>{\pgfkeysalso{#2}}}}
\tikzset{invisible/.code args={<#1>}{\alt<#1>{\pgfkeysalso{transparent}}{\pgfkeysalso{opaque}}}}
\tikzset{uncover/.code args={<#1>}{\alt<#1>{\pgfkeysalso{opaque}}{\pgfkeysalso{opacity=0.25}}}}
\tikzset{visible/.code args={<#1>}{\alt<#1>{\pgfkeysalso{opaque}}{\pgfkeysalso{transparent}}}}
\tikzset{vuncover/.code args=%
	{<#1><#2>}%
	{\alt<#1>%
		{\alt<#2>%
			{\pgfkeysalso{opaque}}%
			{\pgfkeysalso{opacity=0.25}}%
		}{\pgfkeysalso{transparent}}%
	}%
}

\newcommand<%
	>{\tikzhighlight}[2][]{%
	\delegateStyle{\alt#3%
		{\tikz[baseline=0, anchor=base, inner sep=0.2em, text height=, text depth=]{\node[alert.bg, #1]{\inheritStyle{#2}};}}%
		{\tikz[baseline=0, anchor=base, inner sep=0.2em, text height=, text depth=]{\node[#1, fill=none]{\inheritStyle{#2}};}}%
	}%
}

\newcommand{\mathhighlight}{\tikzhighlight}

\newcommand<>{\mhl}[2][]{\mathhighlight#3[inner sep=0.2em, #1]{#2}}


\newcommand<>{\inlineblock}[2][]{{%
	\usebeamercolor*[fg]{block body}%
	\tikzhighlight#3[fill=block body.bg, #1]{#2}%
}}


% a small letter s for plurals of abbreviations
\newcommand*{\s}{{\scriptsize s}\xspace}


\newcommand<>*{\sout}[2][opacity=0.75, ultra thick]{%
	\delegateStyle{%
		\tikz[baseline=0, anchor=base, inner sep=0, outer sep=0]{
			\useasboundingbox node (n) {\inheritStyle{#2}};
			\only#3{
				\node (h) {\inheritStyle{\ifmmode\mathstrut\else\strut\fi}};
				\draw[#1] (n.west |- {$(h.south)!0.5!(h.north)$}) -- (n.east |- {$(h.south)!0.5!(h.north)$});
			}
		}%
	}%
}


% tight style
% Sets outer sep to default inner sep and inner sep to 0.
% Use this style for nodes that are neither drawn nor filled to prevent
% unwanted growth of the bounding box.
\tikzset{tight/.style={inner sep=0, outer sep=0.3333em}}


% rounded tree edges style
% usage: rounded tree edges={⟨direction⟩}{⟨looseness⟩}{⟨strength⟩}
\tikzset{
	rounded tree edges/.style n args={3}{
	edge from parent path={
	let
		\n{direction}={#1},
		\n{looseness}={#2},
		\n{strength}={#3},
		\p1=(\tikzparentnode),
		\p2=(\tikzchildnode),
		\p3=(\n{direction}:1pt),
		\p4=(\x2 - \x1, \y2 - \y1),
		\n{dist}={veclen(\p4)},
		\p4=(\x4 / \n{dist}, \y4 / \n{dist}),
		\n{angle}={atan2(\y4, \x4)},
		\n{delta}={Mod(\n{angle} - \n{direction}, 360)},
		\n{delta}={\n{delta} > 180 ? \n{delta} - 360  : \n{delta}},
		\n{delta}={\n{delta} >  90 ?  180 - \n{delta} : \n{delta}},
		\n{delta}={\n{delta} < -90 ? -180 - \n{delta} : \n{delta}}
	in (\tikzparentnode) .. controls
		+(    \n{angle}+\n{strength}*\n{delta}:\n{looseness}*0.3915*\n{dist}) and
		+(180+\n{angle}-\n{strength}*\n{delta}:\n{looseness}*0.3915*\n{dist}) ..
		(\tikzchildnode)
	}
	}
}


% Tear out snippets from PDFs.
% Usage: \tear[…]{file.pdf}
% The optional parameter is the same as for \includegraphics.
% Useful Arguments:
%   * page=‹pagenumber›
%   * trim=‹left› ‹bottom› ‹right› ‹top›
%   * width=0.98\linewidth
\newcommand*{\tear}[2][]{%
	\begin{tikzpicture}
		\node
			[ blur shadow
			, clip
			, decorate
			, decoration=random steps
			, draw
			, inner sep=0
			, preaction={fill=white}% hide the shadow if paper is transparent
			] {\includegraphics[#1]{#2}};
	\end{tikzpicture}%
}


\makeatletter
\newcommand*{\timeline}[3][0]{%
	\ifcsname timeline@cmd@#3\endcsname%
		\@timeline[#1]{#2}{#3}%
		\PackageWarning{timeline}{redefining timeline \@backslashchar\string#3}%
	\else%
		\ifcsname#3\endcsname%
			\errmessage{Command \@backslashchar\string#3 already defined}%
		\else%
			\@timeline[#1]{#2}{#3}%
		\fi%
	\fi%
}%
\newcommand*{\@timeline}[3][0]{%
	% mark command as timeline command – they can be overwritten
	\expandafter\def\csname timeline@cmd@#3\endcsname{}%
	\setcounter{@timeline}{#1}%
	\def\timeline@cmd{#3}%
	\timeline@reset%
	\timeline@append{0}%
	\@tfor\timeline@next:=#2\do{%
		\if\timeline@next+%
			\stepcounter{@timeline}%
			\timeline@append{,\the@timeline}%
		\else\if\timeline@next-%
			\stepcounter{@timeline}%
		\else%
			%\timeline@append{\timeline@next}%
			\GenericError{}{\protect\timeline: ignoring unknown character: \timeline@next}%
		\fi\fi%
	}%
}%
% \newcommand*{\tl}[1]{%
% 	\ifcsname timeline@cmd@#1\endcsname%
% 		\csname timeline@cmd@#1\endcsname%
% 	\else%
% 		0%
% 		%\GenericError{}{\protect\tl: timeline not defined: #1}%
% 	\fi%
% }%
\newcounter{@timeline}%
\def\timeline@reset{%
	\expandafter\def\csname\timeline@cmd\endcsname{}%
}%
\def\timeline@append#1{%
	\expandafter\edef\csname\timeline@cmd\endcsname{%
		\csname\timeline@cmd\endcsname#1%
	}%
}%
\makeatother


\newcommand*{\xminus}[1]{%
	\mathrel{\tikz[baseline={([yshift=-0.25em]n.south)}, inner sep=0, outer sep=0.2em]{%
		\node (n) {\(\scriptstyle #1\)};
		\draw (n.south west) -- (n.south east);
	}}%
}
\newcommand*{\tikzrightarrow}[1]{%
	\mathrel{\tikz[baseline={([yshift=-0.25em]n.south)}, inner sep=0, outer sep=0.2em]{%
		\node (n) {\(\scriptstyle #1\)};
		\draw[->, > = Computer Modern Rightarrow, line width = 0.4pt] (n.south west) -- (n.south east);
	}}%
}


%%%%%%%%%%%%%%%%%%%%%%%%%%%%%%%%%%%%%%%%%%%%%%%%%%%%%%%%%%%%%%%%%%%%%%%%%%%%%%
% document specific commands
%%%%%%%%%%%%%%%%%%%%%%%%%%%%%%%%%%%%%%%%%%%%%%%%%%%%%%%%%%%%%%%%%%%%%%%%%%%%%%

\newcommand<>*{\mycite}[1]{\uncover#2{{\color{HKS57K100}[\cite{#1}]}}}


\newcommand{\statetree}[1]{
	\tikz
	[ anchor=base
	, baseline=(current bounding box.center)
	, level distance=2em
	, sibling distance=2em
	]{
		\matrix
		[ draw=nt
		, edge from parent/.style={draw=black}
		, inner sep=0
		, nodes={inner sep=0.2em, rounded corners=0}
		, rounded corners
		] {#1\\}
	}
}


\newcommand*{\mylargeleaf}[1]{{\LARGE\color{HKS41K70}#1}}

\definecolor{state s}{named}{HKS57K80}
\definecolor{state t}{named}{HKS41K70}
\newcommand*{\stateS}[1]{{\color{state s}#1}}
\newcommand*{\stateT}[1]{{\color{state t}#1}}

\tikzset{
	subtree/.style =
		{ fill=lightgray
		, inner sep=0.02em
		, isosceles triangle apex angle=60
		, shape=isosceles triangle
		, shape border rotate=90
		}
	, state/.style = {circle, draw, inner sep=0.1em}
	, trans/.style = {rectangle, draw}
}

\newcommand*{\srBool}{\mathbb{B}}
\newcommand*{\srProb}{ℙ}


%%%%%%%%%%%%%%%%%%%%%%%%%%%%%%%%%%%%%%%%%%%%%%%%%%%%%%%%%%%%%%%%%%%%%%%%%%%%%%
% commands for specific notations
%%%%%%%%%%%%%%%%%%%%%%%%%%%%%%%%%%%%%%%%%%%%%%%%%%%%%%%%%%%%%%%%%%%%%%%%%%%%%%

\DeclareMathOperator*{\argmax}{argmax}

\newcommand*{\cardinality}[1]{\lvert#1\rvert}
\newcommand*{\corpussize}[1]{\lvert#1\rvert}

\DeclareMathOperator{\crispOp}{crisp}
\newcommand*        {\crisp}[2][0]{\crispOp\delim{#1}({#2})}

\DeclareMathOperator{\lhsOp}{lhs}
\newcommand*{\lhs}[1]{\lhsOp(#1)}

\DeclareMathOperator{\lklhdOp}{L}
\newcommand*{\lklhd}[2]{\lklhdOp(#1 ∣ #2)}

\DeclareMathOperator{\mleOp}{mle}
\newcommand*{\mle}[2][]{%
	\ifthenelse{\isempty{#1}}{%
		\mleOp(#2)%
	}{%
		\mleOp_{#1}(#2)%
	}%
}

\DeclareMathOperator{\mrg}{merge}

% CVD: color vision deficiencies
\definecolor{CVD light red}   {HTML}{FF8080}
\definecolor{CVD light yellow}{HTML}{FFFF80}
\definecolor{CVD light green} {HTML}{40FFC0}

\definecolor{nt}{named}{HKS41K70}
\newcommand*{\nt}[1]{{\color{nt}#1}}

% set of all probability distributions over #1
\DeclareMathOperator{\pdsOp}{Pd}
\newcommand*{\pds}[1]{\pdsOp(#1)}

\DeclareMathOperator{\positionsOp}{pos}
\newcommand*{\positions}[1]{\positionsOp(#1)}

\DeclareMathOperator{\rankOp}{rk}
\newcommand*{\rank}[1]{\rankOp(#1)}

\DeclareMathOperator{\runsOp}{run}
\newcommand*{\runs}[2][]{%
	\ifthenelse%
		{\isempty{#1}}%
		{\runsOp(#2)}%
		{\runsOp_{#1}(#2)}%
}

\newcommand*{\semantics}[1]{⟦#1⟧}

\DeclareMathOperator{\splt}{split}

\newcommand*{\subtree}[2]{#1|_{#2}}

\DeclareMathOperator{\supportOp}{supp}
\newcommand*{\support}[1]{\supportOp(#1)}

\newcommand*{\symId}{\textsc{\color{gray}Id}}
\newcommand*{\symCons}{\textsc{\color{gray}Cons}}
\newcommand*{\symFlip}{\textsc{\color{gray}Flip}}
\newcommand*{\symNull}{\textsc{\color{gray}Null}}
\newcommand*{\symNullR}{\textsc{\color{gray}N\(\overline{\textsc{ull}}\)}}
\newcommand*{\symSnoc}{\textsc{\color{gray}Snoc}}

\newcommand*{\transWTA}[4][]{#3 \xrightarrow{#1} #2(#4)}

\DeclareMathOperator{\uniqueRunOp}{r}
\newcommand*{\uniqueRun}[2][]{%
	\ifthenelse%
		{\isempty{#1}}%
		{\uniqueRunOp^{#2}}%
		{\uniqueRunOp_{\!#1}^{#2}}%
}

\DeclareMathOperator{\treesOp}{T}
\newcommand*{\trees}[2][]{%
	\ifthenelse%
		{\isempty{#1}}%
		{\treesOp_{\!#2}}%
		{\treesOp_{\!#2}(#1)}%
}
\DeclareMathOperator{\treesUOp}{U}
\newcommand*{\treesU}[2][]{%
	\ifthenelse%
		{\isempty{#1}}%
		{\treesUOp_{#2}}%
		{\treesUOp_{#2}(#1)}%
}


%%%%%%%%%%%%%%%%%%%%%%%%%%%%%%%%%%%%%%%%%%%%%%%%%%%%%%%%%%%%%%%%%%%%%%%%%%%%%%
% metadata
%%%%%%%%%%%%%%%%%%%%%%%%%%%%%%%%%%%%%%%%%%%%%%%%%%%%%%%%%%%%%%%%%%%%%%%%%%%%%%

\ifstandalonebeamer\else
	\title[Defense of Dissertation]{A Formal View on Training of Weighted Tree Automata by Likelihood-Driven State Splitting and Merging}
	\subtitle{Defense of Dissertation}
\fi
\author{Toni Dietze}
\institute[TU Dresden]{%
	\href{https://www.orchid.inf.tu-dresden.de/index.en/}{Chair for Foundations of Programming}
\\	\href{https://tu-dresden.de/ing/informatik/thi}{Institute of Theoretical Computer Science}
\\	\href{https://tu-dresden.de/ing/informatik}{Faculty of Computer Science}
\\	\href{https://tu-dresden.de/}{Technische Universität Dresden}
\\	01062 Dresden, Germany
}
\date[2018-09-27]{September 27, 2018}

\begin{document}
\begin{standaloneframe}{\jobname}
\begin{columns}[T]
\column{0.5\linewidth}
	\begin{overprint}
	\onslide<-3>
		\begin{center}
		\begin{tikzpicture}[anchor=base, level distance=3em]
			\node (t) {}
				child { node {\(σ\)}
					child { node {\(α\)}
						edge from parent node[left, visible=<2->] {\(\nt{A}\)}
					}
					child { node {\(σ\)}
						child { node {\(β\)}
							edge from parent node[left, visible=<2->] {\(\nt{B}\)}
						}
						child { node {\(σ\)}
							child { node {\(β\)}
								edge from parent node[left, visible=<2->] {\(\nt{B}\)}
							}
							child { node {\(β\)}
								edge from parent node[right, visible=<2->] {\(\nt{B}\)}
							}
							edge from parent node[right, visible=<2->] {\(\nt{D}\)}
						}
						edge from parent node[right, visible=<2->] {\(\nt{C}\)}
					}
					edge from parent node[right, visible=<2->] {\(\nt{S}\)}
					edge from parent [draw=none]
				};
				\node[left=1em of t-1] {\(t_1\colon\)};
		\end{tikzpicture}

		\begin{tikzpicture}[anchor=base, level distance=3em]
			\node (t) {}
				child { node {\(σ\)}
					child { node {\(β\)}
						edge from parent node[left, visible=<2->] {\(\nt{B}\)}
					}
					child { node {\(β\)}
						edge from parent node[right, visible=<2->] {\(\nt{B}\)}
					}
					edge from parent node[right, visible=<2->] {\(\nt{D}\)}
					edge from parent [draw=none]
				};
			\node[left=1em of t-1] {\(t_2\colon\)};
		\end{tikzpicture}
		\end{center}
	\onslide<5->
		\begin{align*}
			G_1 & = (N_1, Σ, I_1, Δ_1)
		\\[1.5em]
			N_1 & = \{\mhl<19->[fill=HKS44K60]{\nt{S}}, \mhl<17->{\nt{A}}, \mhl<17->{\nt{B}}, \mhl<5,19->[onslide={<5>fill=HKS65K60}, onslide={<19->fill=HKS44K60}]{\nt{C}}\}
		\\
			Σ & = \{σ^{(2)}, α^{(0)}, β^{(0)}\}
		\\
			I_1 & = \{\mhl<19->[fill=HKS44K60]{\nt{S}}, \mhl<5,19->[onslide={<5>fill=HKS65K60}, onslide={<19->fill=HKS44K60}]{\nt{C}}\}
		\end{align*}
		\begin{align*}
			Δ_1\colon
			\mhl<19->[fill=HKS44K60]{\nt{S}} & → σ(\mhl<17->{\nt{A}}, \mhl<5,19->[onslide={<5>fill=HKS65K60}, onslide={<19->fill=HKS44K60}]{\nt{C}})
		\\
			\mhl<5,19->[onslide={<5>fill=HKS65K60}, onslide={<19->fill=HKS44K60}]{\nt{C}} & → σ(\mhl<17->{\nt{B}}, \mhl<5,19->[onslide={<5>fill=HKS65K60}, onslide={<19->fill=HKS44K60}]{\nt{C}})
		\\
			\mhl<5,19->[onslide={<5>fill=HKS65K60}, onslide={<19->fill=HKS44K60}]{\nt{C}} & → σ(\mhl<17->{\nt{B}}, \mhl<17->{\nt{B}})
		\\
			\mhl<17->{\nt{A}} & → α
		\\
			\mhl<17->{\nt{B}} & → β
		\end{align*}
	\end{overprint}
\column{0.5\linewidth}
	\begin{overprint}
	\onslide<2>
		\begin{center}
		\[
			\nt{S} = \statetree{
			\node {\(σ\)}
				child { node {\(α\)}
				}
				child { node {\(σ\)}
					child { node {\(β\)}
					}
					child { node {\(σ\)}
						child { node {\(β\)}
						}
						child { node {\(β\)}
						}
					}
				};
			}
		\]
		\[
			\nt{C} = \statetree{
			\node {\(σ\)}
				child { node {\(β\)}
				}
				child { node {\(σ\)}
					child { node {\(β\)}
					}
					child { node {\(β\)}
					}
				};
			}
		\]
		\[
			\nt{D} = \statetree{
			\node {\(σ\)}
				child { node {\(β\)}
				}
				child { node {\(β\)}
				};
			}
		\]
		\[
			\nt{A} = \tikz[baseline=(n.base)]{\node[draw=nt, rounded corners] (n) {\(α\)};}
			\qquad
			\nt{B} = \tikz[baseline=(n.base)]{\node[draw=nt, rounded corners] (n) {\(β\)};}
		\]
		\end{center}
	\onslide<3-5>
		\begin{align*}
			G_0 & = (N_0, Σ, I_0, Δ_0)
		\\[1.5em]
			N_0 & = \{\nt{S}, \nt{A}, \nt{B}, \mhl<4->[fill=HKS65K60]{\nt{C}}, \mhl<4->[fill=HKS65K60]{\nt{D}}\}
		\\
			Σ & = \{σ^{(2)}, α^{(0)}, β^{(0)}\}
		\\
			I_0 & = \{\nt{S}, \mhl<4->[fill=HKS65K60]{\nt{D}}\}
		\end{align*}
		\begin{align*}
			Δ_0\colon
			\nt{S} & → σ(\nt{A}, \mhl<4->[fill=HKS65K60]{\nt{C}})
		\\
			\mhl<4->[fill=HKS65K60]{\nt{C}} & → σ(\nt{B}, \mhl<4->[fill=HKS65K60]{\nt{D}})
		\\
			\mhl<4->[fill=HKS65K60]{\nt{D}} & → σ(\nt{B}, \nt{B})
		\\	\nt{A} & → α
		\\	\nt{B} & → β
		\end{align*}
	\onslide<6>
		\begin{block}{regular tree grammar (rtg)}
			tuple \(G = (N, Σ, I, Δ)\) where
			\begin{itemize}
			\item
				\(N\) alphabet \hfill (\emph{non-terminals})
			\item
				\(Σ\) ranked alphabet \hfill (\emph{terminals})
			\item
				\(I ⊆ N\) alphabet \hfill (\emph{initial non-t.})
			\item
				\(Δ\) is a finite set of \emph{rules} of form \(A_0 → σ(A_1, \dots, A_k)\) where \(k ∈ ℕ\), \(σ ∈ Σ^{(k)}\), \(A_i ∈ N\).
			\end{itemize}
		\end{block}
	\onslide<7-15>
		\begin{center}
		\begin{tikzpicture}[anchor=base, level distance=3em]
			\node (t) {}
				child[onslide={<8>opacity=0.25}] { node {\(σ\)}
					child[onslide={<8-9>opacity=0.25}] { node {\(α\)}
						edge from parent node[left, visible=<9->] {\(\nt{A}\)}
					}
					child[onslide={<8-10>opacity=0.25}] { node {\(σ\)}
						child[onslide={<8-11>opacity=0.25}] { node {\(β\)}
							edge from parent node[left, visible=<11->] {\(\nt{B}\)}
						}
						child[onslide={<8-12>opacity=0.25}] { node {\(σ\)}
							child[onslide={<8-13>opacity=0.25}] { node {\(β\)}
								edge from parent node[left, visible=<13->] {\(\nt{B}\)}
							}
							child[onslide={<8-14>opacity=0.25}] { node {\(β\)}
								edge from parent node[right, visible=<13->] {\(\nt{B}\)}
							}
							edge from parent node[right, visible=<11->] {\(\nt{C}\)}
						}
						edge from parent node[right, visible=<9->] {\(\nt{C}\)}
					}
					edge from parent node[right, visible=<8->] {\(\nt{S}\)}
					edge from parent [draw=none]
				};
				\node[left=1em of t-1] {\(t_1\colon\)};
		\end{tikzpicture}
		\end{center}
	\onslide<16>
		\begin{block}{bottom-up deterministic rtg}
			\centering
			\((N, Σ, I, Δ)\) is bottom-up deterministic

			\emph{if}

			for every \(σ(A_1, \dots, A_k)\) there is at most one \(A_0\) such that \(A_0 → σ(A_1, \dots, A_k) ∈ Δ\)
		\end{block}
		\begin{flushright}
			\(\implies\) there is at most one derivation for a tree
		\end{flushright}
% 	\onslide<18-19>
% 		\begin{align*}
% 			G_2 & = (N_2, Σ, I_2, Δ_2)
% 		\\[1.5em]
% 			N_2 & = \{S, \mhl<18->{A}, C\}
% 		\\
% 			Σ & = \{σ^{(2)}, α^{(0)}, β^{(0)}\}
% 		\\
% 			I_2 & = \{S, C\}
% 		\end{align*}
% 		\begin{align*}
% 			Δ_2\colon
% 			\mhl<1>{S} & → σ(\mhl<18->{A}, C)
% 		\\	\mhl<1>{C} & → σ(\mhl<18->{A}, C)
% 		\\	\mhl<1>{C} & → σ(\mhl<18->{A}, \mhl<18->{A})
% 		\\	\mhl<18->{A} & → α
% 		\\	\mhl<18->{A} & → β
% 		\end{align*}
	\onslide<18->
		\begin{align*}
			G_{\alt<20->32} & = (N_{\alt<20->32}, Σ, I_{\alt<20->32}, Δ_{\alt<20->32})
		\\[1.5em]
			N_{\alt<20->32} & = \{\mhl<20->[fill=HKS44K60]{\nt{S}}, \mhl<18->{\nt{A}}\only<-19>{, \nt{C}}\}
		\\
			Σ & = \{σ^{(2)}, α^{(0)}, β^{(0)}\}
		\\
			I_{\alt<20->32} & = \{\mhl<20->[fill=HKS44K60]{\nt{S}}\only<-19>{, \nt{C}}\}
		\end{align*}
		\begin{align*}
			Δ_{\alt<20->32}\colon
			\mhl<20->[fill=HKS44K60]{\nt{S}} & → σ(\mhl<18->{\nt{A}}, \mhl<20->[fill=HKS44K60]{\nt{\alt<20->SC}})
		\\	\uncover<-19>{\mhl<1>{\nt{C}}} & \uncover<-19>{{} → σ(\mhl<18->{\nt{A}}, \mhl<1>{\nt{C}})}
		\\	\mhl<20->[fill=HKS44K60]{\nt{\alt<20->SC}} & → σ(\mhl<18->{\nt{A}}, \mhl<18->{\nt{A}})
		\\	\mhl<18->{\nt{A}} & → α
		\\	\mhl<18->{\nt{A}} & → β
		\end{align*}
	\end{overprint}
\end{columns}%
\only<5>{%
	\tikz[overlay, shift=(current page.center), xshift=-1em, yshift=8.5em]{
		\node[single arrow, fill=HKS41K20, shape border rotate=180] {merge \(\nt{C}\), \(\nt{D}\)};
	}%
}%
\only<18-19>{%
	\tikz[overlay, shift=(current page.center), xshift=-1em, yshift=8.5em]{
		\node[single arrow, fill=HKS41K20] {merge \(\nt{A}\), \(\nt{B}\)};
	}%
}%
\only<20>{%
	\tikz[overlay, shift=(current page.center), xshift=-1em, yshift=8.5em]{
		\node[single arrow, fill=HKS41K20, align=right] {merge \(\nt{A}\), \(\nt{B}\)\\and \(\nt{S}\), \(\nt{C}\)};
	}%
}%
\end{standaloneframe}
\end{document}
%kate: default-dictionary en

\end{frame}


\section{Merging}

\againframe<4-5,17->{frame:ta}


\begin{frame}<1>[label=frame:weighted]{\secname}
	\begin{columns}[T]
\column{0.5\linewidth}
	\begin{overprint}
	\onslide<1->
		\newcommand{\tikzremember}[2]{\tikz[remember picture, baseline=(#1.base)]{\node[inner sep=0.1em] (#1) {#2};}}
		\begin{align*}
			P_1 & = (G_1, ι_1, δ_1)
		\end{align*}
		\begin{align*}
			ι_1\colon
			\mhl<2->[fill=HKS44K10]{\nt{S}} & ↦ \tikzremember{c1}{\(1\)} / 2
		\\	\mhl<2->[fill=HKS44K10]{\nt{C}} & ↦ \tikzremember{c2}{\(1\)} / 2
		\end{align*}
		\begin{alignat*}{2}
			δ_1\colon
			\mhl<2->[fill=HKS44K10]{\nt{S}} & → σ(\mhl<2->[fill=HKS07K10]{\nt{A}}, \mhl<2->[fill=HKS44K10]{\nt{C}})
			&& ↦ \tikzremember{c3}{\(1\)}/1
		\\
			\mhl<2->[fill=HKS44K10]{\nt{C}} & → σ(\mhl<2->[fill=HKS07K10]{\nt{B}}, \mhl<2->[fill=HKS44K10]{\nt{C}})
			&& ↦ \tikzremember{c4}{\(1\)}/3
		\\
			\mhl<2->[fill=HKS44K10]{\nt{C}} & → σ(\mhl<2->[fill=HKS07K10]{\nt{B}}, \mhl<2->[fill=HKS07K10]{\nt{B}})
			&& ↦ 2/3
		\\
			\mhl<2->[fill=HKS07K10]{\nt{A}} & → α
			&& ↦ 1/1
		\\
			\mhl<2->[fill=HKS07K10]{\nt{B}} & → β
			&& ↦ 5/5
		\end{alignat*}
		\only<2->{%
			\begin{tikzpicture}
				[ overlay
				, remember picture
				, every node/.style={rounded corners, fill=HKS65K60, inner sep=0, blend mode=multiply}
				]
				\node[fit=(c1) (c2)] {};
				\node[fit=(c3) (c4)] {};
			\end{tikzpicture}
		}%
	\end{overprint}
\column{0.5\linewidth}
	\begin{overprint}
	\onslide<1>
		\begin{center}
		\begin{tikzpicture}[anchor=base, level distance=3em]
			\node (t) {}
				child { node {\(σ\)}
					child { node {\(α\)}
						edge from parent node[left] {\(\nt{A}\)}
					}
					child { node {\(σ\)}
						child { node {\(β\)}
							edge from parent node[left] {\(\nt{B}\)}
						}
						child { node {\(σ\)}
							child { node {\(β\)}
								edge from parent node[left] {\(\nt{B}\)}
							}
							child { node {\(β\)}
								edge from parent node[right] {\(\nt{B}\)}
							}
							edge from parent node[right] {\(\nt{C}\)}
						}
						edge from parent node[right] {\(\nt{C}\)}
					}
					edge from parent node[right] {\(\nt{S}\)}
					edge from parent [draw=none]
				};
			\node[left=1em of t-1] {\(t_1\colon\)};
			%\node[right=1em of t-1] {\({} \mapsto 1\)};
		\end{tikzpicture}

		\begin{tikzpicture}[anchor=base, level distance=3em]
			\node (t) {}
				child { node {\(σ\)}
					child { node {\(β\)}
						edge from parent node[left] {\(\nt{B}\)}
					}
					child { node {\(β\)}
						edge from parent node[right] {\(\nt{B}\)}
					}
					edge from parent node[right] {\(\nt{C}\)}
					edge from parent [draw=none]
				};
			\node[left=1em of t-1] {\(t_2\colon\)};
			%\node[right=1em of t-1] {\({} \mapsto 1\)};
		\end{tikzpicture}
		\end{center}
	\onslide<2->
		\begin{align*}
			P_3 & = (G_3, ι_3, δ_3)
		\end{align*}
		\begin{align*}
			ι_3\colon
			\mhl<2->[fill=HKS44K10]{\nt{S}} & ↦ 2 / 2
		\\	\phantom{\mhl<1>{\nt{C}}}
		\end{align*}
		\begin{alignat*}{2}
			δ_3\colon
			\mhl<2->[fill=HKS44K10]{\nt{S}} & → σ(\mhl[fill=HKS07K10]<2->{\nt{A}}, \mhl<2->[fill=HKS44K10]{\nt{S}}) && ↦ 2/4
		\\	\phantom{\mhl<1>{\nt{C}}}
		\\	\mhl<2->[fill=HKS44K10]{\nt{S}} & → σ(\mhl[fill=HKS07K10]<2->{\nt{A}}, \mhl<2->[fill=HKS07K10]{\nt{A}}) && ↦ 2/4
		\\	\mhl<2->[fill=HKS07K10]{\nt{A}} & → α && ↦ 1/6
		\\	\mhl<2->[fill=HKS07K10]{\nt{A}} & → β && ↦ 5/6
		\end{alignat*}
	\end{overprint}
\end{columns}%
\only<2->{%
	\tikz[overlay, shift=(current page.center), xshift=-1em, yshift=10em]{
		\node[single arrow, fill=HKS41K20, align=right] {merge \(\nt{A}\), \(\nt{B}\)\\and \(\nt{S}\), \(\nt{C}\)};
	}%
}%

\end{frame}


\section{Merging and mle}

\againframe<2->{frame:weighted}


\section{Count-Based State Merging Algorithm}

\begin{frame}{\secname}
	\documentclass[beamer]{standalone}
% SPDX-License-Identifier: CC-BY-4.0 OR MIT-0
% Copyright 2018 Toni Dietze
%
\usefonttheme{professionalfonts}

% LuaLaTeX specific packages
\usepackage{fontspec}
	\defaultfontfeatures{Ligatures=TeX}
\usepackage{polyglossia}
	\setdefaultlanguage{english}
\usepackage{amsmath}  % has to be loaded before unicode-math
\usepackage[math-style=ISO]{unicode-math}
	\setmathfont{Latin Modern Math}
% 	\setmathfont[range={\mathcal,\mathbfcal},StylisticSet=1]{xits-math.otf}
% 	\setmathfont[range={"029F5}]{XITS Math}  % ⧵
% 	\setmathfont[range={\mathscr,\mathbfscr},StylisticSet=1]{Latin Modern Math}  % make \mathscr use the correct font

\usepackage[noend]{algpseudocode}
	\algrenewcommand\algorithmicrequire{\textbf{Input:}}
	\algrenewcommand\algorithmicensure{\textbf{Output:}}
\usepackage[backend=biber, maxbibnames=42, maxcitenames=42, sorting=ynt, style=authoryear]{biblatex}
\usepackage{csquotes}
\usepackage{mathtools}
\usepackage{media9}
\usepackage{scalerel}
\usepackage{standalone}
\usepackage{tikz}
	\usetikzlibrary{arrows.meta}
	\usetikzlibrary{backgrounds}
	\usetikzlibrary{calc}
	\usetikzlibrary{decorations}
	\usetikzlibrary{decorations.pathmorphing}
	\usetikzlibrary{decorations.pathreplacing}
	\usetikzlibrary{fadings}
	\usetikzlibrary{fit}
	\usetikzlibrary{graphs}
	\usetikzlibrary{graphdrawing}
	\usetikzlibrary{intersections}
	\usetikzlibrary{positioning}
	\usetikzlibrary{quotes}
	\usetikzlibrary{shadows.blur}
	\usetikzlibrary{shapes.arrows}
	\usetikzlibrary{shapes.geometric}
	\usegdlibrary{trees}
\usepackage{xifthen}
\usepackage{xspace}

\usepackage{pgfplots}
	\pgfplotsset
		{ compat = 1.15
		, /pgf/number format/1000 sep = {\,}
		, /pgf/number format/assume math mode = true
		, every axis plot/.append style =
			{ mark options = {fill opacity = 0.25}
			}
		}
	\usepgfplotslibrary{groupplots}
\usepackage{pgfplotstable}

\hypersetup
	{ bookmarksopen
	, pdflang = en
	, unicode
	}


%%%%%%%%%%%%%%%%%%%%%%%%%%%%%%%%%%%%%%%%%%%%%%%%%%%%%%%%%%%%%%%%%%%%%%%%%%%%%%


% always show bad boxes
%\overfullrule=1em


%%%%%%%%%%%%%%%%%%%%%%%%%%%%%%%%%%%%%%%%%%%%%%%%%%%%%%%%%%%%%%%%%%%%%%%%%%%%%%
% biblatex
%%%%%%%%%%%%%%%%%%%%%%%%%%%%%%%%%%%%%%%%%%%%%%%%%%%%%%%%%%%%%%%%%%%%%%%%%%%%%%

\addbibresource{slides-dissertation-defense.bib}
% \renewcommand*{\finalnamedelim}{\addcomma\space}
% \setlength{\bibitemsep}{1em}
% 
\AtEveryBibitem{% Clean up the bibtex rather than editing it
 \clearlist{address}
 \clearfield{date}
 \clearfield{eprint}
 \clearfield{isbn}
 \clearfield{issn}
 \clearlist{language}
 \clearlist{location}
 \clearfield{month}
 \clearfield{series}
%  \clearfield{url}
%  \clearfield{doi}
 \clearfield{organization}

%  \ifentrytype{book}{}{% Remove stuff except for books
%   \clearfield{booktitle}
%   \clearfield{pages}
  \clearlist{publisher}
  \clearname{editor}
%  }
}
% do not print url if doi is present
% http://tex.stackexchange.com/questions/154864/biblatex-use-doi-only-if-there-is-no-url
\DeclareSourcemap{
	\maps[datatype=bibtex]{
		\map{
			\step[fieldsource=doi,final]
			\step[fieldset=url,null]
}	}	}
%
% remove qoutes around titles
\DeclareFieldFormat
	[article,inbook,incollection,inproceedings,patent,thesis,unpublished]
	{title}{#1\isdot}
% 
% \DeclareFieldFormat{url}{\mkbibacro{URL}\addcolon\addnbspace\url{#1}}
% 
% \DeclareNameAlias{sortname}{first-last}
% 
\renewbibmacro{in:}{\ifentrytype{article}{}{}}


%%%%%%%%%%%%%%%%%%%%%%%%%%%%%%%%%%%%%%%%%%%%%%%%%%%%%%%%%%%%%%%%%%%%%%%%%%%%%%
% beamer
%%%%%%%%%%%%%%%%%%%%%%%%%%%%%%%%%%%%%%%%%%%%%%%%%%%%%%%%%%%%%%%%%%%%%%%%%%%%%%

\useoutertheme{infolines}
\makeatletter
% based on
% /usr/share/texmf-dist/tex/latex/beamer/beamerouterthemeinfolines.sty
\setbeamertemplate{footline}
{%
	\leavevmode%
	\hbox{%
	\begin{beamercolorbox}[wd=.333333\paperwidth,ht=2.25ex,dp=1ex,center]{author in head/foot}%
		\usebeamerfont{author in head/foot}\insertshortauthor\expandafter\beamer@ifempty\expandafter{\beamer@shortinstitute}{}{~~(\insertshortinstitute)}
	\end{beamercolorbox}%
	\begin{beamercolorbox}[wd=.333333\paperwidth,ht=2.25ex,dp=1ex,center]{title in head/foot}%
		\usebeamerfont{title in head/foot}\insertshorttitle
	\end{beamercolorbox}%
	\begin{beamercolorbox}[wd=.333333\paperwidth,ht=2.25ex,dp=1ex,right]{date in head/foot}%
		\usebeamerfont{date in head/foot}%
		\hfill\insertshortdate\hfill\hfill%
		%\hspace*{2ex}%
		%\insertshortdate%
		%\hspace{0pt plus 1 filll}%
		%(\insertframenumber.\insertoverlaynumber{} / \insertmainframenumber)%
		%\hspace{0pt plus 1 filll}%
		\phantom{000}\llap{\insertpagenumber} / \insertpresentationendpage%
		\hspace*{2ex}%
	\end{beamercolorbox}}%
	\vskip0pt%
}
\makeatother
\useinnertheme{circles}
\beamertemplatenavigationsymbolsempty
\setbeamertemplate{bibliography item}{}
\setbeamertemplate{headline}[default]

\input{tudcolors.tex}
\setbeamercolor*{alerted text}{fg=HKS07K100}
\usecolortheme[named=HKS41K100]{structure}

\setbeamercolor*{palette primary}{use=structure,fg=white,bg=structure.fg}
\setbeamercolor*{palette secondary}{use=structure,fg=white,bg=structure.fg!80}
\setbeamercolor*{palette tertiary}{use=structure,fg=white,bg=structure.fg!60}
\setbeamercolor*{palette quaternary}{fg=white,bg=black}

\setbeamercolor*{sidebar}{use=structure,bg=structure.fg}

\setbeamercolor*{palette sidebar primary}{use=structure,fg=structure.fg!20}
\setbeamercolor*{palette sidebar secondary}{fg=white}
\setbeamercolor*{palette sidebar tertiary}{use=structure,fg=structure.fg!40}
\setbeamercolor*{palette sidebar quaternary}{fg=white}

\setbeamercolor*{titlelike}{parent=palette primary}

\setbeamercolor*{separation line}{}
\setbeamercolor*{fine separation line}{}

\setbeamercolor{block title}{use=structure,fg=white,bg=structure.fg}
\setbeamercolor{block title alerted}{use=alerted text,fg=white,bg=alerted text.fg!75!black}
\setbeamercolor{block title example}{use=example text,fg=white,bg=example text.fg!75!black}

\setbeamercolor{block body}{parent=normal text,use=block title,bg=block title.bg!10!bg}
\setbeamercolor{block body alerted}{parent=normal text,use=block title alerted,bg=block title alerted.bg!10!bg}
\setbeamercolor{block body example}{parent=normal text,use=block title example,bg=block title example.bg!10!bg}

% \setbeamertemplate{itemize items}[default]


%%%%%%%%%%%%%%%%%%%%%%%%%%%%%%%%%%%%%%%%%%%%%%%%%%%%%%%%%%%%%%%%%%%%%%%%%%%%%%
% TikZ
%%%%%%%%%%%%%%%%%%%%%%%%%%%%%%%%%%%%%%%%%%%%%%%%%%%%%%%%%%%%%%%%%%%%%%%%%%%%%%

\tikzset
	{ > = Stealth
	}


%%%%%%%%%%%%%%%%%%%%%%%%%%%%%%%%%%%%%%%%%%%%%%%%%%%%%%%%%%%%%%%%%%%%%%%%%%%%%%
% general commands and styles
%%%%%%%%%%%%%%%%%%%%%%%%%%%%%%%%%%%%%%%%%%%%%%%%%%%%%%%%%%%%%%%%%%%%%%%%%%%%%%

% \delegateStyle and \inheritStyle command
% usage: \delegateStyle{… \inheritStyle{…} …}
% example: \(X_{\delegateStyle{\fbox{\inheritStyle{X}}}}\)
% Save the current style and regain it in the argument.
% This works both for math and text mode, and can be nested.
% Acknowledgments: Based on \ThisStyle and \SavedStyle from scalerel package.
\makeatletter
\newcommand*{\@inheritStyle@D}[1]{\(\displaystyle      #1\)}
\newcommand*{\@inheritStyle@T}[1]{\(\textstyle         #1\)}
\newcommand*{\@inheritStyle@S}[1]{\(\scriptstyle       #1\)}
\newcommand*{\@inheritStyle@s}[1]{\(\scriptscriptstyle #1\)}
\newcommand*{\@inheritStyle@t}[1]{#1}
\newcommand*{\inheritStyle}{\csname @inheritStyle@\@inheritStyleSwitch\endcsname}
\newcommand*{\delegateStyle}[1]{%
	\ifmmode%
		\mathchoice%
		{\edef\@inheritStyleSwitch{D}#1}%
		{\edef\@inheritStyleSwitch{T}#1}%
		{\edef\@inheritStyleSwitch{S}#1}%
		{\edef\@inheritStyleSwitch{s}#1}%
	\else%
		\edef\@inheritStyleSwitch{t}#1%
	\fi%
}
\makeatother


% \oalt command
% requires: \delegateStyle and \inheritStyle command
% usage: \oalt<…>[…]{…}{…} (cf. \alt)
% Behaves like \alt, but reserves space according to largest overlays.
% The optional argument defines the alignment inside the reserved space;
% it is one of c, l, r, s (cf. \makebox); the default is c.
\makeatletter
\newlength{\oalt@dp}
\newlength{\oalt@ht}
\newlength{\oalt@wd}
\newbox{\oalt@a}
\newbox{\oalt@b}
\newbox{\oalt@empty}
\newcommand<>*{\oalt}[3][c]{%
	\delegateStyle{%
		% based on \setto… in /usr/share/texmf-dist/tex/latex/base/latex.ltx
		\setbox\oalt@a\hbox{\inheritStyle{#2}}%
		\setbox\oalt@b\hbox{\inheritStyle{#3}}%
		\pgfmathsetlength{\oalt@dp}{max(\dp\oalt@a,\dp\oalt@b)}%
		\pgfmathsetlength{\oalt@ht}{max(\ht\oalt@a,\ht\oalt@b)}%
		\pgfmathsetlength{\oalt@wd}{max(\wd\oalt@a,\wd\oalt@b)}%
		\raisebox{0pt}[\oalt@ht][\oalt@dp]{%
			\makebox[\oalt@wd][#1]{%
				\alt#4{\unhbox\oalt@a}{\unhbox\oalt@b}%
			}%
		}%
		\setbox\oalt@a\box\oalt@empty%
		\setbox\oalt@b\box\oalt@empty%
	}%
}
\makeatother


% \otemporal command
% requires: \delegateStyle and \inheritStyle command
% usage: \otemporal<…>[…]{…}{…}{…} (cf. \temporal)
% Behaves like \temporal, but reserves space according to largest overlays.
% The optional argument defines the alignment inside the reserved space;
% it is one of c, l, r, s (cf. \makebox); the default is c.
\makeatletter
\newlength{\ot@dp}
\newlength{\ot@ht}
\newlength{\ot@wd}
\newbox{\ot@a}
\newbox{\ot@b}
\newbox{\ot@c}
\newbox{\ot@empty}
\newcommand<>*{\otemporal}[4][c]{%
	\delegateStyle{%
		% based on \setto… in /usr/share/texmf-dist/tex/latex/base/latex.ltx
		\setbox\ot@a\hbox{\inheritStyle{#2}}%
		\setbox\ot@b\hbox{\inheritStyle{#3}}%
		\setbox\ot@c\hbox{\inheritStyle{#4}}%
		\pgfmathsetlength{\ot@dp}{max(\dp\ot@a,\dp\ot@b,\dp\ot@c)}%
		\pgfmathsetlength{\ot@ht}{max(\ht\ot@a,\ht\ot@b,\ht\ot@c)}%
		\pgfmathsetlength{\ot@wd}{max(\wd\ot@a,\wd\ot@b,\wd\ot@c)}%
		\raisebox{0pt}[\ot@ht][\ot@dp]{%
			\makebox[\ot@wd][#1]{%
				\temporal#5{\unhbox\ot@a}{\unhbox\ot@b}{\unhbox\ot@c}%
			}%
		}%
		\setbox\ot@a\box\ot@empty%
		\setbox\ot@b\box\ot@empty%
		\setbox\ot@c\box\ot@empty%
	}%
}
\makeatother


% Resize delimiters like braces, brackets, etc.
% Parameters: size, left delimiter, formula, right delimiter
% Example: \delim2({\frac{1}{2}})
\newcommand*{\delim}[4]{%
	\ifcase#1%
		#2#3#4%
	\or%
		\bigl#2#3\bigr#4%
	\or%
		\Bigl#2#3\Bigr#4%
	\or%
		\biggl#2#3\biggr#4%
	\or%
		\Biggl#2#3\Biggr#4%
	\else%
		\left#2#3\right#4%
	\fi%
}


% similar to \fullcite, but using the formatting of \printbibliography
\newcommand*{\printfullcite}[1]{%
	\begin{refsection}%
		\nocite{#1}%
		\DeclareNameAlias{author}{first-last}%
		\printbibliography[heading = none]%
	\end{refsection}%
}


\colorlet{light alert}{HKS07K60}
\tikzset{alert.bg/.style={rounded corners, fill=light alert}}
\tikzset{every picture/.style={line cap=round, semithick}}
% http://tex.stackexchange.com/questions/6135/how-to-make-beamer-overlays-with-tikz-node
\tikzset{onslide/.code args={<#1>#2}{\only<#1>{\pgfkeysalso{#2}}}}
\tikzset{invisible/.code args={<#1>}{\alt<#1>{\pgfkeysalso{transparent}}{\pgfkeysalso{opaque}}}}
\tikzset{uncover/.code args={<#1>}{\alt<#1>{\pgfkeysalso{opaque}}{\pgfkeysalso{opacity=0.25}}}}
\tikzset{visible/.code args={<#1>}{\alt<#1>{\pgfkeysalso{opaque}}{\pgfkeysalso{transparent}}}}
\tikzset{vuncover/.code args=%
	{<#1><#2>}%
	{\alt<#1>%
		{\alt<#2>%
			{\pgfkeysalso{opaque}}%
			{\pgfkeysalso{opacity=0.25}}%
		}{\pgfkeysalso{transparent}}%
	}%
}

\newcommand<%
	>{\tikzhighlight}[2][]{%
	\delegateStyle{\alt#3%
		{\tikz[baseline=0, anchor=base, inner sep=0.2em, text height=, text depth=]{\node[alert.bg, #1]{\inheritStyle{#2}};}}%
		{\tikz[baseline=0, anchor=base, inner sep=0.2em, text height=, text depth=]{\node[#1, fill=none]{\inheritStyle{#2}};}}%
	}%
}

\newcommand{\mathhighlight}{\tikzhighlight}

\newcommand<>{\mhl}[2][]{\mathhighlight#3[inner sep=0.2em, #1]{#2}}


\newcommand<>{\inlineblock}[2][]{{%
	\usebeamercolor*[fg]{block body}%
	\tikzhighlight#3[fill=block body.bg, #1]{#2}%
}}


% a small letter s for plurals of abbreviations
\newcommand*{\s}{{\scriptsize s}\xspace}


\newcommand<>*{\sout}[2][opacity=0.75, ultra thick]{%
	\delegateStyle{%
		\tikz[baseline=0, anchor=base, inner sep=0, outer sep=0]{
			\useasboundingbox node (n) {\inheritStyle{#2}};
			\only#3{
				\node (h) {\inheritStyle{\ifmmode\mathstrut\else\strut\fi}};
				\draw[#1] (n.west |- {$(h.south)!0.5!(h.north)$}) -- (n.east |- {$(h.south)!0.5!(h.north)$});
			}
		}%
	}%
}


% tight style
% Sets outer sep to default inner sep and inner sep to 0.
% Use this style for nodes that are neither drawn nor filled to prevent
% unwanted growth of the bounding box.
\tikzset{tight/.style={inner sep=0, outer sep=0.3333em}}


% rounded tree edges style
% usage: rounded tree edges={⟨direction⟩}{⟨looseness⟩}{⟨strength⟩}
\tikzset{
	rounded tree edges/.style n args={3}{
	edge from parent path={
	let
		\n{direction}={#1},
		\n{looseness}={#2},
		\n{strength}={#3},
		\p1=(\tikzparentnode),
		\p2=(\tikzchildnode),
		\p3=(\n{direction}:1pt),
		\p4=(\x2 - \x1, \y2 - \y1),
		\n{dist}={veclen(\p4)},
		\p4=(\x4 / \n{dist}, \y4 / \n{dist}),
		\n{angle}={atan2(\y4, \x4)},
		\n{delta}={Mod(\n{angle} - \n{direction}, 360)},
		\n{delta}={\n{delta} > 180 ? \n{delta} - 360  : \n{delta}},
		\n{delta}={\n{delta} >  90 ?  180 - \n{delta} : \n{delta}},
		\n{delta}={\n{delta} < -90 ? -180 - \n{delta} : \n{delta}}
	in (\tikzparentnode) .. controls
		+(    \n{angle}+\n{strength}*\n{delta}:\n{looseness}*0.3915*\n{dist}) and
		+(180+\n{angle}-\n{strength}*\n{delta}:\n{looseness}*0.3915*\n{dist}) ..
		(\tikzchildnode)
	}
	}
}


% Tear out snippets from PDFs.
% Usage: \tear[…]{file.pdf}
% The optional parameter is the same as for \includegraphics.
% Useful Arguments:
%   * page=‹pagenumber›
%   * trim=‹left› ‹bottom› ‹right› ‹top›
%   * width=0.98\linewidth
\newcommand*{\tear}[2][]{%
	\begin{tikzpicture}
		\node
			[ blur shadow
			, clip
			, decorate
			, decoration=random steps
			, draw
			, inner sep=0
			, preaction={fill=white}% hide the shadow if paper is transparent
			] {\includegraphics[#1]{#2}};
	\end{tikzpicture}%
}


\makeatletter
\newcommand*{\timeline}[3][0]{%
	\ifcsname timeline@cmd@#3\endcsname%
		\@timeline[#1]{#2}{#3}%
		\PackageWarning{timeline}{redefining timeline \@backslashchar\string#3}%
	\else%
		\ifcsname#3\endcsname%
			\errmessage{Command \@backslashchar\string#3 already defined}%
		\else%
			\@timeline[#1]{#2}{#3}%
		\fi%
	\fi%
}%
\newcommand*{\@timeline}[3][0]{%
	% mark command as timeline command – they can be overwritten
	\expandafter\def\csname timeline@cmd@#3\endcsname{}%
	\setcounter{@timeline}{#1}%
	\def\timeline@cmd{#3}%
	\timeline@reset%
	\timeline@append{0}%
	\@tfor\timeline@next:=#2\do{%
		\if\timeline@next+%
			\stepcounter{@timeline}%
			\timeline@append{,\the@timeline}%
		\else\if\timeline@next-%
			\stepcounter{@timeline}%
		\else%
			%\timeline@append{\timeline@next}%
			\GenericError{}{\protect\timeline: ignoring unknown character: \timeline@next}%
		\fi\fi%
	}%
}%
% \newcommand*{\tl}[1]{%
% 	\ifcsname timeline@cmd@#1\endcsname%
% 		\csname timeline@cmd@#1\endcsname%
% 	\else%
% 		0%
% 		%\GenericError{}{\protect\tl: timeline not defined: #1}%
% 	\fi%
% }%
\newcounter{@timeline}%
\def\timeline@reset{%
	\expandafter\def\csname\timeline@cmd\endcsname{}%
}%
\def\timeline@append#1{%
	\expandafter\edef\csname\timeline@cmd\endcsname{%
		\csname\timeline@cmd\endcsname#1%
	}%
}%
\makeatother


\newcommand*{\xminus}[1]{%
	\mathrel{\tikz[baseline={([yshift=-0.25em]n.south)}, inner sep=0, outer sep=0.2em]{%
		\node (n) {\(\scriptstyle #1\)};
		\draw (n.south west) -- (n.south east);
	}}%
}
\newcommand*{\tikzrightarrow}[1]{%
	\mathrel{\tikz[baseline={([yshift=-0.25em]n.south)}, inner sep=0, outer sep=0.2em]{%
		\node (n) {\(\scriptstyle #1\)};
		\draw[->, > = Computer Modern Rightarrow, line width = 0.4pt] (n.south west) -- (n.south east);
	}}%
}


%%%%%%%%%%%%%%%%%%%%%%%%%%%%%%%%%%%%%%%%%%%%%%%%%%%%%%%%%%%%%%%%%%%%%%%%%%%%%%
% document specific commands
%%%%%%%%%%%%%%%%%%%%%%%%%%%%%%%%%%%%%%%%%%%%%%%%%%%%%%%%%%%%%%%%%%%%%%%%%%%%%%

\newcommand<>*{\mycite}[1]{\uncover#2{{\color{HKS57K100}[\cite{#1}]}}}


\newcommand{\statetree}[1]{
	\tikz
	[ anchor=base
	, baseline=(current bounding box.center)
	, level distance=2em
	, sibling distance=2em
	]{
		\matrix
		[ draw=nt
		, edge from parent/.style={draw=black}
		, inner sep=0
		, nodes={inner sep=0.2em, rounded corners=0}
		, rounded corners
		] {#1\\}
	}
}


\newcommand*{\mylargeleaf}[1]{{\LARGE\color{HKS41K70}#1}}

\definecolor{state s}{named}{HKS57K80}
\definecolor{state t}{named}{HKS41K70}
\newcommand*{\stateS}[1]{{\color{state s}#1}}
\newcommand*{\stateT}[1]{{\color{state t}#1}}

\tikzset{
	subtree/.style =
		{ fill=lightgray
		, inner sep=0.02em
		, isosceles triangle apex angle=60
		, shape=isosceles triangle
		, shape border rotate=90
		}
	, state/.style = {circle, draw, inner sep=0.1em}
	, trans/.style = {rectangle, draw}
}

\newcommand*{\srBool}{\mathbb{B}}
\newcommand*{\srProb}{ℙ}


%%%%%%%%%%%%%%%%%%%%%%%%%%%%%%%%%%%%%%%%%%%%%%%%%%%%%%%%%%%%%%%%%%%%%%%%%%%%%%
% commands for specific notations
%%%%%%%%%%%%%%%%%%%%%%%%%%%%%%%%%%%%%%%%%%%%%%%%%%%%%%%%%%%%%%%%%%%%%%%%%%%%%%

\DeclareMathOperator*{\argmax}{argmax}

\newcommand*{\cardinality}[1]{\lvert#1\rvert}
\newcommand*{\corpussize}[1]{\lvert#1\rvert}

\DeclareMathOperator{\crispOp}{crisp}
\newcommand*        {\crisp}[2][0]{\crispOp\delim{#1}({#2})}

\DeclareMathOperator{\lhsOp}{lhs}
\newcommand*{\lhs}[1]{\lhsOp(#1)}

\DeclareMathOperator{\lklhdOp}{L}
\newcommand*{\lklhd}[2]{\lklhdOp(#1 ∣ #2)}

\DeclareMathOperator{\mleOp}{mle}
\newcommand*{\mle}[2][]{%
	\ifthenelse{\isempty{#1}}{%
		\mleOp(#2)%
	}{%
		\mleOp_{#1}(#2)%
	}%
}

\DeclareMathOperator{\mrg}{merge}

% CVD: color vision deficiencies
\definecolor{CVD light red}   {HTML}{FF8080}
\definecolor{CVD light yellow}{HTML}{FFFF80}
\definecolor{CVD light green} {HTML}{40FFC0}

\definecolor{nt}{named}{HKS41K70}
\newcommand*{\nt}[1]{{\color{nt}#1}}

% set of all probability distributions over #1
\DeclareMathOperator{\pdsOp}{Pd}
\newcommand*{\pds}[1]{\pdsOp(#1)}

\DeclareMathOperator{\positionsOp}{pos}
\newcommand*{\positions}[1]{\positionsOp(#1)}

\DeclareMathOperator{\rankOp}{rk}
\newcommand*{\rank}[1]{\rankOp(#1)}

\DeclareMathOperator{\runsOp}{run}
\newcommand*{\runs}[2][]{%
	\ifthenelse%
		{\isempty{#1}}%
		{\runsOp(#2)}%
		{\runsOp_{#1}(#2)}%
}

\newcommand*{\semantics}[1]{⟦#1⟧}

\DeclareMathOperator{\splt}{split}

\newcommand*{\subtree}[2]{#1|_{#2}}

\DeclareMathOperator{\supportOp}{supp}
\newcommand*{\support}[1]{\supportOp(#1)}

\newcommand*{\symId}{\textsc{\color{gray}Id}}
\newcommand*{\symCons}{\textsc{\color{gray}Cons}}
\newcommand*{\symFlip}{\textsc{\color{gray}Flip}}
\newcommand*{\symNull}{\textsc{\color{gray}Null}}
\newcommand*{\symNullR}{\textsc{\color{gray}N\(\overline{\textsc{ull}}\)}}
\newcommand*{\symSnoc}{\textsc{\color{gray}Snoc}}

\newcommand*{\transWTA}[4][]{#3 \xrightarrow{#1} #2(#4)}

\DeclareMathOperator{\uniqueRunOp}{r}
\newcommand*{\uniqueRun}[2][]{%
	\ifthenelse%
		{\isempty{#1}}%
		{\uniqueRunOp^{#2}}%
		{\uniqueRunOp_{\!#1}^{#2}}%
}

\DeclareMathOperator{\treesOp}{T}
\newcommand*{\trees}[2][]{%
	\ifthenelse%
		{\isempty{#1}}%
		{\treesOp_{\!#2}}%
		{\treesOp_{\!#2}(#1)}%
}
\DeclareMathOperator{\treesUOp}{U}
\newcommand*{\treesU}[2][]{%
	\ifthenelse%
		{\isempty{#1}}%
		{\treesUOp_{#2}}%
		{\treesUOp_{#2}(#1)}%
}


%%%%%%%%%%%%%%%%%%%%%%%%%%%%%%%%%%%%%%%%%%%%%%%%%%%%%%%%%%%%%%%%%%%%%%%%%%%%%%
% metadata
%%%%%%%%%%%%%%%%%%%%%%%%%%%%%%%%%%%%%%%%%%%%%%%%%%%%%%%%%%%%%%%%%%%%%%%%%%%%%%

\ifstandalonebeamer\else
	\title[Defense of Dissertation]{A Formal View on Training of Weighted Tree Automata by Likelihood-Driven State Splitting and Merging}
	\subtitle{Defense of Dissertation}
\fi
\author{Toni Dietze}
\institute[TU Dresden]{%
	\href{https://www.orchid.inf.tu-dresden.de/index.en/}{Chair for Foundations of Programming}
\\	\href{https://tu-dresden.de/ing/informatik/thi}{Institute of Theoretical Computer Science}
\\	\href{https://tu-dresden.de/ing/informatik}{Faculty of Computer Science}
\\	\href{https://tu-dresden.de/}{Technische Universität Dresden}
\\	01062 Dresden, Germany
}
\date[2018-09-27]{September 27, 2018}

\title{\jobname}
\begin{document}
\begin{standaloneframe}{\jobname}
	\begin{algorithmic}[1]
		\Require corpus \(c\) over \(\trees{Σ}\)
		\Ensure sequence of bottom-up deterministic pta \(ℳ_0, …, ℳ_n\), \(n ∈ ℕ\)
		\State \(ℳ_0 = (\mathscr{A}_0, ι_0, ρ_0) \gets \text{canonical pta of \(c\)}\)
		\State \(i \gets 0\)
		\While{there exists a non-trivial \(\mathscr{A}_i\)-merger}
			\State
				\(π \gets \Call{bestMerger}{\mathscr{A}_i, c}\)
			\State
				\(i \gets i + 1\)
			\State
				\(\mathscr{A}_i \gets π(\mathscr{A}_{i-1})\)
			\State
				\(ℳ_i \gets \mle[\mathscr{A}_i]{c}\)
		\EndWhile
	\end{algorithmic}
	\begin{overprint}
	\onslide<2>
		\begin{center}
			\emph{Note:} \(\mathcal{L}(\mathscr{A}_0) \subseteq \mathcal{L}(\mathscr{A}_1) \subseteq \mathcal{L}(\mathscr{A}_2) \subseteq \dots\)
		\end{center}
	\onslide<3>
		\begin{block}{compare with}
			\printfullcite{2001CarrascoOncinaCalera-Rubio}
		\end{block}
	\end{overprint}
\end{standaloneframe}
\end{document}

\end{frame}


\section{Choosing the best Merger}

\begin{frame}{\secname}
	% SPDX-License-Identifier: CC-BY-4.0
% Copyright 2018 Toni Dietze
\documentclass[beamer]{standalone}
% SPDX-License-Identifier: CC-BY-4.0 OR MIT-0
% Copyright 2018 Toni Dietze
%
\usefonttheme{professionalfonts}

% LuaLaTeX specific packages
\usepackage{fontspec}
	\defaultfontfeatures{Ligatures=TeX}
\usepackage{polyglossia}
	\setdefaultlanguage{english}
\usepackage{amsmath}  % has to be loaded before unicode-math
\usepackage[math-style=ISO]{unicode-math}
	\setmathfont{Latin Modern Math}
% 	\setmathfont[range={\mathcal,\mathbfcal},StylisticSet=1]{xits-math.otf}
% 	\setmathfont[range={"029F5}]{XITS Math}  % ⧵
% 	\setmathfont[range={\mathscr,\mathbfscr},StylisticSet=1]{Latin Modern Math}  % make \mathscr use the correct font

\usepackage[noend]{algpseudocode}
	\algrenewcommand\algorithmicrequire{\textbf{Input:}}
	\algrenewcommand\algorithmicensure{\textbf{Output:}}
\usepackage[backend=biber, maxbibnames=42, maxcitenames=42, sorting=ynt, style=authoryear]{biblatex}
\usepackage{csquotes}
\usepackage{mathtools}
\usepackage{media9}
\usepackage{scalerel}
\usepackage{standalone}
\usepackage{tikz}
	\usetikzlibrary{arrows.meta}
	\usetikzlibrary{backgrounds}
	\usetikzlibrary{calc}
	\usetikzlibrary{decorations}
	\usetikzlibrary{decorations.pathmorphing}
	\usetikzlibrary{decorations.pathreplacing}
	\usetikzlibrary{fadings}
	\usetikzlibrary{fit}
	\usetikzlibrary{graphs}
	\usetikzlibrary{graphdrawing}
	\usetikzlibrary{intersections}
	\usetikzlibrary{positioning}
	\usetikzlibrary{quotes}
	\usetikzlibrary{shadows.blur}
	\usetikzlibrary{shapes.arrows}
	\usetikzlibrary{shapes.geometric}
	\usegdlibrary{trees}
\usepackage{xifthen}
\usepackage{xspace}

\usepackage{pgfplots}
	\pgfplotsset
		{ compat = 1.15
		, /pgf/number format/1000 sep = {\,}
		, /pgf/number format/assume math mode = true
		, every axis plot/.append style =
			{ mark options = {fill opacity = 0.25}
			}
		}
	\usepgfplotslibrary{groupplots}
\usepackage{pgfplotstable}

\hypersetup
	{ bookmarksopen
	, pdflang = en
	, unicode
	}


%%%%%%%%%%%%%%%%%%%%%%%%%%%%%%%%%%%%%%%%%%%%%%%%%%%%%%%%%%%%%%%%%%%%%%%%%%%%%%


% always show bad boxes
%\overfullrule=1em


%%%%%%%%%%%%%%%%%%%%%%%%%%%%%%%%%%%%%%%%%%%%%%%%%%%%%%%%%%%%%%%%%%%%%%%%%%%%%%
% biblatex
%%%%%%%%%%%%%%%%%%%%%%%%%%%%%%%%%%%%%%%%%%%%%%%%%%%%%%%%%%%%%%%%%%%%%%%%%%%%%%

\addbibresource{slides-dissertation-defense.bib}
% \renewcommand*{\finalnamedelim}{\addcomma\space}
% \setlength{\bibitemsep}{1em}
% 
\AtEveryBibitem{% Clean up the bibtex rather than editing it
 \clearlist{address}
 \clearfield{date}
 \clearfield{eprint}
 \clearfield{isbn}
 \clearfield{issn}
 \clearlist{language}
 \clearlist{location}
 \clearfield{month}
 \clearfield{series}
%  \clearfield{url}
%  \clearfield{doi}
 \clearfield{organization}

%  \ifentrytype{book}{}{% Remove stuff except for books
%   \clearfield{booktitle}
%   \clearfield{pages}
  \clearlist{publisher}
  \clearname{editor}
%  }
}
% do not print url if doi is present
% http://tex.stackexchange.com/questions/154864/biblatex-use-doi-only-if-there-is-no-url
\DeclareSourcemap{
	\maps[datatype=bibtex]{
		\map{
			\step[fieldsource=doi,final]
			\step[fieldset=url,null]
}	}	}
%
% remove qoutes around titles
\DeclareFieldFormat
	[article,inbook,incollection,inproceedings,patent,thesis,unpublished]
	{title}{#1\isdot}
% 
% \DeclareFieldFormat{url}{\mkbibacro{URL}\addcolon\addnbspace\url{#1}}
% 
% \DeclareNameAlias{sortname}{first-last}
% 
\renewbibmacro{in:}{\ifentrytype{article}{}{}}


%%%%%%%%%%%%%%%%%%%%%%%%%%%%%%%%%%%%%%%%%%%%%%%%%%%%%%%%%%%%%%%%%%%%%%%%%%%%%%
% beamer
%%%%%%%%%%%%%%%%%%%%%%%%%%%%%%%%%%%%%%%%%%%%%%%%%%%%%%%%%%%%%%%%%%%%%%%%%%%%%%

\useoutertheme{infolines}
\makeatletter
% based on
% /usr/share/texmf-dist/tex/latex/beamer/beamerouterthemeinfolines.sty
\setbeamertemplate{footline}
{%
	\leavevmode%
	\hbox{%
	\begin{beamercolorbox}[wd=.333333\paperwidth,ht=2.25ex,dp=1ex,center]{author in head/foot}%
		\usebeamerfont{author in head/foot}\insertshortauthor\expandafter\beamer@ifempty\expandafter{\beamer@shortinstitute}{}{~~(\insertshortinstitute)}
	\end{beamercolorbox}%
	\begin{beamercolorbox}[wd=.333333\paperwidth,ht=2.25ex,dp=1ex,center]{title in head/foot}%
		\usebeamerfont{title in head/foot}\insertshorttitle
	\end{beamercolorbox}%
	\begin{beamercolorbox}[wd=.333333\paperwidth,ht=2.25ex,dp=1ex,right]{date in head/foot}%
		\usebeamerfont{date in head/foot}%
		\hfill\insertshortdate\hfill\hfill%
		%\hspace*{2ex}%
		%\insertshortdate%
		%\hspace{0pt plus 1 filll}%
		%(\insertframenumber.\insertoverlaynumber{} / \insertmainframenumber)%
		%\hspace{0pt plus 1 filll}%
		\phantom{000}\llap{\insertpagenumber} / \insertpresentationendpage%
		\hspace*{2ex}%
	\end{beamercolorbox}}%
	\vskip0pt%
}
\makeatother
\useinnertheme{circles}
\beamertemplatenavigationsymbolsempty
\setbeamertemplate{bibliography item}{}
\setbeamertemplate{headline}[default]

\input{tudcolors.tex}
\setbeamercolor*{alerted text}{fg=HKS07K100}
\usecolortheme[named=HKS41K100]{structure}

\setbeamercolor*{palette primary}{use=structure,fg=white,bg=structure.fg}
\setbeamercolor*{palette secondary}{use=structure,fg=white,bg=structure.fg!80}
\setbeamercolor*{palette tertiary}{use=structure,fg=white,bg=structure.fg!60}
\setbeamercolor*{palette quaternary}{fg=white,bg=black}

\setbeamercolor*{sidebar}{use=structure,bg=structure.fg}

\setbeamercolor*{palette sidebar primary}{use=structure,fg=structure.fg!20}
\setbeamercolor*{palette sidebar secondary}{fg=white}
\setbeamercolor*{palette sidebar tertiary}{use=structure,fg=structure.fg!40}
\setbeamercolor*{palette sidebar quaternary}{fg=white}

\setbeamercolor*{titlelike}{parent=palette primary}

\setbeamercolor*{separation line}{}
\setbeamercolor*{fine separation line}{}

\setbeamercolor{block title}{use=structure,fg=white,bg=structure.fg}
\setbeamercolor{block title alerted}{use=alerted text,fg=white,bg=alerted text.fg!75!black}
\setbeamercolor{block title example}{use=example text,fg=white,bg=example text.fg!75!black}

\setbeamercolor{block body}{parent=normal text,use=block title,bg=block title.bg!10!bg}
\setbeamercolor{block body alerted}{parent=normal text,use=block title alerted,bg=block title alerted.bg!10!bg}
\setbeamercolor{block body example}{parent=normal text,use=block title example,bg=block title example.bg!10!bg}

% \setbeamertemplate{itemize items}[default]


%%%%%%%%%%%%%%%%%%%%%%%%%%%%%%%%%%%%%%%%%%%%%%%%%%%%%%%%%%%%%%%%%%%%%%%%%%%%%%
% TikZ
%%%%%%%%%%%%%%%%%%%%%%%%%%%%%%%%%%%%%%%%%%%%%%%%%%%%%%%%%%%%%%%%%%%%%%%%%%%%%%

\tikzset
	{ > = Stealth
	}


%%%%%%%%%%%%%%%%%%%%%%%%%%%%%%%%%%%%%%%%%%%%%%%%%%%%%%%%%%%%%%%%%%%%%%%%%%%%%%
% general commands and styles
%%%%%%%%%%%%%%%%%%%%%%%%%%%%%%%%%%%%%%%%%%%%%%%%%%%%%%%%%%%%%%%%%%%%%%%%%%%%%%

% \delegateStyle and \inheritStyle command
% usage: \delegateStyle{… \inheritStyle{…} …}
% example: \(X_{\delegateStyle{\fbox{\inheritStyle{X}}}}\)
% Save the current style and regain it in the argument.
% This works both for math and text mode, and can be nested.
% Acknowledgments: Based on \ThisStyle and \SavedStyle from scalerel package.
\makeatletter
\newcommand*{\@inheritStyle@D}[1]{\(\displaystyle      #1\)}
\newcommand*{\@inheritStyle@T}[1]{\(\textstyle         #1\)}
\newcommand*{\@inheritStyle@S}[1]{\(\scriptstyle       #1\)}
\newcommand*{\@inheritStyle@s}[1]{\(\scriptscriptstyle #1\)}
\newcommand*{\@inheritStyle@t}[1]{#1}
\newcommand*{\inheritStyle}{\csname @inheritStyle@\@inheritStyleSwitch\endcsname}
\newcommand*{\delegateStyle}[1]{%
	\ifmmode%
		\mathchoice%
		{\edef\@inheritStyleSwitch{D}#1}%
		{\edef\@inheritStyleSwitch{T}#1}%
		{\edef\@inheritStyleSwitch{S}#1}%
		{\edef\@inheritStyleSwitch{s}#1}%
	\else%
		\edef\@inheritStyleSwitch{t}#1%
	\fi%
}
\makeatother


% \oalt command
% requires: \delegateStyle and \inheritStyle command
% usage: \oalt<…>[…]{…}{…} (cf. \alt)
% Behaves like \alt, but reserves space according to largest overlays.
% The optional argument defines the alignment inside the reserved space;
% it is one of c, l, r, s (cf. \makebox); the default is c.
\makeatletter
\newlength{\oalt@dp}
\newlength{\oalt@ht}
\newlength{\oalt@wd}
\newbox{\oalt@a}
\newbox{\oalt@b}
\newbox{\oalt@empty}
\newcommand<>*{\oalt}[3][c]{%
	\delegateStyle{%
		% based on \setto… in /usr/share/texmf-dist/tex/latex/base/latex.ltx
		\setbox\oalt@a\hbox{\inheritStyle{#2}}%
		\setbox\oalt@b\hbox{\inheritStyle{#3}}%
		\pgfmathsetlength{\oalt@dp}{max(\dp\oalt@a,\dp\oalt@b)}%
		\pgfmathsetlength{\oalt@ht}{max(\ht\oalt@a,\ht\oalt@b)}%
		\pgfmathsetlength{\oalt@wd}{max(\wd\oalt@a,\wd\oalt@b)}%
		\raisebox{0pt}[\oalt@ht][\oalt@dp]{%
			\makebox[\oalt@wd][#1]{%
				\alt#4{\unhbox\oalt@a}{\unhbox\oalt@b}%
			}%
		}%
		\setbox\oalt@a\box\oalt@empty%
		\setbox\oalt@b\box\oalt@empty%
	}%
}
\makeatother


% \otemporal command
% requires: \delegateStyle and \inheritStyle command
% usage: \otemporal<…>[…]{…}{…}{…} (cf. \temporal)
% Behaves like \temporal, but reserves space according to largest overlays.
% The optional argument defines the alignment inside the reserved space;
% it is one of c, l, r, s (cf. \makebox); the default is c.
\makeatletter
\newlength{\ot@dp}
\newlength{\ot@ht}
\newlength{\ot@wd}
\newbox{\ot@a}
\newbox{\ot@b}
\newbox{\ot@c}
\newbox{\ot@empty}
\newcommand<>*{\otemporal}[4][c]{%
	\delegateStyle{%
		% based on \setto… in /usr/share/texmf-dist/tex/latex/base/latex.ltx
		\setbox\ot@a\hbox{\inheritStyle{#2}}%
		\setbox\ot@b\hbox{\inheritStyle{#3}}%
		\setbox\ot@c\hbox{\inheritStyle{#4}}%
		\pgfmathsetlength{\ot@dp}{max(\dp\ot@a,\dp\ot@b,\dp\ot@c)}%
		\pgfmathsetlength{\ot@ht}{max(\ht\ot@a,\ht\ot@b,\ht\ot@c)}%
		\pgfmathsetlength{\ot@wd}{max(\wd\ot@a,\wd\ot@b,\wd\ot@c)}%
		\raisebox{0pt}[\ot@ht][\ot@dp]{%
			\makebox[\ot@wd][#1]{%
				\temporal#5{\unhbox\ot@a}{\unhbox\ot@b}{\unhbox\ot@c}%
			}%
		}%
		\setbox\ot@a\box\ot@empty%
		\setbox\ot@b\box\ot@empty%
		\setbox\ot@c\box\ot@empty%
	}%
}
\makeatother


% Resize delimiters like braces, brackets, etc.
% Parameters: size, left delimiter, formula, right delimiter
% Example: \delim2({\frac{1}{2}})
\newcommand*{\delim}[4]{%
	\ifcase#1%
		#2#3#4%
	\or%
		\bigl#2#3\bigr#4%
	\or%
		\Bigl#2#3\Bigr#4%
	\or%
		\biggl#2#3\biggr#4%
	\or%
		\Biggl#2#3\Biggr#4%
	\else%
		\left#2#3\right#4%
	\fi%
}


% similar to \fullcite, but using the formatting of \printbibliography
\newcommand*{\printfullcite}[1]{%
	\begin{refsection}%
		\nocite{#1}%
		\DeclareNameAlias{author}{first-last}%
		\printbibliography[heading = none]%
	\end{refsection}%
}


\colorlet{light alert}{HKS07K60}
\tikzset{alert.bg/.style={rounded corners, fill=light alert}}
\tikzset{every picture/.style={line cap=round, semithick}}
% http://tex.stackexchange.com/questions/6135/how-to-make-beamer-overlays-with-tikz-node
\tikzset{onslide/.code args={<#1>#2}{\only<#1>{\pgfkeysalso{#2}}}}
\tikzset{invisible/.code args={<#1>}{\alt<#1>{\pgfkeysalso{transparent}}{\pgfkeysalso{opaque}}}}
\tikzset{uncover/.code args={<#1>}{\alt<#1>{\pgfkeysalso{opaque}}{\pgfkeysalso{opacity=0.25}}}}
\tikzset{visible/.code args={<#1>}{\alt<#1>{\pgfkeysalso{opaque}}{\pgfkeysalso{transparent}}}}
\tikzset{vuncover/.code args=%
	{<#1><#2>}%
	{\alt<#1>%
		{\alt<#2>%
			{\pgfkeysalso{opaque}}%
			{\pgfkeysalso{opacity=0.25}}%
		}{\pgfkeysalso{transparent}}%
	}%
}

\newcommand<%
	>{\tikzhighlight}[2][]{%
	\delegateStyle{\alt#3%
		{\tikz[baseline=0, anchor=base, inner sep=0.2em, text height=, text depth=]{\node[alert.bg, #1]{\inheritStyle{#2}};}}%
		{\tikz[baseline=0, anchor=base, inner sep=0.2em, text height=, text depth=]{\node[#1, fill=none]{\inheritStyle{#2}};}}%
	}%
}

\newcommand{\mathhighlight}{\tikzhighlight}

\newcommand<>{\mhl}[2][]{\mathhighlight#3[inner sep=0.2em, #1]{#2}}


\newcommand<>{\inlineblock}[2][]{{%
	\usebeamercolor*[fg]{block body}%
	\tikzhighlight#3[fill=block body.bg, #1]{#2}%
}}


% a small letter s for plurals of abbreviations
\newcommand*{\s}{{\scriptsize s}\xspace}


\newcommand<>*{\sout}[2][opacity=0.75, ultra thick]{%
	\delegateStyle{%
		\tikz[baseline=0, anchor=base, inner sep=0, outer sep=0]{
			\useasboundingbox node (n) {\inheritStyle{#2}};
			\only#3{
				\node (h) {\inheritStyle{\ifmmode\mathstrut\else\strut\fi}};
				\draw[#1] (n.west |- {$(h.south)!0.5!(h.north)$}) -- (n.east |- {$(h.south)!0.5!(h.north)$});
			}
		}%
	}%
}


% tight style
% Sets outer sep to default inner sep and inner sep to 0.
% Use this style for nodes that are neither drawn nor filled to prevent
% unwanted growth of the bounding box.
\tikzset{tight/.style={inner sep=0, outer sep=0.3333em}}


% rounded tree edges style
% usage: rounded tree edges={⟨direction⟩}{⟨looseness⟩}{⟨strength⟩}
\tikzset{
	rounded tree edges/.style n args={3}{
	edge from parent path={
	let
		\n{direction}={#1},
		\n{looseness}={#2},
		\n{strength}={#3},
		\p1=(\tikzparentnode),
		\p2=(\tikzchildnode),
		\p3=(\n{direction}:1pt),
		\p4=(\x2 - \x1, \y2 - \y1),
		\n{dist}={veclen(\p4)},
		\p4=(\x4 / \n{dist}, \y4 / \n{dist}),
		\n{angle}={atan2(\y4, \x4)},
		\n{delta}={Mod(\n{angle} - \n{direction}, 360)},
		\n{delta}={\n{delta} > 180 ? \n{delta} - 360  : \n{delta}},
		\n{delta}={\n{delta} >  90 ?  180 - \n{delta} : \n{delta}},
		\n{delta}={\n{delta} < -90 ? -180 - \n{delta} : \n{delta}}
	in (\tikzparentnode) .. controls
		+(    \n{angle}+\n{strength}*\n{delta}:\n{looseness}*0.3915*\n{dist}) and
		+(180+\n{angle}-\n{strength}*\n{delta}:\n{looseness}*0.3915*\n{dist}) ..
		(\tikzchildnode)
	}
	}
}


% Tear out snippets from PDFs.
% Usage: \tear[…]{file.pdf}
% The optional parameter is the same as for \includegraphics.
% Useful Arguments:
%   * page=‹pagenumber›
%   * trim=‹left› ‹bottom› ‹right› ‹top›
%   * width=0.98\linewidth
\newcommand*{\tear}[2][]{%
	\begin{tikzpicture}
		\node
			[ blur shadow
			, clip
			, decorate
			, decoration=random steps
			, draw
			, inner sep=0
			, preaction={fill=white}% hide the shadow if paper is transparent
			] {\includegraphics[#1]{#2}};
	\end{tikzpicture}%
}


\makeatletter
\newcommand*{\timeline}[3][0]{%
	\ifcsname timeline@cmd@#3\endcsname%
		\@timeline[#1]{#2}{#3}%
		\PackageWarning{timeline}{redefining timeline \@backslashchar\string#3}%
	\else%
		\ifcsname#3\endcsname%
			\errmessage{Command \@backslashchar\string#3 already defined}%
		\else%
			\@timeline[#1]{#2}{#3}%
		\fi%
	\fi%
}%
\newcommand*{\@timeline}[3][0]{%
	% mark command as timeline command – they can be overwritten
	\expandafter\def\csname timeline@cmd@#3\endcsname{}%
	\setcounter{@timeline}{#1}%
	\def\timeline@cmd{#3}%
	\timeline@reset%
	\timeline@append{0}%
	\@tfor\timeline@next:=#2\do{%
		\if\timeline@next+%
			\stepcounter{@timeline}%
			\timeline@append{,\the@timeline}%
		\else\if\timeline@next-%
			\stepcounter{@timeline}%
		\else%
			%\timeline@append{\timeline@next}%
			\GenericError{}{\protect\timeline: ignoring unknown character: \timeline@next}%
		\fi\fi%
	}%
}%
% \newcommand*{\tl}[1]{%
% 	\ifcsname timeline@cmd@#1\endcsname%
% 		\csname timeline@cmd@#1\endcsname%
% 	\else%
% 		0%
% 		%\GenericError{}{\protect\tl: timeline not defined: #1}%
% 	\fi%
% }%
\newcounter{@timeline}%
\def\timeline@reset{%
	\expandafter\def\csname\timeline@cmd\endcsname{}%
}%
\def\timeline@append#1{%
	\expandafter\edef\csname\timeline@cmd\endcsname{%
		\csname\timeline@cmd\endcsname#1%
	}%
}%
\makeatother


\newcommand*{\xminus}[1]{%
	\mathrel{\tikz[baseline={([yshift=-0.25em]n.south)}, inner sep=0, outer sep=0.2em]{%
		\node (n) {\(\scriptstyle #1\)};
		\draw (n.south west) -- (n.south east);
	}}%
}
\newcommand*{\tikzrightarrow}[1]{%
	\mathrel{\tikz[baseline={([yshift=-0.25em]n.south)}, inner sep=0, outer sep=0.2em]{%
		\node (n) {\(\scriptstyle #1\)};
		\draw[->, > = Computer Modern Rightarrow, line width = 0.4pt] (n.south west) -- (n.south east);
	}}%
}


%%%%%%%%%%%%%%%%%%%%%%%%%%%%%%%%%%%%%%%%%%%%%%%%%%%%%%%%%%%%%%%%%%%%%%%%%%%%%%
% document specific commands
%%%%%%%%%%%%%%%%%%%%%%%%%%%%%%%%%%%%%%%%%%%%%%%%%%%%%%%%%%%%%%%%%%%%%%%%%%%%%%

\newcommand<>*{\mycite}[1]{\uncover#2{{\color{HKS57K100}[\cite{#1}]}}}


\newcommand{\statetree}[1]{
	\tikz
	[ anchor=base
	, baseline=(current bounding box.center)
	, level distance=2em
	, sibling distance=2em
	]{
		\matrix
		[ draw=nt
		, edge from parent/.style={draw=black}
		, inner sep=0
		, nodes={inner sep=0.2em, rounded corners=0}
		, rounded corners
		] {#1\\}
	}
}


\newcommand*{\mylargeleaf}[1]{{\LARGE\color{HKS41K70}#1}}

\definecolor{state s}{named}{HKS57K80}
\definecolor{state t}{named}{HKS41K70}
\newcommand*{\stateS}[1]{{\color{state s}#1}}
\newcommand*{\stateT}[1]{{\color{state t}#1}}

\tikzset{
	subtree/.style =
		{ fill=lightgray
		, inner sep=0.02em
		, isosceles triangle apex angle=60
		, shape=isosceles triangle
		, shape border rotate=90
		}
	, state/.style = {circle, draw, inner sep=0.1em}
	, trans/.style = {rectangle, draw}
}

\newcommand*{\srBool}{\mathbb{B}}
\newcommand*{\srProb}{ℙ}


%%%%%%%%%%%%%%%%%%%%%%%%%%%%%%%%%%%%%%%%%%%%%%%%%%%%%%%%%%%%%%%%%%%%%%%%%%%%%%
% commands for specific notations
%%%%%%%%%%%%%%%%%%%%%%%%%%%%%%%%%%%%%%%%%%%%%%%%%%%%%%%%%%%%%%%%%%%%%%%%%%%%%%

\DeclareMathOperator*{\argmax}{argmax}

\newcommand*{\cardinality}[1]{\lvert#1\rvert}
\newcommand*{\corpussize}[1]{\lvert#1\rvert}

\DeclareMathOperator{\crispOp}{crisp}
\newcommand*        {\crisp}[2][0]{\crispOp\delim{#1}({#2})}

\DeclareMathOperator{\lhsOp}{lhs}
\newcommand*{\lhs}[1]{\lhsOp(#1)}

\DeclareMathOperator{\lklhdOp}{L}
\newcommand*{\lklhd}[2]{\lklhdOp(#1 ∣ #2)}

\DeclareMathOperator{\mleOp}{mle}
\newcommand*{\mle}[2][]{%
	\ifthenelse{\isempty{#1}}{%
		\mleOp(#2)%
	}{%
		\mleOp_{#1}(#2)%
	}%
}

\DeclareMathOperator{\mrg}{merge}

% CVD: color vision deficiencies
\definecolor{CVD light red}   {HTML}{FF8080}
\definecolor{CVD light yellow}{HTML}{FFFF80}
\definecolor{CVD light green} {HTML}{40FFC0}

\definecolor{nt}{named}{HKS41K70}
\newcommand*{\nt}[1]{{\color{nt}#1}}

% set of all probability distributions over #1
\DeclareMathOperator{\pdsOp}{Pd}
\newcommand*{\pds}[1]{\pdsOp(#1)}

\DeclareMathOperator{\positionsOp}{pos}
\newcommand*{\positions}[1]{\positionsOp(#1)}

\DeclareMathOperator{\rankOp}{rk}
\newcommand*{\rank}[1]{\rankOp(#1)}

\DeclareMathOperator{\runsOp}{run}
\newcommand*{\runs}[2][]{%
	\ifthenelse%
		{\isempty{#1}}%
		{\runsOp(#2)}%
		{\runsOp_{#1}(#2)}%
}

\newcommand*{\semantics}[1]{⟦#1⟧}

\DeclareMathOperator{\splt}{split}

\newcommand*{\subtree}[2]{#1|_{#2}}

\DeclareMathOperator{\supportOp}{supp}
\newcommand*{\support}[1]{\supportOp(#1)}

\newcommand*{\symId}{\textsc{\color{gray}Id}}
\newcommand*{\symCons}{\textsc{\color{gray}Cons}}
\newcommand*{\symFlip}{\textsc{\color{gray}Flip}}
\newcommand*{\symNull}{\textsc{\color{gray}Null}}
\newcommand*{\symNullR}{\textsc{\color{gray}N\(\overline{\textsc{ull}}\)}}
\newcommand*{\symSnoc}{\textsc{\color{gray}Snoc}}

\newcommand*{\transWTA}[4][]{#3 \xrightarrow{#1} #2(#4)}

\DeclareMathOperator{\uniqueRunOp}{r}
\newcommand*{\uniqueRun}[2][]{%
	\ifthenelse%
		{\isempty{#1}}%
		{\uniqueRunOp^{#2}}%
		{\uniqueRunOp_{\!#1}^{#2}}%
}

\DeclareMathOperator{\treesOp}{T}
\newcommand*{\trees}[2][]{%
	\ifthenelse%
		{\isempty{#1}}%
		{\treesOp_{\!#2}}%
		{\treesOp_{\!#2}(#1)}%
}
\DeclareMathOperator{\treesUOp}{U}
\newcommand*{\treesU}[2][]{%
	\ifthenelse%
		{\isempty{#1}}%
		{\treesUOp_{#2}}%
		{\treesUOp_{#2}(#1)}%
}


%%%%%%%%%%%%%%%%%%%%%%%%%%%%%%%%%%%%%%%%%%%%%%%%%%%%%%%%%%%%%%%%%%%%%%%%%%%%%%
% metadata
%%%%%%%%%%%%%%%%%%%%%%%%%%%%%%%%%%%%%%%%%%%%%%%%%%%%%%%%%%%%%%%%%%%%%%%%%%%%%%

\ifstandalonebeamer\else
	\title[Defense of Dissertation]{A Formal View on Training of Weighted Tree Automata by Likelihood-Driven State Splitting and Merging}
	\subtitle{Defense of Dissertation}
\fi
\author{Toni Dietze}
\institute[TU Dresden]{%
	\href{https://www.orchid.inf.tu-dresden.de/index.en/}{Chair for Foundations of Programming}
\\	\href{https://tu-dresden.de/ing/informatik/thi}{Institute of Theoretical Computer Science}
\\	\href{https://tu-dresden.de/ing/informatik}{Faculty of Computer Science}
\\	\href{https://tu-dresden.de/}{Technische Universität Dresden}
\\	01062 Dresden, Germany
}
\date[2018-09-27]{September 27, 2018}

\title{\jobname}
\begin{document}
\begin{standaloneframe}{\jobname}
	\hfill
	{\small\tikz{\node[fill=HKS92K20, inner sep=0]{\begin{tabular}{l@{ \ldots{} }l}
		\(\mathscr{A}\) & bottom-up deterministic ta \((Q, Σ, I, Δ)\)
	\\	\(c\) & corpus over \(\trees{Σ}\)
	\\	\(Π\) & set of all bottom-up determinism preserving mergers for \(\mathscr{A}\)
	\end{tabular}};}}
	\begin{block}{maximize the likelihood (again)}
		\[
			\argmax_{π ∈ Π} \lklhd{c}{\semantics{\mle[π(\mathscr{A})]{c}}}
			\uncover<3->{
				=
				\argmax_{π ∈ Π}
					\frac
						{\lklhd{c}{\semantics{\mle[π(\mathscr{A})]{c}}}}
						{\lklhd{c}{\semantics{\mle[\mathscr{A}]{c}}}}
			}
		\]
	\end{block}
	\vspace*{-1em}
	\begin{overprint}
	\onslide<2-3>
		\begin{block}{combining likelihood and mle}
			\[
				\lklhd{c}{\semantics{\mle[\mathscr{A}]{c}}}
				= \frac
					{∏_{q ∈ Q} {c_{\mathscr{A}}^{\mathrm{I}}(q)}^{c_{\mathscr{A}}^{\mathrm{I}}(q)}}
					{\corpussize{c}}
				· \frac
					{∏_{τ ∈ Δ} {c_{\mathscr{A}}^{\mathrm{Δ}}(τ)}^{c_{\mathscr{A}}^{\mathrm{Δ}}(τ)}}
					{∏_{q ∈ Q} {c_{\mathscr{A}}^{\mathrm{Q}}(q)}^{c_{\mathscr{A}}^{\mathrm{Q}}(q)}}
			\]
		\end{block}
	\onslide<4>
		\begin{block}{}
			Assuming every merger
			\begin{itemize}
			\item does not collapse rules or root states, and
			\item preserves bottom-up determinism,
			\end{itemize}
			it is best to merge the two least frequent non-terminals (w.r.t.\@ \(c_{\mathscr{A}}^{\mathrm{Q}}\)).
		\end{block}
	\end{overprint}
\end{standaloneframe}
\end{document}

\end{frame}

\begin{frame}{\secname}
	\begin{center}
		% SPDX-License-Identifier: CC-BY-4.0
% Copyright 2018 Toni Dietze
\begin{tikzpicture}[anchor=north]
	\path (-1.5em, 2em) rectangle (29.5em, -10em);
	\node at (0, 0) {\(\begin{aligned}
		A & \uncover<2->{{} \mapsto 3}
	\\	B & \uncover<2->{{} \mapsto 4}
	\\	C & \uncover<2->{{} \mapsto 6}
	\\	D & \uncover<2->{{} \mapsto 9}
	\\	\vdots
	\end{aligned}\)};
	\node<2-> at (0.2em, 2em) {counts:};
	\node<3->[single arrow, fill=HKS41K20, align=center] at (4em, -1em) {create \\ mergers};
	\node<3-> (n2) at (10em, 0) {\(\begin{aligned}
		& \text{mrg \(A\), \(B\)}
	\\	& \text{mrg \(A\), \(C\)}
	\\	& \text{mrg \(B\), \(C\)}
	\\	& \text{mrg \(A\), \(D\)}
	\\	& \text{mrg \(B\), \(D\)}
	\\	& \vdots\qquad
	\end{aligned}\)};
	\draw<4->[rounded corners, color=HKS07K100] (n2.north west) rectangle (n2.east);
	\node<5->[single arrow, fill=HKS41K20, align=center] at (15em, -1em) {saturate \\ mergers};
	\node<5-> (n3) at (24em, 0) {\(\begin{alignedat}{2}
		& \text{mrg \(A\), \(B\), \(E\)}        && \uncover<6->{{} \mapsto 0.3}
	\\	& \text{mrg \(A\), \(C\)}               && \uncover<6->{{} \mapsto 0.5}
	\\	& \text{mrg \(B\), \(C\), \(D\), \(F\)} && \uncover<6->{{} \mapsto 0.1}
	\end{alignedat}\)};
	\node<6-> at (26.5em, 2em) {likelihoods:};
	\draw<7->[rounded corners, color=HKS07K100] ($(n3.west) - (0, 0.8em)$) rectangle ($(n3.east) + (0, 0.8em)$);
\end{tikzpicture}
	\end{center}
\end{frame}


\section{Count-Based State Merging Algorithm – Experiments 2}

\begin{frame}{\secname}{}
	\centering
	% SPDX-License-Identifier: CC-BY-4.0
% Copyright 2018 Toni Dietze
\documentclass[beamer]{standalone}
% SPDX-License-Identifier: CC-BY-4.0 OR MIT-0
% Copyright 2018 Toni Dietze
%
\usefonttheme{professionalfonts}

% LuaLaTeX specific packages
\usepackage{fontspec}
	\defaultfontfeatures{Ligatures=TeX}
\usepackage{polyglossia}
	\setdefaultlanguage{english}
\usepackage{amsmath}  % has to be loaded before unicode-math
\usepackage[math-style=ISO]{unicode-math}
	\setmathfont{Latin Modern Math}
% 	\setmathfont[range={\mathcal,\mathbfcal},StylisticSet=1]{xits-math.otf}
% 	\setmathfont[range={"029F5}]{XITS Math}  % ⧵
% 	\setmathfont[range={\mathscr,\mathbfscr},StylisticSet=1]{Latin Modern Math}  % make \mathscr use the correct font

\usepackage[noend]{algpseudocode}
	\algrenewcommand\algorithmicrequire{\textbf{Input:}}
	\algrenewcommand\algorithmicensure{\textbf{Output:}}
\usepackage[backend=biber, maxbibnames=42, maxcitenames=42, sorting=ynt, style=authoryear]{biblatex}
\usepackage{csquotes}
\usepackage{mathtools}
\usepackage{media9}
\usepackage{scalerel}
\usepackage{standalone}
\usepackage{tikz}
	\usetikzlibrary{arrows.meta}
	\usetikzlibrary{backgrounds}
	\usetikzlibrary{calc}
	\usetikzlibrary{decorations}
	\usetikzlibrary{decorations.pathmorphing}
	\usetikzlibrary{decorations.pathreplacing}
	\usetikzlibrary{fadings}
	\usetikzlibrary{fit}
	\usetikzlibrary{graphs}
	\usetikzlibrary{graphdrawing}
	\usetikzlibrary{intersections}
	\usetikzlibrary{positioning}
	\usetikzlibrary{quotes}
	\usetikzlibrary{shadows.blur}
	\usetikzlibrary{shapes.arrows}
	\usetikzlibrary{shapes.geometric}
	\usegdlibrary{trees}
\usepackage{xifthen}
\usepackage{xspace}

\usepackage{pgfplots}
	\pgfplotsset
		{ compat = 1.15
		, /pgf/number format/1000 sep = {\,}
		, /pgf/number format/assume math mode = true
		, every axis plot/.append style =
			{ mark options = {fill opacity = 0.25}
			}
		}
	\usepgfplotslibrary{groupplots}
\usepackage{pgfplotstable}

\hypersetup
	{ bookmarksopen
	, pdflang = en
	, unicode
	}


%%%%%%%%%%%%%%%%%%%%%%%%%%%%%%%%%%%%%%%%%%%%%%%%%%%%%%%%%%%%%%%%%%%%%%%%%%%%%%


% always show bad boxes
%\overfullrule=1em


%%%%%%%%%%%%%%%%%%%%%%%%%%%%%%%%%%%%%%%%%%%%%%%%%%%%%%%%%%%%%%%%%%%%%%%%%%%%%%
% biblatex
%%%%%%%%%%%%%%%%%%%%%%%%%%%%%%%%%%%%%%%%%%%%%%%%%%%%%%%%%%%%%%%%%%%%%%%%%%%%%%

\addbibresource{slides-dissertation-defense.bib}
% \renewcommand*{\finalnamedelim}{\addcomma\space}
% \setlength{\bibitemsep}{1em}
% 
\AtEveryBibitem{% Clean up the bibtex rather than editing it
 \clearlist{address}
 \clearfield{date}
 \clearfield{eprint}
 \clearfield{isbn}
 \clearfield{issn}
 \clearlist{language}
 \clearlist{location}
 \clearfield{month}
 \clearfield{series}
%  \clearfield{url}
%  \clearfield{doi}
 \clearfield{organization}

%  \ifentrytype{book}{}{% Remove stuff except for books
%   \clearfield{booktitle}
%   \clearfield{pages}
  \clearlist{publisher}
  \clearname{editor}
%  }
}
% do not print url if doi is present
% http://tex.stackexchange.com/questions/154864/biblatex-use-doi-only-if-there-is-no-url
\DeclareSourcemap{
	\maps[datatype=bibtex]{
		\map{
			\step[fieldsource=doi,final]
			\step[fieldset=url,null]
}	}	}
%
% remove qoutes around titles
\DeclareFieldFormat
	[article,inbook,incollection,inproceedings,patent,thesis,unpublished]
	{title}{#1\isdot}
% 
% \DeclareFieldFormat{url}{\mkbibacro{URL}\addcolon\addnbspace\url{#1}}
% 
% \DeclareNameAlias{sortname}{first-last}
% 
\renewbibmacro{in:}{\ifentrytype{article}{}{}}


%%%%%%%%%%%%%%%%%%%%%%%%%%%%%%%%%%%%%%%%%%%%%%%%%%%%%%%%%%%%%%%%%%%%%%%%%%%%%%
% beamer
%%%%%%%%%%%%%%%%%%%%%%%%%%%%%%%%%%%%%%%%%%%%%%%%%%%%%%%%%%%%%%%%%%%%%%%%%%%%%%

\useoutertheme{infolines}
\makeatletter
% based on
% /usr/share/texmf-dist/tex/latex/beamer/beamerouterthemeinfolines.sty
\setbeamertemplate{footline}
{%
	\leavevmode%
	\hbox{%
	\begin{beamercolorbox}[wd=.333333\paperwidth,ht=2.25ex,dp=1ex,center]{author in head/foot}%
		\usebeamerfont{author in head/foot}\insertshortauthor\expandafter\beamer@ifempty\expandafter{\beamer@shortinstitute}{}{~~(\insertshortinstitute)}
	\end{beamercolorbox}%
	\begin{beamercolorbox}[wd=.333333\paperwidth,ht=2.25ex,dp=1ex,center]{title in head/foot}%
		\usebeamerfont{title in head/foot}\insertshorttitle
	\end{beamercolorbox}%
	\begin{beamercolorbox}[wd=.333333\paperwidth,ht=2.25ex,dp=1ex,right]{date in head/foot}%
		\usebeamerfont{date in head/foot}%
		\hfill\insertshortdate\hfill\hfill%
		%\hspace*{2ex}%
		%\insertshortdate%
		%\hspace{0pt plus 1 filll}%
		%(\insertframenumber.\insertoverlaynumber{} / \insertmainframenumber)%
		%\hspace{0pt plus 1 filll}%
		\phantom{000}\llap{\insertpagenumber} / \insertpresentationendpage%
		\hspace*{2ex}%
	\end{beamercolorbox}}%
	\vskip0pt%
}
\makeatother
\useinnertheme{circles}
\beamertemplatenavigationsymbolsempty
\setbeamertemplate{bibliography item}{}
\setbeamertemplate{headline}[default]

\input{tudcolors.tex}
\setbeamercolor*{alerted text}{fg=HKS07K100}
\usecolortheme[named=HKS41K100]{structure}

\setbeamercolor*{palette primary}{use=structure,fg=white,bg=structure.fg}
\setbeamercolor*{palette secondary}{use=structure,fg=white,bg=structure.fg!80}
\setbeamercolor*{palette tertiary}{use=structure,fg=white,bg=structure.fg!60}
\setbeamercolor*{palette quaternary}{fg=white,bg=black}

\setbeamercolor*{sidebar}{use=structure,bg=structure.fg}

\setbeamercolor*{palette sidebar primary}{use=structure,fg=structure.fg!20}
\setbeamercolor*{palette sidebar secondary}{fg=white}
\setbeamercolor*{palette sidebar tertiary}{use=structure,fg=structure.fg!40}
\setbeamercolor*{palette sidebar quaternary}{fg=white}

\setbeamercolor*{titlelike}{parent=palette primary}

\setbeamercolor*{separation line}{}
\setbeamercolor*{fine separation line}{}

\setbeamercolor{block title}{use=structure,fg=white,bg=structure.fg}
\setbeamercolor{block title alerted}{use=alerted text,fg=white,bg=alerted text.fg!75!black}
\setbeamercolor{block title example}{use=example text,fg=white,bg=example text.fg!75!black}

\setbeamercolor{block body}{parent=normal text,use=block title,bg=block title.bg!10!bg}
\setbeamercolor{block body alerted}{parent=normal text,use=block title alerted,bg=block title alerted.bg!10!bg}
\setbeamercolor{block body example}{parent=normal text,use=block title example,bg=block title example.bg!10!bg}

% \setbeamertemplate{itemize items}[default]


%%%%%%%%%%%%%%%%%%%%%%%%%%%%%%%%%%%%%%%%%%%%%%%%%%%%%%%%%%%%%%%%%%%%%%%%%%%%%%
% TikZ
%%%%%%%%%%%%%%%%%%%%%%%%%%%%%%%%%%%%%%%%%%%%%%%%%%%%%%%%%%%%%%%%%%%%%%%%%%%%%%

\tikzset
	{ > = Stealth
	}


%%%%%%%%%%%%%%%%%%%%%%%%%%%%%%%%%%%%%%%%%%%%%%%%%%%%%%%%%%%%%%%%%%%%%%%%%%%%%%
% general commands and styles
%%%%%%%%%%%%%%%%%%%%%%%%%%%%%%%%%%%%%%%%%%%%%%%%%%%%%%%%%%%%%%%%%%%%%%%%%%%%%%

% \delegateStyle and \inheritStyle command
% usage: \delegateStyle{… \inheritStyle{…} …}
% example: \(X_{\delegateStyle{\fbox{\inheritStyle{X}}}}\)
% Save the current style and regain it in the argument.
% This works both for math and text mode, and can be nested.
% Acknowledgments: Based on \ThisStyle and \SavedStyle from scalerel package.
\makeatletter
\newcommand*{\@inheritStyle@D}[1]{\(\displaystyle      #1\)}
\newcommand*{\@inheritStyle@T}[1]{\(\textstyle         #1\)}
\newcommand*{\@inheritStyle@S}[1]{\(\scriptstyle       #1\)}
\newcommand*{\@inheritStyle@s}[1]{\(\scriptscriptstyle #1\)}
\newcommand*{\@inheritStyle@t}[1]{#1}
\newcommand*{\inheritStyle}{\csname @inheritStyle@\@inheritStyleSwitch\endcsname}
\newcommand*{\delegateStyle}[1]{%
	\ifmmode%
		\mathchoice%
		{\edef\@inheritStyleSwitch{D}#1}%
		{\edef\@inheritStyleSwitch{T}#1}%
		{\edef\@inheritStyleSwitch{S}#1}%
		{\edef\@inheritStyleSwitch{s}#1}%
	\else%
		\edef\@inheritStyleSwitch{t}#1%
	\fi%
}
\makeatother


% \oalt command
% requires: \delegateStyle and \inheritStyle command
% usage: \oalt<…>[…]{…}{…} (cf. \alt)
% Behaves like \alt, but reserves space according to largest overlays.
% The optional argument defines the alignment inside the reserved space;
% it is one of c, l, r, s (cf. \makebox); the default is c.
\makeatletter
\newlength{\oalt@dp}
\newlength{\oalt@ht}
\newlength{\oalt@wd}
\newbox{\oalt@a}
\newbox{\oalt@b}
\newbox{\oalt@empty}
\newcommand<>*{\oalt}[3][c]{%
	\delegateStyle{%
		% based on \setto… in /usr/share/texmf-dist/tex/latex/base/latex.ltx
		\setbox\oalt@a\hbox{\inheritStyle{#2}}%
		\setbox\oalt@b\hbox{\inheritStyle{#3}}%
		\pgfmathsetlength{\oalt@dp}{max(\dp\oalt@a,\dp\oalt@b)}%
		\pgfmathsetlength{\oalt@ht}{max(\ht\oalt@a,\ht\oalt@b)}%
		\pgfmathsetlength{\oalt@wd}{max(\wd\oalt@a,\wd\oalt@b)}%
		\raisebox{0pt}[\oalt@ht][\oalt@dp]{%
			\makebox[\oalt@wd][#1]{%
				\alt#4{\unhbox\oalt@a}{\unhbox\oalt@b}%
			}%
		}%
		\setbox\oalt@a\box\oalt@empty%
		\setbox\oalt@b\box\oalt@empty%
	}%
}
\makeatother


% \otemporal command
% requires: \delegateStyle and \inheritStyle command
% usage: \otemporal<…>[…]{…}{…}{…} (cf. \temporal)
% Behaves like \temporal, but reserves space according to largest overlays.
% The optional argument defines the alignment inside the reserved space;
% it is one of c, l, r, s (cf. \makebox); the default is c.
\makeatletter
\newlength{\ot@dp}
\newlength{\ot@ht}
\newlength{\ot@wd}
\newbox{\ot@a}
\newbox{\ot@b}
\newbox{\ot@c}
\newbox{\ot@empty}
\newcommand<>*{\otemporal}[4][c]{%
	\delegateStyle{%
		% based on \setto… in /usr/share/texmf-dist/tex/latex/base/latex.ltx
		\setbox\ot@a\hbox{\inheritStyle{#2}}%
		\setbox\ot@b\hbox{\inheritStyle{#3}}%
		\setbox\ot@c\hbox{\inheritStyle{#4}}%
		\pgfmathsetlength{\ot@dp}{max(\dp\ot@a,\dp\ot@b,\dp\ot@c)}%
		\pgfmathsetlength{\ot@ht}{max(\ht\ot@a,\ht\ot@b,\ht\ot@c)}%
		\pgfmathsetlength{\ot@wd}{max(\wd\ot@a,\wd\ot@b,\wd\ot@c)}%
		\raisebox{0pt}[\ot@ht][\ot@dp]{%
			\makebox[\ot@wd][#1]{%
				\temporal#5{\unhbox\ot@a}{\unhbox\ot@b}{\unhbox\ot@c}%
			}%
		}%
		\setbox\ot@a\box\ot@empty%
		\setbox\ot@b\box\ot@empty%
		\setbox\ot@c\box\ot@empty%
	}%
}
\makeatother


% Resize delimiters like braces, brackets, etc.
% Parameters: size, left delimiter, formula, right delimiter
% Example: \delim2({\frac{1}{2}})
\newcommand*{\delim}[4]{%
	\ifcase#1%
		#2#3#4%
	\or%
		\bigl#2#3\bigr#4%
	\or%
		\Bigl#2#3\Bigr#4%
	\or%
		\biggl#2#3\biggr#4%
	\or%
		\Biggl#2#3\Biggr#4%
	\else%
		\left#2#3\right#4%
	\fi%
}


% similar to \fullcite, but using the formatting of \printbibliography
\newcommand*{\printfullcite}[1]{%
	\begin{refsection}%
		\nocite{#1}%
		\DeclareNameAlias{author}{first-last}%
		\printbibliography[heading = none]%
	\end{refsection}%
}


\colorlet{light alert}{HKS07K60}
\tikzset{alert.bg/.style={rounded corners, fill=light alert}}
\tikzset{every picture/.style={line cap=round, semithick}}
% http://tex.stackexchange.com/questions/6135/how-to-make-beamer-overlays-with-tikz-node
\tikzset{onslide/.code args={<#1>#2}{\only<#1>{\pgfkeysalso{#2}}}}
\tikzset{invisible/.code args={<#1>}{\alt<#1>{\pgfkeysalso{transparent}}{\pgfkeysalso{opaque}}}}
\tikzset{uncover/.code args={<#1>}{\alt<#1>{\pgfkeysalso{opaque}}{\pgfkeysalso{opacity=0.25}}}}
\tikzset{visible/.code args={<#1>}{\alt<#1>{\pgfkeysalso{opaque}}{\pgfkeysalso{transparent}}}}
\tikzset{vuncover/.code args=%
	{<#1><#2>}%
	{\alt<#1>%
		{\alt<#2>%
			{\pgfkeysalso{opaque}}%
			{\pgfkeysalso{opacity=0.25}}%
		}{\pgfkeysalso{transparent}}%
	}%
}

\newcommand<%
	>{\tikzhighlight}[2][]{%
	\delegateStyle{\alt#3%
		{\tikz[baseline=0, anchor=base, inner sep=0.2em, text height=, text depth=]{\node[alert.bg, #1]{\inheritStyle{#2}};}}%
		{\tikz[baseline=0, anchor=base, inner sep=0.2em, text height=, text depth=]{\node[#1, fill=none]{\inheritStyle{#2}};}}%
	}%
}

\newcommand{\mathhighlight}{\tikzhighlight}

\newcommand<>{\mhl}[2][]{\mathhighlight#3[inner sep=0.2em, #1]{#2}}


\newcommand<>{\inlineblock}[2][]{{%
	\usebeamercolor*[fg]{block body}%
	\tikzhighlight#3[fill=block body.bg, #1]{#2}%
}}


% a small letter s for plurals of abbreviations
\newcommand*{\s}{{\scriptsize s}\xspace}


\newcommand<>*{\sout}[2][opacity=0.75, ultra thick]{%
	\delegateStyle{%
		\tikz[baseline=0, anchor=base, inner sep=0, outer sep=0]{
			\useasboundingbox node (n) {\inheritStyle{#2}};
			\only#3{
				\node (h) {\inheritStyle{\ifmmode\mathstrut\else\strut\fi}};
				\draw[#1] (n.west |- {$(h.south)!0.5!(h.north)$}) -- (n.east |- {$(h.south)!0.5!(h.north)$});
			}
		}%
	}%
}


% tight style
% Sets outer sep to default inner sep and inner sep to 0.
% Use this style for nodes that are neither drawn nor filled to prevent
% unwanted growth of the bounding box.
\tikzset{tight/.style={inner sep=0, outer sep=0.3333em}}


% rounded tree edges style
% usage: rounded tree edges={⟨direction⟩}{⟨looseness⟩}{⟨strength⟩}
\tikzset{
	rounded tree edges/.style n args={3}{
	edge from parent path={
	let
		\n{direction}={#1},
		\n{looseness}={#2},
		\n{strength}={#3},
		\p1=(\tikzparentnode),
		\p2=(\tikzchildnode),
		\p3=(\n{direction}:1pt),
		\p4=(\x2 - \x1, \y2 - \y1),
		\n{dist}={veclen(\p4)},
		\p4=(\x4 / \n{dist}, \y4 / \n{dist}),
		\n{angle}={atan2(\y4, \x4)},
		\n{delta}={Mod(\n{angle} - \n{direction}, 360)},
		\n{delta}={\n{delta} > 180 ? \n{delta} - 360  : \n{delta}},
		\n{delta}={\n{delta} >  90 ?  180 - \n{delta} : \n{delta}},
		\n{delta}={\n{delta} < -90 ? -180 - \n{delta} : \n{delta}}
	in (\tikzparentnode) .. controls
		+(    \n{angle}+\n{strength}*\n{delta}:\n{looseness}*0.3915*\n{dist}) and
		+(180+\n{angle}-\n{strength}*\n{delta}:\n{looseness}*0.3915*\n{dist}) ..
		(\tikzchildnode)
	}
	}
}


% Tear out snippets from PDFs.
% Usage: \tear[…]{file.pdf}
% The optional parameter is the same as for \includegraphics.
% Useful Arguments:
%   * page=‹pagenumber›
%   * trim=‹left› ‹bottom› ‹right› ‹top›
%   * width=0.98\linewidth
\newcommand*{\tear}[2][]{%
	\begin{tikzpicture}
		\node
			[ blur shadow
			, clip
			, decorate
			, decoration=random steps
			, draw
			, inner sep=0
			, preaction={fill=white}% hide the shadow if paper is transparent
			] {\includegraphics[#1]{#2}};
	\end{tikzpicture}%
}


\makeatletter
\newcommand*{\timeline}[3][0]{%
	\ifcsname timeline@cmd@#3\endcsname%
		\@timeline[#1]{#2}{#3}%
		\PackageWarning{timeline}{redefining timeline \@backslashchar\string#3}%
	\else%
		\ifcsname#3\endcsname%
			\errmessage{Command \@backslashchar\string#3 already defined}%
		\else%
			\@timeline[#1]{#2}{#3}%
		\fi%
	\fi%
}%
\newcommand*{\@timeline}[3][0]{%
	% mark command as timeline command – they can be overwritten
	\expandafter\def\csname timeline@cmd@#3\endcsname{}%
	\setcounter{@timeline}{#1}%
	\def\timeline@cmd{#3}%
	\timeline@reset%
	\timeline@append{0}%
	\@tfor\timeline@next:=#2\do{%
		\if\timeline@next+%
			\stepcounter{@timeline}%
			\timeline@append{,\the@timeline}%
		\else\if\timeline@next-%
			\stepcounter{@timeline}%
		\else%
			%\timeline@append{\timeline@next}%
			\GenericError{}{\protect\timeline: ignoring unknown character: \timeline@next}%
		\fi\fi%
	}%
}%
% \newcommand*{\tl}[1]{%
% 	\ifcsname timeline@cmd@#1\endcsname%
% 		\csname timeline@cmd@#1\endcsname%
% 	\else%
% 		0%
% 		%\GenericError{}{\protect\tl: timeline not defined: #1}%
% 	\fi%
% }%
\newcounter{@timeline}%
\def\timeline@reset{%
	\expandafter\def\csname\timeline@cmd\endcsname{}%
}%
\def\timeline@append#1{%
	\expandafter\edef\csname\timeline@cmd\endcsname{%
		\csname\timeline@cmd\endcsname#1%
	}%
}%
\makeatother


\newcommand*{\xminus}[1]{%
	\mathrel{\tikz[baseline={([yshift=-0.25em]n.south)}, inner sep=0, outer sep=0.2em]{%
		\node (n) {\(\scriptstyle #1\)};
		\draw (n.south west) -- (n.south east);
	}}%
}
\newcommand*{\tikzrightarrow}[1]{%
	\mathrel{\tikz[baseline={([yshift=-0.25em]n.south)}, inner sep=0, outer sep=0.2em]{%
		\node (n) {\(\scriptstyle #1\)};
		\draw[->, > = Computer Modern Rightarrow, line width = 0.4pt] (n.south west) -- (n.south east);
	}}%
}


%%%%%%%%%%%%%%%%%%%%%%%%%%%%%%%%%%%%%%%%%%%%%%%%%%%%%%%%%%%%%%%%%%%%%%%%%%%%%%
% document specific commands
%%%%%%%%%%%%%%%%%%%%%%%%%%%%%%%%%%%%%%%%%%%%%%%%%%%%%%%%%%%%%%%%%%%%%%%%%%%%%%

\newcommand<>*{\mycite}[1]{\uncover#2{{\color{HKS57K100}[\cite{#1}]}}}


\newcommand{\statetree}[1]{
	\tikz
	[ anchor=base
	, baseline=(current bounding box.center)
	, level distance=2em
	, sibling distance=2em
	]{
		\matrix
		[ draw=nt
		, edge from parent/.style={draw=black}
		, inner sep=0
		, nodes={inner sep=0.2em, rounded corners=0}
		, rounded corners
		] {#1\\}
	}
}


\newcommand*{\mylargeleaf}[1]{{\LARGE\color{HKS41K70}#1}}

\definecolor{state s}{named}{HKS57K80}
\definecolor{state t}{named}{HKS41K70}
\newcommand*{\stateS}[1]{{\color{state s}#1}}
\newcommand*{\stateT}[1]{{\color{state t}#1}}

\tikzset{
	subtree/.style =
		{ fill=lightgray
		, inner sep=0.02em
		, isosceles triangle apex angle=60
		, shape=isosceles triangle
		, shape border rotate=90
		}
	, state/.style = {circle, draw, inner sep=0.1em}
	, trans/.style = {rectangle, draw}
}

\newcommand*{\srBool}{\mathbb{B}}
\newcommand*{\srProb}{ℙ}


%%%%%%%%%%%%%%%%%%%%%%%%%%%%%%%%%%%%%%%%%%%%%%%%%%%%%%%%%%%%%%%%%%%%%%%%%%%%%%
% commands for specific notations
%%%%%%%%%%%%%%%%%%%%%%%%%%%%%%%%%%%%%%%%%%%%%%%%%%%%%%%%%%%%%%%%%%%%%%%%%%%%%%

\DeclareMathOperator*{\argmax}{argmax}

\newcommand*{\cardinality}[1]{\lvert#1\rvert}
\newcommand*{\corpussize}[1]{\lvert#1\rvert}

\DeclareMathOperator{\crispOp}{crisp}
\newcommand*        {\crisp}[2][0]{\crispOp\delim{#1}({#2})}

\DeclareMathOperator{\lhsOp}{lhs}
\newcommand*{\lhs}[1]{\lhsOp(#1)}

\DeclareMathOperator{\lklhdOp}{L}
\newcommand*{\lklhd}[2]{\lklhdOp(#1 ∣ #2)}

\DeclareMathOperator{\mleOp}{mle}
\newcommand*{\mle}[2][]{%
	\ifthenelse{\isempty{#1}}{%
		\mleOp(#2)%
	}{%
		\mleOp_{#1}(#2)%
	}%
}

\DeclareMathOperator{\mrg}{merge}

% CVD: color vision deficiencies
\definecolor{CVD light red}   {HTML}{FF8080}
\definecolor{CVD light yellow}{HTML}{FFFF80}
\definecolor{CVD light green} {HTML}{40FFC0}

\definecolor{nt}{named}{HKS41K70}
\newcommand*{\nt}[1]{{\color{nt}#1}}

% set of all probability distributions over #1
\DeclareMathOperator{\pdsOp}{Pd}
\newcommand*{\pds}[1]{\pdsOp(#1)}

\DeclareMathOperator{\positionsOp}{pos}
\newcommand*{\positions}[1]{\positionsOp(#1)}

\DeclareMathOperator{\rankOp}{rk}
\newcommand*{\rank}[1]{\rankOp(#1)}

\DeclareMathOperator{\runsOp}{run}
\newcommand*{\runs}[2][]{%
	\ifthenelse%
		{\isempty{#1}}%
		{\runsOp(#2)}%
		{\runsOp_{#1}(#2)}%
}

\newcommand*{\semantics}[1]{⟦#1⟧}

\DeclareMathOperator{\splt}{split}

\newcommand*{\subtree}[2]{#1|_{#2}}

\DeclareMathOperator{\supportOp}{supp}
\newcommand*{\support}[1]{\supportOp(#1)}

\newcommand*{\symId}{\textsc{\color{gray}Id}}
\newcommand*{\symCons}{\textsc{\color{gray}Cons}}
\newcommand*{\symFlip}{\textsc{\color{gray}Flip}}
\newcommand*{\symNull}{\textsc{\color{gray}Null}}
\newcommand*{\symNullR}{\textsc{\color{gray}N\(\overline{\textsc{ull}}\)}}
\newcommand*{\symSnoc}{\textsc{\color{gray}Snoc}}

\newcommand*{\transWTA}[4][]{#3 \xrightarrow{#1} #2(#4)}

\DeclareMathOperator{\uniqueRunOp}{r}
\newcommand*{\uniqueRun}[2][]{%
	\ifthenelse%
		{\isempty{#1}}%
		{\uniqueRunOp^{#2}}%
		{\uniqueRunOp_{\!#1}^{#2}}%
}

\DeclareMathOperator{\treesOp}{T}
\newcommand*{\trees}[2][]{%
	\ifthenelse%
		{\isempty{#1}}%
		{\treesOp_{\!#2}}%
		{\treesOp_{\!#2}(#1)}%
}
\DeclareMathOperator{\treesUOp}{U}
\newcommand*{\treesU}[2][]{%
	\ifthenelse%
		{\isempty{#1}}%
		{\treesUOp_{#2}}%
		{\treesUOp_{#2}(#1)}%
}


%%%%%%%%%%%%%%%%%%%%%%%%%%%%%%%%%%%%%%%%%%%%%%%%%%%%%%%%%%%%%%%%%%%%%%%%%%%%%%
% metadata
%%%%%%%%%%%%%%%%%%%%%%%%%%%%%%%%%%%%%%%%%%%%%%%%%%%%%%%%%%%%%%%%%%%%%%%%%%%%%%

\ifstandalonebeamer\else
	\title[Defense of Dissertation]{A Formal View on Training of Weighted Tree Automata by Likelihood-Driven State Splitting and Merging}
	\subtitle{Defense of Dissertation}
\fi
\author{Toni Dietze}
\institute[TU Dresden]{%
	\href{https://www.orchid.inf.tu-dresden.de/index.en/}{Chair for Foundations of Programming}
\\	\href{https://tu-dresden.de/ing/informatik/thi}{Institute of Theoretical Computer Science}
\\	\href{https://tu-dresden.de/ing/informatik}{Faculty of Computer Science}
\\	\href{https://tu-dresden.de/}{Technische Universität Dresden}
\\	01062 Dresden, Germany
}
\date[2018-09-27]{September 27, 2018}

\begin{document}
\begin{standaloneframe}{\jobname}
	\centering
	{\small
	\begin{tikzpicture}
	[ data/.style={align=center, draw, minimum width=2em, rounded corners}
	, func/.style={align=center, draw, minimum width=2em}
	, node distance = 2em and 3em
	]
		\node[func] (gen) {\strut generate corpus};
		\node[data, right = of gen] (c) {\strut corpus};

		\node[data, above = 1.5em of gen] (size) {\strut corpus size};
		\node[data, left = of gen] (pta) {\strut pta};

		\node[func, below = of gen] (cmp) {\strut compare};
		\node[data, below = of c] (pta seq) {\strut sequence of pta\s};

		\node[func] (cbsm) at ([xshift = 7em] $(c.south)!0.5!(pta seq.north)$) {\strut cbsm};

		\node[data, left = of cmp] (res) {\strut iteration or “no”};
		\begin{scope}[->, rounded corners]
			\draw (size) -- (gen);
			\draw (pta) -- (gen);
			\draw (gen) -- (c);
			\draw (c) -| (cbsm);
			\draw (cbsm) |- (pta seq);
			\draw (pta seq) -- (cmp);
			\draw (cmp) -- (res);
			\draw (pta) -| ({$(pta.east)!0.5!(gen.west)$} |- {$(gen.south)!0.5!(cmp.north)$}) -| (cmp);
		\end{scope}
	\end{tikzpicture}}
	\\[2em]
	\begin{block}<2->{Result}
		\centering
		\begin{tabular}{r@{\qquad}r@{ }lr@{ }l}
			\textbf{\usebeamercolor*[fg]{description item}Inputs:}
			& 11 & ta
			& \visible<3->{36} & \visible<3->{pta}
		\\[0.5em]
			\visible<4->{\textbf{\usebeamercolor*[fg]{description item}Result:}}
			& \visible<5->{8} & \visible<5->{ta recovered}
			& \visible<4->{31} & \visible<4->{pta recovered}
		\\	&& \hfill\llap{\footnotesize\visible<5->{(each pta recovered)}}
		\end{tabular}
	\end{block}
\end{standaloneframe}
\end{document}

\end{frame}


\section{Tree Automata (ta) – Revisited}

\againframe<15>[t]{frame:automata}


\section{Weighted Unranked Tree Automata (wuta)}

\againframe<16->[t]{frame:automata}

\iffalse
\begin{frame}{\secname}
	\centering
	% SPDX-License-Identifier: CC-BY-4.0
% Copyright 2018 Toni Dietze
\documentclass[beamer]{standalone}
% SPDX-License-Identifier: CC-BY-4.0 OR MIT-0
% Copyright 2018 Toni Dietze
%
\usefonttheme{professionalfonts}

% LuaLaTeX specific packages
\usepackage{fontspec}
	\defaultfontfeatures{Ligatures=TeX}
\usepackage{polyglossia}
	\setdefaultlanguage{english}
\usepackage{amsmath}  % has to be loaded before unicode-math
\usepackage[math-style=ISO]{unicode-math}
	\setmathfont{Latin Modern Math}
% 	\setmathfont[range={\mathcal,\mathbfcal},StylisticSet=1]{xits-math.otf}
% 	\setmathfont[range={"029F5}]{XITS Math}  % ⧵
% 	\setmathfont[range={\mathscr,\mathbfscr},StylisticSet=1]{Latin Modern Math}  % make \mathscr use the correct font

\usepackage[noend]{algpseudocode}
	\algrenewcommand\algorithmicrequire{\textbf{Input:}}
	\algrenewcommand\algorithmicensure{\textbf{Output:}}
\usepackage[backend=biber, maxbibnames=42, maxcitenames=42, sorting=ynt, style=authoryear]{biblatex}
\usepackage{csquotes}
\usepackage{mathtools}
\usepackage{media9}
\usepackage{scalerel}
\usepackage{standalone}
\usepackage{tikz}
	\usetikzlibrary{arrows.meta}
	\usetikzlibrary{backgrounds}
	\usetikzlibrary{calc}
	\usetikzlibrary{decorations}
	\usetikzlibrary{decorations.pathmorphing}
	\usetikzlibrary{decorations.pathreplacing}
	\usetikzlibrary{fadings}
	\usetikzlibrary{fit}
	\usetikzlibrary{graphs}
	\usetikzlibrary{graphdrawing}
	\usetikzlibrary{intersections}
	\usetikzlibrary{positioning}
	\usetikzlibrary{quotes}
	\usetikzlibrary{shadows.blur}
	\usetikzlibrary{shapes.arrows}
	\usetikzlibrary{shapes.geometric}
	\usegdlibrary{trees}
\usepackage{xifthen}
\usepackage{xspace}

\usepackage{pgfplots}
	\pgfplotsset
		{ compat = 1.15
		, /pgf/number format/1000 sep = {\,}
		, /pgf/number format/assume math mode = true
		, every axis plot/.append style =
			{ mark options = {fill opacity = 0.25}
			}
		}
	\usepgfplotslibrary{groupplots}
\usepackage{pgfplotstable}

\hypersetup
	{ bookmarksopen
	, pdflang = en
	, unicode
	}


%%%%%%%%%%%%%%%%%%%%%%%%%%%%%%%%%%%%%%%%%%%%%%%%%%%%%%%%%%%%%%%%%%%%%%%%%%%%%%


% always show bad boxes
%\overfullrule=1em


%%%%%%%%%%%%%%%%%%%%%%%%%%%%%%%%%%%%%%%%%%%%%%%%%%%%%%%%%%%%%%%%%%%%%%%%%%%%%%
% biblatex
%%%%%%%%%%%%%%%%%%%%%%%%%%%%%%%%%%%%%%%%%%%%%%%%%%%%%%%%%%%%%%%%%%%%%%%%%%%%%%

\addbibresource{slides-dissertation-defense.bib}
% \renewcommand*{\finalnamedelim}{\addcomma\space}
% \setlength{\bibitemsep}{1em}
% 
\AtEveryBibitem{% Clean up the bibtex rather than editing it
 \clearlist{address}
 \clearfield{date}
 \clearfield{eprint}
 \clearfield{isbn}
 \clearfield{issn}
 \clearlist{language}
 \clearlist{location}
 \clearfield{month}
 \clearfield{series}
%  \clearfield{url}
%  \clearfield{doi}
 \clearfield{organization}

%  \ifentrytype{book}{}{% Remove stuff except for books
%   \clearfield{booktitle}
%   \clearfield{pages}
  \clearlist{publisher}
  \clearname{editor}
%  }
}
% do not print url if doi is present
% http://tex.stackexchange.com/questions/154864/biblatex-use-doi-only-if-there-is-no-url
\DeclareSourcemap{
	\maps[datatype=bibtex]{
		\map{
			\step[fieldsource=doi,final]
			\step[fieldset=url,null]
}	}	}
%
% remove qoutes around titles
\DeclareFieldFormat
	[article,inbook,incollection,inproceedings,patent,thesis,unpublished]
	{title}{#1\isdot}
% 
% \DeclareFieldFormat{url}{\mkbibacro{URL}\addcolon\addnbspace\url{#1}}
% 
% \DeclareNameAlias{sortname}{first-last}
% 
\renewbibmacro{in:}{\ifentrytype{article}{}{}}


%%%%%%%%%%%%%%%%%%%%%%%%%%%%%%%%%%%%%%%%%%%%%%%%%%%%%%%%%%%%%%%%%%%%%%%%%%%%%%
% beamer
%%%%%%%%%%%%%%%%%%%%%%%%%%%%%%%%%%%%%%%%%%%%%%%%%%%%%%%%%%%%%%%%%%%%%%%%%%%%%%

\useoutertheme{infolines}
\makeatletter
% based on
% /usr/share/texmf-dist/tex/latex/beamer/beamerouterthemeinfolines.sty
\setbeamertemplate{footline}
{%
	\leavevmode%
	\hbox{%
	\begin{beamercolorbox}[wd=.333333\paperwidth,ht=2.25ex,dp=1ex,center]{author in head/foot}%
		\usebeamerfont{author in head/foot}\insertshortauthor\expandafter\beamer@ifempty\expandafter{\beamer@shortinstitute}{}{~~(\insertshortinstitute)}
	\end{beamercolorbox}%
	\begin{beamercolorbox}[wd=.333333\paperwidth,ht=2.25ex,dp=1ex,center]{title in head/foot}%
		\usebeamerfont{title in head/foot}\insertshorttitle
	\end{beamercolorbox}%
	\begin{beamercolorbox}[wd=.333333\paperwidth,ht=2.25ex,dp=1ex,right]{date in head/foot}%
		\usebeamerfont{date in head/foot}%
		\hfill\insertshortdate\hfill\hfill%
		%\hspace*{2ex}%
		%\insertshortdate%
		%\hspace{0pt plus 1 filll}%
		%(\insertframenumber.\insertoverlaynumber{} / \insertmainframenumber)%
		%\hspace{0pt plus 1 filll}%
		\phantom{000}\llap{\insertpagenumber} / \insertpresentationendpage%
		\hspace*{2ex}%
	\end{beamercolorbox}}%
	\vskip0pt%
}
\makeatother
\useinnertheme{circles}
\beamertemplatenavigationsymbolsempty
\setbeamertemplate{bibliography item}{}
\setbeamertemplate{headline}[default]

\input{tudcolors.tex}
\setbeamercolor*{alerted text}{fg=HKS07K100}
\usecolortheme[named=HKS41K100]{structure}

\setbeamercolor*{palette primary}{use=structure,fg=white,bg=structure.fg}
\setbeamercolor*{palette secondary}{use=structure,fg=white,bg=structure.fg!80}
\setbeamercolor*{palette tertiary}{use=structure,fg=white,bg=structure.fg!60}
\setbeamercolor*{palette quaternary}{fg=white,bg=black}

\setbeamercolor*{sidebar}{use=structure,bg=structure.fg}

\setbeamercolor*{palette sidebar primary}{use=structure,fg=structure.fg!20}
\setbeamercolor*{palette sidebar secondary}{fg=white}
\setbeamercolor*{palette sidebar tertiary}{use=structure,fg=structure.fg!40}
\setbeamercolor*{palette sidebar quaternary}{fg=white}

\setbeamercolor*{titlelike}{parent=palette primary}

\setbeamercolor*{separation line}{}
\setbeamercolor*{fine separation line}{}

\setbeamercolor{block title}{use=structure,fg=white,bg=structure.fg}
\setbeamercolor{block title alerted}{use=alerted text,fg=white,bg=alerted text.fg!75!black}
\setbeamercolor{block title example}{use=example text,fg=white,bg=example text.fg!75!black}

\setbeamercolor{block body}{parent=normal text,use=block title,bg=block title.bg!10!bg}
\setbeamercolor{block body alerted}{parent=normal text,use=block title alerted,bg=block title alerted.bg!10!bg}
\setbeamercolor{block body example}{parent=normal text,use=block title example,bg=block title example.bg!10!bg}

% \setbeamertemplate{itemize items}[default]


%%%%%%%%%%%%%%%%%%%%%%%%%%%%%%%%%%%%%%%%%%%%%%%%%%%%%%%%%%%%%%%%%%%%%%%%%%%%%%
% TikZ
%%%%%%%%%%%%%%%%%%%%%%%%%%%%%%%%%%%%%%%%%%%%%%%%%%%%%%%%%%%%%%%%%%%%%%%%%%%%%%

\tikzset
	{ > = Stealth
	}


%%%%%%%%%%%%%%%%%%%%%%%%%%%%%%%%%%%%%%%%%%%%%%%%%%%%%%%%%%%%%%%%%%%%%%%%%%%%%%
% general commands and styles
%%%%%%%%%%%%%%%%%%%%%%%%%%%%%%%%%%%%%%%%%%%%%%%%%%%%%%%%%%%%%%%%%%%%%%%%%%%%%%

% \delegateStyle and \inheritStyle command
% usage: \delegateStyle{… \inheritStyle{…} …}
% example: \(X_{\delegateStyle{\fbox{\inheritStyle{X}}}}\)
% Save the current style and regain it in the argument.
% This works both for math and text mode, and can be nested.
% Acknowledgments: Based on \ThisStyle and \SavedStyle from scalerel package.
\makeatletter
\newcommand*{\@inheritStyle@D}[1]{\(\displaystyle      #1\)}
\newcommand*{\@inheritStyle@T}[1]{\(\textstyle         #1\)}
\newcommand*{\@inheritStyle@S}[1]{\(\scriptstyle       #1\)}
\newcommand*{\@inheritStyle@s}[1]{\(\scriptscriptstyle #1\)}
\newcommand*{\@inheritStyle@t}[1]{#1}
\newcommand*{\inheritStyle}{\csname @inheritStyle@\@inheritStyleSwitch\endcsname}
\newcommand*{\delegateStyle}[1]{%
	\ifmmode%
		\mathchoice%
		{\edef\@inheritStyleSwitch{D}#1}%
		{\edef\@inheritStyleSwitch{T}#1}%
		{\edef\@inheritStyleSwitch{S}#1}%
		{\edef\@inheritStyleSwitch{s}#1}%
	\else%
		\edef\@inheritStyleSwitch{t}#1%
	\fi%
}
\makeatother


% \oalt command
% requires: \delegateStyle and \inheritStyle command
% usage: \oalt<…>[…]{…}{…} (cf. \alt)
% Behaves like \alt, but reserves space according to largest overlays.
% The optional argument defines the alignment inside the reserved space;
% it is one of c, l, r, s (cf. \makebox); the default is c.
\makeatletter
\newlength{\oalt@dp}
\newlength{\oalt@ht}
\newlength{\oalt@wd}
\newbox{\oalt@a}
\newbox{\oalt@b}
\newbox{\oalt@empty}
\newcommand<>*{\oalt}[3][c]{%
	\delegateStyle{%
		% based on \setto… in /usr/share/texmf-dist/tex/latex/base/latex.ltx
		\setbox\oalt@a\hbox{\inheritStyle{#2}}%
		\setbox\oalt@b\hbox{\inheritStyle{#3}}%
		\pgfmathsetlength{\oalt@dp}{max(\dp\oalt@a,\dp\oalt@b)}%
		\pgfmathsetlength{\oalt@ht}{max(\ht\oalt@a,\ht\oalt@b)}%
		\pgfmathsetlength{\oalt@wd}{max(\wd\oalt@a,\wd\oalt@b)}%
		\raisebox{0pt}[\oalt@ht][\oalt@dp]{%
			\makebox[\oalt@wd][#1]{%
				\alt#4{\unhbox\oalt@a}{\unhbox\oalt@b}%
			}%
		}%
		\setbox\oalt@a\box\oalt@empty%
		\setbox\oalt@b\box\oalt@empty%
	}%
}
\makeatother


% \otemporal command
% requires: \delegateStyle and \inheritStyle command
% usage: \otemporal<…>[…]{…}{…}{…} (cf. \temporal)
% Behaves like \temporal, but reserves space according to largest overlays.
% The optional argument defines the alignment inside the reserved space;
% it is one of c, l, r, s (cf. \makebox); the default is c.
\makeatletter
\newlength{\ot@dp}
\newlength{\ot@ht}
\newlength{\ot@wd}
\newbox{\ot@a}
\newbox{\ot@b}
\newbox{\ot@c}
\newbox{\ot@empty}
\newcommand<>*{\otemporal}[4][c]{%
	\delegateStyle{%
		% based on \setto… in /usr/share/texmf-dist/tex/latex/base/latex.ltx
		\setbox\ot@a\hbox{\inheritStyle{#2}}%
		\setbox\ot@b\hbox{\inheritStyle{#3}}%
		\setbox\ot@c\hbox{\inheritStyle{#4}}%
		\pgfmathsetlength{\ot@dp}{max(\dp\ot@a,\dp\ot@b,\dp\ot@c)}%
		\pgfmathsetlength{\ot@ht}{max(\ht\ot@a,\ht\ot@b,\ht\ot@c)}%
		\pgfmathsetlength{\ot@wd}{max(\wd\ot@a,\wd\ot@b,\wd\ot@c)}%
		\raisebox{0pt}[\ot@ht][\ot@dp]{%
			\makebox[\ot@wd][#1]{%
				\temporal#5{\unhbox\ot@a}{\unhbox\ot@b}{\unhbox\ot@c}%
			}%
		}%
		\setbox\ot@a\box\ot@empty%
		\setbox\ot@b\box\ot@empty%
		\setbox\ot@c\box\ot@empty%
	}%
}
\makeatother


% Resize delimiters like braces, brackets, etc.
% Parameters: size, left delimiter, formula, right delimiter
% Example: \delim2({\frac{1}{2}})
\newcommand*{\delim}[4]{%
	\ifcase#1%
		#2#3#4%
	\or%
		\bigl#2#3\bigr#4%
	\or%
		\Bigl#2#3\Bigr#4%
	\or%
		\biggl#2#3\biggr#4%
	\or%
		\Biggl#2#3\Biggr#4%
	\else%
		\left#2#3\right#4%
	\fi%
}


% similar to \fullcite, but using the formatting of \printbibliography
\newcommand*{\printfullcite}[1]{%
	\begin{refsection}%
		\nocite{#1}%
		\DeclareNameAlias{author}{first-last}%
		\printbibliography[heading = none]%
	\end{refsection}%
}


\colorlet{light alert}{HKS07K60}
\tikzset{alert.bg/.style={rounded corners, fill=light alert}}
\tikzset{every picture/.style={line cap=round, semithick}}
% http://tex.stackexchange.com/questions/6135/how-to-make-beamer-overlays-with-tikz-node
\tikzset{onslide/.code args={<#1>#2}{\only<#1>{\pgfkeysalso{#2}}}}
\tikzset{invisible/.code args={<#1>}{\alt<#1>{\pgfkeysalso{transparent}}{\pgfkeysalso{opaque}}}}
\tikzset{uncover/.code args={<#1>}{\alt<#1>{\pgfkeysalso{opaque}}{\pgfkeysalso{opacity=0.25}}}}
\tikzset{visible/.code args={<#1>}{\alt<#1>{\pgfkeysalso{opaque}}{\pgfkeysalso{transparent}}}}
\tikzset{vuncover/.code args=%
	{<#1><#2>}%
	{\alt<#1>%
		{\alt<#2>%
			{\pgfkeysalso{opaque}}%
			{\pgfkeysalso{opacity=0.25}}%
		}{\pgfkeysalso{transparent}}%
	}%
}

\newcommand<%
	>{\tikzhighlight}[2][]{%
	\delegateStyle{\alt#3%
		{\tikz[baseline=0, anchor=base, inner sep=0.2em, text height=, text depth=]{\node[alert.bg, #1]{\inheritStyle{#2}};}}%
		{\tikz[baseline=0, anchor=base, inner sep=0.2em, text height=, text depth=]{\node[#1, fill=none]{\inheritStyle{#2}};}}%
	}%
}

\newcommand{\mathhighlight}{\tikzhighlight}

\newcommand<>{\mhl}[2][]{\mathhighlight#3[inner sep=0.2em, #1]{#2}}


\newcommand<>{\inlineblock}[2][]{{%
	\usebeamercolor*[fg]{block body}%
	\tikzhighlight#3[fill=block body.bg, #1]{#2}%
}}


% a small letter s for plurals of abbreviations
\newcommand*{\s}{{\scriptsize s}\xspace}


\newcommand<>*{\sout}[2][opacity=0.75, ultra thick]{%
	\delegateStyle{%
		\tikz[baseline=0, anchor=base, inner sep=0, outer sep=0]{
			\useasboundingbox node (n) {\inheritStyle{#2}};
			\only#3{
				\node (h) {\inheritStyle{\ifmmode\mathstrut\else\strut\fi}};
				\draw[#1] (n.west |- {$(h.south)!0.5!(h.north)$}) -- (n.east |- {$(h.south)!0.5!(h.north)$});
			}
		}%
	}%
}


% tight style
% Sets outer sep to default inner sep and inner sep to 0.
% Use this style for nodes that are neither drawn nor filled to prevent
% unwanted growth of the bounding box.
\tikzset{tight/.style={inner sep=0, outer sep=0.3333em}}


% rounded tree edges style
% usage: rounded tree edges={⟨direction⟩}{⟨looseness⟩}{⟨strength⟩}
\tikzset{
	rounded tree edges/.style n args={3}{
	edge from parent path={
	let
		\n{direction}={#1},
		\n{looseness}={#2},
		\n{strength}={#3},
		\p1=(\tikzparentnode),
		\p2=(\tikzchildnode),
		\p3=(\n{direction}:1pt),
		\p4=(\x2 - \x1, \y2 - \y1),
		\n{dist}={veclen(\p4)},
		\p4=(\x4 / \n{dist}, \y4 / \n{dist}),
		\n{angle}={atan2(\y4, \x4)},
		\n{delta}={Mod(\n{angle} - \n{direction}, 360)},
		\n{delta}={\n{delta} > 180 ? \n{delta} - 360  : \n{delta}},
		\n{delta}={\n{delta} >  90 ?  180 - \n{delta} : \n{delta}},
		\n{delta}={\n{delta} < -90 ? -180 - \n{delta} : \n{delta}}
	in (\tikzparentnode) .. controls
		+(    \n{angle}+\n{strength}*\n{delta}:\n{looseness}*0.3915*\n{dist}) and
		+(180+\n{angle}-\n{strength}*\n{delta}:\n{looseness}*0.3915*\n{dist}) ..
		(\tikzchildnode)
	}
	}
}


% Tear out snippets from PDFs.
% Usage: \tear[…]{file.pdf}
% The optional parameter is the same as for \includegraphics.
% Useful Arguments:
%   * page=‹pagenumber›
%   * trim=‹left› ‹bottom› ‹right› ‹top›
%   * width=0.98\linewidth
\newcommand*{\tear}[2][]{%
	\begin{tikzpicture}
		\node
			[ blur shadow
			, clip
			, decorate
			, decoration=random steps
			, draw
			, inner sep=0
			, preaction={fill=white}% hide the shadow if paper is transparent
			] {\includegraphics[#1]{#2}};
	\end{tikzpicture}%
}


\makeatletter
\newcommand*{\timeline}[3][0]{%
	\ifcsname timeline@cmd@#3\endcsname%
		\@timeline[#1]{#2}{#3}%
		\PackageWarning{timeline}{redefining timeline \@backslashchar\string#3}%
	\else%
		\ifcsname#3\endcsname%
			\errmessage{Command \@backslashchar\string#3 already defined}%
		\else%
			\@timeline[#1]{#2}{#3}%
		\fi%
	\fi%
}%
\newcommand*{\@timeline}[3][0]{%
	% mark command as timeline command – they can be overwritten
	\expandafter\def\csname timeline@cmd@#3\endcsname{}%
	\setcounter{@timeline}{#1}%
	\def\timeline@cmd{#3}%
	\timeline@reset%
	\timeline@append{0}%
	\@tfor\timeline@next:=#2\do{%
		\if\timeline@next+%
			\stepcounter{@timeline}%
			\timeline@append{,\the@timeline}%
		\else\if\timeline@next-%
			\stepcounter{@timeline}%
		\else%
			%\timeline@append{\timeline@next}%
			\GenericError{}{\protect\timeline: ignoring unknown character: \timeline@next}%
		\fi\fi%
	}%
}%
% \newcommand*{\tl}[1]{%
% 	\ifcsname timeline@cmd@#1\endcsname%
% 		\csname timeline@cmd@#1\endcsname%
% 	\else%
% 		0%
% 		%\GenericError{}{\protect\tl: timeline not defined: #1}%
% 	\fi%
% }%
\newcounter{@timeline}%
\def\timeline@reset{%
	\expandafter\def\csname\timeline@cmd\endcsname{}%
}%
\def\timeline@append#1{%
	\expandafter\edef\csname\timeline@cmd\endcsname{%
		\csname\timeline@cmd\endcsname#1%
	}%
}%
\makeatother


\newcommand*{\xminus}[1]{%
	\mathrel{\tikz[baseline={([yshift=-0.25em]n.south)}, inner sep=0, outer sep=0.2em]{%
		\node (n) {\(\scriptstyle #1\)};
		\draw (n.south west) -- (n.south east);
	}}%
}
\newcommand*{\tikzrightarrow}[1]{%
	\mathrel{\tikz[baseline={([yshift=-0.25em]n.south)}, inner sep=0, outer sep=0.2em]{%
		\node (n) {\(\scriptstyle #1\)};
		\draw[->, > = Computer Modern Rightarrow, line width = 0.4pt] (n.south west) -- (n.south east);
	}}%
}


%%%%%%%%%%%%%%%%%%%%%%%%%%%%%%%%%%%%%%%%%%%%%%%%%%%%%%%%%%%%%%%%%%%%%%%%%%%%%%
% document specific commands
%%%%%%%%%%%%%%%%%%%%%%%%%%%%%%%%%%%%%%%%%%%%%%%%%%%%%%%%%%%%%%%%%%%%%%%%%%%%%%

\newcommand<>*{\mycite}[1]{\uncover#2{{\color{HKS57K100}[\cite{#1}]}}}


\newcommand{\statetree}[1]{
	\tikz
	[ anchor=base
	, baseline=(current bounding box.center)
	, level distance=2em
	, sibling distance=2em
	]{
		\matrix
		[ draw=nt
		, edge from parent/.style={draw=black}
		, inner sep=0
		, nodes={inner sep=0.2em, rounded corners=0}
		, rounded corners
		] {#1\\}
	}
}


\newcommand*{\mylargeleaf}[1]{{\LARGE\color{HKS41K70}#1}}

\definecolor{state s}{named}{HKS57K80}
\definecolor{state t}{named}{HKS41K70}
\newcommand*{\stateS}[1]{{\color{state s}#1}}
\newcommand*{\stateT}[1]{{\color{state t}#1}}

\tikzset{
	subtree/.style =
		{ fill=lightgray
		, inner sep=0.02em
		, isosceles triangle apex angle=60
		, shape=isosceles triangle
		, shape border rotate=90
		}
	, state/.style = {circle, draw, inner sep=0.1em}
	, trans/.style = {rectangle, draw}
}

\newcommand*{\srBool}{\mathbb{B}}
\newcommand*{\srProb}{ℙ}


%%%%%%%%%%%%%%%%%%%%%%%%%%%%%%%%%%%%%%%%%%%%%%%%%%%%%%%%%%%%%%%%%%%%%%%%%%%%%%
% commands for specific notations
%%%%%%%%%%%%%%%%%%%%%%%%%%%%%%%%%%%%%%%%%%%%%%%%%%%%%%%%%%%%%%%%%%%%%%%%%%%%%%

\DeclareMathOperator*{\argmax}{argmax}

\newcommand*{\cardinality}[1]{\lvert#1\rvert}
\newcommand*{\corpussize}[1]{\lvert#1\rvert}

\DeclareMathOperator{\crispOp}{crisp}
\newcommand*        {\crisp}[2][0]{\crispOp\delim{#1}({#2})}

\DeclareMathOperator{\lhsOp}{lhs}
\newcommand*{\lhs}[1]{\lhsOp(#1)}

\DeclareMathOperator{\lklhdOp}{L}
\newcommand*{\lklhd}[2]{\lklhdOp(#1 ∣ #2)}

\DeclareMathOperator{\mleOp}{mle}
\newcommand*{\mle}[2][]{%
	\ifthenelse{\isempty{#1}}{%
		\mleOp(#2)%
	}{%
		\mleOp_{#1}(#2)%
	}%
}

\DeclareMathOperator{\mrg}{merge}

% CVD: color vision deficiencies
\definecolor{CVD light red}   {HTML}{FF8080}
\definecolor{CVD light yellow}{HTML}{FFFF80}
\definecolor{CVD light green} {HTML}{40FFC0}

\definecolor{nt}{named}{HKS41K70}
\newcommand*{\nt}[1]{{\color{nt}#1}}

% set of all probability distributions over #1
\DeclareMathOperator{\pdsOp}{Pd}
\newcommand*{\pds}[1]{\pdsOp(#1)}

\DeclareMathOperator{\positionsOp}{pos}
\newcommand*{\positions}[1]{\positionsOp(#1)}

\DeclareMathOperator{\rankOp}{rk}
\newcommand*{\rank}[1]{\rankOp(#1)}

\DeclareMathOperator{\runsOp}{run}
\newcommand*{\runs}[2][]{%
	\ifthenelse%
		{\isempty{#1}}%
		{\runsOp(#2)}%
		{\runsOp_{#1}(#2)}%
}

\newcommand*{\semantics}[1]{⟦#1⟧}

\DeclareMathOperator{\splt}{split}

\newcommand*{\subtree}[2]{#1|_{#2}}

\DeclareMathOperator{\supportOp}{supp}
\newcommand*{\support}[1]{\supportOp(#1)}

\newcommand*{\symId}{\textsc{\color{gray}Id}}
\newcommand*{\symCons}{\textsc{\color{gray}Cons}}
\newcommand*{\symFlip}{\textsc{\color{gray}Flip}}
\newcommand*{\symNull}{\textsc{\color{gray}Null}}
\newcommand*{\symNullR}{\textsc{\color{gray}N\(\overline{\textsc{ull}}\)}}
\newcommand*{\symSnoc}{\textsc{\color{gray}Snoc}}

\newcommand*{\transWTA}[4][]{#3 \xrightarrow{#1} #2(#4)}

\DeclareMathOperator{\uniqueRunOp}{r}
\newcommand*{\uniqueRun}[2][]{%
	\ifthenelse%
		{\isempty{#1}}%
		{\uniqueRunOp^{#2}}%
		{\uniqueRunOp_{\!#1}^{#2}}%
}

\DeclareMathOperator{\treesOp}{T}
\newcommand*{\trees}[2][]{%
	\ifthenelse%
		{\isempty{#1}}%
		{\treesOp_{\!#2}}%
		{\treesOp_{\!#2}(#1)}%
}
\DeclareMathOperator{\treesUOp}{U}
\newcommand*{\treesU}[2][]{%
	\ifthenelse%
		{\isempty{#1}}%
		{\treesUOp_{#2}}%
		{\treesUOp_{#2}(#1)}%
}


%%%%%%%%%%%%%%%%%%%%%%%%%%%%%%%%%%%%%%%%%%%%%%%%%%%%%%%%%%%%%%%%%%%%%%%%%%%%%%
% metadata
%%%%%%%%%%%%%%%%%%%%%%%%%%%%%%%%%%%%%%%%%%%%%%%%%%%%%%%%%%%%%%%%%%%%%%%%%%%%%%

\ifstandalonebeamer\else
	\title[Defense of Dissertation]{A Formal View on Training of Weighted Tree Automata by Likelihood-Driven State Splitting and Merging}
	\subtitle{Defense of Dissertation}
\fi
\author{Toni Dietze}
\institute[TU Dresden]{%
	\href{https://www.orchid.inf.tu-dresden.de/index.en/}{Chair for Foundations of Programming}
\\	\href{https://tu-dresden.de/ing/informatik/thi}{Institute of Theoretical Computer Science}
\\	\href{https://tu-dresden.de/ing/informatik}{Faculty of Computer Science}
\\	\href{https://tu-dresden.de/}{Technische Universität Dresden}
\\	01062 Dresden, Germany
}
\date[2018-09-27]{September 27, 2018}

\begin{document}
\begin{standaloneframe}{\jobname}
	\only<.(4)>{\transfade}
	\begin{columns}
	\column{0.5\linewidth}\centering
		\setcounter{beamerpauses}{2}
		\begin{tikzpicture}
			[ inner sep=0.25em
			, rounded tree edges={-90}{1}{0.5}
			, tight
			, level distance=4em
			, level 1/.style={sibling distance=5em}
			]
			% σ(α, γ(α), β)
			\node (root) at (20em, -5em) {\(σ\)}
				child {node (leaf1) {\(α\)}
					edge from parent node[left, visible=<.(2)->] (q1) {\(q_1\)}
				}
				child {node {\(γ\)}
					child {node (leaf2) {\(α\)}
						edge from parent node[right, visible=<.(2)->] (q4) {\(q_4\)}
					}
					edge from parent node[right, visible=<.(2)->] (q2) {\(q_2\)}
				}
				child {node (leaf3) {\(β\)}
					edge from parent node[right, visible=<.(2)->] (q3) {\(q_3\)}
				};
			\node[above right=0.4em and 0 of root.north, visible=<.(2)->] {\(q_0\)};
			\begin{scope}[dashed, bend right=5, visible=<.(3)->]
				\draw[|-> ] (q1)++(160:2.9em) to node[below] {\(p_1\)} (q1);
				\draw[ -> ] (q1) to node[below] {\(p_2\)} (q2);
				\draw[ -> ] (q2) to node[below] {\(p_3\)} (q3);
				\draw[ ->|] (q3) to node[below] {\(p_4\)} +(20:2.9em);
				\draw[|->|] (leaf1)++(-1em, -1em) to node[below] {\(p_5\)} ++(2em, 0);
				\draw[|-> ] (q4)++(170:2.9em) to node[below] {\(p_6\)} (q4);
				\draw[ ->|] (q4) to node[below] {\(p_7\)} +(10:2.4em);
				\draw[|->|] (leaf2)++(-1em, -1em) to node[below] {\(p_8\)} ++(2em, 0);
				\draw[|->|] (leaf3)++(-1em, -1em) to node[below] {\(p_9\)} ++(2em, 0);
			\end{scope}
		\end{tikzpicture}

		\uncover<.(2)->{
			\otemporal<.(3)->[r]{wta}{wuta}{} transitions:
			\otemporal<.(3)->[l]{\(Q × Σ × Q^*\)}{\(Q × Σ × \operatorname{wfsa}(Q)\)}{}
		}
	\setcounter{beamerpauses}{1}
	\pause
	\column{0.5\linewidth}\centering
		\setcounter{beamerpauses}{4}
		\begin{tikzpicture}
			[ inner sep=0.25em
			, rounded tree edges={-90}{1}{0.5}
			, tight
			, level distance=3em
			, level 2/.style={sibling distance=7em}
			, level 4/.style={sibling distance=3em}
			]
			\node (root) {\(σ\)}
				child {node[gray] {\(\symCons\)}
					child {node {\(α\)}
						child {node[gray] {\(\symNull\)}
							edge from parent node[right, visible=<.(3)->] {\(p_5\)}
						}
						edge from parent node[sloped, above, visible=<.(2)->] {\(q_1\)}
					}
					child {node[gray] {\(\symCons\)}
						child {node {\(γ\)}
							child {node[gray] {\(\symCons\)}
								child {node {\(α\)}
									child {node[gray] {\(\symNull\)}
										edge from parent node[right, visible=<.(3)->] {\(p_8\)}
									}
								edge from parent node[left, visible=<.(2)->] {\(q_4\)}
								}
								child {node[gray] {\(\symNull\)}
									edge from parent node[right, visible=<.(3)->] {\(p_7\)}
								}
								edge from parent node[right, visible=<.(3)->] {\(p_6\)}
							}
							edge from parent node[sloped, above, visible=<.(2)->] {\(q_2\)}
						}
						child {node[gray] {\(\symCons\)}
							child {node {\(β\)}
								child {node[gray] {\(\symNull\)}
									edge from parent node[right, visible=<.(3)->] {\(p_9\)}
								}
								edge from parent node[left, visible=<.(2)->] {\(q_3\)}
							}
							child {node[gray] {\(\symNull\)}
								edge from parent node[right, visible=<.(3)->] {\(p_4\)}
							}
							edge from parent node[sloped, above, visible=<.(3)->] {\(p_3\)}
						}
						edge from parent node[sloped, above, visible=<.(3)->] {\(p_2\)}
					}
					edge from parent node[right, visible=<.(3)->] {\(p_1\)}
				};
			\node[above right=0.4em and 0 of root.north, visible=<.(2)->] {\(q_0\)};
		\end{tikzpicture}
	\end{columns}
\end{standaloneframe}
\end{document}

\end{frame}
\fi


\section{Right-Branching Binarization}

\begin{frame}{\secname}
	% SPDX-License-Identifier: CC-BY-4.0
% Copyright 2018 Toni Dietze
\documentclass[beamer]{standalone}
% SPDX-License-Identifier: CC-BY-4.0 OR MIT-0
% Copyright 2018 Toni Dietze
%
\usefonttheme{professionalfonts}

% LuaLaTeX specific packages
\usepackage{fontspec}
	\defaultfontfeatures{Ligatures=TeX}
\usepackage{polyglossia}
	\setdefaultlanguage{english}
\usepackage{amsmath}  % has to be loaded before unicode-math
\usepackage[math-style=ISO]{unicode-math}
	\setmathfont{Latin Modern Math}
% 	\setmathfont[range={\mathcal,\mathbfcal},StylisticSet=1]{xits-math.otf}
% 	\setmathfont[range={"029F5}]{XITS Math}  % ⧵
% 	\setmathfont[range={\mathscr,\mathbfscr},StylisticSet=1]{Latin Modern Math}  % make \mathscr use the correct font

\usepackage[noend]{algpseudocode}
	\algrenewcommand\algorithmicrequire{\textbf{Input:}}
	\algrenewcommand\algorithmicensure{\textbf{Output:}}
\usepackage[backend=biber, maxbibnames=42, maxcitenames=42, sorting=ynt, style=authoryear]{biblatex}
\usepackage{csquotes}
\usepackage{mathtools}
\usepackage{media9}
\usepackage{scalerel}
\usepackage{standalone}
\usepackage{tikz}
	\usetikzlibrary{arrows.meta}
	\usetikzlibrary{backgrounds}
	\usetikzlibrary{calc}
	\usetikzlibrary{decorations}
	\usetikzlibrary{decorations.pathmorphing}
	\usetikzlibrary{decorations.pathreplacing}
	\usetikzlibrary{fadings}
	\usetikzlibrary{fit}
	\usetikzlibrary{graphs}
	\usetikzlibrary{graphdrawing}
	\usetikzlibrary{intersections}
	\usetikzlibrary{positioning}
	\usetikzlibrary{quotes}
	\usetikzlibrary{shadows.blur}
	\usetikzlibrary{shapes.arrows}
	\usetikzlibrary{shapes.geometric}
	\usegdlibrary{trees}
\usepackage{xifthen}
\usepackage{xspace}

\usepackage{pgfplots}
	\pgfplotsset
		{ compat = 1.15
		, /pgf/number format/1000 sep = {\,}
		, /pgf/number format/assume math mode = true
		, every axis plot/.append style =
			{ mark options = {fill opacity = 0.25}
			}
		}
	\usepgfplotslibrary{groupplots}
\usepackage{pgfplotstable}

\hypersetup
	{ bookmarksopen
	, pdflang = en
	, unicode
	}


%%%%%%%%%%%%%%%%%%%%%%%%%%%%%%%%%%%%%%%%%%%%%%%%%%%%%%%%%%%%%%%%%%%%%%%%%%%%%%


% always show bad boxes
%\overfullrule=1em


%%%%%%%%%%%%%%%%%%%%%%%%%%%%%%%%%%%%%%%%%%%%%%%%%%%%%%%%%%%%%%%%%%%%%%%%%%%%%%
% biblatex
%%%%%%%%%%%%%%%%%%%%%%%%%%%%%%%%%%%%%%%%%%%%%%%%%%%%%%%%%%%%%%%%%%%%%%%%%%%%%%

\addbibresource{slides-dissertation-defense.bib}
% \renewcommand*{\finalnamedelim}{\addcomma\space}
% \setlength{\bibitemsep}{1em}
% 
\AtEveryBibitem{% Clean up the bibtex rather than editing it
 \clearlist{address}
 \clearfield{date}
 \clearfield{eprint}
 \clearfield{isbn}
 \clearfield{issn}
 \clearlist{language}
 \clearlist{location}
 \clearfield{month}
 \clearfield{series}
%  \clearfield{url}
%  \clearfield{doi}
 \clearfield{organization}

%  \ifentrytype{book}{}{% Remove stuff except for books
%   \clearfield{booktitle}
%   \clearfield{pages}
  \clearlist{publisher}
  \clearname{editor}
%  }
}
% do not print url if doi is present
% http://tex.stackexchange.com/questions/154864/biblatex-use-doi-only-if-there-is-no-url
\DeclareSourcemap{
	\maps[datatype=bibtex]{
		\map{
			\step[fieldsource=doi,final]
			\step[fieldset=url,null]
}	}	}
%
% remove qoutes around titles
\DeclareFieldFormat
	[article,inbook,incollection,inproceedings,patent,thesis,unpublished]
	{title}{#1\isdot}
% 
% \DeclareFieldFormat{url}{\mkbibacro{URL}\addcolon\addnbspace\url{#1}}
% 
% \DeclareNameAlias{sortname}{first-last}
% 
\renewbibmacro{in:}{\ifentrytype{article}{}{}}


%%%%%%%%%%%%%%%%%%%%%%%%%%%%%%%%%%%%%%%%%%%%%%%%%%%%%%%%%%%%%%%%%%%%%%%%%%%%%%
% beamer
%%%%%%%%%%%%%%%%%%%%%%%%%%%%%%%%%%%%%%%%%%%%%%%%%%%%%%%%%%%%%%%%%%%%%%%%%%%%%%

\useoutertheme{infolines}
\makeatletter
% based on
% /usr/share/texmf-dist/tex/latex/beamer/beamerouterthemeinfolines.sty
\setbeamertemplate{footline}
{%
	\leavevmode%
	\hbox{%
	\begin{beamercolorbox}[wd=.333333\paperwidth,ht=2.25ex,dp=1ex,center]{author in head/foot}%
		\usebeamerfont{author in head/foot}\insertshortauthor\expandafter\beamer@ifempty\expandafter{\beamer@shortinstitute}{}{~~(\insertshortinstitute)}
	\end{beamercolorbox}%
	\begin{beamercolorbox}[wd=.333333\paperwidth,ht=2.25ex,dp=1ex,center]{title in head/foot}%
		\usebeamerfont{title in head/foot}\insertshorttitle
	\end{beamercolorbox}%
	\begin{beamercolorbox}[wd=.333333\paperwidth,ht=2.25ex,dp=1ex,right]{date in head/foot}%
		\usebeamerfont{date in head/foot}%
		\hfill\insertshortdate\hfill\hfill%
		%\hspace*{2ex}%
		%\insertshortdate%
		%\hspace{0pt plus 1 filll}%
		%(\insertframenumber.\insertoverlaynumber{} / \insertmainframenumber)%
		%\hspace{0pt plus 1 filll}%
		\phantom{000}\llap{\insertpagenumber} / \insertpresentationendpage%
		\hspace*{2ex}%
	\end{beamercolorbox}}%
	\vskip0pt%
}
\makeatother
\useinnertheme{circles}
\beamertemplatenavigationsymbolsempty
\setbeamertemplate{bibliography item}{}
\setbeamertemplate{headline}[default]

\input{tudcolors.tex}
\setbeamercolor*{alerted text}{fg=HKS07K100}
\usecolortheme[named=HKS41K100]{structure}

\setbeamercolor*{palette primary}{use=structure,fg=white,bg=structure.fg}
\setbeamercolor*{palette secondary}{use=structure,fg=white,bg=structure.fg!80}
\setbeamercolor*{palette tertiary}{use=structure,fg=white,bg=structure.fg!60}
\setbeamercolor*{palette quaternary}{fg=white,bg=black}

\setbeamercolor*{sidebar}{use=structure,bg=structure.fg}

\setbeamercolor*{palette sidebar primary}{use=structure,fg=structure.fg!20}
\setbeamercolor*{palette sidebar secondary}{fg=white}
\setbeamercolor*{palette sidebar tertiary}{use=structure,fg=structure.fg!40}
\setbeamercolor*{palette sidebar quaternary}{fg=white}

\setbeamercolor*{titlelike}{parent=palette primary}

\setbeamercolor*{separation line}{}
\setbeamercolor*{fine separation line}{}

\setbeamercolor{block title}{use=structure,fg=white,bg=structure.fg}
\setbeamercolor{block title alerted}{use=alerted text,fg=white,bg=alerted text.fg!75!black}
\setbeamercolor{block title example}{use=example text,fg=white,bg=example text.fg!75!black}

\setbeamercolor{block body}{parent=normal text,use=block title,bg=block title.bg!10!bg}
\setbeamercolor{block body alerted}{parent=normal text,use=block title alerted,bg=block title alerted.bg!10!bg}
\setbeamercolor{block body example}{parent=normal text,use=block title example,bg=block title example.bg!10!bg}

% \setbeamertemplate{itemize items}[default]


%%%%%%%%%%%%%%%%%%%%%%%%%%%%%%%%%%%%%%%%%%%%%%%%%%%%%%%%%%%%%%%%%%%%%%%%%%%%%%
% TikZ
%%%%%%%%%%%%%%%%%%%%%%%%%%%%%%%%%%%%%%%%%%%%%%%%%%%%%%%%%%%%%%%%%%%%%%%%%%%%%%

\tikzset
	{ > = Stealth
	}


%%%%%%%%%%%%%%%%%%%%%%%%%%%%%%%%%%%%%%%%%%%%%%%%%%%%%%%%%%%%%%%%%%%%%%%%%%%%%%
% general commands and styles
%%%%%%%%%%%%%%%%%%%%%%%%%%%%%%%%%%%%%%%%%%%%%%%%%%%%%%%%%%%%%%%%%%%%%%%%%%%%%%

% \delegateStyle and \inheritStyle command
% usage: \delegateStyle{… \inheritStyle{…} …}
% example: \(X_{\delegateStyle{\fbox{\inheritStyle{X}}}}\)
% Save the current style and regain it in the argument.
% This works both for math and text mode, and can be nested.
% Acknowledgments: Based on \ThisStyle and \SavedStyle from scalerel package.
\makeatletter
\newcommand*{\@inheritStyle@D}[1]{\(\displaystyle      #1\)}
\newcommand*{\@inheritStyle@T}[1]{\(\textstyle         #1\)}
\newcommand*{\@inheritStyle@S}[1]{\(\scriptstyle       #1\)}
\newcommand*{\@inheritStyle@s}[1]{\(\scriptscriptstyle #1\)}
\newcommand*{\@inheritStyle@t}[1]{#1}
\newcommand*{\inheritStyle}{\csname @inheritStyle@\@inheritStyleSwitch\endcsname}
\newcommand*{\delegateStyle}[1]{%
	\ifmmode%
		\mathchoice%
		{\edef\@inheritStyleSwitch{D}#1}%
		{\edef\@inheritStyleSwitch{T}#1}%
		{\edef\@inheritStyleSwitch{S}#1}%
		{\edef\@inheritStyleSwitch{s}#1}%
	\else%
		\edef\@inheritStyleSwitch{t}#1%
	\fi%
}
\makeatother


% \oalt command
% requires: \delegateStyle and \inheritStyle command
% usage: \oalt<…>[…]{…}{…} (cf. \alt)
% Behaves like \alt, but reserves space according to largest overlays.
% The optional argument defines the alignment inside the reserved space;
% it is one of c, l, r, s (cf. \makebox); the default is c.
\makeatletter
\newlength{\oalt@dp}
\newlength{\oalt@ht}
\newlength{\oalt@wd}
\newbox{\oalt@a}
\newbox{\oalt@b}
\newbox{\oalt@empty}
\newcommand<>*{\oalt}[3][c]{%
	\delegateStyle{%
		% based on \setto… in /usr/share/texmf-dist/tex/latex/base/latex.ltx
		\setbox\oalt@a\hbox{\inheritStyle{#2}}%
		\setbox\oalt@b\hbox{\inheritStyle{#3}}%
		\pgfmathsetlength{\oalt@dp}{max(\dp\oalt@a,\dp\oalt@b)}%
		\pgfmathsetlength{\oalt@ht}{max(\ht\oalt@a,\ht\oalt@b)}%
		\pgfmathsetlength{\oalt@wd}{max(\wd\oalt@a,\wd\oalt@b)}%
		\raisebox{0pt}[\oalt@ht][\oalt@dp]{%
			\makebox[\oalt@wd][#1]{%
				\alt#4{\unhbox\oalt@a}{\unhbox\oalt@b}%
			}%
		}%
		\setbox\oalt@a\box\oalt@empty%
		\setbox\oalt@b\box\oalt@empty%
	}%
}
\makeatother


% \otemporal command
% requires: \delegateStyle and \inheritStyle command
% usage: \otemporal<…>[…]{…}{…}{…} (cf. \temporal)
% Behaves like \temporal, but reserves space according to largest overlays.
% The optional argument defines the alignment inside the reserved space;
% it is one of c, l, r, s (cf. \makebox); the default is c.
\makeatletter
\newlength{\ot@dp}
\newlength{\ot@ht}
\newlength{\ot@wd}
\newbox{\ot@a}
\newbox{\ot@b}
\newbox{\ot@c}
\newbox{\ot@empty}
\newcommand<>*{\otemporal}[4][c]{%
	\delegateStyle{%
		% based on \setto… in /usr/share/texmf-dist/tex/latex/base/latex.ltx
		\setbox\ot@a\hbox{\inheritStyle{#2}}%
		\setbox\ot@b\hbox{\inheritStyle{#3}}%
		\setbox\ot@c\hbox{\inheritStyle{#4}}%
		\pgfmathsetlength{\ot@dp}{max(\dp\ot@a,\dp\ot@b,\dp\ot@c)}%
		\pgfmathsetlength{\ot@ht}{max(\ht\ot@a,\ht\ot@b,\ht\ot@c)}%
		\pgfmathsetlength{\ot@wd}{max(\wd\ot@a,\wd\ot@b,\wd\ot@c)}%
		\raisebox{0pt}[\ot@ht][\ot@dp]{%
			\makebox[\ot@wd][#1]{%
				\temporal#5{\unhbox\ot@a}{\unhbox\ot@b}{\unhbox\ot@c}%
			}%
		}%
		\setbox\ot@a\box\ot@empty%
		\setbox\ot@b\box\ot@empty%
		\setbox\ot@c\box\ot@empty%
	}%
}
\makeatother


% Resize delimiters like braces, brackets, etc.
% Parameters: size, left delimiter, formula, right delimiter
% Example: \delim2({\frac{1}{2}})
\newcommand*{\delim}[4]{%
	\ifcase#1%
		#2#3#4%
	\or%
		\bigl#2#3\bigr#4%
	\or%
		\Bigl#2#3\Bigr#4%
	\or%
		\biggl#2#3\biggr#4%
	\or%
		\Biggl#2#3\Biggr#4%
	\else%
		\left#2#3\right#4%
	\fi%
}


% similar to \fullcite, but using the formatting of \printbibliography
\newcommand*{\printfullcite}[1]{%
	\begin{refsection}%
		\nocite{#1}%
		\DeclareNameAlias{author}{first-last}%
		\printbibliography[heading = none]%
	\end{refsection}%
}


\colorlet{light alert}{HKS07K60}
\tikzset{alert.bg/.style={rounded corners, fill=light alert}}
\tikzset{every picture/.style={line cap=round, semithick}}
% http://tex.stackexchange.com/questions/6135/how-to-make-beamer-overlays-with-tikz-node
\tikzset{onslide/.code args={<#1>#2}{\only<#1>{\pgfkeysalso{#2}}}}
\tikzset{invisible/.code args={<#1>}{\alt<#1>{\pgfkeysalso{transparent}}{\pgfkeysalso{opaque}}}}
\tikzset{uncover/.code args={<#1>}{\alt<#1>{\pgfkeysalso{opaque}}{\pgfkeysalso{opacity=0.25}}}}
\tikzset{visible/.code args={<#1>}{\alt<#1>{\pgfkeysalso{opaque}}{\pgfkeysalso{transparent}}}}
\tikzset{vuncover/.code args=%
	{<#1><#2>}%
	{\alt<#1>%
		{\alt<#2>%
			{\pgfkeysalso{opaque}}%
			{\pgfkeysalso{opacity=0.25}}%
		}{\pgfkeysalso{transparent}}%
	}%
}

\newcommand<%
	>{\tikzhighlight}[2][]{%
	\delegateStyle{\alt#3%
		{\tikz[baseline=0, anchor=base, inner sep=0.2em, text height=, text depth=]{\node[alert.bg, #1]{\inheritStyle{#2}};}}%
		{\tikz[baseline=0, anchor=base, inner sep=0.2em, text height=, text depth=]{\node[#1, fill=none]{\inheritStyle{#2}};}}%
	}%
}

\newcommand{\mathhighlight}{\tikzhighlight}

\newcommand<>{\mhl}[2][]{\mathhighlight#3[inner sep=0.2em, #1]{#2}}


\newcommand<>{\inlineblock}[2][]{{%
	\usebeamercolor*[fg]{block body}%
	\tikzhighlight#3[fill=block body.bg, #1]{#2}%
}}


% a small letter s for plurals of abbreviations
\newcommand*{\s}{{\scriptsize s}\xspace}


\newcommand<>*{\sout}[2][opacity=0.75, ultra thick]{%
	\delegateStyle{%
		\tikz[baseline=0, anchor=base, inner sep=0, outer sep=0]{
			\useasboundingbox node (n) {\inheritStyle{#2}};
			\only#3{
				\node (h) {\inheritStyle{\ifmmode\mathstrut\else\strut\fi}};
				\draw[#1] (n.west |- {$(h.south)!0.5!(h.north)$}) -- (n.east |- {$(h.south)!0.5!(h.north)$});
			}
		}%
	}%
}


% tight style
% Sets outer sep to default inner sep and inner sep to 0.
% Use this style for nodes that are neither drawn nor filled to prevent
% unwanted growth of the bounding box.
\tikzset{tight/.style={inner sep=0, outer sep=0.3333em}}


% rounded tree edges style
% usage: rounded tree edges={⟨direction⟩}{⟨looseness⟩}{⟨strength⟩}
\tikzset{
	rounded tree edges/.style n args={3}{
	edge from parent path={
	let
		\n{direction}={#1},
		\n{looseness}={#2},
		\n{strength}={#3},
		\p1=(\tikzparentnode),
		\p2=(\tikzchildnode),
		\p3=(\n{direction}:1pt),
		\p4=(\x2 - \x1, \y2 - \y1),
		\n{dist}={veclen(\p4)},
		\p4=(\x4 / \n{dist}, \y4 / \n{dist}),
		\n{angle}={atan2(\y4, \x4)},
		\n{delta}={Mod(\n{angle} - \n{direction}, 360)},
		\n{delta}={\n{delta} > 180 ? \n{delta} - 360  : \n{delta}},
		\n{delta}={\n{delta} >  90 ?  180 - \n{delta} : \n{delta}},
		\n{delta}={\n{delta} < -90 ? -180 - \n{delta} : \n{delta}}
	in (\tikzparentnode) .. controls
		+(    \n{angle}+\n{strength}*\n{delta}:\n{looseness}*0.3915*\n{dist}) and
		+(180+\n{angle}-\n{strength}*\n{delta}:\n{looseness}*0.3915*\n{dist}) ..
		(\tikzchildnode)
	}
	}
}


% Tear out snippets from PDFs.
% Usage: \tear[…]{file.pdf}
% The optional parameter is the same as for \includegraphics.
% Useful Arguments:
%   * page=‹pagenumber›
%   * trim=‹left› ‹bottom› ‹right› ‹top›
%   * width=0.98\linewidth
\newcommand*{\tear}[2][]{%
	\begin{tikzpicture}
		\node
			[ blur shadow
			, clip
			, decorate
			, decoration=random steps
			, draw
			, inner sep=0
			, preaction={fill=white}% hide the shadow if paper is transparent
			] {\includegraphics[#1]{#2}};
	\end{tikzpicture}%
}


\makeatletter
\newcommand*{\timeline}[3][0]{%
	\ifcsname timeline@cmd@#3\endcsname%
		\@timeline[#1]{#2}{#3}%
		\PackageWarning{timeline}{redefining timeline \@backslashchar\string#3}%
	\else%
		\ifcsname#3\endcsname%
			\errmessage{Command \@backslashchar\string#3 already defined}%
		\else%
			\@timeline[#1]{#2}{#3}%
		\fi%
	\fi%
}%
\newcommand*{\@timeline}[3][0]{%
	% mark command as timeline command – they can be overwritten
	\expandafter\def\csname timeline@cmd@#3\endcsname{}%
	\setcounter{@timeline}{#1}%
	\def\timeline@cmd{#3}%
	\timeline@reset%
	\timeline@append{0}%
	\@tfor\timeline@next:=#2\do{%
		\if\timeline@next+%
			\stepcounter{@timeline}%
			\timeline@append{,\the@timeline}%
		\else\if\timeline@next-%
			\stepcounter{@timeline}%
		\else%
			%\timeline@append{\timeline@next}%
			\GenericError{}{\protect\timeline: ignoring unknown character: \timeline@next}%
		\fi\fi%
	}%
}%
% \newcommand*{\tl}[1]{%
% 	\ifcsname timeline@cmd@#1\endcsname%
% 		\csname timeline@cmd@#1\endcsname%
% 	\else%
% 		0%
% 		%\GenericError{}{\protect\tl: timeline not defined: #1}%
% 	\fi%
% }%
\newcounter{@timeline}%
\def\timeline@reset{%
	\expandafter\def\csname\timeline@cmd\endcsname{}%
}%
\def\timeline@append#1{%
	\expandafter\edef\csname\timeline@cmd\endcsname{%
		\csname\timeline@cmd\endcsname#1%
	}%
}%
\makeatother


\newcommand*{\xminus}[1]{%
	\mathrel{\tikz[baseline={([yshift=-0.25em]n.south)}, inner sep=0, outer sep=0.2em]{%
		\node (n) {\(\scriptstyle #1\)};
		\draw (n.south west) -- (n.south east);
	}}%
}
\newcommand*{\tikzrightarrow}[1]{%
	\mathrel{\tikz[baseline={([yshift=-0.25em]n.south)}, inner sep=0, outer sep=0.2em]{%
		\node (n) {\(\scriptstyle #1\)};
		\draw[->, > = Computer Modern Rightarrow, line width = 0.4pt] (n.south west) -- (n.south east);
	}}%
}


%%%%%%%%%%%%%%%%%%%%%%%%%%%%%%%%%%%%%%%%%%%%%%%%%%%%%%%%%%%%%%%%%%%%%%%%%%%%%%
% document specific commands
%%%%%%%%%%%%%%%%%%%%%%%%%%%%%%%%%%%%%%%%%%%%%%%%%%%%%%%%%%%%%%%%%%%%%%%%%%%%%%

\newcommand<>*{\mycite}[1]{\uncover#2{{\color{HKS57K100}[\cite{#1}]}}}


\newcommand{\statetree}[1]{
	\tikz
	[ anchor=base
	, baseline=(current bounding box.center)
	, level distance=2em
	, sibling distance=2em
	]{
		\matrix
		[ draw=nt
		, edge from parent/.style={draw=black}
		, inner sep=0
		, nodes={inner sep=0.2em, rounded corners=0}
		, rounded corners
		] {#1\\}
	}
}


\newcommand*{\mylargeleaf}[1]{{\LARGE\color{HKS41K70}#1}}

\definecolor{state s}{named}{HKS57K80}
\definecolor{state t}{named}{HKS41K70}
\newcommand*{\stateS}[1]{{\color{state s}#1}}
\newcommand*{\stateT}[1]{{\color{state t}#1}}

\tikzset{
	subtree/.style =
		{ fill=lightgray
		, inner sep=0.02em
		, isosceles triangle apex angle=60
		, shape=isosceles triangle
		, shape border rotate=90
		}
	, state/.style = {circle, draw, inner sep=0.1em}
	, trans/.style = {rectangle, draw}
}

\newcommand*{\srBool}{\mathbb{B}}
\newcommand*{\srProb}{ℙ}


%%%%%%%%%%%%%%%%%%%%%%%%%%%%%%%%%%%%%%%%%%%%%%%%%%%%%%%%%%%%%%%%%%%%%%%%%%%%%%
% commands for specific notations
%%%%%%%%%%%%%%%%%%%%%%%%%%%%%%%%%%%%%%%%%%%%%%%%%%%%%%%%%%%%%%%%%%%%%%%%%%%%%%

\DeclareMathOperator*{\argmax}{argmax}

\newcommand*{\cardinality}[1]{\lvert#1\rvert}
\newcommand*{\corpussize}[1]{\lvert#1\rvert}

\DeclareMathOperator{\crispOp}{crisp}
\newcommand*        {\crisp}[2][0]{\crispOp\delim{#1}({#2})}

\DeclareMathOperator{\lhsOp}{lhs}
\newcommand*{\lhs}[1]{\lhsOp(#1)}

\DeclareMathOperator{\lklhdOp}{L}
\newcommand*{\lklhd}[2]{\lklhdOp(#1 ∣ #2)}

\DeclareMathOperator{\mleOp}{mle}
\newcommand*{\mle}[2][]{%
	\ifthenelse{\isempty{#1}}{%
		\mleOp(#2)%
	}{%
		\mleOp_{#1}(#2)%
	}%
}

\DeclareMathOperator{\mrg}{merge}

% CVD: color vision deficiencies
\definecolor{CVD light red}   {HTML}{FF8080}
\definecolor{CVD light yellow}{HTML}{FFFF80}
\definecolor{CVD light green} {HTML}{40FFC0}

\definecolor{nt}{named}{HKS41K70}
\newcommand*{\nt}[1]{{\color{nt}#1}}

% set of all probability distributions over #1
\DeclareMathOperator{\pdsOp}{Pd}
\newcommand*{\pds}[1]{\pdsOp(#1)}

\DeclareMathOperator{\positionsOp}{pos}
\newcommand*{\positions}[1]{\positionsOp(#1)}

\DeclareMathOperator{\rankOp}{rk}
\newcommand*{\rank}[1]{\rankOp(#1)}

\DeclareMathOperator{\runsOp}{run}
\newcommand*{\runs}[2][]{%
	\ifthenelse%
		{\isempty{#1}}%
		{\runsOp(#2)}%
		{\runsOp_{#1}(#2)}%
}

\newcommand*{\semantics}[1]{⟦#1⟧}

\DeclareMathOperator{\splt}{split}

\newcommand*{\subtree}[2]{#1|_{#2}}

\DeclareMathOperator{\supportOp}{supp}
\newcommand*{\support}[1]{\supportOp(#1)}

\newcommand*{\symId}{\textsc{\color{gray}Id}}
\newcommand*{\symCons}{\textsc{\color{gray}Cons}}
\newcommand*{\symFlip}{\textsc{\color{gray}Flip}}
\newcommand*{\symNull}{\textsc{\color{gray}Null}}
\newcommand*{\symNullR}{\textsc{\color{gray}N\(\overline{\textsc{ull}}\)}}
\newcommand*{\symSnoc}{\textsc{\color{gray}Snoc}}

\newcommand*{\transWTA}[4][]{#3 \xrightarrow{#1} #2(#4)}

\DeclareMathOperator{\uniqueRunOp}{r}
\newcommand*{\uniqueRun}[2][]{%
	\ifthenelse%
		{\isempty{#1}}%
		{\uniqueRunOp^{#2}}%
		{\uniqueRunOp_{\!#1}^{#2}}%
}

\DeclareMathOperator{\treesOp}{T}
\newcommand*{\trees}[2][]{%
	\ifthenelse%
		{\isempty{#1}}%
		{\treesOp_{\!#2}}%
		{\treesOp_{\!#2}(#1)}%
}
\DeclareMathOperator{\treesUOp}{U}
\newcommand*{\treesU}[2][]{%
	\ifthenelse%
		{\isempty{#1}}%
		{\treesUOp_{#2}}%
		{\treesUOp_{#2}(#1)}%
}


%%%%%%%%%%%%%%%%%%%%%%%%%%%%%%%%%%%%%%%%%%%%%%%%%%%%%%%%%%%%%%%%%%%%%%%%%%%%%%
% metadata
%%%%%%%%%%%%%%%%%%%%%%%%%%%%%%%%%%%%%%%%%%%%%%%%%%%%%%%%%%%%%%%%%%%%%%%%%%%%%%

\ifstandalonebeamer\else
	\title[Defense of Dissertation]{A Formal View on Training of Weighted Tree Automata by Likelihood-Driven State Splitting and Merging}
	\subtitle{Defense of Dissertation}
\fi
\author{Toni Dietze}
\institute[TU Dresden]{%
	\href{https://www.orchid.inf.tu-dresden.de/index.en/}{Chair for Foundations of Programming}
\\	\href{https://tu-dresden.de/ing/informatik/thi}{Institute of Theoretical Computer Science}
\\	\href{https://tu-dresden.de/ing/informatik}{Faculty of Computer Science}
\\	\href{https://tu-dresden.de/}{Technische Universität Dresden}
\\	01062 Dresden, Germany
}
\date[2018-09-27]{September 27, 2018}

\begin{document}
\begin{standaloneframe}{\jobname}
\begin{columns}[T]
\column{0.5\linewidth}\centering
	\begin{uncoverenv}<2->
	\only<5->{\scriptsize}%
	\begin{tikzpicture}
	[ anchor=base
	, inner sep=0.25em
	, level distance=3em
	, rounded tree edges={-90}{1}{0.5}
	, sibling distance=5em
	]
		\node (root) {\(a\)}
			child { node {\(\symSnoc\)}
				child { node {\(\symSnoc\)}
					child { node {\(\symSnoc\)}
						child { node {\(\symNull\)}
							edge from parent node[visible=<4->, left] {\(\stateS{p_{\alt<5->10}}\)}
						}
						child { node[subtree] {\(t'_1\)}
							edge from parent node[visible=<3->, right] {\(\stateT{q_1}\)}
						}
						edge from parent node[visible=<4->, left] {\(\stateS{p_{\alt<5->21}}\)}
					}
					child { node[subtree] {\(t'_2\)}
						edge from parent node[visible=<3->, right] {\(\stateT{q_2}\)}
					}
					edge from parent node[visible=<4->, left] {\(\stateS{p_{\alt<5->12}}\)}
				}
				child { node[subtree] {\(t'_3\)}
					edge from parent node[visible=<3->, right] {\(\stateT{q_{\alt<5->13}}\)}
				}
				edge from parent node[visible=<4->, left] {\(\stateS{p_{\alt<5->23}}\)}
			};
		\node[visible=<3->, above left=0.4em and 0 of root.north, tight] {\(\stateT{q_0}\)};
	\end{tikzpicture}
	\end{uncoverenv}
	\begin{onlyenv}<5->
	\vfill%
	\inlineblock<6->{ta \(ℳ'\)}\hfill\null
	\begin{align*}
		\stateT{q_0} & \xrightarrow{a} \stateS{p_2}
	\\
		\stateS{p_2} & \xrightarrow{\symSnoc} \stateT{q_1}\,\stateS{p_1}
	\\
		\stateS{p_1} & \xrightarrow{\symSnoc} \stateT{q_2}\,\stateS{p_2}
	\\
		\stateS{p_1} & \xrightarrow{\symNull} ε
	\end{align*}
	\end{onlyenv}
\column{0.5\linewidth}\centering
	{
	\only<5->{\scriptsize}%
	\uncover<2->{\(\mathllap{\overset{\inlineblock<6>{h_{\mathrm{r}}}}{\longmapsto}}\)}%
	\hfill%
	\begin{tikzpicture}
	[ anchor=base
	, baseline=(current bounding box.center)
	, inner sep=0.25em
	, level distance=4em
	, rounded tree edges={-90}{1}{0.5}
	, sibling distance=5em
	]
		\node (root) {\(a\)}
			child {node[subtree] {\(t_1\)}
				edge from parent node[visible=<3->, pos=0.5, left=0.2em] (q1) {\(\stateT{q_1}\)}
			}
			child {node[subtree] {\(t_2\)}
				edge from parent node[visible=<3->, pos=0.3, right] (q2) {\(\stateT{q_2}\)}
			}
			child {node[subtree] {\(t_3\)}
				edge from parent node[visible=<3->, pos=0.5, right=0.1em] (q3) {\(\stateT{q_{\alt<5->13}}\)}
			};
		\node[visible=<3->, above right=0.4em and 0 of root.north, tight] {\(\stateT{q_0}\)};
		\begin{scope}[visible=<4->, dashed, bend right=5, tight]
			\path[|-> ] (q1)++(160:2.9em) to node[below] (p0) {\(\stateS{p_{\alt<5->10}}\)} (q1);
			\path[ -> ] (q1) to node[below] (p1) {\(\stateS{p_{\alt<5->21}}\)} (q2);
			\path[ -> ] (q2) to node[below] (p2) {\(\stateS{p_{\alt<5->12}}\)} (q3);
			\path[ ->|] (q3) to node[below] (p3) {\(\stateS{p_{\alt<5->23}}\)} +(20:3.1em);
			\draw[ -> ] (p0) to (p1);
			\draw[ -> ] (p1) to (p2);
			\draw[ -> ] (p2) to (p3);
		\end{scope}
	\end{tikzpicture}
	\hfill%
	}

	\begin{onlyenv}<5->
	\vspace{5em}%
	\inlineblock<6->{uta \(ℳ\)}\hfill\null
	\begin{equation*}
		\stateT{q_0} \xrightarrow{a}
			\tikz[baseline = (current bounding box.center)]{\node[draw, rounded corners]{\tikz{
				\node[state] (p1) {\(\stateS{p_1}\)};
				\node[state] (p2) at (3em, 0) {\(\stateS{p_2}\)};
				\draw[<-] (p1) -- ++(-2em, 0);
				\draw[->] (p2) -- ++( 2em, 0);
				\draw[->] (p1) to[bend left] node[above, tight] {\(\stateT{q_1}\)} (p2);
				\draw[->] (p2) to[bend left] node[below, tight] {\(\stateT{q_2}\)} (p1);
			}};}
	\end{equation*}
	\vspace{-1em}%
	\begin{theorem}<6->
		\(⟦ℳ'⟧(h_{\mathrm{r}}^{-1}(t)) = ⟦ℳ⟧(t)\) \hfill \(t ∈ \treesU{Σ}\)
	\end{theorem}
	\end{onlyenv}
\end{columns}
\end{standaloneframe}
\end{document}

\end{frame}


\section{Extension Encoding}

\begin{frame}{\secname}
	% SPDX-License-Identifier: CC-BY-4.0
% Copyright 2018 Toni Dietze
\documentclass[beamer]{standalone}
% SPDX-License-Identifier: CC-BY-4.0 OR MIT-0
% Copyright 2018 Toni Dietze
%
\usefonttheme{professionalfonts}

% LuaLaTeX specific packages
\usepackage{fontspec}
	\defaultfontfeatures{Ligatures=TeX}
\usepackage{polyglossia}
	\setdefaultlanguage{english}
\usepackage{amsmath}  % has to be loaded before unicode-math
\usepackage[math-style=ISO]{unicode-math}
	\setmathfont{Latin Modern Math}
% 	\setmathfont[range={\mathcal,\mathbfcal},StylisticSet=1]{xits-math.otf}
% 	\setmathfont[range={"029F5}]{XITS Math}  % ⧵
% 	\setmathfont[range={\mathscr,\mathbfscr},StylisticSet=1]{Latin Modern Math}  % make \mathscr use the correct font

\usepackage[noend]{algpseudocode}
	\algrenewcommand\algorithmicrequire{\textbf{Input:}}
	\algrenewcommand\algorithmicensure{\textbf{Output:}}
\usepackage[backend=biber, maxbibnames=42, maxcitenames=42, sorting=ynt, style=authoryear]{biblatex}
\usepackage{csquotes}
\usepackage{mathtools}
\usepackage{media9}
\usepackage{scalerel}
\usepackage{standalone}
\usepackage{tikz}
	\usetikzlibrary{arrows.meta}
	\usetikzlibrary{backgrounds}
	\usetikzlibrary{calc}
	\usetikzlibrary{decorations}
	\usetikzlibrary{decorations.pathmorphing}
	\usetikzlibrary{decorations.pathreplacing}
	\usetikzlibrary{fadings}
	\usetikzlibrary{fit}
	\usetikzlibrary{graphs}
	\usetikzlibrary{graphdrawing}
	\usetikzlibrary{intersections}
	\usetikzlibrary{positioning}
	\usetikzlibrary{quotes}
	\usetikzlibrary{shadows.blur}
	\usetikzlibrary{shapes.arrows}
	\usetikzlibrary{shapes.geometric}
	\usegdlibrary{trees}
\usepackage{xifthen}
\usepackage{xspace}

\usepackage{pgfplots}
	\pgfplotsset
		{ compat = 1.15
		, /pgf/number format/1000 sep = {\,}
		, /pgf/number format/assume math mode = true
		, every axis plot/.append style =
			{ mark options = {fill opacity = 0.25}
			}
		}
	\usepgfplotslibrary{groupplots}
\usepackage{pgfplotstable}

\hypersetup
	{ bookmarksopen
	, pdflang = en
	, unicode
	}


%%%%%%%%%%%%%%%%%%%%%%%%%%%%%%%%%%%%%%%%%%%%%%%%%%%%%%%%%%%%%%%%%%%%%%%%%%%%%%


% always show bad boxes
%\overfullrule=1em


%%%%%%%%%%%%%%%%%%%%%%%%%%%%%%%%%%%%%%%%%%%%%%%%%%%%%%%%%%%%%%%%%%%%%%%%%%%%%%
% biblatex
%%%%%%%%%%%%%%%%%%%%%%%%%%%%%%%%%%%%%%%%%%%%%%%%%%%%%%%%%%%%%%%%%%%%%%%%%%%%%%

\addbibresource{slides-dissertation-defense.bib}
% \renewcommand*{\finalnamedelim}{\addcomma\space}
% \setlength{\bibitemsep}{1em}
% 
\AtEveryBibitem{% Clean up the bibtex rather than editing it
 \clearlist{address}
 \clearfield{date}
 \clearfield{eprint}
 \clearfield{isbn}
 \clearfield{issn}
 \clearlist{language}
 \clearlist{location}
 \clearfield{month}
 \clearfield{series}
%  \clearfield{url}
%  \clearfield{doi}
 \clearfield{organization}

%  \ifentrytype{book}{}{% Remove stuff except for books
%   \clearfield{booktitle}
%   \clearfield{pages}
  \clearlist{publisher}
  \clearname{editor}
%  }
}
% do not print url if doi is present
% http://tex.stackexchange.com/questions/154864/biblatex-use-doi-only-if-there-is-no-url
\DeclareSourcemap{
	\maps[datatype=bibtex]{
		\map{
			\step[fieldsource=doi,final]
			\step[fieldset=url,null]
}	}	}
%
% remove qoutes around titles
\DeclareFieldFormat
	[article,inbook,incollection,inproceedings,patent,thesis,unpublished]
	{title}{#1\isdot}
% 
% \DeclareFieldFormat{url}{\mkbibacro{URL}\addcolon\addnbspace\url{#1}}
% 
% \DeclareNameAlias{sortname}{first-last}
% 
\renewbibmacro{in:}{\ifentrytype{article}{}{}}


%%%%%%%%%%%%%%%%%%%%%%%%%%%%%%%%%%%%%%%%%%%%%%%%%%%%%%%%%%%%%%%%%%%%%%%%%%%%%%
% beamer
%%%%%%%%%%%%%%%%%%%%%%%%%%%%%%%%%%%%%%%%%%%%%%%%%%%%%%%%%%%%%%%%%%%%%%%%%%%%%%

\useoutertheme{infolines}
\makeatletter
% based on
% /usr/share/texmf-dist/tex/latex/beamer/beamerouterthemeinfolines.sty
\setbeamertemplate{footline}
{%
	\leavevmode%
	\hbox{%
	\begin{beamercolorbox}[wd=.333333\paperwidth,ht=2.25ex,dp=1ex,center]{author in head/foot}%
		\usebeamerfont{author in head/foot}\insertshortauthor\expandafter\beamer@ifempty\expandafter{\beamer@shortinstitute}{}{~~(\insertshortinstitute)}
	\end{beamercolorbox}%
	\begin{beamercolorbox}[wd=.333333\paperwidth,ht=2.25ex,dp=1ex,center]{title in head/foot}%
		\usebeamerfont{title in head/foot}\insertshorttitle
	\end{beamercolorbox}%
	\begin{beamercolorbox}[wd=.333333\paperwidth,ht=2.25ex,dp=1ex,right]{date in head/foot}%
		\usebeamerfont{date in head/foot}%
		\hfill\insertshortdate\hfill\hfill%
		%\hspace*{2ex}%
		%\insertshortdate%
		%\hspace{0pt plus 1 filll}%
		%(\insertframenumber.\insertoverlaynumber{} / \insertmainframenumber)%
		%\hspace{0pt plus 1 filll}%
		\phantom{000}\llap{\insertpagenumber} / \insertpresentationendpage%
		\hspace*{2ex}%
	\end{beamercolorbox}}%
	\vskip0pt%
}
\makeatother
\useinnertheme{circles}
\beamertemplatenavigationsymbolsempty
\setbeamertemplate{bibliography item}{}
\setbeamertemplate{headline}[default]

\input{tudcolors.tex}
\setbeamercolor*{alerted text}{fg=HKS07K100}
\usecolortheme[named=HKS41K100]{structure}

\setbeamercolor*{palette primary}{use=structure,fg=white,bg=structure.fg}
\setbeamercolor*{palette secondary}{use=structure,fg=white,bg=structure.fg!80}
\setbeamercolor*{palette tertiary}{use=structure,fg=white,bg=structure.fg!60}
\setbeamercolor*{palette quaternary}{fg=white,bg=black}

\setbeamercolor*{sidebar}{use=structure,bg=structure.fg}

\setbeamercolor*{palette sidebar primary}{use=structure,fg=structure.fg!20}
\setbeamercolor*{palette sidebar secondary}{fg=white}
\setbeamercolor*{palette sidebar tertiary}{use=structure,fg=structure.fg!40}
\setbeamercolor*{palette sidebar quaternary}{fg=white}

\setbeamercolor*{titlelike}{parent=palette primary}

\setbeamercolor*{separation line}{}
\setbeamercolor*{fine separation line}{}

\setbeamercolor{block title}{use=structure,fg=white,bg=structure.fg}
\setbeamercolor{block title alerted}{use=alerted text,fg=white,bg=alerted text.fg!75!black}
\setbeamercolor{block title example}{use=example text,fg=white,bg=example text.fg!75!black}

\setbeamercolor{block body}{parent=normal text,use=block title,bg=block title.bg!10!bg}
\setbeamercolor{block body alerted}{parent=normal text,use=block title alerted,bg=block title alerted.bg!10!bg}
\setbeamercolor{block body example}{parent=normal text,use=block title example,bg=block title example.bg!10!bg}

% \setbeamertemplate{itemize items}[default]


%%%%%%%%%%%%%%%%%%%%%%%%%%%%%%%%%%%%%%%%%%%%%%%%%%%%%%%%%%%%%%%%%%%%%%%%%%%%%%
% TikZ
%%%%%%%%%%%%%%%%%%%%%%%%%%%%%%%%%%%%%%%%%%%%%%%%%%%%%%%%%%%%%%%%%%%%%%%%%%%%%%

\tikzset
	{ > = Stealth
	}


%%%%%%%%%%%%%%%%%%%%%%%%%%%%%%%%%%%%%%%%%%%%%%%%%%%%%%%%%%%%%%%%%%%%%%%%%%%%%%
% general commands and styles
%%%%%%%%%%%%%%%%%%%%%%%%%%%%%%%%%%%%%%%%%%%%%%%%%%%%%%%%%%%%%%%%%%%%%%%%%%%%%%

% \delegateStyle and \inheritStyle command
% usage: \delegateStyle{… \inheritStyle{…} …}
% example: \(X_{\delegateStyle{\fbox{\inheritStyle{X}}}}\)
% Save the current style and regain it in the argument.
% This works both for math and text mode, and can be nested.
% Acknowledgments: Based on \ThisStyle and \SavedStyle from scalerel package.
\makeatletter
\newcommand*{\@inheritStyle@D}[1]{\(\displaystyle      #1\)}
\newcommand*{\@inheritStyle@T}[1]{\(\textstyle         #1\)}
\newcommand*{\@inheritStyle@S}[1]{\(\scriptstyle       #1\)}
\newcommand*{\@inheritStyle@s}[1]{\(\scriptscriptstyle #1\)}
\newcommand*{\@inheritStyle@t}[1]{#1}
\newcommand*{\inheritStyle}{\csname @inheritStyle@\@inheritStyleSwitch\endcsname}
\newcommand*{\delegateStyle}[1]{%
	\ifmmode%
		\mathchoice%
		{\edef\@inheritStyleSwitch{D}#1}%
		{\edef\@inheritStyleSwitch{T}#1}%
		{\edef\@inheritStyleSwitch{S}#1}%
		{\edef\@inheritStyleSwitch{s}#1}%
	\else%
		\edef\@inheritStyleSwitch{t}#1%
	\fi%
}
\makeatother


% \oalt command
% requires: \delegateStyle and \inheritStyle command
% usage: \oalt<…>[…]{…}{…} (cf. \alt)
% Behaves like \alt, but reserves space according to largest overlays.
% The optional argument defines the alignment inside the reserved space;
% it is one of c, l, r, s (cf. \makebox); the default is c.
\makeatletter
\newlength{\oalt@dp}
\newlength{\oalt@ht}
\newlength{\oalt@wd}
\newbox{\oalt@a}
\newbox{\oalt@b}
\newbox{\oalt@empty}
\newcommand<>*{\oalt}[3][c]{%
	\delegateStyle{%
		% based on \setto… in /usr/share/texmf-dist/tex/latex/base/latex.ltx
		\setbox\oalt@a\hbox{\inheritStyle{#2}}%
		\setbox\oalt@b\hbox{\inheritStyle{#3}}%
		\pgfmathsetlength{\oalt@dp}{max(\dp\oalt@a,\dp\oalt@b)}%
		\pgfmathsetlength{\oalt@ht}{max(\ht\oalt@a,\ht\oalt@b)}%
		\pgfmathsetlength{\oalt@wd}{max(\wd\oalt@a,\wd\oalt@b)}%
		\raisebox{0pt}[\oalt@ht][\oalt@dp]{%
			\makebox[\oalt@wd][#1]{%
				\alt#4{\unhbox\oalt@a}{\unhbox\oalt@b}%
			}%
		}%
		\setbox\oalt@a\box\oalt@empty%
		\setbox\oalt@b\box\oalt@empty%
	}%
}
\makeatother


% \otemporal command
% requires: \delegateStyle and \inheritStyle command
% usage: \otemporal<…>[…]{…}{…}{…} (cf. \temporal)
% Behaves like \temporal, but reserves space according to largest overlays.
% The optional argument defines the alignment inside the reserved space;
% it is one of c, l, r, s (cf. \makebox); the default is c.
\makeatletter
\newlength{\ot@dp}
\newlength{\ot@ht}
\newlength{\ot@wd}
\newbox{\ot@a}
\newbox{\ot@b}
\newbox{\ot@c}
\newbox{\ot@empty}
\newcommand<>*{\otemporal}[4][c]{%
	\delegateStyle{%
		% based on \setto… in /usr/share/texmf-dist/tex/latex/base/latex.ltx
		\setbox\ot@a\hbox{\inheritStyle{#2}}%
		\setbox\ot@b\hbox{\inheritStyle{#3}}%
		\setbox\ot@c\hbox{\inheritStyle{#4}}%
		\pgfmathsetlength{\ot@dp}{max(\dp\ot@a,\dp\ot@b,\dp\ot@c)}%
		\pgfmathsetlength{\ot@ht}{max(\ht\ot@a,\ht\ot@b,\ht\ot@c)}%
		\pgfmathsetlength{\ot@wd}{max(\wd\ot@a,\wd\ot@b,\wd\ot@c)}%
		\raisebox{0pt}[\ot@ht][\ot@dp]{%
			\makebox[\ot@wd][#1]{%
				\temporal#5{\unhbox\ot@a}{\unhbox\ot@b}{\unhbox\ot@c}%
			}%
		}%
		\setbox\ot@a\box\ot@empty%
		\setbox\ot@b\box\ot@empty%
		\setbox\ot@c\box\ot@empty%
	}%
}
\makeatother


% Resize delimiters like braces, brackets, etc.
% Parameters: size, left delimiter, formula, right delimiter
% Example: \delim2({\frac{1}{2}})
\newcommand*{\delim}[4]{%
	\ifcase#1%
		#2#3#4%
	\or%
		\bigl#2#3\bigr#4%
	\or%
		\Bigl#2#3\Bigr#4%
	\or%
		\biggl#2#3\biggr#4%
	\or%
		\Biggl#2#3\Biggr#4%
	\else%
		\left#2#3\right#4%
	\fi%
}


% similar to \fullcite, but using the formatting of \printbibliography
\newcommand*{\printfullcite}[1]{%
	\begin{refsection}%
		\nocite{#1}%
		\DeclareNameAlias{author}{first-last}%
		\printbibliography[heading = none]%
	\end{refsection}%
}


\colorlet{light alert}{HKS07K60}
\tikzset{alert.bg/.style={rounded corners, fill=light alert}}
\tikzset{every picture/.style={line cap=round, semithick}}
% http://tex.stackexchange.com/questions/6135/how-to-make-beamer-overlays-with-tikz-node
\tikzset{onslide/.code args={<#1>#2}{\only<#1>{\pgfkeysalso{#2}}}}
\tikzset{invisible/.code args={<#1>}{\alt<#1>{\pgfkeysalso{transparent}}{\pgfkeysalso{opaque}}}}
\tikzset{uncover/.code args={<#1>}{\alt<#1>{\pgfkeysalso{opaque}}{\pgfkeysalso{opacity=0.25}}}}
\tikzset{visible/.code args={<#1>}{\alt<#1>{\pgfkeysalso{opaque}}{\pgfkeysalso{transparent}}}}
\tikzset{vuncover/.code args=%
	{<#1><#2>}%
	{\alt<#1>%
		{\alt<#2>%
			{\pgfkeysalso{opaque}}%
			{\pgfkeysalso{opacity=0.25}}%
		}{\pgfkeysalso{transparent}}%
	}%
}

\newcommand<%
	>{\tikzhighlight}[2][]{%
	\delegateStyle{\alt#3%
		{\tikz[baseline=0, anchor=base, inner sep=0.2em, text height=, text depth=]{\node[alert.bg, #1]{\inheritStyle{#2}};}}%
		{\tikz[baseline=0, anchor=base, inner sep=0.2em, text height=, text depth=]{\node[#1, fill=none]{\inheritStyle{#2}};}}%
	}%
}

\newcommand{\mathhighlight}{\tikzhighlight}

\newcommand<>{\mhl}[2][]{\mathhighlight#3[inner sep=0.2em, #1]{#2}}


\newcommand<>{\inlineblock}[2][]{{%
	\usebeamercolor*[fg]{block body}%
	\tikzhighlight#3[fill=block body.bg, #1]{#2}%
}}


% a small letter s for plurals of abbreviations
\newcommand*{\s}{{\scriptsize s}\xspace}


\newcommand<>*{\sout}[2][opacity=0.75, ultra thick]{%
	\delegateStyle{%
		\tikz[baseline=0, anchor=base, inner sep=0, outer sep=0]{
			\useasboundingbox node (n) {\inheritStyle{#2}};
			\only#3{
				\node (h) {\inheritStyle{\ifmmode\mathstrut\else\strut\fi}};
				\draw[#1] (n.west |- {$(h.south)!0.5!(h.north)$}) -- (n.east |- {$(h.south)!0.5!(h.north)$});
			}
		}%
	}%
}


% tight style
% Sets outer sep to default inner sep and inner sep to 0.
% Use this style for nodes that are neither drawn nor filled to prevent
% unwanted growth of the bounding box.
\tikzset{tight/.style={inner sep=0, outer sep=0.3333em}}


% rounded tree edges style
% usage: rounded tree edges={⟨direction⟩}{⟨looseness⟩}{⟨strength⟩}
\tikzset{
	rounded tree edges/.style n args={3}{
	edge from parent path={
	let
		\n{direction}={#1},
		\n{looseness}={#2},
		\n{strength}={#3},
		\p1=(\tikzparentnode),
		\p2=(\tikzchildnode),
		\p3=(\n{direction}:1pt),
		\p4=(\x2 - \x1, \y2 - \y1),
		\n{dist}={veclen(\p4)},
		\p4=(\x4 / \n{dist}, \y4 / \n{dist}),
		\n{angle}={atan2(\y4, \x4)},
		\n{delta}={Mod(\n{angle} - \n{direction}, 360)},
		\n{delta}={\n{delta} > 180 ? \n{delta} - 360  : \n{delta}},
		\n{delta}={\n{delta} >  90 ?  180 - \n{delta} : \n{delta}},
		\n{delta}={\n{delta} < -90 ? -180 - \n{delta} : \n{delta}}
	in (\tikzparentnode) .. controls
		+(    \n{angle}+\n{strength}*\n{delta}:\n{looseness}*0.3915*\n{dist}) and
		+(180+\n{angle}-\n{strength}*\n{delta}:\n{looseness}*0.3915*\n{dist}) ..
		(\tikzchildnode)
	}
	}
}


% Tear out snippets from PDFs.
% Usage: \tear[…]{file.pdf}
% The optional parameter is the same as for \includegraphics.
% Useful Arguments:
%   * page=‹pagenumber›
%   * trim=‹left› ‹bottom› ‹right› ‹top›
%   * width=0.98\linewidth
\newcommand*{\tear}[2][]{%
	\begin{tikzpicture}
		\node
			[ blur shadow
			, clip
			, decorate
			, decoration=random steps
			, draw
			, inner sep=0
			, preaction={fill=white}% hide the shadow if paper is transparent
			] {\includegraphics[#1]{#2}};
	\end{tikzpicture}%
}


\makeatletter
\newcommand*{\timeline}[3][0]{%
	\ifcsname timeline@cmd@#3\endcsname%
		\@timeline[#1]{#2}{#3}%
		\PackageWarning{timeline}{redefining timeline \@backslashchar\string#3}%
	\else%
		\ifcsname#3\endcsname%
			\errmessage{Command \@backslashchar\string#3 already defined}%
		\else%
			\@timeline[#1]{#2}{#3}%
		\fi%
	\fi%
}%
\newcommand*{\@timeline}[3][0]{%
	% mark command as timeline command – they can be overwritten
	\expandafter\def\csname timeline@cmd@#3\endcsname{}%
	\setcounter{@timeline}{#1}%
	\def\timeline@cmd{#3}%
	\timeline@reset%
	\timeline@append{0}%
	\@tfor\timeline@next:=#2\do{%
		\if\timeline@next+%
			\stepcounter{@timeline}%
			\timeline@append{,\the@timeline}%
		\else\if\timeline@next-%
			\stepcounter{@timeline}%
		\else%
			%\timeline@append{\timeline@next}%
			\GenericError{}{\protect\timeline: ignoring unknown character: \timeline@next}%
		\fi\fi%
	}%
}%
% \newcommand*{\tl}[1]{%
% 	\ifcsname timeline@cmd@#1\endcsname%
% 		\csname timeline@cmd@#1\endcsname%
% 	\else%
% 		0%
% 		%\GenericError{}{\protect\tl: timeline not defined: #1}%
% 	\fi%
% }%
\newcounter{@timeline}%
\def\timeline@reset{%
	\expandafter\def\csname\timeline@cmd\endcsname{}%
}%
\def\timeline@append#1{%
	\expandafter\edef\csname\timeline@cmd\endcsname{%
		\csname\timeline@cmd\endcsname#1%
	}%
}%
\makeatother


\newcommand*{\xminus}[1]{%
	\mathrel{\tikz[baseline={([yshift=-0.25em]n.south)}, inner sep=0, outer sep=0.2em]{%
		\node (n) {\(\scriptstyle #1\)};
		\draw (n.south west) -- (n.south east);
	}}%
}
\newcommand*{\tikzrightarrow}[1]{%
	\mathrel{\tikz[baseline={([yshift=-0.25em]n.south)}, inner sep=0, outer sep=0.2em]{%
		\node (n) {\(\scriptstyle #1\)};
		\draw[->, > = Computer Modern Rightarrow, line width = 0.4pt] (n.south west) -- (n.south east);
	}}%
}


%%%%%%%%%%%%%%%%%%%%%%%%%%%%%%%%%%%%%%%%%%%%%%%%%%%%%%%%%%%%%%%%%%%%%%%%%%%%%%
% document specific commands
%%%%%%%%%%%%%%%%%%%%%%%%%%%%%%%%%%%%%%%%%%%%%%%%%%%%%%%%%%%%%%%%%%%%%%%%%%%%%%

\newcommand<>*{\mycite}[1]{\uncover#2{{\color{HKS57K100}[\cite{#1}]}}}


\newcommand{\statetree}[1]{
	\tikz
	[ anchor=base
	, baseline=(current bounding box.center)
	, level distance=2em
	, sibling distance=2em
	]{
		\matrix
		[ draw=nt
		, edge from parent/.style={draw=black}
		, inner sep=0
		, nodes={inner sep=0.2em, rounded corners=0}
		, rounded corners
		] {#1\\}
	}
}


\newcommand*{\mylargeleaf}[1]{{\LARGE\color{HKS41K70}#1}}

\definecolor{state s}{named}{HKS57K80}
\definecolor{state t}{named}{HKS41K70}
\newcommand*{\stateS}[1]{{\color{state s}#1}}
\newcommand*{\stateT}[1]{{\color{state t}#1}}

\tikzset{
	subtree/.style =
		{ fill=lightgray
		, inner sep=0.02em
		, isosceles triangle apex angle=60
		, shape=isosceles triangle
		, shape border rotate=90
		}
	, state/.style = {circle, draw, inner sep=0.1em}
	, trans/.style = {rectangle, draw}
}

\newcommand*{\srBool}{\mathbb{B}}
\newcommand*{\srProb}{ℙ}


%%%%%%%%%%%%%%%%%%%%%%%%%%%%%%%%%%%%%%%%%%%%%%%%%%%%%%%%%%%%%%%%%%%%%%%%%%%%%%
% commands for specific notations
%%%%%%%%%%%%%%%%%%%%%%%%%%%%%%%%%%%%%%%%%%%%%%%%%%%%%%%%%%%%%%%%%%%%%%%%%%%%%%

\DeclareMathOperator*{\argmax}{argmax}

\newcommand*{\cardinality}[1]{\lvert#1\rvert}
\newcommand*{\corpussize}[1]{\lvert#1\rvert}

\DeclareMathOperator{\crispOp}{crisp}
\newcommand*        {\crisp}[2][0]{\crispOp\delim{#1}({#2})}

\DeclareMathOperator{\lhsOp}{lhs}
\newcommand*{\lhs}[1]{\lhsOp(#1)}

\DeclareMathOperator{\lklhdOp}{L}
\newcommand*{\lklhd}[2]{\lklhdOp(#1 ∣ #2)}

\DeclareMathOperator{\mleOp}{mle}
\newcommand*{\mle}[2][]{%
	\ifthenelse{\isempty{#1}}{%
		\mleOp(#2)%
	}{%
		\mleOp_{#1}(#2)%
	}%
}

\DeclareMathOperator{\mrg}{merge}

% CVD: color vision deficiencies
\definecolor{CVD light red}   {HTML}{FF8080}
\definecolor{CVD light yellow}{HTML}{FFFF80}
\definecolor{CVD light green} {HTML}{40FFC0}

\definecolor{nt}{named}{HKS41K70}
\newcommand*{\nt}[1]{{\color{nt}#1}}

% set of all probability distributions over #1
\DeclareMathOperator{\pdsOp}{Pd}
\newcommand*{\pds}[1]{\pdsOp(#1)}

\DeclareMathOperator{\positionsOp}{pos}
\newcommand*{\positions}[1]{\positionsOp(#1)}

\DeclareMathOperator{\rankOp}{rk}
\newcommand*{\rank}[1]{\rankOp(#1)}

\DeclareMathOperator{\runsOp}{run}
\newcommand*{\runs}[2][]{%
	\ifthenelse%
		{\isempty{#1}}%
		{\runsOp(#2)}%
		{\runsOp_{#1}(#2)}%
}

\newcommand*{\semantics}[1]{⟦#1⟧}

\DeclareMathOperator{\splt}{split}

\newcommand*{\subtree}[2]{#1|_{#2}}

\DeclareMathOperator{\supportOp}{supp}
\newcommand*{\support}[1]{\supportOp(#1)}

\newcommand*{\symId}{\textsc{\color{gray}Id}}
\newcommand*{\symCons}{\textsc{\color{gray}Cons}}
\newcommand*{\symFlip}{\textsc{\color{gray}Flip}}
\newcommand*{\symNull}{\textsc{\color{gray}Null}}
\newcommand*{\symNullR}{\textsc{\color{gray}N\(\overline{\textsc{ull}}\)}}
\newcommand*{\symSnoc}{\textsc{\color{gray}Snoc}}

\newcommand*{\transWTA}[4][]{#3 \xrightarrow{#1} #2(#4)}

\DeclareMathOperator{\uniqueRunOp}{r}
\newcommand*{\uniqueRun}[2][]{%
	\ifthenelse%
		{\isempty{#1}}%
		{\uniqueRunOp^{#2}}%
		{\uniqueRunOp_{\!#1}^{#2}}%
}

\DeclareMathOperator{\treesOp}{T}
\newcommand*{\trees}[2][]{%
	\ifthenelse%
		{\isempty{#1}}%
		{\treesOp_{\!#2}}%
		{\treesOp_{\!#2}(#1)}%
}
\DeclareMathOperator{\treesUOp}{U}
\newcommand*{\treesU}[2][]{%
	\ifthenelse%
		{\isempty{#1}}%
		{\treesUOp_{#2}}%
		{\treesUOp_{#2}(#1)}%
}


%%%%%%%%%%%%%%%%%%%%%%%%%%%%%%%%%%%%%%%%%%%%%%%%%%%%%%%%%%%%%%%%%%%%%%%%%%%%%%
% metadata
%%%%%%%%%%%%%%%%%%%%%%%%%%%%%%%%%%%%%%%%%%%%%%%%%%%%%%%%%%%%%%%%%%%%%%%%%%%%%%

\ifstandalonebeamer\else
	\title[Defense of Dissertation]{A Formal View on Training of Weighted Tree Automata by Likelihood-Driven State Splitting and Merging}
	\subtitle{Defense of Dissertation}
\fi
\author{Toni Dietze}
\institute[TU Dresden]{%
	\href{https://www.orchid.inf.tu-dresden.de/index.en/}{Chair for Foundations of Programming}
\\	\href{https://tu-dresden.de/ing/informatik/thi}{Institute of Theoretical Computer Science}
\\	\href{https://tu-dresden.de/ing/informatik}{Faculty of Computer Science}
\\	\href{https://tu-dresden.de/}{Technische Universität Dresden}
\\	01062 Dresden, Germany
}
\date[2018-09-27]{September 27, 2018}

\begin{document}
\begin{standaloneframe}{\jobname}
	\tikzset{mytree/.style=
		{ anchor=base
		, baseline=0
		, every node/.style={minimum width=0.9em, tight}
		, sibling distance=2em
		, text depth=0.2em
		, text height=0.7em
	}}
	\begin{columns}[onlytextwidth]
	\column{0.3\linewidth}\centering
		unranked tree

		\begin{tikzpicture}[mytree]
			\node {\(σ\)}
				child { node {\(α\)}
				}
				child { node {\(γ\)}
					child { node {\(α\)}
					}
				}
				child { node {\(β\)}
				};
		\end{tikzpicture}
		\pause
	\column{0.7\linewidth}\centering
		\alt<-.(4)>
			{binarizing the tree}
			{extension encoding}

		\begin{tikzpicture}
		[ mytree
		, onslide={<.(2)->
			  level 1/.style={sibling distance=4em}
			, level 2/.style={sibling distance=4em}
			, level 3/.style={sibling distance=2em}
			}
		]
			\path (-5.6em, -7.4em) rectangle (2.6em, 0);

			\node<+> {\(σ\)}
			child { node {\(α\)}
			}
			child { node {\(γ\)}
				child { node {\(α\)}
				}
			}
			child { node {\(β\)}
			};

			\node<+> {\(\$\)}
			child { node {\(σ\)}
				child { node {\(α\)}
				}
				child { node {\(γ\)}
					child { node {\(α\)}
					}
				}
			}
			child { node {\(β\)}
			};

			\node<+> {\(\$\)}
			child { node {\(\$\)}
				child { node {\(σ\)}
					child { node {\(α\)}
					}
				}
				child { node {\(γ\)}
					child { node {\(α\)}
					}
				}
			}
			child { node {\(β\)}
			};

			\node<+> {\(\$\)}
			child { node {\(\$\)}
				child { node {\(\$\)}
					child { node {\(σ\)}
					}
					child { node {\(α\)}
					}
				}
				child { node {\(γ\)}
					child { node {\(α\)}
					}
				}
			}
			child { node {\(β\)}
			};

			\node<+> {\(\$\)}
			child { node {\(\$\)}
				child { node {\(\$\)}
					child { node {\(σ\)}
					}
					child { node {\(α\)}
					}
				}
				child { node {\(\$\)}
					child { node {\(γ\)}
					}
					child { node {\(α\)}
					}
				}
			}
			child { node {\(β\)}
			};
		\end{tikzpicture}

		\mycite<.->{2004CarmeNiehrenTommasi}

% 		% first-child-next-sibling encoding
% 		\begin{tikzpicture}[mytree, sibling distance=4em]
% 			\node {\(σ\)}
% 				child { node {\(α\)}
% 					child[missing]
% 					child { node {\(γ\)}
% 						child { node {\(α\)}
% 						}
% 						child { node {\(β\)}
% 						}
% 					}
% 				}
% 				child[missing];
% 		\end{tikzpicture}
	\end{columns}
\end{standaloneframe}
\end{document}

\end{frame}


\end{document}

% kate: default-dictionary en
